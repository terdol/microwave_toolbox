%% Generated by Sphinx.
\def\sphinxdocclass{report}
\documentclass[letterpaper,10pt,english]{sphinxmanual}
\ifdefined\pdfpxdimen
   \let\sphinxpxdimen\pdfpxdimen\else\newdimen\sphinxpxdimen
\fi \sphinxpxdimen=.75bp\relax
\ifdefined\pdfimageresolution
    \pdfimageresolution= \numexpr \dimexpr1in\relax/\sphinxpxdimen\relax
\fi
%% let collapsible pdf bookmarks panel have high depth per default
\PassOptionsToPackage{bookmarksdepth=5}{hyperref}

\PassOptionsToPackage{warn}{textcomp}
\usepackage[utf8]{inputenc}
\ifdefined\DeclareUnicodeCharacter
% support both utf8 and utf8x syntaxes
  \ifdefined\DeclareUnicodeCharacterAsOptional
    \def\sphinxDUC#1{\DeclareUnicodeCharacter{"#1}}
  \else
    \let\sphinxDUC\DeclareUnicodeCharacter
  \fi
  \sphinxDUC{00A0}{\nobreakspace}
  \sphinxDUC{2500}{\sphinxunichar{2500}}
  \sphinxDUC{2502}{\sphinxunichar{2502}}
  \sphinxDUC{2514}{\sphinxunichar{2514}}
  \sphinxDUC{251C}{\sphinxunichar{251C}}
  \sphinxDUC{2572}{\textbackslash}
\fi
\usepackage{cmap}
\usepackage[T1]{fontenc}
\usepackage{amsmath,amssymb,amstext}
\usepackage{babel}



\usepackage{tgtermes}
\usepackage{tgheros}
\renewcommand{\ttdefault}{txtt}



\usepackage[Bjarne]{fncychap}
\usepackage{sphinx}

\fvset{fontsize=auto}
\usepackage{geometry}


% Include hyperref last.
\usepackage{hyperref}
% Fix anchor placement for figures with captions.
\usepackage{hypcap}% it must be loaded after hyperref.
% Set up styles of URL: it should be placed after hyperref.
\urlstyle{same}


\usepackage{sphinxmessages}
\setcounter{tocdepth}{3}
\setcounter{secnumdepth}{3}


\title{spfile}
\date{Jul 06, 2022}
\release{0.1}
\author{Tuncay Erdöl}
\newcommand{\sphinxlogo}{\vbox{}}
\renewcommand{\releasename}{Release}
\makeindex
\begin{document}

\ifdefined\shorthandoff
  \ifnum\catcode`\=\string=\active\shorthandoff{=}\fi
  \ifnum\catcode`\"=\active\shorthandoff{"}\fi
\fi

\pagestyle{empty}
\sphinxmaketitle
\pagestyle{plain}
\sphinxtableofcontents
\pagestyle{normal}
\phantomsection\label{\detokenize{index::doc}}


\sphinxstepscope


\chapter{touchstone module}
\label{\detokenize{touchstone:module-touchstone}}\label{\detokenize{touchstone:touchstone-module}}\label{\detokenize{touchstone::doc}}\index{module@\spxentry{module}!touchstone@\spxentry{touchstone}}\index{touchstone@\spxentry{touchstone}!module@\spxentry{module}}\index{average\_networks() (in module touchstone)@\spxentry{average\_networks()}\spxextra{in module touchstone}}

\begin{fulllineitems}
\phantomsection\label{\detokenize{touchstone:touchstone.average_networks}}
\pysigstartsignatures
\pysiglinewithargsret{\sphinxcode{\sphinxupquote{touchstone.}}\sphinxbfcode{\sphinxupquote{average\_networks}}}{\emph{\DUrole{n}{networks}}}{}
\pysigstopsignatures
\end{fulllineitems}

\index{cascade\_2ports() (in module touchstone)@\spxentry{cascade\_2ports()}\spxextra{in module touchstone}}

\begin{fulllineitems}
\phantomsection\label{\detokenize{touchstone:touchstone.cascade_2ports}}
\pysigstartsignatures
\pysiglinewithargsret{\sphinxcode{\sphinxupquote{touchstone.}}\sphinxbfcode{\sphinxupquote{cascade\_2ports}}}{\emph{\DUrole{n}{filenames}}}{}
\pysigstopsignatures
\end{fulllineitems}

\index{extract\_gamma\_ereff() (in module touchstone)@\spxentry{extract\_gamma\_ereff()}\spxextra{in module touchstone}}

\begin{fulllineitems}
\phantomsection\label{\detokenize{touchstone:touchstone.extract_gamma_ereff}}
\pysigstartsignatures
\pysiglinewithargsret{\sphinxcode{\sphinxupquote{touchstone.}}\sphinxbfcode{\sphinxupquote{extract\_gamma\_ereff}}}{\emph{\DUrole{n}{filename\_long\_line}}, \emph{\DUrole{n}{filename\_short\_line}}, \emph{\DUrole{n}{dL}}, \emph{\DUrole{n}{sm}\DUrole{o}{=}\DUrole{default_value}{1}}}{}
\pysigstopsignatures
\sphinxAtStartPar
Extraction of complex propagation constant (gamma) and complex effective permittivity from the S\sphinxhyphen{}Parameters of 2 uniform transmission lines with different lengths.
\begin{quote}\begin{description}
\sphinxlineitem{Parameters}\begin{itemize}
\item {} 
\sphinxAtStartPar
\sphinxstyleliteralstrong{\sphinxupquote{filename\_long\_line}} (\sphinxstyleliteralemphasis{\sphinxupquote{str}}) \textendash{} S\sphinxhyphen{}Parameter filename of longer line.

\item {} 
\sphinxAtStartPar
\sphinxstyleliteralstrong{\sphinxupquote{filename\_short\_line}} (\sphinxstyleliteralemphasis{\sphinxupquote{str}}) \textendash{} S\sphinxhyphen{}Parameter filename of shorter line. If None is given, only long line is used for extraction.

\item {} 
\sphinxAtStartPar
\sphinxstyleliteralstrong{\sphinxupquote{dL}} (\sphinxstyleliteralemphasis{\sphinxupquote{float}}) \textendash{} Difference of lengths of two lines (positive in meter). If filename\_short\_line is None, then this is the length of long line.

\item {} 
\sphinxAtStartPar
\sphinxstyleliteralstrong{\sphinxupquote{sm}} (\sphinxstyleliteralemphasis{\sphinxupquote{int}}\sphinxstyleliteralemphasis{\sphinxupquote{, }}\sphinxstyleliteralemphasis{\sphinxupquote{optional}}) \textendash{} If this is larger than 1, this is used as number of points for smoothing. Defaults to 1.

\end{itemize}

\sphinxlineitem{Returns}
\sphinxAtStartPar
tuple of two complex numpy arrays (gamma, er\_eff).

\sphinxlineitem{Return type}
\sphinxAtStartPar
tuple

\end{description}\end{quote}

\end{fulllineitems}

\index{extract\_gamma\_ereff\_all() (in module touchstone)@\spxentry{extract\_gamma\_ereff\_all()}\spxextra{in module touchstone}}

\begin{fulllineitems}
\phantomsection\label{\detokenize{touchstone:touchstone.extract_gamma_ereff_all}}
\pysigstartsignatures
\pysiglinewithargsret{\sphinxcode{\sphinxupquote{touchstone.}}\sphinxbfcode{\sphinxupquote{extract\_gamma\_ereff\_all}}}{\emph{\DUrole{n}{files}}, \emph{\DUrole{n}{Ls}}, \emph{\DUrole{n}{sm}\DUrole{o}{=}\DUrole{default_value}{1}}}{}
\pysigstopsignatures
\sphinxAtStartPar
Extraction of average complex propagation constant (gamma) and complex effective permittivity from the S\sphinxhyphen{}Parameters of multiple uniform transmission lines with different lengths.
\begin{quote}\begin{description}
\sphinxlineitem{Parameters}\begin{itemize}
\item {} 
\sphinxAtStartPar
\sphinxstyleliteralstrong{\sphinxupquote{files}} (\sphinxstyleliteralemphasis{\sphinxupquote{list}}) \textendash{} List of S\sphinxhyphen{}Parameter filenames of transmission lines.

\item {} 
\sphinxAtStartPar
\sphinxstyleliteralstrong{\sphinxupquote{Ls}} (\sphinxstyleliteralemphasis{\sphinxupquote{list}}) \textendash{} List of lengths of transmission lines in the same order as \sphinxstyleemphasis{files} parameter.

\item {} 
\sphinxAtStartPar
\sphinxstyleliteralstrong{\sphinxupquote{sm}} (\sphinxstyleliteralemphasis{\sphinxupquote{int}}\sphinxstyleliteralemphasis{\sphinxupquote{, }}\sphinxstyleliteralemphasis{\sphinxupquote{optional}}) \textendash{} If this is larger than 1, this is used as number of points for smoothing. Defaults to 1.

\end{itemize}

\sphinxlineitem{Returns}
\sphinxAtStartPar
tuple of two complex numpy arrays (gamma, er\_eff).

\sphinxlineitem{Return type}
\sphinxAtStartPar
tuple

\end{description}\end{quote}

\end{fulllineitems}

\index{extract\_rlgc() (in module touchstone)@\spxentry{extract\_rlgc()}\spxextra{in module touchstone}}

\begin{fulllineitems}
\phantomsection\label{\detokenize{touchstone:touchstone.extract_rlgc}}
\pysigstartsignatures
\pysiglinewithargsret{\sphinxcode{\sphinxupquote{touchstone.}}\sphinxbfcode{\sphinxupquote{extract\_rlgc}}}{\emph{\DUrole{n}{spr}}, \emph{\DUrole{n}{length}}}{}
\pysigstopsignatures
\sphinxAtStartPar
Extraction of RLGC parameters from S\sphinxhyphen{}parameters of a uniform transmission line.
\begin{quote}\begin{description}
\sphinxlineitem{Parameters}\begin{itemize}
\item {} 
\sphinxAtStartPar
\sphinxstyleliteralstrong{\sphinxupquote{spr}} (\sphinxstyleliteralemphasis{\sphinxupquote{SPFILE}}) \textendash{} SPFILE object of two port network.

\item {} 
\sphinxAtStartPar
\sphinxstyleliteralstrong{\sphinxupquote{length}} (\sphinxstyleliteralemphasis{\sphinxupquote{float}}) \textendash{} Length of transmission line.

\end{itemize}

\sphinxlineitem{Returns}
\sphinxAtStartPar
tuple of two complex numpy arrays (Inductance per unit length, Characteristic impedance of the line).

\sphinxlineitem{Return type}
\sphinxAtStartPar
tuple

\end{description}\end{quote}

\end{fulllineitems}

\index{generate\_multiport\_spfile() (in module touchstone)@\spxentry{generate\_multiport\_spfile()}\spxextra{in module touchstone}}

\begin{fulllineitems}
\phantomsection\label{\detokenize{touchstone:touchstone.generate_multiport_spfile}}
\pysigstartsignatures
\pysiglinewithargsret{\sphinxcode{\sphinxupquote{touchstone.}}\sphinxbfcode{\sphinxupquote{generate\_multiport\_spfile}}}{\emph{\DUrole{n}{conf\_filename}}, \emph{\DUrole{n}{output\_filename}}}{}
\pysigstopsignatures
\sphinxAtStartPar
Configuration file format:
\sphinxhyphen{} comments start by “\#”
\sphinxhyphen{} every line’s format is:
\begin{quote}

\sphinxAtStartPar
i,j ? filename ? is, js
meaning:
S(is,js) of touchstone file filename is S(i,j) of output\_filename
\end{quote}
\begin{quote}\begin{description}
\sphinxlineitem{Parameters}\begin{itemize}
\item {} 
\sphinxAtStartPar
\sphinxstyleliteralstrong{\sphinxupquote{conf\_filename}} (\sphinxstyleliteralemphasis{\sphinxupquote{str}}) \textendash{} Name of the configuration filename.

\item {} 
\sphinxAtStartPar
\sphinxstyleliteralstrong{\sphinxupquote{output\_filename}} (\sphinxstyleliteralemphasis{\sphinxupquote{str}}) \textendash{} Name of the output filename.

\end{itemize}

\end{description}\end{quote}

\end{fulllineitems}

\index{parse\_format() (in module touchstone)@\spxentry{parse\_format()}\spxextra{in module touchstone}}

\begin{fulllineitems}
\phantomsection\label{\detokenize{touchstone:touchstone.parse_format}}
\pysigstartsignatures
\pysiglinewithargsret{\sphinxcode{\sphinxupquote{touchstone.}}\sphinxbfcode{\sphinxupquote{parse\_format}}}{\emph{\DUrole{n}{line}}}{}
\pysigstopsignatures
\end{fulllineitems}

\index{spfile (class in touchstone)@\spxentry{spfile}\spxextra{class in touchstone}}

\begin{fulllineitems}
\phantomsection\label{\detokenize{touchstone:touchstone.spfile}}
\pysigstartsignatures
\pysiglinewithargsret{\sphinxbfcode{\sphinxupquote{class\DUrole{w}{  }}}\sphinxcode{\sphinxupquote{touchstone.}}\sphinxbfcode{\sphinxupquote{spfile}}}{\emph{\DUrole{n}{filename}\DUrole{o}{=}\DUrole{default_value}{\textquotesingle{}\textquotesingle{}}}, \emph{\DUrole{n}{freqs}\DUrole{o}{=}\DUrole{default_value}{None}}, \emph{\DUrole{n}{n\_ports}\DUrole{o}{=}\DUrole{default_value}{1}}, \emph{\DUrole{n}{skiplines}\DUrole{o}{=}\DUrole{default_value}{0}}}{}
\pysigstopsignatures
\sphinxAtStartPar
Bases: \sphinxcode{\sphinxupquote{object}}

\sphinxAtStartPar
Class to process Touchstone files.
\begin{quote}

\sphinxAtStartPar
odo
\end{quote}

\sphinxAtStartPar
TODO:
\index{Extraction() (touchstone.spfile method)@\spxentry{Extraction()}\spxextra{touchstone.spfile method}}

\begin{fulllineitems}
\phantomsection\label{\detokenize{touchstone:touchstone.spfile.Extraction}}
\pysigstartsignatures
\pysiglinewithargsret{\sphinxbfcode{\sphinxupquote{Extraction}}}{\emph{\DUrole{n}{measspfile}}}{}
\pysigstopsignatures
\sphinxAtStartPar
Extract die S\sphinxhyphen{}Parameters using measurement data and simulated S\sphinxhyphen{}Parameters
Port ordering in \sphinxstyleemphasis{measspfile} is assumed to be the same as this \sphinxstyleemphasis{spfile}.
Remaining ports are ports of block to be extracted.
See “Extracting multiport S\sphinxhyphen{}Parameters of chip” in technical document.
\begin{quote}\begin{description}
\sphinxlineitem{Parameters}
\sphinxAtStartPar
\sphinxstyleliteralstrong{\sphinxupquote{measspfile}} ({\hyperref[\detokenize{touchstone:touchstone.spfile}]{\sphinxcrossref{\sphinxstyleliteralemphasis{\sphinxupquote{spfile}}}}}) \textendash{} \sphinxstyleemphasis{SPFILE} object of measured S\sphinxhyphen{}Parameters of first k ports

\sphinxlineitem{Returns}
\sphinxAtStartPar
\sphinxstyleemphasis{SPFILE} object of die’s S\sphinxhyphen{}Parameters

\sphinxlineitem{Return type}
\sphinxAtStartPar
{\hyperref[\detokenize{touchstone:touchstone.spfile}]{\sphinxcrossref{spfile}}}

\end{description}\end{quote}

\end{fulllineitems}

\index{Ffunc() (touchstone.spfile method)@\spxentry{Ffunc()}\spxextra{touchstone.spfile method}}

\begin{fulllineitems}
\phantomsection\label{\detokenize{touchstone:touchstone.spfile.Ffunc}}
\pysigstartsignatures
\pysiglinewithargsret{\sphinxbfcode{\sphinxupquote{Ffunc}}}{\emph{\DUrole{n}{imp}}}{}
\pysigstopsignatures
\sphinxAtStartPar
Calculates F\sphinxhyphen{}matrix in a, b definition of S\sphinxhyphen{}Parameters. For internal use of the library.
\begin{quote}
\begin{align*}\!\begin{aligned}
a=F(V+Z_rI)\\
b=F(V-Z_r^*I)\\
\end{aligned}\end{align*}\end{quote}
\begin{quote}\begin{description}
\sphinxlineitem{Parameters}
\sphinxAtStartPar
\sphinxstyleliteralstrong{\sphinxupquote{imp}} (\sphinxstyleliteralemphasis{\sphinxupquote{numpy.ndarray}}) \textendash{} Zref, Reference impedance array for which includes the reference impedance for each port.

\sphinxlineitem{Returns}
\sphinxAtStartPar
F\sphinxhyphen{}Matrix

\sphinxlineitem{Return type}
\sphinxAtStartPar
numpy.matrix

\end{description}\end{quote}

\end{fulllineitems}

\index{ImpulseResponse() (touchstone.spfile method)@\spxentry{ImpulseResponse()}\spxextra{touchstone.spfile method}}

\begin{fulllineitems}
\phantomsection\label{\detokenize{touchstone:touchstone.spfile.ImpulseResponse}}
\pysigstartsignatures
\pysiglinewithargsret{\sphinxbfcode{\sphinxupquote{ImpulseResponse}}}{\emph{\DUrole{n}{i}\DUrole{o}{=}\DUrole{default_value}{2}}, \emph{\DUrole{n}{j}\DUrole{o}{=}\DUrole{default_value}{1}}, \emph{\DUrole{n}{dc\_interp}\DUrole{o}{=}\DUrole{default_value}{1}}, \emph{\DUrole{n}{dc\_value}\DUrole{o}{=}\DUrole{default_value}{0.0}}, \emph{\DUrole{n}{max\_time\_step}\DUrole{o}{=}\DUrole{default_value}{1.0}}, \emph{\DUrole{n}{freq\_res\_coef}\DUrole{o}{=}\DUrole{default_value}{1.0}}, \emph{\DUrole{n}{window\_name}\DUrole{o}{=}\DUrole{default_value}{\textquotesingle{}blackman\textquotesingle{}}}}{}
\pysigstopsignatures
\sphinxAtStartPar
Calculates impulse response of \(S_{i j}\)
\begin{quote}\begin{description}
\sphinxlineitem{Parameters}\begin{itemize}
\item {} 
\sphinxAtStartPar
\sphinxstyleliteralstrong{\sphinxupquote{i}} (\sphinxstyleliteralemphasis{\sphinxupquote{int}}\sphinxstyleliteralemphasis{\sphinxupquote{, }}\sphinxstyleliteralemphasis{\sphinxupquote{optional}}) \textendash{} Port\sphinxhyphen{}1. Defaults to 2.

\item {} 
\sphinxAtStartPar
\sphinxstyleliteralstrong{\sphinxupquote{j}} (\sphinxstyleliteralemphasis{\sphinxupquote{int}}\sphinxstyleliteralemphasis{\sphinxupquote{, }}\sphinxstyleliteralemphasis{\sphinxupquote{optional}}) \textendash{} Port\sphinxhyphen{}2. Defaults to 1.

\item {} 
\sphinxAtStartPar
\sphinxstyleliteralstrong{\sphinxupquote{dc\_interp}} (\sphinxstyleliteralemphasis{\sphinxupquote{int}}\sphinxstyleliteralemphasis{\sphinxupquote{, }}\sphinxstyleliteralemphasis{\sphinxupquote{optional}}) \textendash{} If 1, add DC point to interpolation. Defaults to 1.

\item {} 
\sphinxAtStartPar
\sphinxstyleliteralstrong{\sphinxupquote{dc\_value}} (\sphinxstyleliteralemphasis{\sphinxupquote{float}}\sphinxstyleliteralemphasis{\sphinxupquote{, }}\sphinxstyleliteralemphasis{\sphinxupquote{optional}}) \textendash{} dc\_value to be used at interpolation if \sphinxstyleemphasis{dc\_interp=0}. Defaults to 0.0. This value is appended to \(S_{i j}\) and the rest is left to interpolation in \sphinxstyleemphasis{data\_array} function.

\item {} 
\sphinxAtStartPar
\sphinxstyleliteralstrong{\sphinxupquote{max\_time\_step}} (\sphinxstyleliteralemphasis{\sphinxupquote{float}}\sphinxstyleliteralemphasis{\sphinxupquote{, }}\sphinxstyleliteralemphasis{\sphinxupquote{optional}}) \textendash{} Not used for now. Defaults to 1.0.

\item {} 
\sphinxAtStartPar
\sphinxstyleliteralstrong{\sphinxupquote{freq\_res\_coef}} (\sphinxstyleliteralemphasis{\sphinxupquote{float}}\sphinxstyleliteralemphasis{\sphinxupquote{, }}\sphinxstyleliteralemphasis{\sphinxupquote{optional}}) \textendash{} Coeeficient to increase the frequency resolution by interpolation. Defaults to 1.0 (no interpolation).

\item {} 
\sphinxAtStartPar
\sphinxstyleliteralstrong{\sphinxupquote{window}} (\sphinxstyleliteralemphasis{\sphinxupquote{str}}\sphinxstyleliteralemphasis{\sphinxupquote{, }}\sphinxstyleliteralemphasis{\sphinxupquote{optional}}) \textendash{} Windows function to prevent ringing. Defaults to “blackman”. Other windows will be added later.

\end{itemize}

\sphinxlineitem{Returns}
\sphinxAtStartPar
\begin{description}
\sphinxlineitem{The elements of the tuple are the following in order:}\begin{enumerate}
\sphinxsetlistlabels{\arabic}{enumi}{enumii}{}{.}%
\item {} 
\sphinxAtStartPar
Raw frequency data used as input

\item {} 
\sphinxAtStartPar
Window array

\item {} 
\sphinxAtStartPar
Time array

\item {} 
\sphinxAtStartPar
Time\sphinxhyphen{}Domain Waveform of Impulse Response

\item {} 
\sphinxAtStartPar
Time\sphinxhyphen{}Domain Waveform of Impulse Input

\item {} 
\sphinxAtStartPar
Time step

\item {} 
\sphinxAtStartPar
Frequency step

\item {} 
\sphinxAtStartPar
Size of input array

\item {} 
\sphinxAtStartPar
Max Value of Impulse Input

\end{enumerate}

\end{description}


\sphinxlineitem{Return type}
\sphinxAtStartPar
9\sphinxhyphen{}tuple

\end{description}\end{quote}

\end{fulllineitems}

\index{ImpulseResponseBanded() (touchstone.spfile method)@\spxentry{ImpulseResponseBanded()}\spxextra{touchstone.spfile method}}

\begin{fulllineitems}
\phantomsection\label{\detokenize{touchstone:touchstone.spfile.ImpulseResponseBanded}}
\pysigstartsignatures
\pysiglinewithargsret{\sphinxbfcode{\sphinxupquote{ImpulseResponseBanded}}}{\emph{\DUrole{n}{i}\DUrole{o}{=}\DUrole{default_value}{2}}, \emph{\DUrole{n}{j}\DUrole{o}{=}\DUrole{default_value}{1}}, \emph{\DUrole{n}{dc\_interp}\DUrole{o}{=}\DUrole{default_value}{1}}, \emph{\DUrole{n}{dc\_value}\DUrole{o}{=}\DUrole{default_value}{0.0}}, \emph{\DUrole{n}{max\_time\_step}\DUrole{o}{=}\DUrole{default_value}{1.0}}, \emph{\DUrole{n}{freq\_res\_coef}\DUrole{o}{=}\DUrole{default_value}{1.0}}, \emph{\DUrole{n}{Window}\DUrole{o}{=}\DUrole{default_value}{\textquotesingle{}blackman\textquotesingle{}}}}{}
\pysigstopsignatures
\sphinxAtStartPar
Calculates impulse response of \(S_{i j}\)
\begin{quote}\begin{description}
\sphinxlineitem{Parameters}\begin{itemize}
\item {} 
\sphinxAtStartPar
\sphinxstyleliteralstrong{\sphinxupquote{i}} (\sphinxstyleliteralemphasis{\sphinxupquote{int}}\sphinxstyleliteralemphasis{\sphinxupquote{, }}\sphinxstyleliteralemphasis{\sphinxupquote{optional}}) \textendash{} Port\sphinxhyphen{}1. Defaults to 2.

\item {} 
\sphinxAtStartPar
\sphinxstyleliteralstrong{\sphinxupquote{j}} (\sphinxstyleliteralemphasis{\sphinxupquote{int}}\sphinxstyleliteralemphasis{\sphinxupquote{, }}\sphinxstyleliteralemphasis{\sphinxupquote{optional}}) \textendash{} Port\sphinxhyphen{}2. Defaults to 1.

\item {} 
\sphinxAtStartPar
\sphinxstyleliteralstrong{\sphinxupquote{dc\_interp}} (\sphinxstyleliteralemphasis{\sphinxupquote{int}}\sphinxstyleliteralemphasis{\sphinxupquote{, }}\sphinxstyleliteralemphasis{\sphinxupquote{optional}}) \textendash{} If 1, add DC point to interpolation. Defaults to 1.

\item {} 
\sphinxAtStartPar
\sphinxstyleliteralstrong{\sphinxupquote{dc\_value}} (\sphinxstyleliteralemphasis{\sphinxupquote{float}}\sphinxstyleliteralemphasis{\sphinxupquote{, }}\sphinxstyleliteralemphasis{\sphinxupquote{optional}}) \textendash{} dc\_value to be used at interpolation if \sphinxstyleemphasis{dc\_interp=0}. Defaults to 0.0. This value is appended to \(S_{i j}\) and the rest is left to interpolation in \sphinxstyleemphasis{data\_array} function.

\item {} 
\sphinxAtStartPar
\sphinxstyleliteralstrong{\sphinxupquote{max\_time\_step}} (\sphinxstyleliteralemphasis{\sphinxupquote{float}}\sphinxstyleliteralemphasis{\sphinxupquote{, }}\sphinxstyleliteralemphasis{\sphinxupquote{optional}}) \textendash{} Not used for now. Defaults to 1.0.

\item {} 
\sphinxAtStartPar
\sphinxstyleliteralstrong{\sphinxupquote{freq\_res\_coef}} (\sphinxstyleliteralemphasis{\sphinxupquote{float}}\sphinxstyleliteralemphasis{\sphinxupquote{, }}\sphinxstyleliteralemphasis{\sphinxupquote{optional}}) \textendash{} Coeeficient to increase the frequency resolution by interpolation. Defaults to 1.0 (no interpolation).

\item {} 
\sphinxAtStartPar
\sphinxstyleliteralstrong{\sphinxupquote{Window}} (\sphinxstyleliteralemphasis{\sphinxupquote{str}}\sphinxstyleliteralemphasis{\sphinxupquote{, }}\sphinxstyleliteralemphasis{\sphinxupquote{optional}}) \textendash{} Windows function to prevent ringing. Defaults to “blackman”. Other windows will be added later.

\end{itemize}

\sphinxlineitem{Returns}
\sphinxAtStartPar
\begin{description}
\sphinxlineitem{The elements of the tuple are the following in order:}\begin{enumerate}
\sphinxsetlistlabels{\arabic}{enumi}{enumii}{}{.}%
\item {} 
\sphinxAtStartPar
Raw frequency data used as input

\item {} 
\sphinxAtStartPar
Window array

\item {} 
\sphinxAtStartPar
Time array

\item {} 
\sphinxAtStartPar
Time\sphinxhyphen{}Domain Waveform of Impulse Response

\item {} 
\sphinxAtStartPar
Time\sphinxhyphen{}Domain Waveform of Impulse Input

\item {} 
\sphinxAtStartPar
Time step

\item {} 
\sphinxAtStartPar
Frequency step

\item {} 
\sphinxAtStartPar
Size of input array

\item {} 
\sphinxAtStartPar
Max Value of Impulse Input

\end{enumerate}

\end{description}


\sphinxlineitem{Return type}
\sphinxAtStartPar
9\sphinxhyphen{}tuple

\end{description}\end{quote}

\end{fulllineitems}

\index{S() (touchstone.spfile method)@\spxentry{S()}\spxextra{touchstone.spfile method}}

\begin{fulllineitems}
\phantomsection\label{\detokenize{touchstone:touchstone.spfile.S}}
\pysigstartsignatures
\pysiglinewithargsret{\sphinxbfcode{\sphinxupquote{S}}}{\emph{\DUrole{n}{i}\DUrole{o}{=}\DUrole{default_value}{1}}, \emph{\DUrole{n}{j}\DUrole{o}{=}\DUrole{default_value}{1}}, \emph{\DUrole{n}{data\_format}\DUrole{o}{=}\DUrole{default_value}{\textquotesingle{}COMPLEX\textquotesingle{}}}, \emph{\DUrole{n}{freqs}\DUrole{o}{=}\DUrole{default_value}{None}}}{}
\pysigstopsignatures
\sphinxAtStartPar
Return \(S_{i j}\) in format \sphinxstyleemphasis{data\_format}
Uses \sphinxstyleemphasis{data\_array} method internally. A convenience function for practical use.
\begin{quote}\begin{description}
\sphinxlineitem{Parameters}\begin{itemize}
\item {} 
\sphinxAtStartPar
\sphinxstyleliteralstrong{\sphinxupquote{i}} (\sphinxstyleliteralemphasis{\sphinxupquote{int}}\sphinxstyleliteralemphasis{\sphinxupquote{, }}\sphinxstyleliteralemphasis{\sphinxupquote{optional}}) \textendash{} Port\sphinxhyphen{}1. Defaults to 1.

\item {} 
\sphinxAtStartPar
\sphinxstyleliteralstrong{\sphinxupquote{j}} (\sphinxstyleliteralemphasis{\sphinxupquote{int}}\sphinxstyleliteralemphasis{\sphinxupquote{, }}\sphinxstyleliteralemphasis{\sphinxupquote{optional}}) \textendash{} Port\sphinxhyphen{}2. Defaults to 1.

\item {} 
\sphinxAtStartPar
\sphinxstyleliteralstrong{\sphinxupquote{data\_format}} (\sphinxstyleliteralemphasis{\sphinxupquote{str}}\sphinxstyleliteralemphasis{\sphinxupquote{, }}\sphinxstyleliteralemphasis{\sphinxupquote{optional}}) \textendash{} See \sphinxstyleemphasis{data\_format} parameter of \sphinxstyleemphasis{data\_array} method. Defaults to “COMPLEX”.

\end{itemize}

\sphinxlineitem{Returns}
\sphinxAtStartPar
\(S_{i j}\) as \sphinxstyleemphasis{data\_format}

\sphinxlineitem{Return type}
\sphinxAtStartPar
numpy.array

\end{description}\end{quote}

\end{fulllineitems}

\index{T() (touchstone.spfile method)@\spxentry{T()}\spxextra{touchstone.spfile method}}

\begin{fulllineitems}
\phantomsection\label{\detokenize{touchstone:touchstone.spfile.T}}
\pysigstartsignatures
\pysiglinewithargsret{\sphinxbfcode{\sphinxupquote{T}}}{\emph{\DUrole{n}{i}\DUrole{o}{=}\DUrole{default_value}{1}}, \emph{\DUrole{n}{j}\DUrole{o}{=}\DUrole{default_value}{1}}, \emph{\DUrole{n}{data\_format}\DUrole{o}{=}\DUrole{default_value}{\textquotesingle{}COMPLEX\textquotesingle{}}}, \emph{\DUrole{n}{freqs}\DUrole{o}{=}\DUrole{default_value}{None}}}{}
\pysigstopsignatures
\sphinxAtStartPar
Return \(T_{i j}\) in format \sphinxstyleemphasis{data\_format}
Uses \sphinxstyleemphasis{data\_array} method internally. A convenience function for practical use.
\begin{quote}\begin{description}
\sphinxlineitem{Parameters}\begin{itemize}
\item {} 
\sphinxAtStartPar
\sphinxstyleliteralstrong{\sphinxupquote{i}} (\sphinxstyleliteralemphasis{\sphinxupquote{int}}\sphinxstyleliteralemphasis{\sphinxupquote{, }}\sphinxstyleliteralemphasis{\sphinxupquote{optional}}) \textendash{} Port\sphinxhyphen{}1. Defaults to 1.

\item {} 
\sphinxAtStartPar
\sphinxstyleliteralstrong{\sphinxupquote{j}} (\sphinxstyleliteralemphasis{\sphinxupquote{int}}\sphinxstyleliteralemphasis{\sphinxupquote{, }}\sphinxstyleliteralemphasis{\sphinxupquote{optional}}) \textendash{} Port\sphinxhyphen{}2. Defaults to 1.

\item {} 
\sphinxAtStartPar
\sphinxstyleliteralstrong{\sphinxupquote{data\_format}} (\sphinxstyleliteralemphasis{\sphinxupquote{str}}\sphinxstyleliteralemphasis{\sphinxupquote{, }}\sphinxstyleliteralemphasis{\sphinxupquote{optional}}) \textendash{} See \sphinxstyleemphasis{data\_format} parameter of \sphinxstyleemphasis{data\_array} method. Defaults to “COMPLEX”.

\end{itemize}

\sphinxlineitem{Returns}
\sphinxAtStartPar
\(T_{i j}\) as \sphinxstyleemphasis{data\_format}

\sphinxlineitem{Return type}
\sphinxAtStartPar
numpy.array

\end{description}\end{quote}

\end{fulllineitems}

\index{UniformDeembed() (touchstone.spfile method)@\spxentry{UniformDeembed()}\spxextra{touchstone.spfile method}}

\begin{fulllineitems}
\phantomsection\label{\detokenize{touchstone:touchstone.spfile.UniformDeembed}}
\pysigstartsignatures
\pysiglinewithargsret{\sphinxbfcode{\sphinxupquote{UniformDeembed}}}{\emph{\DUrole{n}{quantity}}, \emph{\DUrole{n}{ports}\DUrole{o}{=}\DUrole{default_value}{\textquotesingle{}all\textquotesingle{}}}, \emph{\DUrole{n}{kind}\DUrole{o}{=}\DUrole{default_value}{\textquotesingle{}degrees\textquotesingle{}}}, \emph{\DUrole{n}{inplace}\DUrole{o}{=}\DUrole{default_value}{\sphinxhyphen{} 1}}}{}
\pysigstopsignatures\begin{description}
\sphinxlineitem{This function deembeds some of the ports of S\sphinxhyphen{}Parameters. Deembedding quantity can be:}\begin{itemize}
\item {} 
\sphinxAtStartPar
Phase in degrees

\item {} 
\sphinxAtStartPar
Phase in radians

\item {} 
\sphinxAtStartPar
Length in meters

\item {} 
\sphinxAtStartPar
Delay in seconds

\end{itemize}

\end{description}

\sphinxAtStartPar
A positive quantity means deembedding into the circuit.
The Zc of de\sphinxhyphen{}embedding lines is the reference impedances of each port.
\begin{quote}\begin{description}
\sphinxlineitem{Parameters}\begin{itemize}
\item {} 
\sphinxAtStartPar
\sphinxstyleliteralstrong{\sphinxupquote{quantity}} (\sphinxstyleliteralemphasis{\sphinxupquote{float}}\sphinxstyleliteralemphasis{\sphinxupquote{ or }}\sphinxstyleliteralemphasis{\sphinxupquote{list}}) \textendash{} Quantity to be deembedded.
\sphinxhyphen{} If a number is given, it is used for all frequencies and ports
\sphinxhyphen{} If a list is given, if its size is 1, its element is used for all ports. If its size is equal to number of ports, the list is used for all frequencies.
Otherwise its size should be equal to the number of frequencies. If an element of list is number, it is used for all ports. If an element of the list is also a list, the elements size should be same as the number of ports.

\item {} 
\sphinxAtStartPar
\sphinxstyleliteralstrong{\sphinxupquote{ports}} (\sphinxstyleliteralemphasis{\sphinxupquote{list}}\sphinxstyleliteralemphasis{\sphinxupquote{, }}\sphinxstyleliteralemphasis{\sphinxupquote{optional}}) \textendash{} List of port numbers to be deembedded. If not given all ports are deembedded.

\item {} 
\sphinxAtStartPar
\sphinxstyleliteralstrong{\sphinxupquote{kind}} (\sphinxstyleliteralemphasis{\sphinxupquote{string}}\sphinxstyleliteralemphasis{\sphinxupquote{, }}\sphinxstyleliteralemphasis{\sphinxupquote{optional}}) \textendash{} One of the following values, “degrees”, “radians”, “length” and “delay”. Defaults to “degrees”.

\item {} 
\sphinxAtStartPar
\sphinxstyleliteralstrong{\sphinxupquote{inplace}} (\sphinxstyleliteralemphasis{\sphinxupquote{int}}\sphinxstyleliteralemphasis{\sphinxupquote{, }}\sphinxstyleliteralemphasis{\sphinxupquote{optional}}) \textendash{} Object editing mode. Defaults to \sphinxhyphen{}1.

\end{itemize}

\sphinxlineitem{Returns}
\sphinxAtStartPar
De\sphinxhyphen{}embedded spfile

\sphinxlineitem{Return type}
\sphinxAtStartPar
{\hyperref[\detokenize{touchstone:touchstone.spfile}]{\sphinxcrossref{spfile}}}

\end{description}\end{quote}

\end{fulllineitems}

\index{Y() (touchstone.spfile method)@\spxentry{Y()}\spxextra{touchstone.spfile method}}

\begin{fulllineitems}
\phantomsection\label{\detokenize{touchstone:touchstone.spfile.Y}}
\pysigstartsignatures
\pysiglinewithargsret{\sphinxbfcode{\sphinxupquote{Y}}}{\emph{\DUrole{n}{i}\DUrole{o}{=}\DUrole{default_value}{1}}, \emph{\DUrole{n}{j}\DUrole{o}{=}\DUrole{default_value}{1}}, \emph{\DUrole{n}{data\_format}\DUrole{o}{=}\DUrole{default_value}{\textquotesingle{}COMPLEX\textquotesingle{}}}, \emph{\DUrole{n}{freqs}\DUrole{o}{=}\DUrole{default_value}{None}}}{}
\pysigstopsignatures
\sphinxAtStartPar
Return \(Y_{i j}\) in format \sphinxstyleemphasis{data\_format}
Uses \sphinxstyleemphasis{data\_array} method internally. A convenience function for practical use.
\begin{quote}\begin{description}
\sphinxlineitem{Parameters}\begin{itemize}
\item {} 
\sphinxAtStartPar
\sphinxstyleliteralstrong{\sphinxupquote{i}} (\sphinxstyleliteralemphasis{\sphinxupquote{int}}\sphinxstyleliteralemphasis{\sphinxupquote{, }}\sphinxstyleliteralemphasis{\sphinxupquote{optional}}) \textendash{} Port\sphinxhyphen{}1. Defaults to 1.

\item {} 
\sphinxAtStartPar
\sphinxstyleliteralstrong{\sphinxupquote{j}} (\sphinxstyleliteralemphasis{\sphinxupquote{int}}\sphinxstyleliteralemphasis{\sphinxupquote{, }}\sphinxstyleliteralemphasis{\sphinxupquote{optional}}) \textendash{} Port\sphinxhyphen{}2. Defaults to 1.

\item {} 
\sphinxAtStartPar
\sphinxstyleliteralstrong{\sphinxupquote{data\_format}} (\sphinxstyleliteralemphasis{\sphinxupquote{str}}\sphinxstyleliteralemphasis{\sphinxupquote{, }}\sphinxstyleliteralemphasis{\sphinxupquote{optional}}) \textendash{} See \sphinxstyleemphasis{data\_format} parameter of \sphinxstyleemphasis{data\_array} method. Defaults to “COMPLEX”.

\end{itemize}

\sphinxlineitem{Returns}
\sphinxAtStartPar
\(Y_{i j}\) as \sphinxstyleemphasis{data\_format}

\sphinxlineitem{Return type}
\sphinxAtStartPar
numpy.array

\end{description}\end{quote}

\end{fulllineitems}

\index{Z() (touchstone.spfile method)@\spxentry{Z()}\spxextra{touchstone.spfile method}}

\begin{fulllineitems}
\phantomsection\label{\detokenize{touchstone:touchstone.spfile.Z}}
\pysigstartsignatures
\pysiglinewithargsret{\sphinxbfcode{\sphinxupquote{Z}}}{\emph{\DUrole{n}{i}\DUrole{o}{=}\DUrole{default_value}{1}}, \emph{\DUrole{n}{j}\DUrole{o}{=}\DUrole{default_value}{1}}, \emph{\DUrole{n}{data\_format}\DUrole{o}{=}\DUrole{default_value}{\textquotesingle{}COMPLEX\textquotesingle{}}}, \emph{\DUrole{n}{freqs}\DUrole{o}{=}\DUrole{default_value}{None}}}{}
\pysigstopsignatures
\sphinxAtStartPar
Return \(Z_{i j}\) in format \sphinxstyleemphasis{data\_format}
Uses \sphinxstyleemphasis{data\_array} method internally. A convenience function for practical use.
\begin{quote}\begin{description}
\sphinxlineitem{Parameters}\begin{itemize}
\item {} 
\sphinxAtStartPar
\sphinxstyleliteralstrong{\sphinxupquote{i}} (\sphinxstyleliteralemphasis{\sphinxupquote{int}}\sphinxstyleliteralemphasis{\sphinxupquote{, }}\sphinxstyleliteralemphasis{\sphinxupquote{optional}}) \textendash{} Port\sphinxhyphen{}1. Defaults to 1.

\item {} 
\sphinxAtStartPar
\sphinxstyleliteralstrong{\sphinxupquote{j}} (\sphinxstyleliteralemphasis{\sphinxupquote{int}}\sphinxstyleliteralemphasis{\sphinxupquote{, }}\sphinxstyleliteralemphasis{\sphinxupquote{optional}}) \textendash{} Port\sphinxhyphen{}2. Defaults to 1.

\item {} 
\sphinxAtStartPar
\sphinxstyleliteralstrong{\sphinxupquote{data\_format}} (\sphinxstyleliteralemphasis{\sphinxupquote{str}}\sphinxstyleliteralemphasis{\sphinxupquote{, }}\sphinxstyleliteralemphasis{\sphinxupquote{optional}}) \textendash{} See \sphinxstyleemphasis{data\_format} parameter of \sphinxstyleemphasis{data\_array} method. Defaults to “COMPLEX”.

\end{itemize}

\sphinxlineitem{Returns}
\sphinxAtStartPar
\(Z_{i j}\) as \sphinxstyleemphasis{data\_format}

\sphinxlineitem{Return type}
\sphinxAtStartPar
numpy.array

\end{description}\end{quote}

\end{fulllineitems}

\index{Z\_conjmatch() (touchstone.spfile method)@\spxentry{Z\_conjmatch()}\spxextra{touchstone.spfile method}}

\begin{fulllineitems}
\phantomsection\label{\detokenize{touchstone:touchstone.spfile.Z_conjmatch}}
\pysigstartsignatures
\pysiglinewithargsret{\sphinxbfcode{\sphinxupquote{Z\_conjmatch}}}{\emph{\DUrole{n}{port1}\DUrole{o}{=}\DUrole{default_value}{1}}, \emph{\DUrole{n}{port2}\DUrole{o}{=}\DUrole{default_value}{2}}}{}
\pysigstopsignatures
\sphinxAtStartPar
Calculates source and load reflection coefficients for simultaneous conjugate match.
\begin{quote}\begin{description}
\sphinxlineitem{Parameters}\begin{itemize}
\item {} 
\sphinxAtStartPar
\sphinxstyleliteralstrong{\sphinxupquote{port1}} (\sphinxstyleliteralemphasis{\sphinxupquote{int}}\sphinxstyleliteralemphasis{\sphinxupquote{, }}\sphinxstyleliteralemphasis{\sphinxupquote{optional}}) \textendash{} {[}description{]}. Defaults to 1.

\item {} 
\sphinxAtStartPar
\sphinxstyleliteralstrong{\sphinxupquote{port2}} (\sphinxstyleliteralemphasis{\sphinxupquote{int}}\sphinxstyleliteralemphasis{\sphinxupquote{, }}\sphinxstyleliteralemphasis{\sphinxupquote{optional}}) \textendash{} {[}description{]}. Defaults to 2.

\end{itemize}

\sphinxlineitem{Returns}
\sphinxAtStartPar
\begin{itemize}
\item {} 
\sphinxAtStartPar
GS: Reflection coefficient at Port\sphinxhyphen{}1

\item {} 
\sphinxAtStartPar
GL: Reflection coefficient at Port\sphinxhyphen{}2

\end{itemize}


\sphinxlineitem{Return type}
\sphinxAtStartPar
2\sphinxhyphen{}tuple of numpy.arrays (GS, GL)

\end{description}\end{quote}

\end{fulllineitems}

\index{\_\_add\_\_() (touchstone.spfile method)@\spxentry{\_\_add\_\_()}\spxextra{touchstone.spfile method}}

\begin{fulllineitems}
\phantomsection\label{\detokenize{touchstone:touchstone.spfile.__add__}}
\pysigstartsignatures
\pysiglinewithargsret{\sphinxbfcode{\sphinxupquote{\_\_add\_\_}}}{\emph{\DUrole{n}{SP2}}}{}
\pysigstopsignatures\begin{description}
\sphinxlineitem{Implements SP1+SP2. Cascades port\sphinxhyphen{}1 of SP2 to port\sphinxhyphen{}2 of SP1. Port ordering is as follows:}
\noindent\sphinxincludegraphics{{./ditaa-991c97d11f881cf866de60e5b2ee483754c8ff0a}.png}

\end{description}

\sphinxAtStartPar
SP1 is \sphinxstyleemphasis{self}.
\begin{quote}\begin{description}
\sphinxlineitem{Parameters}
\sphinxAtStartPar
\sphinxstyleliteralstrong{\sphinxupquote{SP2}} ({\hyperref[\detokenize{touchstone:touchstone.spfile}]{\sphinxcrossref{\sphinxstyleliteralemphasis{\sphinxupquote{spfile}}}}}) \textendash{} Appended spfile network

\sphinxlineitem{Returns}
\sphinxAtStartPar
The result of cascade of 2 networks

\sphinxlineitem{Return type}
\sphinxAtStartPar
{\hyperref[\detokenize{touchstone:touchstone.spfile}]{\sphinxcrossref{spfile}}}

\end{description}\end{quote}

\end{fulllineitems}

\index{\_\_neg\_\_() (touchstone.spfile method)@\spxentry{\_\_neg\_\_()}\spxextra{touchstone.spfile method}}

\begin{fulllineitems}
\phantomsection\label{\detokenize{touchstone:touchstone.spfile.__neg__}}
\pysigstartsignatures
\pysiglinewithargsret{\sphinxbfcode{\sphinxupquote{\_\_neg\_\_}}}{}{}
\pysigstopsignatures
\sphinxAtStartPar
Calculates an spfile object for two\sphinxhyphen{}port networks which is the inverse of this network. This is used to use + and \sphinxhyphen{} signs to cascade or deembed 2\sphinxhyphen{}port blocks.
\begin{quote}\begin{description}
\sphinxlineitem{Returns}
\sphinxAtStartPar
\begin{enumerate}
\sphinxsetlistlabels{\arabic}{enumi}{enumii}{}{.}%
\item {} 
\sphinxAtStartPar
\sphinxstyleemphasis{None} if number of ports is not 2.

\item {} 
\sphinxAtStartPar
\sphinxstyleemphasis{spfile} which is the inverse of the spfile object operated on.

\end{enumerate}


\sphinxlineitem{Return type}
\sphinxAtStartPar
{\hyperref[\detokenize{touchstone:touchstone.spfile}]{\sphinxcrossref{spfile}}}

\end{description}\end{quote}

\end{fulllineitems}

\index{\_\_sub\_\_() (touchstone.spfile method)@\spxentry{\_\_sub\_\_()}\spxextra{touchstone.spfile method}}

\begin{fulllineitems}
\phantomsection\label{\detokenize{touchstone:touchstone.spfile.__sub__}}
\pysigstartsignatures
\pysiglinewithargsret{\sphinxbfcode{\sphinxupquote{\_\_sub\_\_}}}{\emph{\DUrole{n}{SP2}}}{}
\pysigstopsignatures
\sphinxAtStartPar
Implements SP1\sphinxhyphen{}SP2.
Deembeds SP2 from port\sphinxhyphen{}2 of SP1.
Port ordering is as follows:
(1)\sphinxhyphen{}SP1\sphinxhyphen{}(2)—(1)\sphinxhyphen{}SP2\sphinxhyphen{}(2)
SP1 is \sphinxstyleemphasis{self}.
\begin{quote}\begin{description}
\sphinxlineitem{Parameters}
\sphinxAtStartPar
\sphinxstyleliteralstrong{\sphinxupquote{SP2}} ({\hyperref[\detokenize{touchstone:touchstone.spfile}]{\sphinxcrossref{\sphinxstyleliteralemphasis{\sphinxupquote{spfile}}}}}) \textendash{} Deembedded spfile network

\sphinxlineitem{Returns}
\sphinxAtStartPar
The resulting of deembedding process

\sphinxlineitem{Return type}
\sphinxAtStartPar
{\hyperref[\detokenize{touchstone:touchstone.spfile}]{\sphinxcrossref{spfile}}}

\end{description}\end{quote}

\end{fulllineitems}

\index{add\_abs\_noise() (touchstone.spfile method)@\spxentry{add\_abs\_noise()}\spxextra{touchstone.spfile method}}

\begin{fulllineitems}
\phantomsection\label{\detokenize{touchstone:touchstone.spfile.add_abs_noise}}
\pysigstartsignatures
\pysiglinewithargsret{\sphinxbfcode{\sphinxupquote{add\_abs\_noise}}}{\emph{\DUrole{n}{dbnoise}\DUrole{o}{=}\DUrole{default_value}{0.1}}, \emph{\DUrole{n}{phasenoise}\DUrole{o}{=}\DUrole{default_value}{0.1}}, \emph{\DUrole{n}{inplace}\DUrole{o}{=}\DUrole{default_value}{\sphinxhyphen{} 1}}}{}
\pysigstopsignatures
\sphinxAtStartPar
This method adds random amplitude and phase noise to the s\sphinxhyphen{}parameter data.
Mean value for both noises are 0.
\begin{quote}\begin{description}
\sphinxlineitem{Parameters}\begin{itemize}
\item {} 
\sphinxAtStartPar
\sphinxstyleliteralstrong{\sphinxupquote{dbnoise}} (\sphinxstyleliteralemphasis{\sphinxupquote{float}}\sphinxstyleliteralemphasis{\sphinxupquote{, }}\sphinxstyleliteralemphasis{\sphinxupquote{optional}}) \textendash{} Standard deviation of amplitude noise in dB. Defaults to 0.1.

\item {} 
\sphinxAtStartPar
\sphinxstyleliteralstrong{\sphinxupquote{phasenoise}} (\sphinxstyleliteralemphasis{\sphinxupquote{float}}\sphinxstyleliteralemphasis{\sphinxupquote{, }}\sphinxstyleliteralemphasis{\sphinxupquote{optional}}) \textendash{} Standard deviation of phase noise in degrees. Defaults to 0.1.

\item {} 
\sphinxAtStartPar
\sphinxstyleliteralstrong{\sphinxupquote{inplace}} (\sphinxstyleliteralemphasis{\sphinxupquote{int}}\sphinxstyleliteralemphasis{\sphinxupquote{, }}\sphinxstyleliteralemphasis{\sphinxupquote{optional}}) \textendash{} object editing mode. Defaults to \sphinxhyphen{}1.

\end{itemize}

\sphinxlineitem{Returns}
\sphinxAtStartPar
object with noisy data

\sphinxlineitem{Return type}
\sphinxAtStartPar
{\hyperref[\detokenize{touchstone:touchstone.spfile}]{\sphinxcrossref{spfile}}}

\end{description}\end{quote}

\end{fulllineitems}

\index{calc\_syz() (touchstone.spfile method)@\spxentry{calc\_syz()}\spxextra{touchstone.spfile method}}

\begin{fulllineitems}
\phantomsection\label{\detokenize{touchstone:touchstone.spfile.calc_syz}}
\pysigstartsignatures
\pysiglinewithargsret{\sphinxbfcode{\sphinxupquote{calc\_syz}}}{\emph{\DUrole{n}{input}\DUrole{o}{=}\DUrole{default_value}{\textquotesingle{}S\textquotesingle{}}}, \emph{\DUrole{n}{indices}\DUrole{o}{=}\DUrole{default_value}{None}}}{}
\pysigstopsignatures
\sphinxAtStartPar
This function calculates 2 of S, Y and Z parameters using the remaining parameter given.
Y and Z\sphinxhyphen{}matrices calculated separately instead of calculating one and taking inverse. Because one of them may be undefined for some circuits.
\begin{quote}\begin{description}
\sphinxlineitem{Parameters}\begin{itemize}
\item {} 
\sphinxAtStartPar
\sphinxstyleliteralstrong{\sphinxupquote{input}} (\sphinxstyleliteralemphasis{\sphinxupquote{str}}\sphinxstyleliteralemphasis{\sphinxupquote{, }}\sphinxstyleliteralemphasis{\sphinxupquote{optional}}) \textendash{} Input parameter type (should be S, Y or Z). Defaults to “S”.

\item {} 
\sphinxAtStartPar
\sphinxstyleliteralstrong{\sphinxupquote{indices}} (\sphinxstyleliteralemphasis{\sphinxupquote{list}}\sphinxstyleliteralemphasis{\sphinxupquote{, }}\sphinxstyleliteralemphasis{\sphinxupquote{optional}}) \textendash{} If given, output matrices are calculated only at the indices given by this list. If it is None, then output matrices are calculated at all frequencies. Defaults to None.

\end{itemize}

\end{description}\end{quote}

\end{fulllineitems}

\index{calc\_t\_eigs() (touchstone.spfile method)@\spxentry{calc\_t\_eigs()}\spxextra{touchstone.spfile method}}

\begin{fulllineitems}
\phantomsection\label{\detokenize{touchstone:touchstone.spfile.calc_t_eigs}}
\pysigstartsignatures
\pysiglinewithargsret{\sphinxbfcode{\sphinxupquote{calc\_t\_eigs}}}{\emph{\DUrole{n}{port1}\DUrole{o}{=}\DUrole{default_value}{1}}, \emph{\DUrole{n}{port2}\DUrole{o}{=}\DUrole{default_value}{2}}}{}
\pysigstopsignatures
\sphinxAtStartPar
Eigenfunctions and Eigenvector of T\sphinxhyphen{}Matrix is calculated.
Only power\sphinxhyphen{}wave formulation is implemented

\end{fulllineitems}

\index{change\_formulation() (touchstone.spfile method)@\spxentry{change\_formulation()}\spxextra{touchstone.spfile method}}

\begin{fulllineitems}
\phantomsection\label{\detokenize{touchstone:touchstone.spfile.change_formulation}}
\pysigstartsignatures
\pysiglinewithargsret{\sphinxbfcode{\sphinxupquote{change\_formulation}}}{\emph{\DUrole{n}{formulation}}}{}
\pysigstopsignatures
\end{fulllineitems}

\index{change\_ref\_impedance() (touchstone.spfile method)@\spxentry{change\_ref\_impedance()}\spxextra{touchstone.spfile method}}

\begin{fulllineitems}
\phantomsection\label{\detokenize{touchstone:touchstone.spfile.change_ref_impedance}}
\pysigstartsignatures
\pysiglinewithargsret{\sphinxbfcode{\sphinxupquote{change\_ref\_impedance}}}{\emph{\DUrole{n}{Znewinput}}, \emph{\DUrole{n}{inplace}\DUrole{o}{=}\DUrole{default_value}{\sphinxhyphen{} 1}}}{}
\pysigstopsignatures
\sphinxAtStartPar
Changes reference impedance and re\sphinxhyphen{}calculates S\sphinxhyphen{}Parameters.
\begin{quote}\begin{description}
\sphinxlineitem{Parameters}
\sphinxAtStartPar
\sphinxstyleliteralstrong{\sphinxupquote{Znew}} (\sphinxstyleliteralemphasis{\sphinxupquote{float}}\sphinxstyleliteralemphasis{\sphinxupquote{ or }}\sphinxstyleliteralemphasis{\sphinxupquote{list}}) \textendash{} New Reference Impedance. Its type can be:
\sphinxhyphen{} float: In this case Znew value is used for all ports
\sphinxhyphen{} list: In this case each element of this list is assgined to different ports in order as reference impedance. Length of \sphinxstyleemphasis{Znew} should be equal to number of ports. If an element of the list is None, then the reference impedance for corresponding port is not changed.

\sphinxlineitem{Returns}
\sphinxAtStartPar
The spfile object with new reference impedance

\sphinxlineitem{Return type}
\sphinxAtStartPar
{\hyperref[\detokenize{touchstone:touchstone.spfile}]{\sphinxcrossref{spfile}}}

\end{description}\end{quote}

\end{fulllineitems}

\index{check\_passivity() (touchstone.spfile method)@\spxentry{check\_passivity()}\spxextra{touchstone.spfile method}}

\begin{fulllineitems}
\phantomsection\label{\detokenize{touchstone:touchstone.spfile.check_passivity}}
\pysigstartsignatures
\pysiglinewithargsret{\sphinxbfcode{\sphinxupquote{check\_passivity}}}{}{}
\pysigstopsignatures
\sphinxAtStartPar
This method determines the frequencies and frequency indices at which the network is not passive.
Reference: Fast Passivity Enforcement of S\sphinxhyphen{}Parameter Macromodels by Pole Perturbation.pdf
For a better discussion: “S\sphinxhyphen{}Parameter Quality Metrics (Yuriy Shlepnev)”
\begin{quote}\begin{description}
\sphinxlineitem{Returns}
\sphinxAtStartPar
For non\sphinxhyphen{}passive frequencies (indices, frequencies, eigenvalues)

\sphinxlineitem{Return type}
\sphinxAtStartPar
3\sphinxhyphen{}tuple of lists

\end{description}\end{quote}

\end{fulllineitems}

\index{column\_of\_data() (touchstone.spfile method)@\spxentry{column\_of\_data()}\spxextra{touchstone.spfile method}}

\begin{fulllineitems}
\phantomsection\label{\detokenize{touchstone:touchstone.spfile.column_of_data}}
\pysigstartsignatures
\pysiglinewithargsret{\sphinxbfcode{\sphinxupquote{column\_of\_data}}}{\emph{\DUrole{n}{i}}, \emph{\DUrole{n}{j}}}{}
\pysigstopsignatures
\sphinxAtStartPar
Gets the indice of column at \sphinxstyleemphasis{sdata} matrix corresponding to \(S_{i j}\)
For internal use of the library.
\begin{quote}\begin{description}
\sphinxlineitem{Parameters}\begin{itemize}
\item {} 
\sphinxAtStartPar
\sphinxstyleliteralstrong{\sphinxupquote{i}} (\sphinxstyleliteralemphasis{\sphinxupquote{int}}) \textendash{} First index

\item {} 
\sphinxAtStartPar
\sphinxstyleliteralstrong{\sphinxupquote{j}} (\sphinxstyleliteralemphasis{\sphinxupquote{int}}) \textendash{} Second index

\end{itemize}

\sphinxlineitem{Returns}
\sphinxAtStartPar
Index of column

\sphinxlineitem{Return type}
\sphinxAtStartPar
int

\end{description}\end{quote}

\end{fulllineitems}

\index{conj\_match\_uncoupled() (touchstone.spfile method)@\spxentry{conj\_match\_uncoupled()}\spxextra{touchstone.spfile method}}

\begin{fulllineitems}
\phantomsection\label{\detokenize{touchstone:touchstone.spfile.conj_match_uncoupled}}
\pysigstartsignatures
\pysiglinewithargsret{\sphinxbfcode{\sphinxupquote{conj\_match\_uncoupled}}}{\emph{\DUrole{n}{ports}\DUrole{o}{=}\DUrole{default_value}{{[}{]}}}, \emph{\DUrole{n}{inplace}\DUrole{o}{=}\DUrole{default_value}{\sphinxhyphen{} 1}}, \emph{\DUrole{n}{noofiters}\DUrole{o}{=}\DUrole{default_value}{50}}}{}
\pysigstopsignatures
\sphinxAtStartPar
Sets the reference impedance for given ports as the complex conjugate of output impedance at each port.
The ports are assumed to be uncoupled. Coupling is taken care of by doing the same operation multiple times.
\begin{quote}\begin{description}
\sphinxlineitem{Parameters}\begin{itemize}
\item {} 
\sphinxAtStartPar
\sphinxstyleliteralstrong{\sphinxupquote{ports}} (\sphinxstyleliteralemphasis{\sphinxupquote{list}}\sphinxstyleliteralemphasis{\sphinxupquote{,}}\sphinxstyleliteralemphasis{\sphinxupquote{optional}}) \textendash{} {[}description{]}. Defaults to all ports.

\item {} 
\sphinxAtStartPar
\sphinxstyleliteralstrong{\sphinxupquote{inplace}} (\sphinxstyleliteralemphasis{\sphinxupquote{int}}\sphinxstyleliteralemphasis{\sphinxupquote{, }}\sphinxstyleliteralemphasis{\sphinxupquote{optional}}) \textendash{} Object editing mode. Defaults to \sphinxhyphen{}1.

\item {} 
\sphinxAtStartPar
\sphinxstyleliteralstrong{\sphinxupquote{noofiters}} (\sphinxstyleliteralemphasis{\sphinxupquote{int}}\sphinxstyleliteralemphasis{\sphinxupquote{, }}\sphinxstyleliteralemphasis{\sphinxupquote{optional}}) \textendash{} Numberof iterations. Defaults to 50.

\end{itemize}

\sphinxlineitem{Returns}
\sphinxAtStartPar
spfile object with new s\sphinxhyphen{}parameters

\end{description}\end{quote}

\end{fulllineitems}

\index{connect\_2\_ports() (touchstone.spfile method)@\spxentry{connect\_2\_ports()}\spxextra{touchstone.spfile method}}

\begin{fulllineitems}
\phantomsection\label{\detokenize{touchstone:touchstone.spfile.connect_2_ports}}
\pysigstartsignatures
\pysiglinewithargsret{\sphinxbfcode{\sphinxupquote{connect\_2\_ports}}}{\emph{\DUrole{n}{k}}, \emph{\DUrole{n}{m}}, \emph{\DUrole{n}{inplace}\DUrole{o}{=}\DUrole{default_value}{\sphinxhyphen{} 1}}}{}
\pysigstopsignatures
\sphinxAtStartPar
Port\sphinxhyphen{}m is connected to port\sphinxhyphen{}k and both ports are removed.
Reference: QUCS technical.pdf, S\sphinxhyphen{}parameters in CAE programs, p.29
\begin{quote}\begin{description}
\sphinxlineitem{Parameters}\begin{itemize}
\item {} 
\sphinxAtStartPar
\sphinxstyleliteralstrong{\sphinxupquote{k}} (\sphinxstyleliteralemphasis{\sphinxupquote{int}}) \textendash{} First port index to be connected.

\item {} 
\sphinxAtStartPar
\sphinxstyleliteralstrong{\sphinxupquote{m}} (\sphinxstyleliteralemphasis{\sphinxupquote{int}}) \textendash{} Second port index to be connected.

\item {} 
\sphinxAtStartPar
\sphinxstyleliteralstrong{\sphinxupquote{inplace}} (\sphinxstyleliteralemphasis{\sphinxupquote{int}}\sphinxstyleliteralemphasis{\sphinxupquote{, }}\sphinxstyleliteralemphasis{\sphinxupquote{optional}}) \textendash{} Object editing mode. Defaults to \sphinxhyphen{}1.

\end{itemize}

\sphinxlineitem{Returns}
\sphinxAtStartPar
New spfile object

\sphinxlineitem{Return type}
\sphinxAtStartPar
{\hyperref[\detokenize{touchstone:touchstone.spfile}]{\sphinxcrossref{spfile}}}

\end{description}\end{quote}

\end{fulllineitems}

\index{connect\_2\_ports\_list() (touchstone.spfile method)@\spxentry{connect\_2\_ports\_list()}\spxextra{touchstone.spfile method}}

\begin{fulllineitems}
\phantomsection\label{\detokenize{touchstone:touchstone.spfile.connect_2_ports_list}}
\pysigstartsignatures
\pysiglinewithargsret{\sphinxbfcode{\sphinxupquote{connect\_2\_ports\_list}}}{\emph{\DUrole{n}{conns}}, \emph{\DUrole{n}{inplace}\DUrole{o}{=}\DUrole{default_value}{\sphinxhyphen{} 1}}}{}
\pysigstopsignatures
\sphinxAtStartPar
Short circuit ports together one\sphinxhyphen{}to\sphinxhyphen{}one. Short circuited ports are removed.
Ports that will be connected are given as tuples in list \sphinxstyleemphasis{conns}
i.e. conns={[}(p1,p2),(p3,p4),..{]}
The order of remaining ports is kept.
Reference: QUCS technical.pdf, S\sphinxhyphen{}parameters in CAE programs, p.29
\begin{quote}\begin{description}
\sphinxlineitem{Parameters}\begin{itemize}
\item {} 
\sphinxAtStartPar
\sphinxstyleliteralstrong{\sphinxupquote{conns}} (\sphinxstyleliteralemphasis{\sphinxupquote{list of tuples}}) \textendash{} A list of 2\sphinxhyphen{}tuples of integers showing the ports connected

\item {} 
\sphinxAtStartPar
\sphinxstyleliteralstrong{\sphinxupquote{inplace}} (\sphinxstyleliteralemphasis{\sphinxupquote{int}}\sphinxstyleliteralemphasis{\sphinxupquote{, }}\sphinxstyleliteralemphasis{\sphinxupquote{optional}}) \textendash{} Object editing mode. Defaults to \sphinxhyphen{}1.

\end{itemize}

\sphinxlineitem{Returns}
\sphinxAtStartPar
New spfile object

\sphinxlineitem{Return type}
\sphinxAtStartPar
{\hyperref[\detokenize{touchstone:touchstone.spfile}]{\sphinxcrossref{spfile}}}

\end{description}\end{quote}

\end{fulllineitems}

\index{connect\_2\_ports\_retain() (touchstone.spfile method)@\spxentry{connect\_2\_ports\_retain()}\spxextra{touchstone.spfile method}}

\begin{fulllineitems}
\phantomsection\label{\detokenize{touchstone:touchstone.spfile.connect_2_ports_retain}}
\pysigstartsignatures
\pysiglinewithargsret{\sphinxbfcode{\sphinxupquote{connect\_2\_ports\_retain}}}{\emph{\DUrole{n}{k}}, \emph{\DUrole{n}{m}}, \emph{\DUrole{n}{inplace}\DUrole{o}{=}\DUrole{default_value}{\sphinxhyphen{} 1}}}{}
\pysigstopsignatures
\sphinxAtStartPar
Port\sphinxhyphen{}m is connected to port\sphinxhyphen{}k and both ports are removed. New port becomes the last port of the circuit.
Reference: QUCS technical.pdf, S\sphinxhyphen{}parameters in CAE programs, p.29
\begin{quote}\begin{description}
\sphinxlineitem{Parameters}\begin{itemize}
\item {} 
\sphinxAtStartPar
\sphinxstyleliteralstrong{\sphinxupquote{k}} (\sphinxstyleliteralemphasis{\sphinxupquote{int}}) \textendash{} First port index to be connected.

\item {} 
\sphinxAtStartPar
\sphinxstyleliteralstrong{\sphinxupquote{m}} (\sphinxstyleliteralemphasis{\sphinxupquote{int}}) \textendash{} Second port index to be connected.

\item {} 
\sphinxAtStartPar
\sphinxstyleliteralstrong{\sphinxupquote{inplace}} (\sphinxstyleliteralemphasis{\sphinxupquote{int}}\sphinxstyleliteralemphasis{\sphinxupquote{, }}\sphinxstyleliteralemphasis{\sphinxupquote{optional}}) \textendash{} Object editing mode. Defaults to \sphinxhyphen{}1.

\end{itemize}

\sphinxlineitem{Returns}
\sphinxAtStartPar
New \sphinxstyleemphasis{spfile} object

\sphinxlineitem{Return type}
\sphinxAtStartPar
{\hyperref[\detokenize{touchstone:touchstone.spfile}]{\sphinxcrossref{spfile}}}

\end{description}\end{quote}

\end{fulllineitems}

\index{connect\_network\_1\_conn() (touchstone.spfile method)@\spxentry{connect\_network\_1\_conn()}\spxextra{touchstone.spfile method}}

\begin{fulllineitems}
\phantomsection\label{\detokenize{touchstone:touchstone.spfile.connect_network_1_conn}}
\pysigstartsignatures
\pysiglinewithargsret{\sphinxbfcode{\sphinxupquote{connect\_network\_1\_conn}}}{\emph{\DUrole{n}{EX}}, \emph{\DUrole{n}{k}}, \emph{\DUrole{n}{m}}, \emph{\DUrole{n}{preserveportnumbers}\DUrole{o}{=}\DUrole{default_value}{False}}, \emph{\DUrole{n}{inplace}\DUrole{o}{=}\DUrole{default_value}{\sphinxhyphen{} 1}}}{}
\pysigstopsignatures
\sphinxAtStartPar
Port\sphinxhyphen{}m of EX circuit is connected to port\sphinxhyphen{}k of this circuit. Both of these ports will be removed.
Remaining ports of EX are added to the port list of this circuit in order.
Reference: QUCS technical.pdf, S\sphinxhyphen{}parameters in CAE programs, p.29
\begin{quote}\begin{description}
\sphinxlineitem{Parameters}\begin{itemize}
\item {} 
\sphinxAtStartPar
\sphinxstyleliteralstrong{\sphinxupquote{EX}} ({\hyperref[\detokenize{touchstone:touchstone.spfile}]{\sphinxcrossref{\sphinxstyleliteralemphasis{\sphinxupquote{spfile}}}}}) \textendash{} External network to be connected to this.

\item {} 
\sphinxAtStartPar
\sphinxstyleliteralstrong{\sphinxupquote{k}} (\sphinxstyleliteralemphasis{\sphinxupquote{int}}) \textendash{} Port number of self to be connected.

\item {} 
\sphinxAtStartPar
\sphinxstyleliteralstrong{\sphinxupquote{m}} (\sphinxstyleliteralemphasis{\sphinxupquote{int}}) \textendash{} Port number of EX to be connected.

\item {} 
\sphinxAtStartPar
\sphinxstyleliteralstrong{\sphinxupquote{inplace}} (\sphinxstyleliteralemphasis{\sphinxupquote{int}}\sphinxstyleliteralemphasis{\sphinxupquote{, }}\sphinxstyleliteralemphasis{\sphinxupquote{optional}}) \textendash{} Object editing mode. Defaults to \sphinxhyphen{}1.

\item {} 
\sphinxAtStartPar
\sphinxstyleliteralstrong{\sphinxupquote{preserveportnumbers}} (\sphinxstyleliteralemphasis{\sphinxupquote{bool}}\sphinxstyleliteralemphasis{\sphinxupquote{, }}\sphinxstyleliteralemphasis{\sphinxupquote{optional}}) \textendash{} if True, the number of the first added port will be k. Defaults to False.

\end{itemize}

\sphinxlineitem{Returns}
\sphinxAtStartPar
Connected network

\sphinxlineitem{Return type}
\sphinxAtStartPar
{\hyperref[\detokenize{touchstone:touchstone.spfile}]{\sphinxcrossref{spfile}}}

\end{description}\end{quote}

\end{fulllineitems}

\index{connect\_network\_1\_conn\_retain() (touchstone.spfile method)@\spxentry{connect\_network\_1\_conn\_retain()}\spxextra{touchstone.spfile method}}

\begin{fulllineitems}
\phantomsection\label{\detokenize{touchstone:touchstone.spfile.connect_network_1_conn_retain}}
\pysigstartsignatures
\pysiglinewithargsret{\sphinxbfcode{\sphinxupquote{connect\_network\_1\_conn\_retain}}}{\emph{\DUrole{n}{EX}}, \emph{\DUrole{n}{k}}, \emph{\DUrole{n}{m}}, \emph{\DUrole{n}{inplace}\DUrole{o}{=}\DUrole{default_value}{\sphinxhyphen{} 1}}}{}
\pysigstopsignatures
\sphinxAtStartPar
Port\sphinxhyphen{}m of EX circuit is connected to port\sphinxhyphen{}k of this circuit. This connection point will also be a port. Remaining ports of EX are added to the port list of this circuit in order. The port of connection point will be the last port of the final network.
Reference: QUCS technical.pdf, S\sphinxhyphen{}parameters in CAE programs, p.29
\begin{quote}\begin{description}
\sphinxlineitem{Parameters}\begin{itemize}
\item {} 
\sphinxAtStartPar
\sphinxstyleliteralstrong{\sphinxupquote{EX}} ({\hyperref[\detokenize{touchstone:touchstone.spfile}]{\sphinxcrossref{\sphinxstyleliteralemphasis{\sphinxupquote{spfile}}}}}) \textendash{} External network to be connected to this.

\item {} 
\sphinxAtStartPar
\sphinxstyleliteralstrong{\sphinxupquote{k}} (\sphinxstyleliteralemphasis{\sphinxupquote{int}}) \textendash{} Port number of self to be connected.

\item {} 
\sphinxAtStartPar
\sphinxstyleliteralstrong{\sphinxupquote{m}} (\sphinxstyleliteralemphasis{\sphinxupquote{int}}) \textendash{} Port number of EX to be connected.

\item {} 
\sphinxAtStartPar
\sphinxstyleliteralstrong{\sphinxupquote{inplace}} (\sphinxstyleliteralemphasis{\sphinxupquote{int}}\sphinxstyleliteralemphasis{\sphinxupquote{, }}\sphinxstyleliteralemphasis{\sphinxupquote{optional}}) \textendash{} Object editing mode. Defaults to \sphinxhyphen{}1.

\item {} 
\sphinxAtStartPar
\sphinxstyleliteralstrong{\sphinxupquote{preserveportnumbers1}} (\sphinxstyleliteralemphasis{\sphinxupquote{bool}}\sphinxstyleliteralemphasis{\sphinxupquote{, }}\sphinxstyleliteralemphasis{\sphinxupquote{optional}}) \textendash{} if True, the number of the first added port will be k. Defaults to False.

\end{itemize}

\sphinxlineitem{Returns}
\sphinxAtStartPar
Connected network

\sphinxlineitem{Return type}
\sphinxAtStartPar
{\hyperref[\detokenize{touchstone:touchstone.spfile}]{\sphinxcrossref{spfile}}}

\end{description}\end{quote}

\end{fulllineitems}

\index{convert\_s1p\_to\_s2p() (touchstone.spfile method)@\spxentry{convert\_s1p\_to\_s2p()}\spxextra{touchstone.spfile method}}

\begin{fulllineitems}
\phantomsection\label{\detokenize{touchstone:touchstone.spfile.convert_s1p_to_s2p}}
\pysigstartsignatures
\pysiglinewithargsret{\sphinxbfcode{\sphinxupquote{convert\_s1p\_to\_s2p}}}{}{}
\pysigstopsignatures
\end{fulllineitems}

\index{copy() (touchstone.spfile method)@\spxentry{copy()}\spxextra{touchstone.spfile method}}

\begin{fulllineitems}
\phantomsection\label{\detokenize{touchstone:touchstone.spfile.copy}}
\pysigstartsignatures
\pysiglinewithargsret{\sphinxbfcode{\sphinxupquote{copy}}}{}{}
\pysigstopsignatures
\end{fulllineitems}

\index{copy\_data\_from\_spfile() (touchstone.spfile method)@\spxentry{copy\_data\_from\_spfile()}\spxextra{touchstone.spfile method}}

\begin{fulllineitems}
\phantomsection\label{\detokenize{touchstone:touchstone.spfile.copy_data_from_spfile}}
\pysigstartsignatures
\pysiglinewithargsret{\sphinxbfcode{\sphinxupquote{copy\_data\_from\_spfile}}}{\emph{\DUrole{n}{local\_i}}, \emph{\DUrole{n}{local\_j}}, \emph{\DUrole{n}{source\_i}}, \emph{\DUrole{n}{source\_j}}, \emph{\DUrole{n}{sourcespfile}}}{}
\pysigstopsignatures
\sphinxAtStartPar
This method copies S\sphinxhyphen{}Parameter data from another SPFILE object

\end{fulllineitems}

\index{cpwgline() (touchstone.spfile class method)@\spxentry{cpwgline()}\spxextra{touchstone.spfile class method}}

\begin{fulllineitems}
\phantomsection\label{\detokenize{touchstone:touchstone.spfile.cpwgline}}
\pysigstartsignatures
\pysiglinewithargsret{\sphinxbfcode{\sphinxupquote{classmethod\DUrole{w}{  }}}\sphinxbfcode{\sphinxupquote{cpwgline}}}{\emph{\DUrole{n}{length}}, \emph{\DUrole{n}{w}}, \emph{\DUrole{n}{th}}, \emph{\DUrole{n}{er}}, \emph{\DUrole{n}{s}}, \emph{\DUrole{n}{h}}, \emph{\DUrole{n}{freqs}\DUrole{o}{=}\DUrole{default_value}{None}}}{}
\pysigstopsignatures
\sphinxAtStartPar
Create an \sphinxcode{\sphinxupquote{spfile}} object corresponding to a cpwg transmission line.
\begin{quote}\begin{description}
\sphinxlineitem{Parameters}\begin{itemize}
\item {} 
\sphinxAtStartPar
\sphinxstyleliteralstrong{\sphinxupquote{length}} (\sphinxstyleliteralemphasis{\sphinxupquote{float}}) \textendash{} Length of cpwg line.

\item {} 
\sphinxAtStartPar
\sphinxstyleliteralstrong{\sphinxupquote{w}} (\sphinxstyleliteralemphasis{\sphinxupquote{float}}) \textendash{} Width of cpwg line.

\item {} 
\sphinxAtStartPar
\sphinxstyleliteralstrong{\sphinxupquote{th}} (\sphinxstyleliteralemphasis{\sphinxupquote{float}}) \textendash{} Thickness of metal.

\item {} 
\sphinxAtStartPar
\sphinxstyleliteralstrong{\sphinxupquote{er}} (\sphinxstyleliteralemphasis{\sphinxupquote{float}}) \textendash{} Relative permittivity of substrate.

\item {} 
\sphinxAtStartPar
\sphinxstyleliteralstrong{\sphinxupquote{s}} (\sphinxstyleliteralemphasis{\sphinxupquote{float}}) \textendash{} Gap of cpwg line.

\item {} 
\sphinxAtStartPar
\sphinxstyleliteralstrong{\sphinxupquote{h}} (\sphinxstyleliteralemphasis{\sphinxupquote{float}}) \textendash{} Thickness of substrate.

\item {} 
\sphinxAtStartPar
\sphinxstyleliteralstrong{\sphinxupquote{freqs}} (\sphinxstyleliteralemphasis{\sphinxupquote{float}}\sphinxstyleliteralemphasis{\sphinxupquote{, }}\sphinxstyleliteralemphasis{\sphinxupquote{optional}}) \textendash{} Frequency list of object. Defaults to None. If None, frequencies should be set later.

\end{itemize}

\sphinxlineitem{Returns}
\sphinxAtStartPar
An spfile object.

\sphinxlineitem{Return type}
\sphinxAtStartPar
{\hyperref[\detokenize{touchstone:touchstone.spfile}]{\sphinxcrossref{spfile}}}

\end{description}\end{quote}

\end{fulllineitems}

\index{crop\_with\_frequency() (touchstone.spfile method)@\spxentry{crop\_with\_frequency()}\spxextra{touchstone.spfile method}}

\begin{fulllineitems}
\phantomsection\label{\detokenize{touchstone:touchstone.spfile.crop_with_frequency}}
\pysigstartsignatures
\pysiglinewithargsret{\sphinxbfcode{\sphinxupquote{crop\_with\_frequency}}}{\emph{\DUrole{n}{fstart}\DUrole{o}{=}\DUrole{default_value}{None}}, \emph{\DUrole{n}{fstop}\DUrole{o}{=}\DUrole{default_value}{None}}, \emph{\DUrole{n}{inplace}\DUrole{o}{=}\DUrole{default_value}{\sphinxhyphen{} 1}}}{}
\pysigstopsignatures
\sphinxAtStartPar
Crop the points below fstart and above fstop. No recalculation or interpolation occurs.
\begin{quote}\begin{description}
\sphinxlineitem{Parameters}\begin{itemize}
\item {} 
\sphinxAtStartPar
\sphinxstyleliteralstrong{\sphinxupquote{fstart}} (\sphinxstyleliteralemphasis{\sphinxupquote{float}}\sphinxstyleliteralemphasis{\sphinxupquote{, }}\sphinxstyleliteralemphasis{\sphinxupquote{optional}}) \textendash{} Lower frequency for cropping. Default value is None which means no cropping will occur at lower frequency side.

\item {} 
\sphinxAtStartPar
\sphinxstyleliteralstrong{\sphinxupquote{fstop}} (\sphinxstyleliteralemphasis{\sphinxupquote{float}}\sphinxstyleliteralemphasis{\sphinxupquote{, }}\sphinxstyleliteralemphasis{\sphinxupquote{optional}}) \textendash{} Higher frequency for cropping. Default value is None which means no cropping will occur at higher frequency side.

\item {} 
\sphinxAtStartPar
\sphinxstyleliteralstrong{\sphinxupquote{inplace}} (\sphinxstyleliteralemphasis{\sphinxupquote{int}}\sphinxstyleliteralemphasis{\sphinxupquote{, }}\sphinxstyleliteralemphasis{\sphinxupquote{optional}}) \textendash{} Object editing mode. Defaults to \sphinxhyphen{}1.

\end{itemize}

\sphinxlineitem{Returns}
\sphinxAtStartPar
spfile object with new frequency points.

\sphinxlineitem{Return type}
\sphinxAtStartPar
{\hyperref[\detokenize{touchstone:touchstone.spfile}]{\sphinxcrossref{spfile}}}

\end{description}\end{quote}

\end{fulllineitems}

\index{data\_array() (touchstone.spfile method)@\spxentry{data\_array()}\spxextra{touchstone.spfile method}}

\begin{fulllineitems}
\phantomsection\label{\detokenize{touchstone:touchstone.spfile.data_array}}
\pysigstartsignatures
\pysiglinewithargsret{\sphinxbfcode{\sphinxupquote{data\_array}}}{\emph{\DUrole{n}{data\_format}\DUrole{o}{=}\DUrole{default_value}{\textquotesingle{}DB\textquotesingle{}}}, \emph{\DUrole{n}{M}\DUrole{o}{=}\DUrole{default_value}{\textquotesingle{}S\textquotesingle{}}}, \emph{\DUrole{n}{i}\DUrole{o}{=}\DUrole{default_value}{1}}, \emph{\DUrole{n}{j}\DUrole{o}{=}\DUrole{default_value}{1}}, \emph{\DUrole{n}{frequencies}\DUrole{o}{=}\DUrole{default_value}{None}}, \emph{\DUrole{n}{ref}\DUrole{o}{=}\DUrole{default_value}{None}}, \emph{\DUrole{n}{DCInt}\DUrole{o}{=}\DUrole{default_value}{0}}, \emph{\DUrole{n}{DCValue}\DUrole{o}{=}\DUrole{default_value}{0.0}}, \emph{\DUrole{n}{smoothing}\DUrole{o}{=}\DUrole{default_value}{0}}, \emph{\DUrole{n}{InterpolationConstant}\DUrole{o}{=}\DUrole{default_value}{0}}}{}
\pysigstopsignatures
\sphinxAtStartPar
Return a network parameter between ports \sphinxstyleemphasis{i} and \sphinxstyleemphasis{j} (\(M_{i j}\)) at specified frequencies in specified format.
\begin{quote}\begin{description}
\sphinxlineitem{Parameters}\begin{itemize}
\item {} 
\sphinxAtStartPar
\sphinxstyleliteralstrong{\sphinxupquote{data\_format}} (\sphinxstyleliteralemphasis{\sphinxupquote{str}}\sphinxstyleliteralemphasis{\sphinxupquote{, }}\sphinxstyleliteralemphasis{\sphinxupquote{optional}}) \textendash{} Defaults to “DB”. The format of the data returned. Possible values (case insensitive):
\sphinxhyphen{}   “K”: Stability factor of 2\sphinxhyphen{}port
\sphinxhyphen{}   “MU1”: Input stability factor of 2\sphinxhyphen{}port
\sphinxhyphen{}   “MU2”: Output stability factor of 2\sphinxhyphen{}port
\sphinxhyphen{}   “VSWR”: VSWR ar port i
\sphinxhyphen{}   “MAG”: Magnitude of \(M_{i j}\)
\sphinxhyphen{}   “DB”: Magnitude of \(M_{i j}\) in dB
\sphinxhyphen{}   “REAL”: Real part of \(M_{i j}\)
\sphinxhyphen{}   “IMAG”: Imaginary part of \(M_{i j}\)
\sphinxhyphen{}   “PHASE”: Phase of \(M_{i j}\) in degrees between 0\sphinxhyphen{}360
\sphinxhyphen{}   “UNWRAPPEDPHASE”: Unwrapped Phase of \(M_{i j}\) in degrees
\sphinxhyphen{}   “GROUPDELAY”: Group Delay of \(M_{i j}\) in degrees

\item {} 
\sphinxAtStartPar
\sphinxstyleliteralstrong{\sphinxupquote{M}} (\sphinxstyleliteralemphasis{\sphinxupquote{str}}\sphinxstyleliteralemphasis{\sphinxupquote{, }}\sphinxstyleliteralemphasis{\sphinxupquote{optional}}) \textendash{} Defaults to “S”. Possible values (case insensitive):
\sphinxhyphen{}   “S”: Return S\sphinxhyphen{}parameter data
\sphinxhyphen{}   “Y”: Return Y\sphinxhyphen{}parameter data
\sphinxhyphen{}   “Z”: Return Z\sphinxhyphen{}parameter data
\sphinxhyphen{}   “ABCD”: Return ABCD\sphinxhyphen{}parameter data

\item {} 
\sphinxAtStartPar
\sphinxstyleliteralstrong{\sphinxupquote{i}} (\sphinxstyleliteralemphasis{\sphinxupquote{int}}\sphinxstyleliteralemphasis{\sphinxupquote{, }}\sphinxstyleliteralemphasis{\sphinxupquote{optional}}) \textendash{} First port number. Defaults to 1.

\item {} 
\sphinxAtStartPar
\sphinxstyleliteralstrong{\sphinxupquote{j}} (\sphinxstyleliteralemphasis{\sphinxupquote{int}}\sphinxstyleliteralemphasis{\sphinxupquote{, }}\sphinxstyleliteralemphasis{\sphinxupquote{optional}}) \textendash{} Second port number. Defaults to 1. Ignored for {\color{red}\bfseries{}*}data\_format*=”VSWR”

\item {} 
\sphinxAtStartPar
\sphinxstyleliteralstrong{\sphinxupquote{frequencies}} (\sphinxstyleliteralemphasis{\sphinxupquote{list}}\sphinxstyleliteralemphasis{\sphinxupquote{, }}\sphinxstyleliteralemphasis{\sphinxupquote{optional}}) \textendash{} Defaults to {[}{]}. List of frequencies in Hz. If an empty list is given, networks whole frequency range is used.

\item {} 
\sphinxAtStartPar
\sphinxstyleliteralstrong{\sphinxupquote{ref}} ({\hyperref[\detokenize{touchstone:touchstone.spfile}]{\sphinxcrossref{\sphinxstyleliteralemphasis{\sphinxupquote{spfile}}}}}\sphinxstyleliteralemphasis{\sphinxupquote{, }}\sphinxstyleliteralemphasis{\sphinxupquote{optional}}) \textendash{} Defaults to None. If given the data of this network is subtracted from the same data of \sphinxstyleemphasis{ref} object.

\item {} 
\sphinxAtStartPar
\sphinxstyleliteralstrong{\sphinxupquote{DCInt}} (\sphinxstyleliteralemphasis{\sphinxupquote{int}}\sphinxstyleliteralemphasis{\sphinxupquote{, }}\sphinxstyleliteralemphasis{\sphinxupquote{optional}}) \textendash{} Defaults to 0. If 1, DC point given by \sphinxstyleemphasis{DCValue} is used at frequency interpolation if \sphinxstyleemphasis{frequencies} is not {[}{]}.

\item {} 
\sphinxAtStartPar
\sphinxstyleliteralstrong{\sphinxupquote{DCValue}} (\sphinxstyleliteralemphasis{\sphinxupquote{complex}}\sphinxstyleliteralemphasis{\sphinxupquote{, }}\sphinxstyleliteralemphasis{\sphinxupquote{optional}}) \textendash{} Defaults to 0.0. DCValue that can be used for interpolation over frequency.

\item {} 
\sphinxAtStartPar
\sphinxstyleliteralstrong{\sphinxupquote{smoothing}} (\sphinxstyleliteralemphasis{\sphinxupquote{int}}\sphinxstyleliteralemphasis{\sphinxupquote{, }}\sphinxstyleliteralemphasis{\sphinxupquote{optional}}) \textendash{} Defaults to 0. if this is higher than 0, it is used as the number of points for smoothing.

\item {} 
\sphinxAtStartPar
\sphinxstyleliteralstrong{\sphinxupquote{InterpolationConstant}} (\sphinxstyleliteralemphasis{\sphinxupquote{int}}\sphinxstyleliteralemphasis{\sphinxupquote{, }}\sphinxstyleliteralemphasis{\sphinxupquote{optional}}) \textendash{} Defaults to 0. If this is higher than 0, it is taken as the number of frequencies that will be added between 2 consecutive frequency points. By this way, number of frequencies is increased by interpolation.

\end{itemize}

\sphinxlineitem{Returns}
\sphinxAtStartPar
Network data array

\sphinxlineitem{Return type}
\sphinxAtStartPar
numpy.array

\end{description}\end{quote}

\end{fulllineitems}

\index{gav() (touchstone.spfile method)@\spxentry{gav()}\spxextra{touchstone.spfile method}}

\begin{fulllineitems}
\phantomsection\label{\detokenize{touchstone:touchstone.spfile.gav}}
\pysigstartsignatures
\pysiglinewithargsret{\sphinxbfcode{\sphinxupquote{gav}}}{\emph{\DUrole{n}{port1}\DUrole{o}{=}\DUrole{default_value}{1}}, \emph{\DUrole{n}{port2}\DUrole{o}{=}\DUrole{default_value}{2}}, \emph{\DUrole{n}{ZS}\DUrole{o}{=}\DUrole{default_value}{{[}{]}}}, \emph{\DUrole{n}{dB}\DUrole{o}{=}\DUrole{default_value}{True}}}{}
\pysigstopsignatures
\sphinxAtStartPar
Available gain from port1 to port2. If dB=True, output is in dB, otherwise it is a power ratio.
\begin{quote}
\begin{equation*}
\begin{split}G_{av}=\frac{P_{av,toLoad}}{P_{av,fromSource}}\end{split}
\end{equation*}\end{quote}
\begin{quote}\begin{description}
\sphinxlineitem{Parameters}\begin{itemize}
\item {} 
\sphinxAtStartPar
\sphinxstyleliteralstrong{\sphinxupquote{port1}} (\sphinxstyleliteralemphasis{\sphinxupquote{int}}\sphinxstyleliteralemphasis{\sphinxupquote{, }}\sphinxstyleliteralemphasis{\sphinxupquote{optional}}) \textendash{} Index of input port. Defaults to 1.

\item {} 
\sphinxAtStartPar
\sphinxstyleliteralstrong{\sphinxupquote{port2}} (\sphinxstyleliteralemphasis{\sphinxupquote{int}}\sphinxstyleliteralemphasis{\sphinxupquote{, }}\sphinxstyleliteralemphasis{\sphinxupquote{optional}}) \textendash{} Index of output port. Defaults to 2.

\item {} 
\sphinxAtStartPar
\sphinxstyleliteralstrong{\sphinxupquote{ZS}} (\sphinxstyleliteralemphasis{\sphinxupquote{list}}\sphinxstyleliteralemphasis{\sphinxupquote{ or }}\sphinxstyleliteralemphasis{\sphinxupquote{numpy.ndarray}}\sphinxstyleliteralemphasis{\sphinxupquote{, }}\sphinxstyleliteralemphasis{\sphinxupquote{optional}}) \textendash{} Impedance of input port. Defaults to current reference impedance.

\item {} 
\sphinxAtStartPar
\sphinxstyleliteralstrong{\sphinxupquote{dB}} (\sphinxstyleliteralemphasis{\sphinxupquote{bool}}\sphinxstyleliteralemphasis{\sphinxupquote{, }}\sphinxstyleliteralemphasis{\sphinxupquote{optional}}) \textendash{} Enable dB output. Defaults to True.

\end{itemize}

\sphinxlineitem{Returns}
\sphinxAtStartPar
Array of Gmax values for all frequencies

\sphinxlineitem{Return type}
\sphinxAtStartPar
numpy.ndarray

\end{description}\end{quote}

\end{fulllineitems}

\index{get\_formulation() (touchstone.spfile method)@\spxentry{get\_formulation()}\spxextra{touchstone.spfile method}}

\begin{fulllineitems}
\phantomsection\label{\detokenize{touchstone:touchstone.spfile.get_formulation}}
\pysigstartsignatures
\pysiglinewithargsret{\sphinxbfcode{\sphinxupquote{get\_formulation}}}{}{}
\pysigstopsignatures
\end{fulllineitems}

\index{get\_frequency\_list() (touchstone.spfile method)@\spxentry{get\_frequency\_list()}\spxextra{touchstone.spfile method}}

\begin{fulllineitems}
\phantomsection\label{\detokenize{touchstone:touchstone.spfile.get_frequency_list}}
\pysigstartsignatures
\pysiglinewithargsret{\sphinxbfcode{\sphinxupquote{get\_frequency\_list}}}{}{}
\pysigstopsignatures
\sphinxAtStartPar
Returns the frequency list of network
\begin{quote}\begin{description}
\sphinxlineitem{Returns}
\sphinxAtStartPar
Frequency list of network

\sphinxlineitem{Return type}
\sphinxAtStartPar
numpy.array

\end{description}\end{quote}

\end{fulllineitems}

\index{get\_no\_of\_ports() (touchstone.spfile method)@\spxentry{get\_no\_of\_ports()}\spxextra{touchstone.spfile method}}

\begin{fulllineitems}
\phantomsection\label{\detokenize{touchstone:touchstone.spfile.get_no_of_ports}}
\pysigstartsignatures
\pysiglinewithargsret{\sphinxbfcode{\sphinxupquote{get\_no\_of\_ports}}}{}{}
\pysigstopsignatures
\end{fulllineitems}

\index{get\_port\_names() (touchstone.spfile method)@\spxentry{get\_port\_names()}\spxextra{touchstone.spfile method}}

\begin{fulllineitems}
\phantomsection\label{\detokenize{touchstone:touchstone.spfile.get_port_names}}
\pysigstartsignatures
\pysiglinewithargsret{\sphinxbfcode{\sphinxupquote{get\_port\_names}}}{}{}
\pysigstopsignatures
\sphinxAtStartPar
Get list of port names.

\end{fulllineitems}

\index{get\_port\_number\_from\_name() (touchstone.spfile method)@\spxentry{get\_port\_number\_from\_name()}\spxextra{touchstone.spfile method}}

\begin{fulllineitems}
\phantomsection\label{\detokenize{touchstone:touchstone.spfile.get_port_number_from_name}}
\pysigstartsignatures
\pysiglinewithargsret{\sphinxbfcode{\sphinxupquote{get\_port\_number\_from\_name}}}{\emph{\DUrole{n}{isim}}}{}
\pysigstopsignatures
\sphinxAtStartPar
Index of first port index with name \sphinxstyleemphasis{isim}
\begin{quote}\begin{description}
\sphinxlineitem{Parameters}
\sphinxAtStartPar
\sphinxstyleliteralstrong{\sphinxupquote{isim}} (\sphinxstyleliteralemphasis{\sphinxupquote{bool}}) \textendash{} Name of the port

\sphinxlineitem{Returns}
\sphinxAtStartPar
Port index if port is found, 0 otherwise

\sphinxlineitem{Return type}
\sphinxAtStartPar
int

\end{description}\end{quote}

\end{fulllineitems}

\index{get\_sym\_parameters() (touchstone.spfile method)@\spxentry{get\_sym\_parameters()}\spxextra{touchstone.spfile method}}

\begin{fulllineitems}
\phantomsection\label{\detokenize{touchstone:touchstone.spfile.get_sym_parameters}}
\pysigstartsignatures
\pysiglinewithargsret{\sphinxbfcode{\sphinxupquote{get\_sym\_parameters}}}{}{}
\pysigstopsignatures
\sphinxAtStartPar
This function is used to get the values of symbolic variables of the network.
\begin{quote}\begin{description}
\sphinxlineitem{Returns}
\sphinxAtStartPar
This is a dictionary containing the values of symbolic variables of the network

\sphinxlineitem{Return type}
\sphinxAtStartPar
dict

\end{description}\end{quote}

\end{fulllineitems}

\index{get\_sym\_smatrix() (touchstone.spfile method)@\spxentry{get\_sym\_smatrix()}\spxextra{touchstone.spfile method}}

\begin{fulllineitems}
\phantomsection\label{\detokenize{touchstone:touchstone.spfile.get_sym_smatrix}}
\pysigstartsignatures
\pysiglinewithargsret{\sphinxbfcode{\sphinxupquote{get\_sym\_smatrix}}}{}{}
\pysigstopsignatures
\end{fulllineitems}

\index{get\_undefinedYindices() (touchstone.spfile method)@\spxentry{get\_undefinedYindices()}\spxextra{touchstone.spfile method}}

\begin{fulllineitems}
\phantomsection\label{\detokenize{touchstone:touchstone.spfile.get_undefinedYindices}}
\pysigstartsignatures
\pysiglinewithargsret{\sphinxbfcode{\sphinxupquote{get\_undefinedYindices}}}{}{}
\pysigstopsignatures
\end{fulllineitems}

\index{get\_undefinedZindices() (touchstone.spfile method)@\spxentry{get\_undefinedZindices()}\spxextra{touchstone.spfile method}}

\begin{fulllineitems}
\phantomsection\label{\detokenize{touchstone:touchstone.spfile.get_undefinedZindices}}
\pysigstartsignatures
\pysiglinewithargsret{\sphinxbfcode{\sphinxupquote{get\_undefinedZindices}}}{}{}
\pysigstopsignatures
\end{fulllineitems}

\index{getdata\_format() (touchstone.spfile method)@\spxentry{getdata\_format()}\spxextra{touchstone.spfile method}}

\begin{fulllineitems}
\phantomsection\label{\detokenize{touchstone:touchstone.spfile.getdata_format}}
\pysigstartsignatures
\pysiglinewithargsret{\sphinxbfcode{\sphinxupquote{getdata\_format}}}{}{}
\pysigstopsignatures
\end{fulllineitems}

\index{getfilename() (touchstone.spfile method)@\spxentry{getfilename()}\spxextra{touchstone.spfile method}}

\begin{fulllineitems}
\phantomsection\label{\detokenize{touchstone:touchstone.spfile.getfilename}}
\pysigstartsignatures
\pysiglinewithargsret{\sphinxbfcode{\sphinxupquote{getfilename}}}{}{}
\pysigstopsignatures
\end{fulllineitems}

\index{gmax() (touchstone.spfile method)@\spxentry{gmax()}\spxextra{touchstone.spfile method}}

\begin{fulllineitems}
\phantomsection\label{\detokenize{touchstone:touchstone.spfile.gmax}}
\pysigstartsignatures
\pysiglinewithargsret{\sphinxbfcode{\sphinxupquote{gmax}}}{\emph{\DUrole{n}{port1}\DUrole{o}{=}\DUrole{default_value}{1}}, \emph{\DUrole{n}{port2}\DUrole{o}{=}\DUrole{default_value}{2}}, \emph{\DUrole{n}{dB}\DUrole{o}{=}\DUrole{default_value}{True}}}{}
\pysigstopsignatures
\sphinxAtStartPar
Calculates Gmax from port1 to port2. Other ports are terminated with current reference impedances. If dB=True, output is in dB, otherwise it is a power ratio.
\begin{quote}\begin{description}
\sphinxlineitem{Parameters}\begin{itemize}
\item {} 
\sphinxAtStartPar
\sphinxstyleliteralstrong{\sphinxupquote{port1}} (\sphinxstyleliteralemphasis{\sphinxupquote{int}}\sphinxstyleliteralemphasis{\sphinxupquote{, }}\sphinxstyleliteralemphasis{\sphinxupquote{optional}}) \textendash{} Index of input port. Defaults to 1.

\item {} 
\sphinxAtStartPar
\sphinxstyleliteralstrong{\sphinxupquote{port2}} (\sphinxstyleliteralemphasis{\sphinxupquote{int}}\sphinxstyleliteralemphasis{\sphinxupquote{, }}\sphinxstyleliteralemphasis{\sphinxupquote{optional}}) \textendash{} Index of output port. Defaults to 2.

\item {} 
\sphinxAtStartPar
\sphinxstyleliteralstrong{\sphinxupquote{dB}} (\sphinxstyleliteralemphasis{\sphinxupquote{bool}}\sphinxstyleliteralemphasis{\sphinxupquote{, }}\sphinxstyleliteralemphasis{\sphinxupquote{optional}}) \textendash{} Enable dB output. Defaults to True.

\end{itemize}

\sphinxlineitem{Returns}
\sphinxAtStartPar
Array of Gmax values for all frequencies

\sphinxlineitem{Return type}
\sphinxAtStartPar
numpy.ndarray

\end{description}\end{quote}

\end{fulllineitems}

\index{gop() (touchstone.spfile method)@\spxentry{gop()}\spxextra{touchstone.spfile method}}

\begin{fulllineitems}
\phantomsection\label{\detokenize{touchstone:touchstone.spfile.gop}}
\pysigstartsignatures
\pysiglinewithargsret{\sphinxbfcode{\sphinxupquote{gop}}}{\emph{\DUrole{n}{port1}\DUrole{o}{=}\DUrole{default_value}{1}}, \emph{\DUrole{n}{port2}\DUrole{o}{=}\DUrole{default_value}{2}}, \emph{\DUrole{n}{ZL}\DUrole{o}{=}\DUrole{default_value}{None}}, \emph{\DUrole{n}{dB}\DUrole{o}{=}\DUrole{default_value}{True}}}{}
\pysigstopsignatures
\sphinxAtStartPar
Operating power gain from port1 to port2 with load impedance of ZL. If dB=True, output is in dB, otherwise it is a power ratio.
\begin{quote}
\begin{equation*}
\begin{split}G_{op}=\frac{P_{toLoad}}{P_{toNetwork}}\end{split}
\end{equation*}\end{quote}
\begin{quote}\begin{description}
\sphinxlineitem{Parameters}\begin{itemize}
\item {} 
\sphinxAtStartPar
\sphinxstyleliteralstrong{\sphinxupquote{port1}} (\sphinxstyleliteralemphasis{\sphinxupquote{int}}\sphinxstyleliteralemphasis{\sphinxupquote{, }}\sphinxstyleliteralemphasis{\sphinxupquote{optional}}) \textendash{} Index of input port. Defaults to 1.

\item {} 
\sphinxAtStartPar
\sphinxstyleliteralstrong{\sphinxupquote{port2}} (\sphinxstyleliteralemphasis{\sphinxupquote{int}}\sphinxstyleliteralemphasis{\sphinxupquote{, }}\sphinxstyleliteralemphasis{\sphinxupquote{optional}}) \textendash{} Index of output port. Defaults to 2.

\item {} 
\sphinxAtStartPar
\sphinxstyleliteralstrong{\sphinxupquote{ZL}} (\sphinxstyleliteralemphasis{\sphinxupquote{ndarray}}\sphinxstyleliteralemphasis{\sphinxupquote{ or }}\sphinxstyleliteralemphasis{\sphinxupquote{float}}\sphinxstyleliteralemphasis{\sphinxupquote{, }}\sphinxstyleliteralemphasis{\sphinxupquote{optional}}) \textendash{} Load impedance. Defaults to current port impedance at port2.

\item {} 
\sphinxAtStartPar
\sphinxstyleliteralstrong{\sphinxupquote{dB}} (\sphinxstyleliteralemphasis{\sphinxupquote{bool}}\sphinxstyleliteralemphasis{\sphinxupquote{, }}\sphinxstyleliteralemphasis{\sphinxupquote{optional}}) \textendash{} Enable dB output. Defaults to True.

\end{itemize}

\sphinxlineitem{Returns}
\sphinxAtStartPar
Array of Gop values for all frequencies

\sphinxlineitem{Return type}
\sphinxAtStartPar
numpy.ndarray

\end{description}\end{quote}

\end{fulllineitems}

\index{gop2() (touchstone.spfile method)@\spxentry{gop2()}\spxextra{touchstone.spfile method}}

\begin{fulllineitems}
\phantomsection\label{\detokenize{touchstone:touchstone.spfile.gop2}}
\pysigstartsignatures
\pysiglinewithargsret{\sphinxbfcode{\sphinxupquote{gop2}}}{\emph{\DUrole{n}{port1}\DUrole{o}{=}\DUrole{default_value}{1}}, \emph{\DUrole{n}{port2}\DUrole{o}{=}\DUrole{default_value}{2}}, \emph{\DUrole{n}{ZL}\DUrole{o}{=}\DUrole{default_value}{50.0}}, \emph{\DUrole{n}{dB}\DUrole{o}{=}\DUrole{default_value}{True}}}{}
\pysigstopsignatures
\sphinxAtStartPar
Operating power gain from port1 to port2 with load impedance of ZL. If dB=True, output is in dB, otherwise it is a power ratio.
\begin{quote}
\begin{equation*}
\begin{split}G_{op}=\frac{P_{toLoad}}{P_{toNetwork}}\end{split}
\end{equation*}\end{quote}
\begin{quote}\begin{description}
\sphinxlineitem{Parameters}\begin{itemize}
\item {} 
\sphinxAtStartPar
\sphinxstyleliteralstrong{\sphinxupquote{port1}} (\sphinxstyleliteralemphasis{\sphinxupquote{int}}\sphinxstyleliteralemphasis{\sphinxupquote{, }}\sphinxstyleliteralemphasis{\sphinxupquote{optional}}) \textendash{} Index of input port. Defaults to 1.

\item {} 
\sphinxAtStartPar
\sphinxstyleliteralstrong{\sphinxupquote{port2}} (\sphinxstyleliteralemphasis{\sphinxupquote{int}}\sphinxstyleliteralemphasis{\sphinxupquote{, }}\sphinxstyleliteralemphasis{\sphinxupquote{optional}}) \textendash{} Index of output port. Defaults to 2.

\item {} 
\sphinxAtStartPar
\sphinxstyleliteralstrong{\sphinxupquote{ZL}} (\sphinxstyleliteralemphasis{\sphinxupquote{ndarray}}\sphinxstyleliteralemphasis{\sphinxupquote{ or }}\sphinxstyleliteralemphasis{\sphinxupquote{float}}\sphinxstyleliteralemphasis{\sphinxupquote{, }}\sphinxstyleliteralemphasis{\sphinxupquote{optional}}) \textendash{} Load impedance. Defaults to current port impedance at port2.

\item {} 
\sphinxAtStartPar
\sphinxstyleliteralstrong{\sphinxupquote{dB}} (\sphinxstyleliteralemphasis{\sphinxupquote{bool}}\sphinxstyleliteralemphasis{\sphinxupquote{, }}\sphinxstyleliteralemphasis{\sphinxupquote{optional}}) \textendash{} Enable dB output. Defaults to True.

\end{itemize}

\sphinxlineitem{Returns}
\sphinxAtStartPar
Array of Gop values for all frequencies

\sphinxlineitem{Return type}
\sphinxAtStartPar
numpy.ndarray

\end{description}\end{quote}

\end{fulllineitems}

\index{gt() (touchstone.spfile method)@\spxentry{gt()}\spxextra{touchstone.spfile method}}

\begin{fulllineitems}
\phantomsection\label{\detokenize{touchstone:touchstone.spfile.gt}}
\pysigstartsignatures
\pysiglinewithargsret{\sphinxbfcode{\sphinxupquote{gt}}}{\emph{\DUrole{n}{port1}\DUrole{o}{=}\DUrole{default_value}{1}}, \emph{\DUrole{n}{port2}\DUrole{o}{=}\DUrole{default_value}{2}}, \emph{\DUrole{n}{ZS}\DUrole{o}{=}\DUrole{default_value}{{[}{]}}}, \emph{\DUrole{n}{ZL}\DUrole{o}{=}\DUrole{default_value}{{[}{]}}}, \emph{\DUrole{n}{dB}\DUrole{o}{=}\DUrole{default_value}{True}}}{}
\pysigstopsignatures
\sphinxAtStartPar
This method calculates transducer gain (GT) from port1 to port2. Source and load impedances can be specified independently. If any one of them is not specified, current reference impedance is used for that port. Other ports are terminated by reference impedances. This calculation can also be done using impedance renormalization.
\begin{quote}
\begin{equation*}
\begin{split}G_{av}=\frac{P_{load}}{P_{av,fromSource}}\end{split}
\end{equation*}\end{quote}
\begin{quote}\begin{description}
\sphinxlineitem{Parameters}\begin{itemize}
\item {} 
\sphinxAtStartPar
\sphinxstyleliteralstrong{\sphinxupquote{port1}} (\sphinxstyleliteralemphasis{\sphinxupquote{int}}\sphinxstyleliteralemphasis{\sphinxupquote{, }}\sphinxstyleliteralemphasis{\sphinxupquote{optional}}) \textendash{} Index of source port. Defaults to 1.

\item {} 
\sphinxAtStartPar
\sphinxstyleliteralstrong{\sphinxupquote{port2}} (\sphinxstyleliteralemphasis{\sphinxupquote{int}}\sphinxstyleliteralemphasis{\sphinxupquote{, }}\sphinxstyleliteralemphasis{\sphinxupquote{optional}}) \textendash{} Index of load port. Defaults to 2.

\item {} 
\sphinxAtStartPar
\sphinxstyleliteralstrong{\sphinxupquote{dB}} (\sphinxstyleliteralemphasis{\sphinxupquote{bool}}\sphinxstyleliteralemphasis{\sphinxupquote{, }}\sphinxstyleliteralemphasis{\sphinxupquote{optional}}) \textendash{} Enable dB output. Defaults to True.

\item {} 
\sphinxAtStartPar
\sphinxstyleliteralstrong{\sphinxupquote{ZS}} (\sphinxstyleliteralemphasis{\sphinxupquote{float}}\sphinxstyleliteralemphasis{\sphinxupquote{, }}\sphinxstyleliteralemphasis{\sphinxupquote{optional}}) \textendash{} Source impedance. Defaults to 50.0.

\item {} 
\sphinxAtStartPar
\sphinxstyleliteralstrong{\sphinxupquote{ZL}} (\sphinxstyleliteralemphasis{\sphinxupquote{float}}\sphinxstyleliteralemphasis{\sphinxupquote{, }}\sphinxstyleliteralemphasis{\sphinxupquote{optional}}) \textendash{} Load impedance. Defaults to 50.0.

\end{itemize}

\sphinxlineitem{Returns}
\sphinxAtStartPar
Array of GT values for all frequencies

\sphinxlineitem{Return type}
\sphinxAtStartPar
numpy.ndarray

\end{description}\end{quote}

\end{fulllineitems}

\index{input\_impedance() (touchstone.spfile method)@\spxentry{input\_impedance()}\spxextra{touchstone.spfile method}}

\begin{fulllineitems}
\phantomsection\label{\detokenize{touchstone:touchstone.spfile.input_impedance}}
\pysigstartsignatures
\pysiglinewithargsret{\sphinxbfcode{\sphinxupquote{input\_impedance}}}{\emph{\DUrole{n}{k}}}{}
\pysigstopsignatures
\sphinxAtStartPar
Input impedance at port k. All ports are terminated with reference impedances.
\begin{quote}\begin{description}
\sphinxlineitem{Parameters}
\sphinxAtStartPar
\sphinxstyleliteralstrong{\sphinxupquote{port}} (\sphinxstyleliteralemphasis{\sphinxupquote{int}}) \textendash{} Port number for input impedance.

\sphinxlineitem{Returns}
\sphinxAtStartPar
Array of impedance values for all frequencies

\sphinxlineitem{Return type}
\sphinxAtStartPar
numpy.ndarray

\end{description}\end{quote}

\end{fulllineitems}

\index{interpolate() (touchstone.spfile method)@\spxentry{interpolate()}\spxextra{touchstone.spfile method}}

\begin{fulllineitems}
\phantomsection\label{\detokenize{touchstone:touchstone.spfile.interpolate}}
\pysigstartsignatures
\pysiglinewithargsret{\sphinxbfcode{\sphinxupquote{interpolate}}}{\emph{\DUrole{n}{number\_of\_points}\DUrole{o}{=}\DUrole{default_value}{5}}, \emph{\DUrole{n}{inplace}\DUrole{o}{=}\DUrole{default_value}{\sphinxhyphen{} 1}}}{}
\pysigstopsignatures
\sphinxAtStartPar
This method increases the number of frequencies through interpolation.
\begin{quote}\begin{description}
\sphinxlineitem{Parameters}\begin{itemize}
\item {} 
\sphinxAtStartPar
\sphinxstyleliteralstrong{\sphinxupquote{number\_of\_points}} (\sphinxstyleliteralemphasis{\sphinxupquote{int}}\sphinxstyleliteralemphasis{\sphinxupquote{, }}\sphinxstyleliteralemphasis{\sphinxupquote{optional}}) \textendash{} Number of points used for interpolation. Defaults to 5.

\item {} 
\sphinxAtStartPar
\sphinxstyleliteralstrong{\sphinxupquote{inplace}} (\sphinxstyleliteralemphasis{\sphinxupquote{int}}\sphinxstyleliteralemphasis{\sphinxupquote{, }}\sphinxstyleliteralemphasis{\sphinxupquote{optional}}) \textendash{} object editing mode. Defaults to \sphinxhyphen{}1.

\end{itemize}

\sphinxlineitem{Returns}
\sphinxAtStartPar
Network object with smooth data

\sphinxlineitem{Return type}
\sphinxAtStartPar
{\hyperref[\detokenize{touchstone:touchstone.spfile}]{\sphinxcrossref{spfile}}}

\end{description}\end{quote}

\end{fulllineitems}

\index{interpolate\_data() (touchstone.spfile method)@\spxentry{interpolate\_data()}\spxextra{touchstone.spfile method}}

\begin{fulllineitems}
\phantomsection\label{\detokenize{touchstone:touchstone.spfile.interpolate_data}}
\pysigstartsignatures
\pysiglinewithargsret{\sphinxbfcode{\sphinxupquote{interpolate\_data}}}{\emph{\DUrole{n}{datain}}, \emph{\DUrole{n}{freqs}}}{}
\pysigstopsignatures
\sphinxAtStartPar
Calculate new data corresponding to new frequency points \sphinxstyleemphasis{freqs} by interpolation from original data corresponding to current frequency points of the network.
\begin{quote}\begin{description}
\sphinxlineitem{Parameters}\begin{itemize}
\item {} 
\sphinxAtStartPar
\sphinxstyleliteralstrong{\sphinxupquote{data}} (\sphinxstyleliteralemphasis{\sphinxupquote{numpy.ndarray}}\sphinxstyleliteralemphasis{\sphinxupquote{ or }}\sphinxstyleliteralemphasis{\sphinxupquote{list}}) \textendash{} Original data specified at current frequency points of the network.

\item {} 
\sphinxAtStartPar
\sphinxstyleliteralstrong{\sphinxupquote{freqs}} (\sphinxstyleliteralemphasis{\sphinxupquote{numpy.ndarray}}\sphinxstyleliteralemphasis{\sphinxupquote{ or }}\sphinxstyleliteralemphasis{\sphinxupquote{list}}) \textendash{} New frequency list.

\end{itemize}

\sphinxlineitem{Returns}
\sphinxAtStartPar
New data corresponding to \sphinxstyleemphasis{freqs}

\sphinxlineitem{Return type}
\sphinxAtStartPar
numpy.ndarray

\end{description}\end{quote}

\end{fulllineitems}

\index{inverse\_2port() (touchstone.spfile method)@\spxentry{inverse\_2port()}\spxextra{touchstone.spfile method}}

\begin{fulllineitems}
\phantomsection\label{\detokenize{touchstone:touchstone.spfile.inverse_2port}}
\pysigstartsignatures
\pysiglinewithargsret{\sphinxbfcode{\sphinxupquote{inverse\_2port}}}{\emph{\DUrole{n}{inplace}\DUrole{o}{=}\DUrole{default_value}{\sphinxhyphen{} 1}}}{}
\pysigstopsignatures
\sphinxAtStartPar
Take inverse of 2\sphinxhyphen{}port data for de\sphinxhyphen{}embedding purposes. The reference impedance of the network is not changed.
\begin{quote}\begin{description}
\sphinxlineitem{Parameters}
\sphinxAtStartPar
\sphinxstyleliteralstrong{\sphinxupquote{inplace}} (\sphinxstyleliteralemphasis{\sphinxupquote{int}}\sphinxstyleliteralemphasis{\sphinxupquote{, }}\sphinxstyleliteralemphasis{\sphinxupquote{optional}}) \textendash{} Object editing mode. Defaults to \sphinxhyphen{}1.

\sphinxlineitem{Returns}
\sphinxAtStartPar
Inverted 2\sphinxhyphen{}port spfile

\sphinxlineitem{Return type}
\sphinxAtStartPar
{\hyperref[\detokenize{touchstone:touchstone.spfile}]{\sphinxcrossref{spfile}}}

\end{description}\end{quote}

\end{fulllineitems}

\index{load\_impedance() (touchstone.spfile method)@\spxentry{load\_impedance()}\spxextra{touchstone.spfile method}}

\begin{fulllineitems}
\phantomsection\label{\detokenize{touchstone:touchstone.spfile.load_impedance}}
\pysigstartsignatures
\pysiglinewithargsret{\sphinxbfcode{\sphinxupquote{load\_impedance}}}{\emph{\DUrole{n}{Gamma\_in}}, \emph{\DUrole{n}{port1}\DUrole{o}{=}\DUrole{default_value}{1}}, \emph{\DUrole{n}{port2}\DUrole{o}{=}\DUrole{default_value}{2}}}{}
\pysigstopsignatures
\sphinxAtStartPar
Calculates termination impedance at port2 that gives Gamma\_in reflection coefficient at port1.
\begin{quote}\begin{description}
\sphinxlineitem{Parameters}\begin{itemize}
\item {} 
\sphinxAtStartPar
\sphinxstyleliteralstrong{\sphinxupquote{Gamma\_in}} (\sphinxstyleliteralemphasis{\sphinxupquote{float}}\sphinxstyleliteralemphasis{\sphinxupquote{,}}\sphinxstyleliteralemphasis{\sphinxupquote{ndarray}}) \textendash{} Required reflection coefficient.

\item {} 
\sphinxAtStartPar
\sphinxstyleliteralstrong{\sphinxupquote{port1}} (\sphinxstyleliteralemphasis{\sphinxupquote{int}}) \textendash{} Source port.

\item {} 
\sphinxAtStartPar
\sphinxstyleliteralstrong{\sphinxupquote{port2}} (\sphinxstyleliteralemphasis{\sphinxupquote{int}}) \textendash{} Load port.

\end{itemize}

\sphinxlineitem{Returns}
\sphinxAtStartPar
Array of reflection coeeficient of termination at port2

\sphinxlineitem{Return type}
\sphinxAtStartPar
numpy.ndarray

\end{description}\end{quote}

\end{fulllineitems}

\index{make\_symmetric() (touchstone.spfile method)@\spxentry{make\_symmetric()}\spxextra{touchstone.spfile method}}

\begin{fulllineitems}
\phantomsection\label{\detokenize{touchstone:touchstone.spfile.make_symmetric}}
\pysigstartsignatures
\pysiglinewithargsret{\sphinxbfcode{\sphinxupquote{make\_symmetric}}}{\emph{\DUrole{n}{kind}\DUrole{o}{=}\DUrole{default_value}{1}}, \emph{\DUrole{n}{inplace}\DUrole{o}{=}\DUrole{default_value}{\sphinxhyphen{} 1}}}{}
\pysigstopsignatures
\sphinxAtStartPar
Make SPFILE symmetric by taking the average of S11 and S22. S12=S21 assumed.
\begin{quote}\begin{description}
\sphinxlineitem{Parameters}
\sphinxAtStartPar
\sphinxstyleliteralstrong{\sphinxupquote{inplace}} (\sphinxstyleliteralemphasis{\sphinxupquote{int}}\sphinxstyleliteralemphasis{\sphinxupquote{, }}\sphinxstyleliteralemphasis{\sphinxupquote{optional}}) \textendash{} Object editing mode. Defaults to \sphinxhyphen{}1.

\sphinxlineitem{Returns}
\sphinxAtStartPar
Modified spfile object

\sphinxlineitem{Return type}
\sphinxAtStartPar
{\hyperref[\detokenize{touchstone:touchstone.spfile}]{\sphinxcrossref{spfile}}}

\end{description}\end{quote}

\end{fulllineitems}

\index{microstripline() (touchstone.spfile class method)@\spxentry{microstripline()}\spxextra{touchstone.spfile class method}}

\begin{fulllineitems}
\phantomsection\label{\detokenize{touchstone:touchstone.spfile.microstripline}}
\pysigstartsignatures
\pysiglinewithargsret{\sphinxbfcode{\sphinxupquote{classmethod\DUrole{w}{  }}}\sphinxbfcode{\sphinxupquote{microstripline}}}{\emph{\DUrole{n}{length}}, \emph{\DUrole{n}{w}}, \emph{\DUrole{n}{h}}, \emph{\DUrole{n}{t}}, \emph{\DUrole{n}{er}}, \emph{\DUrole{n}{freqs}\DUrole{o}{=}\DUrole{default_value}{None}}}{}
\pysigstopsignatures
\sphinxAtStartPar
Create an \sphinxcode{\sphinxupquote{spfile}} object corresponding to a microstrip line.
\begin{quote}\begin{description}
\sphinxlineitem{Parameters}\begin{itemize}
\item {} 
\sphinxAtStartPar
\sphinxstyleliteralstrong{\sphinxupquote{length}} (\sphinxstyleliteralemphasis{\sphinxupquote{float}}) \textendash{} Length of microstrip line.

\item {} 
\sphinxAtStartPar
\sphinxstyleliteralstrong{\sphinxupquote{w}} (\sphinxstyleliteralemphasis{\sphinxupquote{float}}) \textendash{} Width of microstrip line.

\item {} 
\sphinxAtStartPar
\sphinxstyleliteralstrong{\sphinxupquote{h}} (\sphinxstyleliteralemphasis{\sphinxupquote{float}}) \textendash{} Thickness of substrate.

\item {} 
\sphinxAtStartPar
\sphinxstyleliteralstrong{\sphinxupquote{t}} (\sphinxstyleliteralemphasis{\sphinxupquote{float}}) \textendash{} Thickness of metal.

\item {} 
\sphinxAtStartPar
\sphinxstyleliteralstrong{\sphinxupquote{er}} (\sphinxstyleliteralemphasis{\sphinxupquote{float}}) \textendash{} Relative permittivity of microstrip substrate.

\item {} 
\sphinxAtStartPar
\sphinxstyleliteralstrong{\sphinxupquote{freqs}} (\sphinxstyleliteralemphasis{\sphinxupquote{float}}\sphinxstyleliteralemphasis{\sphinxupquote{, }}\sphinxstyleliteralemphasis{\sphinxupquote{optional}}) \textendash{} Frequency list of object. Defaults to None. If None, frequencies should be set later.

\end{itemize}

\sphinxlineitem{Returns}
\sphinxAtStartPar
An spfile object.

\sphinxlineitem{Return type}
\sphinxAtStartPar
{\hyperref[\detokenize{touchstone:touchstone.spfile}]{\sphinxcrossref{spfile}}}

\end{description}\end{quote}

\end{fulllineitems}

\index{microstripstep() (touchstone.spfile class method)@\spxentry{microstripstep()}\spxextra{touchstone.spfile class method}}

\begin{fulllineitems}
\phantomsection\label{\detokenize{touchstone:touchstone.spfile.microstripstep}}
\pysigstartsignatures
\pysiglinewithargsret{\sphinxbfcode{\sphinxupquote{classmethod\DUrole{w}{  }}}\sphinxbfcode{\sphinxupquote{microstripstep}}}{\emph{\DUrole{n}{w1}}, \emph{\DUrole{n}{w2}}, \emph{\DUrole{n}{eps\_r}}, \emph{\DUrole{n}{h}}, \emph{\DUrole{n}{t}}, \emph{\DUrole{n}{freqs}\DUrole{o}{=}\DUrole{default_value}{None}}}{}
\pysigstopsignatures
\sphinxAtStartPar
Create an \sphinxcode{\sphinxupquote{spfile}} object corresponding to a microstrip step.
\begin{quote}\begin{description}
\sphinxlineitem{Parameters}\begin{itemize}
\item {} 
\sphinxAtStartPar
\sphinxstyleliteralstrong{\sphinxupquote{w1}} (\sphinxstyleliteralemphasis{\sphinxupquote{float}}) \textendash{} Width of microstrip line at port\sphinxhyphen{}1.

\item {} 
\sphinxAtStartPar
\sphinxstyleliteralstrong{\sphinxupquote{w2}} (\sphinxstyleliteralemphasis{\sphinxupquote{float}}) \textendash{} Width of microstrip line at port\sphinxhyphen{}2.

\item {} 
\sphinxAtStartPar
\sphinxstyleliteralstrong{\sphinxupquote{t}} (\sphinxstyleliteralemphasis{\sphinxupquote{float}}) \textendash{} Thickness of metal.

\item {} 
\sphinxAtStartPar
\sphinxstyleliteralstrong{\sphinxupquote{freqs}} (\sphinxstyleliteralemphasis{\sphinxupquote{float}}\sphinxstyleliteralemphasis{\sphinxupquote{, }}\sphinxstyleliteralemphasis{\sphinxupquote{optional}}) \textendash{} Frequency list of object. Defaults to None. If None, frequencies should be set later.

\end{itemize}

\sphinxlineitem{Returns}
\sphinxAtStartPar
An spfile object equivalent to microstrip step.

\sphinxlineitem{Return type}
\sphinxAtStartPar
{\hyperref[\detokenize{touchstone:touchstone.spfile}]{\sphinxcrossref{spfile}}}

\end{description}\end{quote}

\end{fulllineitems}

\index{prepare\_ref\_impedance\_array() (touchstone.spfile method)@\spxentry{prepare\_ref\_impedance\_array()}\spxextra{touchstone.spfile method}}

\begin{fulllineitems}
\phantomsection\label{\detokenize{touchstone:touchstone.spfile.prepare_ref_impedance_array}}
\pysigstartsignatures
\pysiglinewithargsret{\sphinxbfcode{\sphinxupquote{prepare\_ref\_impedance\_array}}}{\emph{\DUrole{n}{imparray}\DUrole{o}{=}\DUrole{default_value}{None}}}{}
\pysigstopsignatures\begin{description}
\sphinxlineitem{Turns reference impedance array which is composed of numbers,arrays, functions or 1\sphinxhyphen{}ports to numerical array which}
\sphinxAtStartPar
is composed of numbers and arrays. It is made sure that :math:{\color{red}\bfseries{}\textasciigrave{}}Re(Z)

\end{description}

\sphinxAtStartPar
eq 0\textasciigrave{}. Mainly for internal use.
\begin{quote}
\begin{description}
\sphinxlineitem{Args:}
\sphinxAtStartPar
imparray (list): List of impedance array

\sphinxlineitem{Returns:}
\sphinxAtStartPar
numpy.ndarray: Calculated impedance array

\end{description}
\end{quote}

\end{fulllineitems}

\index{read\_file() (touchstone.spfile method)@\spxentry{read\_file()}\spxextra{touchstone.spfile method}}

\begin{fulllineitems}
\phantomsection\label{\detokenize{touchstone:touchstone.spfile.read_file}}
\pysigstartsignatures
\pysiglinewithargsret{\sphinxbfcode{\sphinxupquote{read\_file}}}{\emph{\DUrole{n}{file\_name}}, \emph{\DUrole{n}{skiplines}\DUrole{o}{=}\DUrole{default_value}{0}}}{}
\pysigstopsignatures
\sphinxAtStartPar
Network data is read from file. \sphinxstyleemphasis{filename} attribute of object is set with given argument.
\begin{quote}\begin{description}
\sphinxlineitem{Parameters}\begin{itemize}
\item {} 
\sphinxAtStartPar
\sphinxstyleliteralstrong{\sphinxupquote{filename}} (\sphinxstyleliteralemphasis{\sphinxupquote{str}}) \textendash{} Name of the file to be read. Its extension should be either “ts” of in the form of “sNp” or “sN”.

\item {} 
\sphinxAtStartPar
\sphinxstyleliteralstrong{\sphinxupquote{skiplines}} (\sphinxstyleliteralemphasis{\sphinxupquote{int}}\sphinxstyleliteralemphasis{\sphinxupquote{, }}\sphinxstyleliteralemphasis{\sphinxupquote{optional}}) \textendash{} This option is used if some beginning lines will be ignored. Default value is 0.

\end{itemize}

\end{description}\end{quote}

\end{fulllineitems}

\index{read\_file\_again() (touchstone.spfile method)@\spxentry{read\_file\_again()}\spxextra{touchstone.spfile method}}

\begin{fulllineitems}
\phantomsection\label{\detokenize{touchstone:touchstone.spfile.read_file_again}}
\pysigstartsignatures
\pysiglinewithargsret{\sphinxbfcode{\sphinxupquote{read\_file\_again}}}{}{}
\pysigstopsignatures
\sphinxAtStartPar
Network data is read from the file named \sphinxstyleemphasis{filename}.

\end{fulllineitems}

\index{restore\_passivity() (touchstone.spfile method)@\spxentry{restore\_passivity()}\spxextra{touchstone.spfile method}}

\begin{fulllineitems}
\phantomsection\label{\detokenize{touchstone:touchstone.spfile.restore_passivity}}
\pysigstartsignatures
\pysiglinewithargsret{\sphinxbfcode{\sphinxupquote{restore\_passivity}}}{\emph{\DUrole{n}{inplace}\DUrole{o}{=}\DUrole{default_value}{\sphinxhyphen{} 1}}}{}
\pysigstopsignatures
\sphinxAtStartPar
Make the network passive by minimum modification.
Reference: Fast and Optimal Algorithms for Enforcing Reciprocity, Passivity and Causality in S\sphinxhyphen{}parameters.pdf
\begin{quote}\begin{description}
\sphinxlineitem{Parameters}
\sphinxAtStartPar
\sphinxstyleliteralstrong{\sphinxupquote{inplace}} (\sphinxstyleliteralemphasis{\sphinxupquote{int}}\sphinxstyleliteralemphasis{\sphinxupquote{, }}\sphinxstyleliteralemphasis{\sphinxupquote{optional}}) \textendash{} Object editing mode. Defaults to \sphinxhyphen{}1.

\sphinxlineitem{Returns}
\sphinxAtStartPar
Passive network object

\sphinxlineitem{Return type}
\sphinxAtStartPar
{\hyperref[\detokenize{touchstone:touchstone.spfile}]{\sphinxcrossref{spfile}}}

\end{description}\end{quote}

\end{fulllineitems}

\index{restore\_passivity2() (touchstone.spfile method)@\spxentry{restore\_passivity2()}\spxextra{touchstone.spfile method}}

\begin{fulllineitems}
\phantomsection\label{\detokenize{touchstone:touchstone.spfile.restore_passivity2}}
\pysigstartsignatures
\pysiglinewithargsret{\sphinxbfcode{\sphinxupquote{restore\_passivity2}}}{}{}
\pysigstopsignatures
\sphinxAtStartPar
\sphinxstylestrong{Obsolete}
Bu metod S\sphinxhyphen{}parametre datasinin pasif olmadigi frequenciesda
S\sphinxhyphen{}parametre datasina mumkun olan en kucuk degisikligi yaparak
S\sphinxhyphen{}parametre datasini pasif hale getirir.
Referans:
Restoration of Passivity In S\sphinxhyphen{}parameter Data of Microwave Measurements.pdf

\end{fulllineitems}

\index{return\_s2p() (touchstone.spfile method)@\spxentry{return\_s2p()}\spxextra{touchstone.spfile method}}

\begin{fulllineitems}
\phantomsection\label{\detokenize{touchstone:touchstone.spfile.return_s2p}}
\pysigstartsignatures
\pysiglinewithargsret{\sphinxbfcode{\sphinxupquote{return\_s2p}}}{\emph{\DUrole{n}{port1}\DUrole{o}{=}\DUrole{default_value}{1}}, \emph{\DUrole{n}{port2}\DUrole{o}{=}\DUrole{default_value}{2}}}{}
\pysigstopsignatures
\end{fulllineitems}

\index{s2abcd() (touchstone.spfile method)@\spxentry{s2abcd()}\spxextra{touchstone.spfile method}}

\begin{fulllineitems}
\phantomsection\label{\detokenize{touchstone:touchstone.spfile.s2abcd}}
\pysigstartsignatures
\pysiglinewithargsret{\sphinxbfcode{\sphinxupquote{s2abcd}}}{\emph{\DUrole{n}{port1}\DUrole{o}{=}\DUrole{default_value}{1}}, \emph{\DUrole{n}{port2}\DUrole{o}{=}\DUrole{default_value}{2}}}{}
\pysigstopsignatures
\sphinxAtStartPar
S\sphinxhyphen{}Matrix to ABCD matrix conversion between 2 chosen ports. Other ports are terminated with reference impedances
\begin{quote}\begin{description}
\sphinxlineitem{Parameters}\begin{itemize}
\item {} 
\sphinxAtStartPar
\sphinxstyleliteralstrong{\sphinxupquote{port1}} (\sphinxstyleliteralemphasis{\sphinxupquote{int}}\sphinxstyleliteralemphasis{\sphinxupquote{, }}\sphinxstyleliteralemphasis{\sphinxupquote{optional}}) \textendash{} Index of Port\sphinxhyphen{}1. Defaults to 1.

\item {} 
\sphinxAtStartPar
\sphinxstyleliteralstrong{\sphinxupquote{port2}} (\sphinxstyleliteralemphasis{\sphinxupquote{int}}\sphinxstyleliteralemphasis{\sphinxupquote{, }}\sphinxstyleliteralemphasis{\sphinxupquote{optional}}) \textendash{} Index of Port\sphinxhyphen{}2. Defaults to 2.

\end{itemize}

\sphinxlineitem{Returns}
\sphinxAtStartPar
ABCD data. Numpy.matrix of size (ns,4) (ns: number of frequencies). Each row contains (A,B,C,D) numbers in order.

\sphinxlineitem{Return type}
\sphinxAtStartPar
numpy.matrix

\end{description}\end{quote}

\end{fulllineitems}

\index{s2t() (touchstone.spfile method)@\spxentry{s2t()}\spxextra{touchstone.spfile method}}

\begin{fulllineitems}
\phantomsection\label{\detokenize{touchstone:touchstone.spfile.s2t}}
\pysigstartsignatures
\pysiglinewithargsret{\sphinxbfcode{\sphinxupquote{s2t}}}{}{}
\pysigstopsignatures
\sphinxAtStartPar
Calculate transmission matrix for 2\sphinxhyphen{}port networks.
\begin{quote}\begin{description}
\sphinxlineitem{Returns}
\sphinxAtStartPar
SPFILE object

\sphinxlineitem{Return type}
\sphinxAtStartPar
{\hyperref[\detokenize{touchstone:touchstone.spfile}]{\sphinxcrossref{spfile}}}

\end{description}\end{quote}

\end{fulllineitems}

\index{scaledata() (touchstone.spfile method)@\spxentry{scaledata()}\spxextra{touchstone.spfile method}}

\begin{fulllineitems}
\phantomsection\label{\detokenize{touchstone:touchstone.spfile.scaledata}}
\pysigstartsignatures
\pysiglinewithargsret{\sphinxbfcode{\sphinxupquote{scaledata}}}{\emph{\DUrole{n}{scale}\DUrole{o}{=}\DUrole{default_value}{1.0}}, \emph{\DUrole{n}{dataindices}\DUrole{o}{=}\DUrole{default_value}{None}}}{}
\pysigstopsignatures
\end{fulllineitems}

\index{series\_impedance() (touchstone.spfile class method)@\spxentry{series\_impedance()}\spxextra{touchstone.spfile class method}}

\begin{fulllineitems}
\phantomsection\label{\detokenize{touchstone:touchstone.spfile.series_impedance}}
\pysigstartsignatures
\pysiglinewithargsret{\sphinxbfcode{\sphinxupquote{classmethod\DUrole{w}{  }}}\sphinxbfcode{\sphinxupquote{series\_impedance}}}{\emph{\DUrole{n}{Z}}, \emph{\DUrole{n}{freqs}\DUrole{o}{=}\DUrole{default_value}{None}}}{}
\pysigstopsignatures
\sphinxAtStartPar
Create an \sphinxcode{\sphinxupquote{spfile}} object corresponding to a stripline step
\begin{quote}\begin{description}
\sphinxlineitem{Parameters}\begin{itemize}
\item {} 
\sphinxAtStartPar
\sphinxstyleliteralstrong{\sphinxupquote{R}} (\sphinxstyleliteralemphasis{\sphinxupquote{float}}) \textendash{} Shunt resistance.

\item {} 
\sphinxAtStartPar
\sphinxstyleliteralstrong{\sphinxupquote{freqs}} (\sphinxstyleliteralemphasis{\sphinxupquote{float}}\sphinxstyleliteralemphasis{\sphinxupquote{, }}\sphinxstyleliteralemphasis{\sphinxupquote{optional}}) \textendash{} Frequency list of object. Defaults to None. If None, frequencies should be set later.

\end{itemize}

\sphinxlineitem{Returns}
\sphinxAtStartPar
An spfile object.

\sphinxlineitem{Return type}
\sphinxAtStartPar
{\hyperref[\detokenize{touchstone:touchstone.spfile}]{\sphinxcrossref{spfile}}}

\end{description}\end{quote}

\end{fulllineitems}

\index{set\_formulation() (touchstone.spfile method)@\spxentry{set\_formulation()}\spxextra{touchstone.spfile method}}

\begin{fulllineitems}
\phantomsection\label{\detokenize{touchstone:touchstone.spfile.set_formulation}}
\pysigstartsignatures
\pysiglinewithargsret{\sphinxbfcode{\sphinxupquote{set\_formulation}}}{\emph{\DUrole{n}{formulation}}}{}
\pysigstopsignatures
\end{fulllineitems}

\index{set\_frequencies\_wo\_recalc() (touchstone.spfile method)@\spxentry{set\_frequencies\_wo\_recalc()}\spxextra{touchstone.spfile method}}

\begin{fulllineitems}
\phantomsection\label{\detokenize{touchstone:touchstone.spfile.set_frequencies_wo_recalc}}
\pysigstartsignatures
\pysiglinewithargsret{\sphinxbfcode{\sphinxupquote{set\_frequencies\_wo\_recalc}}}{\emph{\DUrole{n}{freqs}}}{}
\pysigstopsignatures
\sphinxAtStartPar
Directly sets the frequencies of this network, but does not re\sphinxhyphen{}calculate s\sphinxhyphen{}parameters.
\begin{quote}\begin{description}
\sphinxlineitem{Parameters}
\sphinxAtStartPar
\sphinxstyleliteralstrong{\sphinxupquote{freqs}} (\sphinxstyleliteralemphasis{\sphinxupquote{list}}\sphinxstyleliteralemphasis{\sphinxupquote{ or }}\sphinxstyleliteralemphasis{\sphinxupquote{ndarray}}) \textendash{} New frequency values

\end{description}\end{quote}

\end{fulllineitems}

\index{set\_frequency\_limits() (touchstone.spfile method)@\spxentry{set\_frequency\_limits()}\spxextra{touchstone.spfile method}}

\begin{fulllineitems}
\phantomsection\label{\detokenize{touchstone:touchstone.spfile.set_frequency_limits}}
\pysigstartsignatures
\pysiglinewithargsret{\sphinxbfcode{\sphinxupquote{set\_frequency\_limits}}}{\emph{\DUrole{n}{flow}}, \emph{\DUrole{n}{fhigh}}, \emph{\DUrole{n}{inplace}\DUrole{o}{=}\DUrole{default_value}{\sphinxhyphen{} 1}}}{}
\pysigstopsignatures
\sphinxAtStartPar
Remove frequency points higher than \sphinxstyleemphasis{fhigh} and lower than \sphinxstyleemphasis{flow}.
\begin{quote}\begin{description}
\sphinxlineitem{Parameters}\begin{itemize}
\item {} 
\sphinxAtStartPar
\sphinxstyleliteralstrong{\sphinxupquote{flow}} (\sphinxstyleliteralemphasis{\sphinxupquote{float}}) \textendash{} Lowest Frequency (Hz)

\item {} 
\sphinxAtStartPar
\sphinxstyleliteralstrong{\sphinxupquote{fhigh}} (\sphinxstyleliteralemphasis{\sphinxupquote{float}}) \textendash{} Highest Frequency (Hz)

\item {} 
\sphinxAtStartPar
\sphinxstyleliteralstrong{\sphinxupquote{inplace}} (\sphinxstyleliteralemphasis{\sphinxupquote{int}}\sphinxstyleliteralemphasis{\sphinxupquote{, }}\sphinxstyleliteralemphasis{\sphinxupquote{optional}}) \textendash{} Object editing mode. Defaults to \sphinxhyphen{}1.

\end{itemize}

\sphinxlineitem{Returns}
\sphinxAtStartPar
spfile object with new frequency points.

\sphinxlineitem{Return type}
\sphinxAtStartPar
{\hyperref[\detokenize{touchstone:touchstone.spfile}]{\sphinxcrossref{spfile}}}

\end{description}\end{quote}

\end{fulllineitems}

\index{set\_frequency\_points() (touchstone.spfile method)@\spxentry{set\_frequency\_points()}\spxextra{touchstone.spfile method}}

\begin{fulllineitems}
\phantomsection\label{\detokenize{touchstone:touchstone.spfile.set_frequency_points}}
\pysigstartsignatures
\pysiglinewithargsret{\sphinxbfcode{\sphinxupquote{set\_frequency\_points}}}{\emph{\DUrole{n}{frequencies}}, \emph{\DUrole{n}{inplace}\DUrole{o}{=}\DUrole{default_value}{\sphinxhyphen{} 1}}}{}
\pysigstopsignatures
\sphinxAtStartPar
Set new frequency points. if S\sphinxhyphen{}Parameter data generator function is
available, use that to calculate new s\sphinxhyphen{}parameter data. If not, use
interpolation/extrapolation. For new frequency points, S\sphinxhyphen{}Parameters
and reference impedances which are in the form of array are re\sphinxhyphen{}calculated.
\begin{quote}\begin{description}
\sphinxlineitem{Parameters}\begin{itemize}
\item {} 
\sphinxAtStartPar
\sphinxstyleliteralstrong{\sphinxupquote{frequencies}} (\sphinxstyleliteralemphasis{\sphinxupquote{list}}) \textendash{} New frequency array

\item {} 
\sphinxAtStartPar
\sphinxstyleliteralstrong{\sphinxupquote{inplace}} (\sphinxstyleliteralemphasis{\sphinxupquote{int}}\sphinxstyleliteralemphasis{\sphinxupquote{, }}\sphinxstyleliteralemphasis{\sphinxupquote{optional}}) \textendash{} Object editing mode. Defaults to \sphinxhyphen{}1.

\end{itemize}

\sphinxlineitem{Returns}
\sphinxAtStartPar
spfile object with new frequency points.

\sphinxlineitem{Return type}
\sphinxAtStartPar
{\hyperref[\detokenize{touchstone:touchstone.spfile}]{\sphinxcrossref{spfile}}}

\end{description}\end{quote}

\end{fulllineitems}

\index{set\_frequency\_points\_array() (touchstone.spfile method)@\spxentry{set\_frequency\_points\_array()}\spxextra{touchstone.spfile method}}

\begin{fulllineitems}
\phantomsection\label{\detokenize{touchstone:touchstone.spfile.set_frequency_points_array}}
\pysigstartsignatures
\pysiglinewithargsret{\sphinxbfcode{\sphinxupquote{set\_frequency\_points\_array}}}{\emph{\DUrole{n}{fstart}}, \emph{\DUrole{n}{fstop}}, \emph{\DUrole{n}{NumberOfPoints}}, \emph{\DUrole{n}{inplace}\DUrole{o}{=}\DUrole{default_value}{\sphinxhyphen{} 1}}}{}
\pysigstopsignatures
\sphinxAtStartPar
Set the frequencies of the object using start\sphinxhyphen{}end frequencies and number of points.
\begin{quote}\begin{description}
\sphinxlineitem{Parameters}\begin{itemize}
\item {} 
\sphinxAtStartPar
\sphinxstyleliteralstrong{\sphinxupquote{fstart}} (\sphinxstyleliteralemphasis{\sphinxupquote{{[}}}\sphinxstyleliteralemphasis{\sphinxupquote{type}}\sphinxstyleliteralemphasis{\sphinxupquote{{]}}}) \textendash{} Start frequency.

\item {} 
\sphinxAtStartPar
\sphinxstyleliteralstrong{\sphinxupquote{fstop}} (\sphinxstyleliteralemphasis{\sphinxupquote{{[}}}\sphinxstyleliteralemphasis{\sphinxupquote{type}}\sphinxstyleliteralemphasis{\sphinxupquote{{]}}}) \textendash{} End frequency.

\item {} 
\sphinxAtStartPar
\sphinxstyleliteralstrong{\sphinxupquote{NumberOfPoints}} (\sphinxstyleliteralemphasis{\sphinxupquote{int}}) \textendash{} Number of frequencies.

\item {} 
\sphinxAtStartPar
\sphinxstyleliteralstrong{\sphinxupquote{inplace}} (\sphinxstyleliteralemphasis{\sphinxupquote{int}}\sphinxstyleliteralemphasis{\sphinxupquote{, }}\sphinxstyleliteralemphasis{\sphinxupquote{optional}}) \textendash{} Object editing mode. Defaults to \sphinxhyphen{}1.

\end{itemize}

\sphinxlineitem{Returns}
\sphinxAtStartPar
spfile object with new frequency points.

\sphinxlineitem{Return type}
\sphinxAtStartPar
{\hyperref[\detokenize{touchstone:touchstone.spfile}]{\sphinxcrossref{spfile}}}

\end{description}\end{quote}

\end{fulllineitems}

\index{set\_inplace() (touchstone.spfile method)@\spxentry{set\_inplace()}\spxextra{touchstone.spfile method}}

\begin{fulllineitems}
\phantomsection\label{\detokenize{touchstone:touchstone.spfile.set_inplace}}
\pysigstartsignatures
\pysiglinewithargsret{\sphinxbfcode{\sphinxupquote{set\_inplace}}}{\emph{\DUrole{n}{inplace}}}{}
\pysigstopsignatures
\end{fulllineitems}

\index{set\_port\_name() (touchstone.spfile method)@\spxentry{set\_port\_name()}\spxextra{touchstone.spfile method}}

\begin{fulllineitems}
\phantomsection\label{\detokenize{touchstone:touchstone.spfile.set_port_name}}
\pysigstartsignatures
\pysiglinewithargsret{\sphinxbfcode{\sphinxupquote{set\_port\_name}}}{\emph{\DUrole{n}{name}}, \emph{\DUrole{n}{i}}}{}
\pysigstopsignatures
\sphinxAtStartPar
Set name of a specific port.
\begin{quote}\begin{description}
\sphinxlineitem{Parameters}\begin{itemize}
\item {} 
\sphinxAtStartPar
\sphinxstyleliteralstrong{\sphinxupquote{name}} (\sphinxstyleliteralemphasis{\sphinxupquote{str}}) \textendash{} New name of the port

\item {} 
\sphinxAtStartPar
\sphinxstyleliteralstrong{\sphinxupquote{i}} (\sphinxstyleliteralemphasis{\sphinxupquote{int}}) \textendash{} Port number

\end{itemize}

\end{description}\end{quote}

\end{fulllineitems}

\index{set\_port\_names() (touchstone.spfile method)@\spxentry{set\_port\_names()}\spxextra{touchstone.spfile method}}

\begin{fulllineitems}
\phantomsection\label{\detokenize{touchstone:touchstone.spfile.set_port_names}}
\pysigstartsignatures
\pysiglinewithargsret{\sphinxbfcode{\sphinxupquote{set\_port\_names}}}{\emph{\DUrole{n}{names}}}{}
\pysigstopsignatures
\sphinxAtStartPar
Set port names with a list.
\begin{quote}\begin{description}
\sphinxlineitem{Parameters}
\sphinxAtStartPar
\sphinxstyleliteralstrong{\sphinxupquote{names}} (\sphinxstyleliteralemphasis{\sphinxupquote{list}}) \textendash{} List of new names of the ports

\end{description}\end{quote}

\end{fulllineitems}

\index{set\_smatrix\_at\_frequency\_point() (touchstone.spfile method)@\spxentry{set\_smatrix\_at\_frequency\_point()}\spxextra{touchstone.spfile method}}

\begin{fulllineitems}
\phantomsection\label{\detokenize{touchstone:touchstone.spfile.set_smatrix_at_frequency_point}}
\pysigstartsignatures
\pysiglinewithargsret{\sphinxbfcode{\sphinxupquote{set\_smatrix\_at\_frequency\_point}}}{\emph{\DUrole{n}{indices}}, \emph{\DUrole{n}{smatrix}}}{}
\pysigstopsignatures
\sphinxAtStartPar
Set S\sphinxhyphen{}Matrix at frequency indices
\begin{quote}\begin{description}
\sphinxlineitem{Parameters}\begin{itemize}
\item {} 
\sphinxAtStartPar
\sphinxstyleliteralstrong{\sphinxupquote{indices}} (\sphinxstyleliteralemphasis{\sphinxupquote{list}}) \textendash{} List of frequency indices

\item {} 
\sphinxAtStartPar
\sphinxstyleliteralstrong{\sphinxupquote{smatrix}} (\sphinxstyleliteralemphasis{\sphinxupquote{numpy.matrix}}) \textendash{} New S\sphinxhyphen{}Matrix value which is to be set at all \sphinxstyleemphasis{indices}

\end{itemize}

\end{description}\end{quote}

\end{fulllineitems}

\index{set\_sparam\_gen\_func() (touchstone.spfile method)@\spxentry{set\_sparam\_gen\_func()}\spxextra{touchstone.spfile method}}

\begin{fulllineitems}
\phantomsection\label{\detokenize{touchstone:touchstone.spfile.set_sparam_gen_func}}
\pysigstartsignatures
\pysiglinewithargsret{\sphinxbfcode{\sphinxupquote{set\_sparam\_gen\_func}}}{\emph{\DUrole{n}{func}\DUrole{o}{=}\DUrole{default_value}{None}}}{}
\pysigstopsignatures
\sphinxAtStartPar
This function is used to set the function that generates s\sphinxhyphen{}parameters from frequency.
\begin{quote}\begin{description}
\sphinxlineitem{Parameters}
\sphinxAtStartPar
\sphinxstyleliteralstrong{\sphinxupquote{func}} (\sphinxstyleliteralemphasis{\sphinxupquote{function}}\sphinxstyleliteralemphasis{\sphinxupquote{, }}\sphinxstyleliteralemphasis{\sphinxupquote{optional}}) \textendash{} function to be set. Defaults to None.

\end{description}\end{quote}

\end{fulllineitems}

\index{set\_sparam\_mod\_func() (touchstone.spfile method)@\spxentry{set\_sparam\_mod\_func()}\spxextra{touchstone.spfile method}}

\begin{fulllineitems}
\phantomsection\label{\detokenize{touchstone:touchstone.spfile.set_sparam_mod_func}}
\pysigstartsignatures
\pysiglinewithargsret{\sphinxbfcode{\sphinxupquote{set\_sparam\_mod\_func}}}{\emph{\DUrole{n}{func}\DUrole{o}{=}\DUrole{default_value}{None}}}{}
\pysigstopsignatures
\sphinxAtStartPar
This function is used to set the function that generates s\sphinxhyphen{}parameters from frequency.
\begin{quote}\begin{description}
\sphinxlineitem{Parameters}
\sphinxAtStartPar
\sphinxstyleliteralstrong{\sphinxupquote{func}} (\sphinxstyleliteralemphasis{\sphinxupquote{function}}\sphinxstyleliteralemphasis{\sphinxupquote{, }}\sphinxstyleliteralemphasis{\sphinxupquote{optional}}) \textendash{} function to be set. Defaults to None.

\end{description}\end{quote}

\end{fulllineitems}

\index{set\_sym\_parameters() (touchstone.spfile method)@\spxentry{set\_sym\_parameters()}\spxextra{touchstone.spfile method}}

\begin{fulllineitems}
\phantomsection\label{\detokenize{touchstone:touchstone.spfile.set_sym_parameters}}
\pysigstartsignatures
\pysiglinewithargsret{\sphinxbfcode{\sphinxupquote{set\_sym\_parameters}}}{\emph{\DUrole{n}{paramdict}}}{}
\pysigstopsignatures
\sphinxAtStartPar
This function is used to set the values of symbolic variables of the network. This is used if the S\sphinxhyphen{}Matrix of the network is defined by an arithmetic expression containing symbolic variables. This property is used in conjunction with \sphinxstyleemphasis{sympy} library for symbolic manipulation. Arithmetic expression for S\sphinxhyphen{}Matrix is defined by \sphinxcode{\sphinxupquote{set\_sym\_smatrix}} function.
\begin{quote}\begin{description}
\sphinxlineitem{Parameters}
\sphinxAtStartPar
\sphinxstyleliteralstrong{\sphinxupquote{paramdict}} (\sphinxstyleliteralemphasis{\sphinxupquote{dict}}) \textendash{} This is a dictionary containing the values of symbolic variables of the network

\end{description}\end{quote}

\end{fulllineitems}

\index{set\_sym\_smatrix() (touchstone.spfile method)@\spxentry{set\_sym\_smatrix()}\spxextra{touchstone.spfile method}}

\begin{fulllineitems}
\phantomsection\label{\detokenize{touchstone:touchstone.spfile.set_sym_smatrix}}
\pysigstartsignatures
\pysiglinewithargsret{\sphinxbfcode{\sphinxupquote{set\_sym\_smatrix}}}{\emph{\DUrole{n}{SM}}}{}
\pysigstopsignatures
\sphinxAtStartPar
This function is used to set arithmetic expression for S\sphinxhyphen{}Matrix, if S\sphinxhyphen{}Matrix is defined using symbolic variables.
\begin{quote}\begin{description}
\sphinxlineitem{Parameters}
\sphinxAtStartPar
\sphinxstyleliteralstrong{\sphinxupquote{SM}} (\sphinxstyleliteralemphasis{\sphinxupquote{sympy.Matrix}}) \textendash{} Symbolic \sphinxcode{\sphinxupquote{sympy.Matrix}} expression for S\sphinxhyphen{}Parameter matrix

\end{description}\end{quote}

\end{fulllineitems}

\index{setdata\_format() (touchstone.spfile method)@\spxentry{setdata\_format()}\spxextra{touchstone.spfile method}}

\begin{fulllineitems}
\phantomsection\label{\detokenize{touchstone:touchstone.spfile.setdata_format}}
\pysigstartsignatures
\pysiglinewithargsret{\sphinxbfcode{\sphinxupquote{setdata\_format}}}{\emph{\DUrole{n}{data\_format}}}{}
\pysigstopsignatures
\end{fulllineitems}

\index{setdatapoint() (touchstone.spfile method)@\spxentry{setdatapoint()}\spxextra{touchstone.spfile method}}

\begin{fulllineitems}
\phantomsection\label{\detokenize{touchstone:touchstone.spfile.setdatapoint}}
\pysigstartsignatures
\pysiglinewithargsret{\sphinxbfcode{\sphinxupquote{setdatapoint}}}{\emph{\DUrole{n}{m}}, \emph{\DUrole{n}{indices}}, \emph{\DUrole{n}{x}}}{}
\pysigstopsignatures
\sphinxAtStartPar
Set the value for some part of S\sphinxhyphen{}Parameter data.
\begin{quote}
\begin{equation*}
\begin{split}S_{i j}[m:m+len(x)]=x\end{split}
\end{equation*}\end{quote}
\begin{quote}\begin{description}
\sphinxlineitem{Parameters}\begin{itemize}
\item {} 
\sphinxAtStartPar
\sphinxstyleliteralstrong{\sphinxupquote{m}} (\sphinxstyleliteralemphasis{\sphinxupquote{int}}) \textendash{} Starting frequency indice

\item {} 
\sphinxAtStartPar
\sphinxstyleliteralstrong{\sphinxupquote{indices}} (\sphinxstyleliteralemphasis{\sphinxupquote{tuple of int}}) \textendash{} Parameters to be set (i,j)

\item {} 
\sphinxAtStartPar
\sphinxstyleliteralstrong{\sphinxupquote{x}} (\sphinxstyleliteralemphasis{\sphinxupquote{number}}\sphinxstyleliteralemphasis{\sphinxupquote{ or }}\sphinxstyleliteralemphasis{\sphinxupquote{list}}) \textendash{} New value. If this is a number, it is converted to a list.

\end{itemize}

\end{description}\end{quote}

\end{fulllineitems}

\index{shunt\_impedance() (touchstone.spfile class method)@\spxentry{shunt\_impedance()}\spxextra{touchstone.spfile class method}}

\begin{fulllineitems}
\phantomsection\label{\detokenize{touchstone:touchstone.spfile.shunt_impedance}}
\pysigstartsignatures
\pysiglinewithargsret{\sphinxbfcode{\sphinxupquote{classmethod\DUrole{w}{  }}}\sphinxbfcode{\sphinxupquote{shunt\_impedance}}}{\emph{\DUrole{n}{Z}}, \emph{\DUrole{n}{freqs}\DUrole{o}{=}\DUrole{default_value}{None}}}{}
\pysigstopsignatures
\sphinxAtStartPar
Create an \sphinxcode{\sphinxupquote{spfile}} object corresponding to a stripline step
\begin{quote}\begin{description}
\sphinxlineitem{Parameters}\begin{itemize}
\item {} 
\sphinxAtStartPar
\sphinxstyleliteralstrong{\sphinxupquote{R}} (\sphinxstyleliteralemphasis{\sphinxupquote{float}}) \textendash{} Shunt resistance.

\item {} 
\sphinxAtStartPar
\sphinxstyleliteralstrong{\sphinxupquote{freqs}} (\sphinxstyleliteralemphasis{\sphinxupquote{float}}\sphinxstyleliteralemphasis{\sphinxupquote{, }}\sphinxstyleliteralemphasis{\sphinxupquote{optional}}) \textendash{} Frequency list of object. Defaults to None. If None, frequencies should be set later.

\end{itemize}

\sphinxlineitem{Returns}
\sphinxAtStartPar
An spfile object.

\sphinxlineitem{Return type}
\sphinxAtStartPar
{\hyperref[\detokenize{touchstone:touchstone.spfile}]{\sphinxcrossref{spfile}}}

\end{description}\end{quote}

\end{fulllineitems}

\index{smoothing() (touchstone.spfile method)@\spxentry{smoothing()}\spxextra{touchstone.spfile method}}

\begin{fulllineitems}
\phantomsection\label{\detokenize{touchstone:touchstone.spfile.smoothing}}
\pysigstartsignatures
\pysiglinewithargsret{\sphinxbfcode{\sphinxupquote{smoothing}}}{\emph{\DUrole{n}{smoothing\_length}\DUrole{o}{=}\DUrole{default_value}{5}}, \emph{\DUrole{n}{inplace}\DUrole{o}{=}\DUrole{default_value}{\sphinxhyphen{} 1}}}{}
\pysigstopsignatures
\sphinxAtStartPar
This method applies moving average smoothing to the s\sphinxhyphen{}parameter data
\begin{quote}\begin{description}
\sphinxlineitem{Parameters}\begin{itemize}
\item {} 
\sphinxAtStartPar
\sphinxstyleliteralstrong{\sphinxupquote{smoothing\_length}} (\sphinxstyleliteralemphasis{\sphinxupquote{int}}\sphinxstyleliteralemphasis{\sphinxupquote{, }}\sphinxstyleliteralemphasis{\sphinxupquote{optional}}) \textendash{} Number of points used for smoothing. Defaults to 5.

\item {} 
\sphinxAtStartPar
\sphinxstyleliteralstrong{\sphinxupquote{inplace}} (\sphinxstyleliteralemphasis{\sphinxupquote{int}}\sphinxstyleliteralemphasis{\sphinxupquote{, }}\sphinxstyleliteralemphasis{\sphinxupquote{optional}}) \textendash{} object editing mode. Defaults to \sphinxhyphen{}1.

\end{itemize}

\sphinxlineitem{Returns}
\sphinxAtStartPar
Network object with smooth data

\sphinxlineitem{Return type}
\sphinxAtStartPar
{\hyperref[\detokenize{touchstone:touchstone.spfile}]{\sphinxcrossref{spfile}}}

\end{description}\end{quote}

\end{fulllineitems}

\index{snp2smp() (touchstone.spfile method)@\spxentry{snp2smp()}\spxextra{touchstone.spfile method}}

\begin{fulllineitems}
\phantomsection\label{\detokenize{touchstone:touchstone.spfile.snp2smp}}
\pysigstartsignatures
\pysiglinewithargsret{\sphinxbfcode{\sphinxupquote{snp2smp}}}{\emph{\DUrole{n}{ports}}, \emph{\DUrole{n}{inplace}\DUrole{o}{=}\DUrole{default_value}{\sphinxhyphen{} 1}}}{}
\pysigstopsignatures
\sphinxAtStartPar
This method changes the port numbering of the network
port j of new network corresponds to ports{[}j{]} in old network.

\sphinxAtStartPar
if the length of “ports” argument is lower than number of ports, remaining ports are terminated with current reference impedances and number of ports are reduced.
\begin{quote}\begin{description}
\sphinxlineitem{Parameters}\begin{itemize}
\item {} 
\sphinxAtStartPar
\sphinxstyleliteralstrong{\sphinxupquote{ports}} (\sphinxstyleliteralemphasis{\sphinxupquote{list}}) \textendash{} New port order

\item {} 
\sphinxAtStartPar
\sphinxstyleliteralstrong{\sphinxupquote{inplace}} (\sphinxstyleliteralemphasis{\sphinxupquote{int}}\sphinxstyleliteralemphasis{\sphinxupquote{, }}\sphinxstyleliteralemphasis{\sphinxupquote{optional}}) \textendash{} Object editing mode. Defaults to \sphinxhyphen{}1.

\end{itemize}

\sphinxlineitem{Returns}
\sphinxAtStartPar
Modified spfile object

\sphinxlineitem{Return type}
\sphinxAtStartPar
{\hyperref[\detokenize{touchstone:touchstone.spfile}]{\sphinxcrossref{spfile}}}

\end{description}\end{quote}

\end{fulllineitems}

\index{sqrt\_network() (touchstone.spfile method)@\spxentry{sqrt\_network()}\spxextra{touchstone.spfile method}}

\begin{fulllineitems}
\phantomsection\label{\detokenize{touchstone:touchstone.spfile.sqrt_network}}
\pysigstartsignatures
\pysiglinewithargsret{\sphinxbfcode{\sphinxupquote{sqrt\_network}}}{}{}
\pysigstopsignatures
\sphinxAtStartPar
Calculate the spfile, when two of which are cascaded, this spfile is obtained.
\begin{quote}\begin{description}
\sphinxlineitem{Returns}
\sphinxAtStartPar
SPFILE object

\sphinxlineitem{Return type}
\sphinxAtStartPar
{\hyperref[\detokenize{touchstone:touchstone.spfile}]{\sphinxcrossref{spfile}}}

\end{description}\end{quote}

\end{fulllineitems}

\index{stability\_factor\_k() (touchstone.spfile method)@\spxentry{stability\_factor\_k()}\spxextra{touchstone.spfile method}}

\begin{fulllineitems}
\phantomsection\label{\detokenize{touchstone:touchstone.spfile.stability_factor_k}}
\pysigstartsignatures
\pysiglinewithargsret{\sphinxbfcode{\sphinxupquote{stability\_factor\_k}}}{\emph{\DUrole{n}{port1}\DUrole{o}{=}\DUrole{default_value}{1}}, \emph{\DUrole{n}{port2}\DUrole{o}{=}\DUrole{default_value}{2}}}{}
\pysigstopsignatures
\sphinxAtStartPar
Calculates \sphinxstyleemphasis{k} stability factor, from port1 to port2. Other ports are terminated with reference impedances.
\begin{quote}\begin{description}
\sphinxlineitem{Parameters}\begin{itemize}
\item {} 
\sphinxAtStartPar
\sphinxstyleliteralstrong{\sphinxupquote{port1}} (\sphinxstyleliteralemphasis{\sphinxupquote{int}}\sphinxstyleliteralemphasis{\sphinxupquote{, }}\sphinxstyleliteralemphasis{\sphinxupquote{optional}}) \textendash{} Index of source port. Defaults to 1.

\item {} 
\sphinxAtStartPar
\sphinxstyleliteralstrong{\sphinxupquote{port2}} (\sphinxstyleliteralemphasis{\sphinxupquote{int}}\sphinxstyleliteralemphasis{\sphinxupquote{, }}\sphinxstyleliteralemphasis{\sphinxupquote{optional}}) \textendash{} Index of load port. Defaults to 2.

\end{itemize}

\sphinxlineitem{Returns}
\sphinxAtStartPar
Array of stability factor for all frequencies

\sphinxlineitem{Return type}
\sphinxAtStartPar
numpy.ndarray

\end{description}\end{quote}

\end{fulllineitems}

\index{stability\_factor\_mu1() (touchstone.spfile method)@\spxentry{stability\_factor\_mu1()}\spxextra{touchstone.spfile method}}

\begin{fulllineitems}
\phantomsection\label{\detokenize{touchstone:touchstone.spfile.stability_factor_mu1}}
\pysigstartsignatures
\pysiglinewithargsret{\sphinxbfcode{\sphinxupquote{stability\_factor\_mu1}}}{\emph{\DUrole{n}{port1}\DUrole{o}{=}\DUrole{default_value}{1}}, \emph{\DUrole{n}{port2}\DUrole{o}{=}\DUrole{default_value}{2}}}{}
\pysigstopsignatures
\sphinxAtStartPar
Calculates \(\mu_1\) stability factor, from port1 to port2. Other ports are terminated with reference impedances.
\begin{quote}\begin{description}
\sphinxlineitem{Parameters}\begin{itemize}
\item {} 
\sphinxAtStartPar
\sphinxstyleliteralstrong{\sphinxupquote{port1}} (\sphinxstyleliteralemphasis{\sphinxupquote{int}}\sphinxstyleliteralemphasis{\sphinxupquote{, }}\sphinxstyleliteralemphasis{\sphinxupquote{optional}}) \textendash{} Index of source port. Defaults to 1.

\item {} 
\sphinxAtStartPar
\sphinxstyleliteralstrong{\sphinxupquote{port2}} (\sphinxstyleliteralemphasis{\sphinxupquote{int}}\sphinxstyleliteralemphasis{\sphinxupquote{, }}\sphinxstyleliteralemphasis{\sphinxupquote{optional}}) \textendash{} Index of load port. Defaults to 2.

\end{itemize}

\sphinxlineitem{Returns}
\sphinxAtStartPar
Array of stability factor for all frequencies

\sphinxlineitem{Return type}
\sphinxAtStartPar
numpy.ndarray

\end{description}\end{quote}

\end{fulllineitems}

\index{stability\_factor\_mu2() (touchstone.spfile method)@\spxentry{stability\_factor\_mu2()}\spxextra{touchstone.spfile method}}

\begin{fulllineitems}
\phantomsection\label{\detokenize{touchstone:touchstone.spfile.stability_factor_mu2}}
\pysigstartsignatures
\pysiglinewithargsret{\sphinxbfcode{\sphinxupquote{stability\_factor\_mu2}}}{\emph{\DUrole{n}{port1}\DUrole{o}{=}\DUrole{default_value}{1}}, \emph{\DUrole{n}{port2}\DUrole{o}{=}\DUrole{default_value}{2}}}{}
\pysigstopsignatures
\sphinxAtStartPar
Calculates \(\mu_2\) stability factor, from port1 to port2. Other ports are terminated with reference impedances.
\begin{quote}\begin{description}
\sphinxlineitem{Parameters}\begin{itemize}
\item {} 
\sphinxAtStartPar
\sphinxstyleliteralstrong{\sphinxupquote{port1}} (\sphinxstyleliteralemphasis{\sphinxupquote{int}}\sphinxstyleliteralemphasis{\sphinxupquote{, }}\sphinxstyleliteralemphasis{\sphinxupquote{optional}}) \textendash{} Index of source port. Defaults to 1.

\item {} 
\sphinxAtStartPar
\sphinxstyleliteralstrong{\sphinxupquote{port2}} (\sphinxstyleliteralemphasis{\sphinxupquote{int}}\sphinxstyleliteralemphasis{\sphinxupquote{, }}\sphinxstyleliteralemphasis{\sphinxupquote{optional}}) \textendash{} Index of load port. Defaults to 2.

\end{itemize}

\sphinxlineitem{Returns}
\sphinxAtStartPar
Array of stability factor for all frequencies

\sphinxlineitem{Return type}
\sphinxAtStartPar
numpy.ndarray

\end{description}\end{quote}

\end{fulllineitems}

\index{stripline() (touchstone.spfile class method)@\spxentry{stripline()}\spxextra{touchstone.spfile class method}}

\begin{fulllineitems}
\phantomsection\label{\detokenize{touchstone:touchstone.spfile.stripline}}
\pysigstartsignatures
\pysiglinewithargsret{\sphinxbfcode{\sphinxupquote{classmethod\DUrole{w}{  }}}\sphinxbfcode{\sphinxupquote{stripline}}}{\emph{\DUrole{n}{length}}, \emph{\DUrole{n}{w}}, \emph{\DUrole{n}{er}}, \emph{\DUrole{n}{h1}}, \emph{\DUrole{n}{h2}}, \emph{\DUrole{n}{t}}, \emph{\DUrole{n}{freqs}\DUrole{o}{=}\DUrole{default_value}{None}}}{}
\pysigstopsignatures
\sphinxAtStartPar
Create an \sphinxcode{\sphinxupquote{spfile}} object corresponding to a stripline transmission line.
\begin{quote}\begin{description}
\sphinxlineitem{Parameters}\begin{itemize}
\item {} 
\sphinxAtStartPar
\sphinxstyleliteralstrong{\sphinxupquote{length}} (\sphinxstyleliteralemphasis{\sphinxupquote{float}}) \textendash{} Length of cpwg line.

\item {} 
\sphinxAtStartPar
\sphinxstyleliteralstrong{\sphinxupquote{w}} (\sphinxstyleliteralemphasis{\sphinxupquote{float}}) \textendash{} Width of stripline.

\item {} 
\sphinxAtStartPar
\sphinxstyleliteralstrong{\sphinxupquote{er}} (\sphinxstyleliteralemphasis{\sphinxupquote{float}}) \textendash{} Relative permittivity of substrate.

\item {} 
\sphinxAtStartPar
\sphinxstyleliteralstrong{\sphinxupquote{h1}} (\sphinxstyleliteralemphasis{\sphinxupquote{float}}) \textendash{} Thickness of substrate from bottom ground to bottom of line.

\item {} 
\sphinxAtStartPar
\sphinxstyleliteralstrong{\sphinxupquote{h2}} (\sphinxstyleliteralemphasis{\sphinxupquote{float}}) \textendash{} Thickness of substrate from top line to top ground.

\item {} 
\sphinxAtStartPar
\sphinxstyleliteralstrong{\sphinxupquote{t}} (\sphinxstyleliteralemphasis{\sphinxupquote{float}}) \textendash{} Thickness of metal.

\item {} 
\sphinxAtStartPar
\sphinxstyleliteralstrong{\sphinxupquote{freqs}} (\sphinxstyleliteralemphasis{\sphinxupquote{float}}\sphinxstyleliteralemphasis{\sphinxupquote{, }}\sphinxstyleliteralemphasis{\sphinxupquote{optional}}) \textendash{} Frequency list of object. Defaults to None. If None, frequencies should be set later.

\end{itemize}

\sphinxlineitem{Returns}
\sphinxAtStartPar
An spfile object.

\sphinxlineitem{Return type}
\sphinxAtStartPar
{\hyperref[\detokenize{touchstone:touchstone.spfile}]{\sphinxcrossref{spfile}}}

\end{description}\end{quote}

\end{fulllineitems}

\index{striplinestep() (touchstone.spfile class method)@\spxentry{striplinestep()}\spxextra{touchstone.spfile class method}}

\begin{fulllineitems}
\phantomsection\label{\detokenize{touchstone:touchstone.spfile.striplinestep}}
\pysigstartsignatures
\pysiglinewithargsret{\sphinxbfcode{\sphinxupquote{classmethod\DUrole{w}{  }}}\sphinxbfcode{\sphinxupquote{striplinestep}}}{\emph{\DUrole{n}{w1}}, \emph{\DUrole{n}{w2}}, \emph{\DUrole{n}{eps\_r}}, \emph{\DUrole{n}{h1}}, \emph{\DUrole{n}{h2}}, \emph{\DUrole{n}{t}}, \emph{\DUrole{n}{freqs}\DUrole{o}{=}\DUrole{default_value}{None}}}{}
\pysigstopsignatures
\sphinxAtStartPar
Create an \sphinxcode{\sphinxupquote{spfile}} object corresponding to a stripline step
\begin{quote}\begin{description}
\sphinxlineitem{Parameters}\begin{itemize}
\item {} 
\sphinxAtStartPar
\sphinxstyleliteralstrong{\sphinxupquote{w1}} (\sphinxstyleliteralemphasis{\sphinxupquote{float}}) \textendash{} Width of stripline line at port\sphinxhyphen{}1.

\item {} 
\sphinxAtStartPar
\sphinxstyleliteralstrong{\sphinxupquote{w2}} (\sphinxstyleliteralemphasis{\sphinxupquote{float}}) \textendash{} Width of stripline line at port\sphinxhyphen{}2.

\item {} 
\sphinxAtStartPar
\sphinxstyleliteralstrong{\sphinxupquote{eps\_r}} (\sphinxstyleliteralemphasis{\sphinxupquote{float}}) \textendash{} Relative permittivity of stripline substrate.

\item {} 
\sphinxAtStartPar
\sphinxstyleliteralstrong{\sphinxupquote{h}} (\sphinxstyleliteralemphasis{\sphinxupquote{float}}) \textendash{} Thickness of stripline substrate.

\item {} 
\sphinxAtStartPar
\sphinxstyleliteralstrong{\sphinxupquote{t}} (\sphinxstyleliteralemphasis{\sphinxupquote{float}}) \textendash{} Thickness of metal.

\item {} 
\sphinxAtStartPar
\sphinxstyleliteralstrong{\sphinxupquote{freqs}} (\sphinxstyleliteralemphasis{\sphinxupquote{float}}\sphinxstyleliteralemphasis{\sphinxupquote{, }}\sphinxstyleliteralemphasis{\sphinxupquote{optional}}) \textendash{} Frequency list of object. Defaults to None. If None, frequencies should be set later.

\end{itemize}

\sphinxlineitem{Returns}
\sphinxAtStartPar
An spfile object.

\sphinxlineitem{Return type}
\sphinxAtStartPar
{\hyperref[\detokenize{touchstone:touchstone.spfile}]{\sphinxcrossref{spfile}}}

\end{description}\end{quote}

\end{fulllineitems}

\index{write2file() (touchstone.spfile method)@\spxentry{write2file()}\spxextra{touchstone.spfile method}}

\begin{fulllineitems}
\phantomsection\label{\detokenize{touchstone:touchstone.spfile.write2file}}
\pysigstartsignatures
\pysiglinewithargsret{\sphinxbfcode{\sphinxupquote{write2file}}}{\emph{\DUrole{n}{filename}\DUrole{o}{=}\DUrole{default_value}{\textquotesingle{}\textquotesingle{}}}, \emph{\DUrole{n}{parameter}\DUrole{o}{=}\DUrole{default_value}{\textquotesingle{}S\textquotesingle{}}}, \emph{\DUrole{n}{freq\_unit}\DUrole{o}{=}\DUrole{default_value}{\textquotesingle{}\textquotesingle{}}}, \emph{\DUrole{n}{data\_format}\DUrole{o}{=}\DUrole{default_value}{\textquotesingle{}\textquotesingle{}}}, \emph{\DUrole{n}{normalized}\DUrole{o}{=}\DUrole{default_value}{True}}}{}
\pysigstopsignatures
\sphinxAtStartPar
This function writes a parameter (S, Y or Z) file. If the filename given does not have the proper filename extension, it is corrected.
\begin{quote}\begin{description}
\sphinxlineitem{Parameters}\begin{itemize}
\item {} 
\sphinxAtStartPar
\sphinxstyleliteralstrong{\sphinxupquote{filename}} (\sphinxstyleliteralemphasis{\sphinxupquote{str}}\sphinxstyleliteralemphasis{\sphinxupquote{, }}\sphinxstyleliteralemphasis{\sphinxupquote{optional}}) \textendash{} Filename to be written. Defaults to “”.

\item {} 
\sphinxAtStartPar
\sphinxstyleliteralstrong{\sphinxupquote{parameter}} (\sphinxstyleliteralemphasis{\sphinxupquote{str}}\sphinxstyleliteralemphasis{\sphinxupquote{, }}\sphinxstyleliteralemphasis{\sphinxupquote{optional}}) \textendash{} Parameter to be written (S, Y or Z). Defaults to “S”.

\item {} 
\sphinxAtStartPar
\sphinxstyleliteralstrong{\sphinxupquote{freq\_unit}} (\sphinxstyleliteralemphasis{\sphinxupquote{str}}\sphinxstyleliteralemphasis{\sphinxupquote{, }}\sphinxstyleliteralemphasis{\sphinxupquote{optional}}) \textendash{} Frequency unit (GHz, MHz, kHz or Hz). Defaults to “Hz”.

\item {} 
\sphinxAtStartPar
\sphinxstyleliteralstrong{\sphinxupquote{data\_format}} (\sphinxstyleliteralemphasis{\sphinxupquote{str}}\sphinxstyleliteralemphasis{\sphinxupquote{, }}\sphinxstyleliteralemphasis{\sphinxupquote{optional}}) \textendash{} Format of file DB, RI or MA. Defaults to “”.

\end{itemize}

\end{description}\end{quote}

\end{fulllineitems}


\end{fulllineitems}

\index{thru\_line\_deembedding() (in module touchstone)@\spxentry{thru\_line\_deembedding()}\spxextra{in module touchstone}}

\begin{fulllineitems}
\phantomsection\label{\detokenize{touchstone:touchstone.thru_line_deembedding}}
\pysigstartsignatures
\pysiglinewithargsret{\sphinxcode{\sphinxupquote{touchstone.}}\sphinxbfcode{\sphinxupquote{thru\_line\_deembedding}}}{\emph{\DUrole{n}{thru\_filename}}, \emph{\DUrole{n}{line\_filename}}, \emph{\DUrole{n}{make\_symmetric}\DUrole{o}{=}\DUrole{default_value}{True}}}{}
\pysigstopsignatures
\sphinxAtStartPar
Extraction of transition s\sphinxhyphen{}parameters from THRU and LINE measurements. Transitions on both sides are assumed to be identical. For output \sphinxstyleemphasis{spfile} objects, port\sphinxhyphen{}1 is launcher side and port\sphinxhyphen{}2 is transmission line side. The length difference between LINE and THRU should be ideally \(\lambda/4\).
The reference impedance for the 2. port of the transition should be the same as the characteristic impedance of the interconnecting line. So the reference impedances of the output \sphinxstyleemphasis{spfile} should be adjusted (without renormalizing s\sphinxhyphen{}parameters) after calling this function. The minimum frequency in the S\sphinxhyphen{}Parameter files should be such that the phase difference between the measurements should be smaller than 2:math:\sphinxtitleref{pi}.
\begin{quote}\begin{description}
\sphinxlineitem{Parameters}\begin{itemize}
\item {} 
\sphinxAtStartPar
\sphinxstyleliteralstrong{\sphinxupquote{thru\_filename}} (\sphinxstyleliteralemphasis{\sphinxupquote{str}}) \textendash{} 2\sphinxhyphen{}Port S\sphinxhyphen{}Parameter filename of THRU measurement

\item {} 
\sphinxAtStartPar
\sphinxstyleliteralstrong{\sphinxupquote{line\_filename}} (\sphinxstyleliteralemphasis{\sphinxupquote{str}}) \textendash{} 2\sphinxhyphen{}Port S\sphinxhyphen{}Parameter filename of LINE measurement

\end{itemize}

\sphinxlineitem{Returns}
\sphinxAtStartPar
2\sphinxhyphen{}Element tuple of (transition spfile, complex phase vector (\(-\gamma l\)) of connecting line of LINE standard (in radian))

\sphinxlineitem{Return type}
\sphinxAtStartPar
tuple({\hyperref[\detokenize{touchstone:touchstone.spfile}]{\sphinxcrossref{spfile}}}, numpy.ndarray)

\end{description}\end{quote}

\end{fulllineitems}

\index{trl\_launcher\_extraction() (in module touchstone)@\spxentry{trl\_launcher\_extraction()}\spxextra{in module touchstone}}

\begin{fulllineitems}
\phantomsection\label{\detokenize{touchstone:touchstone.trl_launcher_extraction}}
\pysigstartsignatures
\pysiglinewithargsret{\sphinxcode{\sphinxupquote{touchstone.}}\sphinxbfcode{\sphinxupquote{trl\_launcher\_extraction}}}{\emph{\DUrole{n}{thru\_file}}, \emph{\DUrole{n}{line\_file}}, \emph{\DUrole{n}{reflect\_file}}, \emph{\DUrole{n}{refstd}\DUrole{o}{=}\DUrole{default_value}{False}}}{}
\pysigstopsignatures
\sphinxAtStartPar
Extraction of launcher s\sphinxhyphen{}parameters by THRU, LINE, REFLECT calibration. For both output \sphinxstyleemphasis{spfile} objects, port\sphinxhyphen{}1 is launcher side and port\sphinxhyphen{}2 is transmission line side.
Reference: TRL algorithm to de\sphinxhyphen{}embed a RF test fixture.pdf (Note that the T\sphinxhyphen{}Matrix definiton in the reference document is different than this library.)
\begin{quote}\begin{description}
\sphinxlineitem{Parameters}\begin{itemize}
\item {} 
\sphinxAtStartPar
\sphinxstyleliteralstrong{\sphinxupquote{thru\_file}} (\sphinxstyleliteralemphasis{\sphinxupquote{str}}) \textendash{} 2\sphinxhyphen{}Port S\sphinxhyphen{}Parameter filename or \sphinxstyleemphasis{spfile} of THRU measurement

\item {} 
\sphinxAtStartPar
\sphinxstyleliteralstrong{\sphinxupquote{line\_file}} (\sphinxstyleliteralemphasis{\sphinxupquote{str}}) \textendash{} 2\sphinxhyphen{}Port S\sphinxhyphen{}Parameter filename or \sphinxstyleemphasis{spfile} of LINE measurement

\item {} 
\sphinxAtStartPar
\sphinxstyleliteralstrong{\sphinxupquote{reflect\_file}} (\sphinxstyleliteralemphasis{\sphinxupquote{str}}) \textendash{} 2\sphinxhyphen{}Port S\sphinxhyphen{}Parameter filename or \sphinxstyleemphasis{spfile} of REFLECT measurement

\item {} 
\sphinxAtStartPar
\sphinxstyleliteralstrong{\sphinxupquote{refstd}} (\sphinxstyleliteralemphasis{\sphinxupquote{boolean}}) \textendash{} True if OPEN is used as REFLECT standard and False (default) if SHORT is used

\end{itemize}

\sphinxlineitem{Returns}
\sphinxAtStartPar
3\sphinxhyphen{}Element tuple of (left side launcher spfile, right side launcher spfile, positive phase vector of connecting line of LINE standard (in radian) )

\sphinxlineitem{Return type}
\sphinxAtStartPar
tuple({\hyperref[\detokenize{touchstone:touchstone.spfile}]{\sphinxcrossref{spfile}}}, {\hyperref[\detokenize{touchstone:touchstone.spfile}]{\sphinxcrossref{spfile}}}, numpy.ndarray)

\end{description}\end{quote}

\end{fulllineitems}

\index{untermination\_method() (in module touchstone)@\spxentry{untermination\_method()}\spxextra{in module touchstone}}

\begin{fulllineitems}
\phantomsection\label{\detokenize{touchstone:touchstone.untermination_method}}
\pysigstartsignatures
\pysiglinewithargsret{\sphinxcode{\sphinxupquote{touchstone.}}\sphinxbfcode{\sphinxupquote{untermination\_method}}}{\emph{\DUrole{n}{g1}}, \emph{\DUrole{n}{g2}}, \emph{\DUrole{n}{g3}}, \emph{\DUrole{n}{gL1}}, \emph{\DUrole{n}{gL2}}, \emph{\DUrole{n}{gL3}}, \emph{\DUrole{n}{returnS2P}\DUrole{o}{=}\DUrole{default_value}{False}}, \emph{\DUrole{n}{freqs}\DUrole{o}{=}\DUrole{default_value}{None}}}{}
\pysigstopsignatures
\sphinxAtStartPar
Determination of \(S_{11}\), \(S_{22}\) and \(S_{21}=S_{12}\) for a 2\sphinxhyphen{}port network network using 3 reflection coefficient values at port\sphinxhyphen{}1 for 3 terminations at port\sphinxhyphen{}2. \(S_{21}\) can only be calculated with a sign ambiguity because it exists only as square in the formulae.

\sphinxAtStartPar
Port\sphinxhyphen{}1: Input port.
Port\sphinxhyphen{}2: Output port where load impedances are switched.
\begin{quote}\begin{description}
\sphinxlineitem{Parameters}\begin{itemize}
\item {} 
\sphinxAtStartPar
\sphinxstyleliteralstrong{\sphinxupquote{g1}} (\sphinxstyleliteralemphasis{\sphinxupquote{float}}\sphinxstyleliteralemphasis{\sphinxupquote{, }}\sphinxstyleliteralemphasis{\sphinxupquote{complex}}\sphinxstyleliteralemphasis{\sphinxupquote{ or }}\sphinxstyleliteralemphasis{\sphinxupquote{ndarray}}) \textendash{} Reflection coefficient at port\sphinxhyphen{}1 when port\sphinxhyphen{}2 is terminated by a load with reflection coefficient gL1

\item {} 
\sphinxAtStartPar
\sphinxstyleliteralstrong{\sphinxupquote{g2}} (\sphinxstyleliteralemphasis{\sphinxupquote{float}}\sphinxstyleliteralemphasis{\sphinxupquote{, }}\sphinxstyleliteralemphasis{\sphinxupquote{complex}}\sphinxstyleliteralemphasis{\sphinxupquote{ or }}\sphinxstyleliteralemphasis{\sphinxupquote{ndarray}}) \textendash{} Reflection coefficient at port\sphinxhyphen{}1 when port\sphinxhyphen{}2 is terminated by a load with reflection coefficient gL2

\item {} 
\sphinxAtStartPar
\sphinxstyleliteralstrong{\sphinxupquote{g3}} (\sphinxstyleliteralemphasis{\sphinxupquote{float}}\sphinxstyleliteralemphasis{\sphinxupquote{, }}\sphinxstyleliteralemphasis{\sphinxupquote{complex}}\sphinxstyleliteralemphasis{\sphinxupquote{ or }}\sphinxstyleliteralemphasis{\sphinxupquote{ndarray}}) \textendash{} Reflection coefficient at port\sphinxhyphen{}1 when port\sphinxhyphen{}2 is terminated by a load with reflection coefficient gL3

\item {} 
\sphinxAtStartPar
\sphinxstyleliteralstrong{\sphinxupquote{gL1}} (\sphinxstyleliteralemphasis{\sphinxupquote{float}}\sphinxstyleliteralemphasis{\sphinxupquote{, }}\sphinxstyleliteralemphasis{\sphinxupquote{complex}}\sphinxstyleliteralemphasis{\sphinxupquote{ or }}\sphinxstyleliteralemphasis{\sphinxupquote{ndarray}}) \textendash{} Reflection coefficient of load at port\sphinxhyphen{}2 that gives g1 reflection coefficient at port\sphinxhyphen{}1

\item {} 
\sphinxAtStartPar
\sphinxstyleliteralstrong{\sphinxupquote{gL2}} (\sphinxstyleliteralemphasis{\sphinxupquote{float}}\sphinxstyleliteralemphasis{\sphinxupquote{, }}\sphinxstyleliteralemphasis{\sphinxupquote{complex}}\sphinxstyleliteralemphasis{\sphinxupquote{ or }}\sphinxstyleliteralemphasis{\sphinxupquote{ndarray}}) \textendash{} Reflection coefficient of load at port\sphinxhyphen{}2 that gives g2 reflection coefficient at port\sphinxhyphen{}1

\item {} 
\sphinxAtStartPar
\sphinxstyleliteralstrong{\sphinxupquote{gL3}} (\sphinxstyleliteralemphasis{\sphinxupquote{float}}\sphinxstyleliteralemphasis{\sphinxupquote{, }}\sphinxstyleliteralemphasis{\sphinxupquote{complex}}\sphinxstyleliteralemphasis{\sphinxupquote{ or }}\sphinxstyleliteralemphasis{\sphinxupquote{ndarray}}) \textendash{} Reflection coefficient of load at port\sphinxhyphen{}2 that gives g3 reflection coefficient at port\sphinxhyphen{}1

\item {} 
\sphinxAtStartPar
\sphinxstyleliteralstrong{\sphinxupquote{returnS2P}} (\sphinxstyleliteralemphasis{\sphinxupquote{boolean}}) \textendash{} If True, function returns an \sphinxstyleemphasis{spfile} object of the 2\sphinxhyphen{}port network, if False, it returns 3\sphinxhyphen{}tuple of S\sphinxhyphen{}Parameter arrays. Default is False.

\item {} 
\sphinxAtStartPar
\sphinxstyleliteralstrong{\sphinxupquote{freqs}} (\sphinxstyleliteralemphasis{\sphinxupquote{numpy.ndarray}}\sphinxstyleliteralemphasis{\sphinxupquote{, }}\sphinxstyleliteralemphasis{\sphinxupquote{list}}) \textendash{} If returnS2P is True, this input is used as the frequency points of the returned \sphinxstyleemphasis{spfile} object. Default is None.

\end{itemize}

\sphinxlineitem{Returns}
\sphinxAtStartPar
Either 3\sphinxhyphen{}Element tuple of (S11, S22, S21) or \sphinxstyleemphasis{spfile} object, depending on returnS2P input.

\sphinxlineitem{Return type}
\sphinxAtStartPar
tuple

\end{description}\end{quote}

\end{fulllineitems}

\index{untermination\_method\_old() (in module touchstone)@\spxentry{untermination\_method\_old()}\spxextra{in module touchstone}}

\begin{fulllineitems}
\phantomsection\label{\detokenize{touchstone:touchstone.untermination_method_old}}
\pysigstartsignatures
\pysiglinewithargsret{\sphinxcode{\sphinxupquote{touchstone.}}\sphinxbfcode{\sphinxupquote{untermination\_method\_old}}}{\emph{\DUrole{n}{g1}}, \emph{\DUrole{n}{g2}}, \emph{\DUrole{n}{g3}}, \emph{\DUrole{n}{gL1}}, \emph{\DUrole{n}{gL2}}, \emph{\DUrole{n}{gL3}}, \emph{\DUrole{n}{returnS2P}\DUrole{o}{=}\DUrole{default_value}{False}}, \emph{\DUrole{n}{freqs}\DUrole{o}{=}\DUrole{default_value}{None}}}{}
\pysigstopsignatures
\sphinxAtStartPar
Determination of \(S_{11}\), \(S_{22}\) and \(S_{21}=S_{12}\) for a 2\sphinxhyphen{}port network network using 3 reflection coefficient values at port\sphinxhyphen{}1 for 3 terminations at port\sphinxhyphen{}2. \(S_{21}\) can only be calculated with a sign ambiguity because it exists only as square in the formulae.

\sphinxAtStartPar
Port\sphinxhyphen{}1: Input port.
Port\sphinxhyphen{}2: Output port where load impedances are switched.
\begin{quote}\begin{description}
\sphinxlineitem{Parameters}\begin{itemize}
\item {} 
\sphinxAtStartPar
\sphinxstyleliteralstrong{\sphinxupquote{g1}} (\sphinxstyleliteralemphasis{\sphinxupquote{float}}\sphinxstyleliteralemphasis{\sphinxupquote{, }}\sphinxstyleliteralemphasis{\sphinxupquote{complex}}\sphinxstyleliteralemphasis{\sphinxupquote{ or }}\sphinxstyleliteralemphasis{\sphinxupquote{ndarray}}) \textendash{} Reflection coefficient at port\sphinxhyphen{}1 when port\sphinxhyphen{}2 is terminated by a load with reflection coefficient gL1

\item {} 
\sphinxAtStartPar
\sphinxstyleliteralstrong{\sphinxupquote{g2}} (\sphinxstyleliteralemphasis{\sphinxupquote{float}}\sphinxstyleliteralemphasis{\sphinxupquote{, }}\sphinxstyleliteralemphasis{\sphinxupquote{complex}}\sphinxstyleliteralemphasis{\sphinxupquote{ or }}\sphinxstyleliteralemphasis{\sphinxupquote{ndarray}}) \textendash{} Reflection coefficient at port\sphinxhyphen{}1 when port\sphinxhyphen{}2 is terminated by a load with reflection coefficient gL2

\item {} 
\sphinxAtStartPar
\sphinxstyleliteralstrong{\sphinxupquote{g3}} (\sphinxstyleliteralemphasis{\sphinxupquote{float}}\sphinxstyleliteralemphasis{\sphinxupquote{, }}\sphinxstyleliteralemphasis{\sphinxupquote{complex}}\sphinxstyleliteralemphasis{\sphinxupquote{ or }}\sphinxstyleliteralemphasis{\sphinxupquote{ndarray}}) \textendash{} Reflection coefficient at port\sphinxhyphen{}1 when port\sphinxhyphen{}2 is terminated by a load with reflection coefficient gL3

\item {} 
\sphinxAtStartPar
\sphinxstyleliteralstrong{\sphinxupquote{gL1}} (\sphinxstyleliteralemphasis{\sphinxupquote{float}}\sphinxstyleliteralemphasis{\sphinxupquote{, }}\sphinxstyleliteralemphasis{\sphinxupquote{complex}}\sphinxstyleliteralemphasis{\sphinxupquote{ or }}\sphinxstyleliteralemphasis{\sphinxupquote{ndarray}}) \textendash{} Reflection coefficient of load at port\sphinxhyphen{}2 that gives g1 reflection coefficient at port\sphinxhyphen{}1

\item {} 
\sphinxAtStartPar
\sphinxstyleliteralstrong{\sphinxupquote{gL2}} (\sphinxstyleliteralemphasis{\sphinxupquote{float}}\sphinxstyleliteralemphasis{\sphinxupquote{, }}\sphinxstyleliteralemphasis{\sphinxupquote{complex}}\sphinxstyleliteralemphasis{\sphinxupquote{ or }}\sphinxstyleliteralemphasis{\sphinxupquote{ndarray}}) \textendash{} Reflection coefficient of load at port\sphinxhyphen{}2 that gives g2 reflection coefficient at port\sphinxhyphen{}1

\item {} 
\sphinxAtStartPar
\sphinxstyleliteralstrong{\sphinxupquote{gL3}} (\sphinxstyleliteralemphasis{\sphinxupquote{float}}\sphinxstyleliteralemphasis{\sphinxupquote{, }}\sphinxstyleliteralemphasis{\sphinxupquote{complex}}\sphinxstyleliteralemphasis{\sphinxupquote{ or }}\sphinxstyleliteralemphasis{\sphinxupquote{ndarray}}) \textendash{} Reflection coefficient of load at port\sphinxhyphen{}2 that gives g3 reflection coefficient at port\sphinxhyphen{}1

\item {} 
\sphinxAtStartPar
\sphinxstyleliteralstrong{\sphinxupquote{returnS2P}} (\sphinxstyleliteralemphasis{\sphinxupquote{boolean}}) \textendash{} If True, function returns an \sphinxstyleemphasis{spfile} object of the 2\sphinxhyphen{}port network, if False, it returns 3\sphinxhyphen{}tuple of S\sphinxhyphen{}Parameter arrays. Default is False.

\item {} 
\sphinxAtStartPar
\sphinxstyleliteralstrong{\sphinxupquote{freqs}} (\sphinxstyleliteralemphasis{\sphinxupquote{numpy.ndarray}}\sphinxstyleliteralemphasis{\sphinxupquote{, }}\sphinxstyleliteralemphasis{\sphinxupquote{list}}) \textendash{} If returnS2P is True, this input is used as the frequency points of the returned \sphinxstyleemphasis{spfile} object. Default is None.

\end{itemize}

\sphinxlineitem{Returns}
\sphinxAtStartPar
Either 3\sphinxhyphen{}Element tuple of (S11, S22, S21) or \sphinxstyleemphasis{spfile} object, depending on returnS2P input.

\sphinxlineitem{Return type}
\sphinxAtStartPar
tuple

\end{description}\end{quote}

\end{fulllineitems}

\index{write\_impedance\_as\_s1p() (in module touchstone)@\spxentry{write\_impedance\_as\_s1p()}\spxextra{in module touchstone}}

\begin{fulllineitems}
\phantomsection\label{\detokenize{touchstone:touchstone.write_impedance_as_s1p}}
\pysigstartsignatures
\pysiglinewithargsret{\sphinxcode{\sphinxupquote{touchstone.}}\sphinxbfcode{\sphinxupquote{write\_impedance\_as\_s1p}}}{\emph{\DUrole{n}{filename}}, \emph{\DUrole{n}{frequencies}}, \emph{\DUrole{n}{Z}}}{}
\pysigstopsignatures
\end{fulllineitems}

\index{write\_impedance\_as\_table() (in module touchstone)@\spxentry{write\_impedance\_as\_table()}\spxextra{in module touchstone}}

\begin{fulllineitems}
\phantomsection\label{\detokenize{touchstone:touchstone.write_impedance_as_table}}
\pysigstartsignatures
\pysiglinewithargsret{\sphinxcode{\sphinxupquote{touchstone.}}\sphinxbfcode{\sphinxupquote{write\_impedance\_as\_table}}}{\emph{\DUrole{n}{filename}}, \emph{\DUrole{n}{frequencies}}, \emph{\DUrole{n}{Z}}}{}
\pysigstopsignatures
\end{fulllineitems}


\sphinxstepscope


\chapter{components module}
\label{\detokenize{components:module-components}}\label{\detokenize{components:components-module}}\label{\detokenize{components::doc}}\index{module@\spxentry{module}!components@\spxentry{components}}\index{components@\spxentry{components}!module@\spxentry{module}}
\sphinxAtStartPar
Created on Tue Nov 17 11:52:33 2009

\sphinxAtStartPar
@author: Tuncay
\index{AWG2Dia() (in module components)@\spxentry{AWG2Dia()}\spxextra{in module components}}

\begin{fulllineitems}
\phantomsection\label{\detokenize{components:components.AWG2Dia}}
\pysigstartsignatures
\pysiglinewithargsret{\sphinxcode{\sphinxupquote{components.}}\sphinxbfcode{\sphinxupquote{AWG2Dia}}}{\emph{\DUrole{n}{arg}}, \emph{\DUrole{n}{defaultunits}\DUrole{o}{=}\DUrole{default_value}{{[}{]}}}}{}
\pysigstopsignatures
\sphinxAtStartPar
Convert AWG to Diameter.
Reference:  Wikipedia, Current rating is calculated through curve fit from online data.
\begin{quote}\begin{description}
\sphinxlineitem{Parameters}\begin{itemize}
\item {} 
\sphinxAtStartPar
\sphinxstyleliteralstrong{\sphinxupquote{arg}} (\sphinxstyleliteralemphasis{\sphinxupquote{list}}) \textendash{} 
\sphinxAtStartPar
First 1 arguments are inputs.
\begin{enumerate}
\sphinxsetlistlabels{\arabic}{enumi}{enumii}{}{.}%
\item {} 
\sphinxAtStartPar
AWG ;

\item {} 
\sphinxAtStartPar
Diameter ;length

\item {} 
\sphinxAtStartPar
Current rating in still air ; current

\end{enumerate}


\item {} 
\sphinxAtStartPar
\sphinxstyleliteralstrong{\sphinxupquote{defaultunits}} (\sphinxstyleliteralemphasis{\sphinxupquote{list}}\sphinxstyleliteralemphasis{\sphinxupquote{, }}\sphinxstyleliteralemphasis{\sphinxupquote{optional}}) \textendash{} Default units for quantities in \sphinxstyleemphasis{arg} list. Default is {[}{]} which means SI units will be used if no unit is given in \sphinxstyleemphasis{arg}.

\end{itemize}

\sphinxlineitem{Returns}
\sphinxAtStartPar
arg

\sphinxlineitem{Return type}
\sphinxAtStartPar
list

\end{description}\end{quote}

\end{fulllineitems}

\index{Absorptive\_Filter\_Equalizer() (in module components)@\spxentry{Absorptive\_Filter\_Equalizer()}\spxextra{in module components}}

\begin{fulllineitems}
\phantomsection\label{\detokenize{components:components.Absorptive_Filter_Equalizer}}
\pysigstartsignatures
\pysiglinewithargsret{\sphinxcode{\sphinxupquote{components.}}\sphinxbfcode{\sphinxupquote{Absorptive\_Filter\_Equalizer}}}{\emph{\DUrole{n}{arg}}, \emph{\DUrole{n}{defaultunits}\DUrole{o}{=}\DUrole{default_value}{{[}{]}}}}{}
\pysigstopsignatures
\sphinxAtStartPar
Equalizer using an absorptive filter composed of two coupled lines.
\begin{quote}\begin{description}
\sphinxlineitem{Parameters}\begin{itemize}
\item {} 
\sphinxAtStartPar
\sphinxstyleliteralstrong{\sphinxupquote{arg}} (\sphinxstyleliteralemphasis{\sphinxupquote{list}}) \textendash{} 
\sphinxAtStartPar
First 4 arguments are inputs.
\begin{enumerate}
\sphinxsetlistlabels{\arabic}{enumi}{enumii}{}{.}%
\item {} 
\sphinxAtStartPar
Reference Impedance ; impedance

\item {} 
\sphinxAtStartPar
Coupling (dB) ;

\item {} 
\sphinxAtStartPar
Center Frequency ; frequency

\item {} 
\sphinxAtStartPar
Test Frequency ; frequency

\item {} 
\sphinxAtStartPar
S21 (dB) ;

\item {} 
\sphinxAtStartPar
Zeven ;  impedance

\item {} 
\sphinxAtStartPar
Zodd ;  impedance

\end{enumerate}

\sphinxAtStartPar
Reference:


\item {} 
\sphinxAtStartPar
\sphinxstyleliteralstrong{\sphinxupquote{defaultunits}} (\sphinxstyleliteralemphasis{\sphinxupquote{list}}\sphinxstyleliteralemphasis{\sphinxupquote{, }}\sphinxstyleliteralemphasis{\sphinxupquote{optional}}) \textendash{} Default units for quantities in \sphinxstyleemphasis{arg} list. Default is {[}{]} which means SI units will be used if no unit is given in \sphinxstyleemphasis{arg}.

\end{itemize}

\sphinxlineitem{Returns}
\sphinxAtStartPar
arg

\sphinxlineitem{Return type}
\sphinxAtStartPar
list

\end{description}\end{quote}

\end{fulllineitems}

\index{Binomial\_QWave\_Impedance\_Transformer() (in module components)@\spxentry{Binomial\_QWave\_Impedance\_Transformer()}\spxextra{in module components}}

\begin{fulllineitems}
\phantomsection\label{\detokenize{components:components.Binomial_QWave_Impedance_Transformer}}
\pysigstartsignatures
\pysiglinewithargsret{\sphinxcode{\sphinxupquote{components.}}\sphinxbfcode{\sphinxupquote{Binomial\_QWave\_Impedance\_Transformer}}}{\emph{\DUrole{n}{arg}}, \emph{\DUrole{n}{defaultunits}\DUrole{o}{=}\DUrole{default_value}{{[}{]}}}}{}
\pysigstopsignatures
\sphinxAtStartPar
Binomial Quarter Wave Impedance Transformer.
\begin{quote}\begin{description}
\sphinxlineitem{Parameters}\begin{itemize}
\item {} 
\sphinxAtStartPar
\sphinxstyleliteralstrong{\sphinxupquote{arg}} (\sphinxstyleliteralemphasis{\sphinxupquote{list}}) \textendash{} 
\sphinxAtStartPar
First 5 arguments are inputs.
\begin{enumerate}
\sphinxsetlistlabels{\arabic}{enumi}{enumii}{}{.}%
\item {} 
\sphinxAtStartPar
Source Impedance;impedance

\item {} 
\sphinxAtStartPar
Load Impedance;impedance

\item {} 
\sphinxAtStartPar
Number Of Matching Sections;

\item {} 
\sphinxAtStartPar
Max(dB(S\textless{}sub\textgreater{}11\textless{}/sub\textgreater{})) In Frequency Band ;

\item {} 
\sphinxAtStartPar
Center Frequency ; frequency

\item {} 
\sphinxAtStartPar
Impedances ; impedance

\end{enumerate}

\sphinxAtStartPar
7.  Bandwidth ; frequency
Reference:  Impedance Matching and Transformation.pdf


\item {} 
\sphinxAtStartPar
\sphinxstyleliteralstrong{\sphinxupquote{defaultunits}} (\sphinxstyleliteralemphasis{\sphinxupquote{list}}\sphinxstyleliteralemphasis{\sphinxupquote{, }}\sphinxstyleliteralemphasis{\sphinxupquote{optional}}) \textendash{} Default units for quantities in \sphinxstyleemphasis{arg} list. Default is {[}{]} which means SI units will be used if no unit is given in \sphinxstyleemphasis{arg}.

\end{itemize}

\sphinxlineitem{Returns}
\sphinxAtStartPar
arg

\sphinxlineitem{Return type}
\sphinxAtStartPar
list

\end{description}\end{quote}

\end{fulllineitems}

\index{Bridged\_Tee\_Attenuator\_Analysis() (in module components)@\spxentry{Bridged\_Tee\_Attenuator\_Analysis()}\spxextra{in module components}}

\begin{fulllineitems}
\phantomsection\label{\detokenize{components:components.Bridged_Tee_Attenuator_Analysis}}
\pysigstartsignatures
\pysiglinewithargsret{\sphinxcode{\sphinxupquote{components.}}\sphinxbfcode{\sphinxupquote{Bridged\_Tee\_Attenuator\_Analysis}}}{\emph{\DUrole{n}{arg}}, \emph{\DUrole{n}{defaultunits}\DUrole{o}{=}\DUrole{default_value}{{[}{]}}}}{}
\pysigstopsignatures
\sphinxAtStartPar
Bridged Tee Attenuator Analysis.
\begin{quote}\begin{description}
\sphinxlineitem{Parameters}\begin{itemize}
\item {} 
\sphinxAtStartPar
\sphinxstyleliteralstrong{\sphinxupquote{arg}} (\sphinxstyleliteralemphasis{\sphinxupquote{list}}) \textendash{} 
\sphinxAtStartPar
First 3 arguments are inputs.
\begin{enumerate}
\sphinxsetlistlabels{\arabic}{enumi}{enumii}{}{.}%
\item {} 
\sphinxAtStartPar
Reference Impedance (Zo); impedance

\item {} 
\sphinxAtStartPar
Series Impedance (Rs); impedance

\item {} 
\sphinxAtStartPar
Parallel Impedance (Rp); impedance

\item {} 
\sphinxAtStartPar
S(1,1) ;

\end{enumerate}

\sphinxAtStartPar
5. S(2,1) ;
Reference:


\item {} 
\sphinxAtStartPar
\sphinxstyleliteralstrong{\sphinxupquote{defaultunits}} (\sphinxstyleliteralemphasis{\sphinxupquote{list}}\sphinxstyleliteralemphasis{\sphinxupquote{, }}\sphinxstyleliteralemphasis{\sphinxupquote{optional}}) \textendash{} Default units for quantities in \sphinxstyleemphasis{arg} list. Default is {[}{]} which means SI units will be used if no unit is given in \sphinxstyleemphasis{arg}.

\end{itemize}

\sphinxlineitem{Returns}
\sphinxAtStartPar
arg

\sphinxlineitem{Return type}
\sphinxAtStartPar
list

\end{description}\end{quote}

\end{fulllineitems}

\index{Bridged\_Tee\_Attenuator\_Synthesis() (in module components)@\spxentry{Bridged\_Tee\_Attenuator\_Synthesis()}\spxextra{in module components}}

\begin{fulllineitems}
\phantomsection\label{\detokenize{components:components.Bridged_Tee_Attenuator_Synthesis}}
\pysigstartsignatures
\pysiglinewithargsret{\sphinxcode{\sphinxupquote{components.}}\sphinxbfcode{\sphinxupquote{Bridged\_Tee\_Attenuator\_Synthesis}}}{\emph{\DUrole{n}{arg}}, \emph{\DUrole{n}{defaultunits}\DUrole{o}{=}\DUrole{default_value}{{[}{]}}}}{}
\pysigstopsignatures
\sphinxAtStartPar
Bridged Tee Attenuator Synthesis.
\begin{quote}\begin{description}
\sphinxlineitem{Parameters}\begin{itemize}
\item {} 
\sphinxAtStartPar
\sphinxstyleliteralstrong{\sphinxupquote{arg}} (\sphinxstyleliteralemphasis{\sphinxupquote{list}}) \textendash{} 
\sphinxAtStartPar
First 3 arguments are inputs.
\begin{enumerate}
\sphinxsetlistlabels{\arabic}{enumi}{enumii}{}{.}%
\item {} 
\sphinxAtStartPar
Reference Impedance (Zo); impedance

\item {} 
\sphinxAtStartPar
Series Impedance (Rs); impedance

\item {} 
\sphinxAtStartPar
Parallel Impedance (Rp); impedance

\item {} 
\sphinxAtStartPar
S(1,1) ;

\end{enumerate}

\sphinxAtStartPar
5. S(2,1) ;
Reference:


\item {} 
\sphinxAtStartPar
\sphinxstyleliteralstrong{\sphinxupquote{defaultunits}} (\sphinxstyleliteralemphasis{\sphinxupquote{list}}\sphinxstyleliteralemphasis{\sphinxupquote{, }}\sphinxstyleliteralemphasis{\sphinxupquote{optional}}) \textendash{} Default units for quantities in \sphinxstyleemphasis{arg} list. Default is {[}{]} which means SI units will be used if no unit is given in \sphinxstyleemphasis{arg}.

\end{itemize}

\sphinxlineitem{Returns}
\sphinxAtStartPar
arg

\sphinxlineitem{Return type}
\sphinxAtStartPar
list

\end{description}\end{quote}

\end{fulllineitems}

\index{Chebyshev\_QWave\_Impedance\_Transformer() (in module components)@\spxentry{Chebyshev\_QWave\_Impedance\_Transformer()}\spxextra{in module components}}

\begin{fulllineitems}
\phantomsection\label{\detokenize{components:components.Chebyshev_QWave_Impedance_Transformer}}
\pysigstartsignatures
\pysiglinewithargsret{\sphinxcode{\sphinxupquote{components.}}\sphinxbfcode{\sphinxupquote{Chebyshev\_QWave\_Impedance\_Transformer}}}{\emph{\DUrole{n}{arg}}, \emph{\DUrole{n}{defaultunits}\DUrole{o}{=}\DUrole{default_value}{{[}{]}}}}{}
\pysigstopsignatures
\sphinxAtStartPar
Chebyshev Quarter Wave Impedance Transformer.
\begin{quote}\begin{description}
\sphinxlineitem{Parameters}\begin{itemize}
\item {} 
\sphinxAtStartPar
\sphinxstyleliteralstrong{\sphinxupquote{arg}} (\sphinxstyleliteralemphasis{\sphinxupquote{list}}) \textendash{} 
\sphinxAtStartPar
First 6 arguments are inputs.
\begin{enumerate}
\sphinxsetlistlabels{\arabic}{enumi}{enumii}{}{.}%
\item {} 
\sphinxAtStartPar
Source Impedance ; impedance

\item {} 
\sphinxAtStartPar
Load Impedance ; impedance

\item {} 
\sphinxAtStartPar
Number Of Matching Sections ;

\item {} 
\sphinxAtStartPar
Minimum Frequency ; frequency

\item {} 
\sphinxAtStartPar
Maximum Frequency ; frequency

\item {} 
\sphinxAtStartPar
Test Frequency ; frequency

\item {} 
\sphinxAtStartPar
Impedances ; impedance

\end{enumerate}

\sphinxAtStartPar
8.  Return Loss at Test Frequency ;
Reference:  Impedance Matching and Transformation.pdf + eski kod


\item {} 
\sphinxAtStartPar
\sphinxstyleliteralstrong{\sphinxupquote{defaultunits}} (\sphinxstyleliteralemphasis{\sphinxupquote{list}}\sphinxstyleliteralemphasis{\sphinxupquote{, }}\sphinxstyleliteralemphasis{\sphinxupquote{optional}}) \textendash{} Default units for quantities in \sphinxstyleemphasis{arg} list. Default is {[}{]} which means SI units will be used if no unit is given in \sphinxstyleemphasis{arg}.

\end{itemize}

\sphinxlineitem{Returns}
\sphinxAtStartPar
arg

\sphinxlineitem{Return type}
\sphinxAtStartPar
list

\end{description}\end{quote}

\end{fulllineitems}

\index{Chebyshev\_Taper\_Impedance\_Transformer() (in module components)@\spxentry{Chebyshev\_Taper\_Impedance\_Transformer()}\spxextra{in module components}}

\begin{fulllineitems}
\phantomsection\label{\detokenize{components:components.Chebyshev_Taper_Impedance_Transformer}}
\pysigstartsignatures
\pysiglinewithargsret{\sphinxcode{\sphinxupquote{components.}}\sphinxbfcode{\sphinxupquote{Chebyshev\_Taper\_Impedance\_Transformer}}}{\emph{\DUrole{n}{arg}}, \emph{\DUrole{n}{defaultunits}\DUrole{o}{=}\DUrole{default_value}{{[}{]}}}}{}
\pysigstopsignatures
\sphinxAtStartPar
Calculates performance and impedance values for an N\sphinxhyphen{}section Chebyshev Impedance Taper.
Reference:  Foundations for Microwave Engineering, Collin
\begin{quote}\begin{description}
\sphinxlineitem{Parameters}\begin{itemize}
\item {} 
\sphinxAtStartPar
\sphinxstyleliteralstrong{\sphinxupquote{arg}} (\sphinxstyleliteralemphasis{\sphinxupquote{list}}) \textendash{} 
\sphinxAtStartPar
First 5 arguments are inputs.
\begin{enumerate}
\sphinxsetlistlabels{\arabic}{enumi}{enumii}{}{.}%
\item {} 
\sphinxAtStartPar
Source Impedance ; impedance

\item {} 
\sphinxAtStartPar
Load Impedance ; impedance

\item {} 
\sphinxAtStartPar
Number Of Sections (Even) ;

\item {} 
\sphinxAtStartPar
Fractional Bandwidth (F2/F1) ;

\item {} 
\sphinxAtStartPar
Length (normalized to Lambda at fcenter) ;

\item {} 
\sphinxAtStartPar
Impedances ; impedance

\item {} 
\sphinxAtStartPar
Return Loss ;

\end{enumerate}


\item {} 
\sphinxAtStartPar
\sphinxstyleliteralstrong{\sphinxupquote{defaultunits}} (\sphinxstyleliteralemphasis{\sphinxupquote{list}}\sphinxstyleliteralemphasis{\sphinxupquote{, }}\sphinxstyleliteralemphasis{\sphinxupquote{optional}}) \textendash{} Default units for quantities in \sphinxstyleemphasis{arg} list. Default is {[}{]} which means SI units will be used if no unit is given in \sphinxstyleemphasis{arg}.

\end{itemize}

\sphinxlineitem{Returns}
\sphinxAtStartPar
arg

\sphinxlineitem{Return type}
\sphinxAtStartPar
list

\end{description}\end{quote}

\end{fulllineitems}

\index{CircularPlateCap() (in module components)@\spxentry{CircularPlateCap()}\spxextra{in module components}}

\begin{fulllineitems}
\phantomsection\label{\detokenize{components:components.CircularPlateCap}}
\pysigstartsignatures
\pysiglinewithargsret{\sphinxcode{\sphinxupquote{components.}}\sphinxbfcode{\sphinxupquote{CircularPlateCap}}}{\emph{\DUrole{n}{arg}}, \emph{\DUrole{n}{defaultunits}\DUrole{o}{=}\DUrole{default_value}{{[}{]}}}}{}
\pysigstopsignatures
\sphinxAtStartPar
Circular Plate Capacitance.
\begin{quote}\begin{description}
\sphinxlineitem{Parameters}\begin{itemize}
\item {} 
\sphinxAtStartPar
\sphinxstyleliteralstrong{\sphinxupquote{arg}} (\sphinxstyleliteralemphasis{\sphinxupquote{list}}) \textendash{} 
\sphinxAtStartPar
First 3 arguments are inputs.
\begin{enumerate}
\sphinxsetlistlabels{\arabic}{enumi}{enumii}{}{.}%
\item {} 
\sphinxAtStartPar
Radius;length

\item {} 
\sphinxAtStartPar
Height;length

\item {} 
\sphinxAtStartPar
Dielectric Permittivity;

\item {} 
\sphinxAtStartPar
Frequency; frequency

\item {} 
\sphinxAtStartPar
Capacitance; capacitance

\item {} 
\sphinxAtStartPar
Impedance; impedance

\end{enumerate}


\item {} 
\sphinxAtStartPar
\sphinxstyleliteralstrong{\sphinxupquote{defaultunits}} (\sphinxstyleliteralemphasis{\sphinxupquote{list}}\sphinxstyleliteralemphasis{\sphinxupquote{, }}\sphinxstyleliteralemphasis{\sphinxupquote{optional}}) \textendash{} Default units for quantities in \sphinxstyleemphasis{arg} list. Default is {[}{]} which means SI units will be used if no unit is given in \sphinxstyleemphasis{arg}.

\end{itemize}

\sphinxlineitem{Returns}
\sphinxAtStartPar
arg

\sphinxlineitem{Return type}
\sphinxAtStartPar
list

\end{description}\end{quote}

\end{fulllineitems}

\index{Dia2AWG() (in module components)@\spxentry{Dia2AWG()}\spxextra{in module components}}

\begin{fulllineitems}
\phantomsection\label{\detokenize{components:components.Dia2AWG}}
\pysigstartsignatures
\pysiglinewithargsret{\sphinxcode{\sphinxupquote{components.}}\sphinxbfcode{\sphinxupquote{Dia2AWG}}}{\emph{\DUrole{n}{arg}}, \emph{\DUrole{n}{defaultunits}\DUrole{o}{=}\DUrole{default_value}{{[}{]}}}}{}
\pysigstopsignatures
\sphinxAtStartPar
Convert Diameter to AWG.
Reference:  Wikipedia
\begin{quote}\begin{description}
\sphinxlineitem{Parameters}\begin{itemize}
\item {} 
\sphinxAtStartPar
\sphinxstyleliteralstrong{\sphinxupquote{arg}} (\sphinxstyleliteralemphasis{\sphinxupquote{list}}) \textendash{} 
\sphinxAtStartPar
First 1 arguments are inputs.
\begin{enumerate}
\sphinxsetlistlabels{\arabic}{enumi}{enumii}{}{.}%
\item {} 
\sphinxAtStartPar
AWG ;

\item {} 
\sphinxAtStartPar
Diameter ;length

\item {} 
\sphinxAtStartPar
Current rating in still air ; current

\end{enumerate}


\item {} 
\sphinxAtStartPar
\sphinxstyleliteralstrong{\sphinxupquote{defaultunits}} (\sphinxstyleliteralemphasis{\sphinxupquote{list}}\sphinxstyleliteralemphasis{\sphinxupquote{, }}\sphinxstyleliteralemphasis{\sphinxupquote{optional}}) \textendash{} Default units for quantities in \sphinxstyleemphasis{arg} list. Default is {[}{]} which means SI units will be used if no unit is given in \sphinxstyleemphasis{arg}.

\end{itemize}

\sphinxlineitem{Returns}
\sphinxAtStartPar
arg

\sphinxlineitem{Return type}
\sphinxAtStartPar
list

\end{description}\end{quote}

\end{fulllineitems}

\index{DualFrequencyTransformer() (in module components)@\spxentry{DualFrequencyTransformer()}\spxextra{in module components}}

\begin{fulllineitems}
\phantomsection\label{\detokenize{components:components.DualFrequencyTransformer}}
\pysigstartsignatures
\pysiglinewithargsret{\sphinxcode{\sphinxupquote{components.}}\sphinxbfcode{\sphinxupquote{DualFrequencyTransformer}}}{\emph{\DUrole{n}{arg}}, \emph{\DUrole{n}{defaultunits}\DUrole{o}{=}\DUrole{default_value}{{[}{]}}}}{}
\pysigstopsignatures
\sphinxAtStartPar
Dual Frequency Transformer.
\begin{quote}\begin{description}
\sphinxlineitem{Parameters}\begin{itemize}
\item {} 
\sphinxAtStartPar
\sphinxstyleliteralstrong{\sphinxupquote{arg}} (\sphinxstyleliteralemphasis{\sphinxupquote{list}}) \textendash{} 
\sphinxAtStartPar
First 4 arguments are inputs.
\begin{enumerate}
\sphinxsetlistlabels{\arabic}{enumi}{enumii}{}{.}%
\item {} 
\sphinxAtStartPar
Source Impedance; impedance

\item {} 
\sphinxAtStartPar
Load Impedance; impedance

\item {} 
\sphinxAtStartPar
f1 Lower Frequency; frequency

\item {} 
\sphinxAtStartPar
f2 Higher Frequency; frequency

\item {} 
\sphinxAtStartPar
Z1; impedance

\item {} 
\sphinxAtStartPar
Z2; impedance

\end{enumerate}

\sphinxAtStartPar
7. Electrical Length ; angle
Reference:  A Small Dual Frequency Transformer in Two Sections


\item {} 
\sphinxAtStartPar
\sphinxstyleliteralstrong{\sphinxupquote{defaultunits}} (\sphinxstyleliteralemphasis{\sphinxupquote{list}}\sphinxstyleliteralemphasis{\sphinxupquote{, }}\sphinxstyleliteralemphasis{\sphinxupquote{optional}}) \textendash{} Default units for quantities in \sphinxstyleemphasis{arg} list. Default is {[}{]} which means SI units will be used if no unit is given in \sphinxstyleemphasis{arg}.

\end{itemize}

\sphinxlineitem{Returns}
\sphinxAtStartPar
arg

\sphinxlineitem{Return type}
\sphinxAtStartPar
list

\end{description}\end{quote}

\end{fulllineitems}

\index{DualTransformation1() (in module components)@\spxentry{DualTransformation1()}\spxextra{in module components}}

\begin{fulllineitems}
\phantomsection\label{\detokenize{components:components.DualTransformation1}}
\pysigstartsignatures
\pysiglinewithargsret{\sphinxcode{\sphinxupquote{components.}}\sphinxbfcode{\sphinxupquote{DualTransformation1}}}{\emph{\DUrole{n}{arg}}, \emph{\DUrole{n}{defaultunits}\DUrole{o}{=}\DUrole{default_value}{{[}{]}}}}{}
\pysigstopsignatures
\sphinxAtStartPar
Dual Transformation 1.
\begin{quote}\begin{description}
\sphinxlineitem{Parameters}\begin{itemize}
\item {} 
\sphinxAtStartPar
\sphinxstyleliteralstrong{\sphinxupquote{arg}} (\sphinxstyleliteralemphasis{\sphinxupquote{list}}) \textendash{} 
\sphinxAtStartPar
First 4 arguments are inputs.
\begin{enumerate}
\sphinxsetlistlabels{\arabic}{enumi}{enumii}{}{.}%
\item {} 
\sphinxAtStartPar
L1 ; inductance

\item {} 
\sphinxAtStartPar
C1 ; capacitance

\item {} 
\sphinxAtStartPar
L2 ; inductance

\item {} 
\sphinxAtStartPar
C2 ; capacitance

\item {} 
\sphinxAtStartPar
L1’ ; inductance

\item {} 
\sphinxAtStartPar
C1’ ; capacitance

\item {} 
\sphinxAtStartPar
L2’ ; inductance

\end{enumerate}

\sphinxAtStartPar
8.  C2’ ; capacitance
Reference:  Microstrip Filters for RF\sphinxhyphen{}Microwave Applications, s.25, Figure 2.6a


\item {} 
\sphinxAtStartPar
\sphinxstyleliteralstrong{\sphinxupquote{defaultunits}} (\sphinxstyleliteralemphasis{\sphinxupquote{list}}\sphinxstyleliteralemphasis{\sphinxupquote{, }}\sphinxstyleliteralemphasis{\sphinxupquote{optional}}) \textendash{} Default units for quantities in \sphinxstyleemphasis{arg} list. Default is {[}{]} which means SI units will be used if no unit is given in \sphinxstyleemphasis{arg}.

\end{itemize}

\sphinxlineitem{Returns}
\sphinxAtStartPar
arg

\sphinxlineitem{Return type}
\sphinxAtStartPar
list

\end{description}\end{quote}

\end{fulllineitems}

\index{DualTransformation2() (in module components)@\spxentry{DualTransformation2()}\spxextra{in module components}}

\begin{fulllineitems}
\phantomsection\label{\detokenize{components:components.DualTransformation2}}
\pysigstartsignatures
\pysiglinewithargsret{\sphinxcode{\sphinxupquote{components.}}\sphinxbfcode{\sphinxupquote{DualTransformation2}}}{\emph{\DUrole{n}{arg}}, \emph{\DUrole{n}{defaultunits}\DUrole{o}{=}\DUrole{default_value}{{[}{]}}}}{}
\pysigstopsignatures
\sphinxAtStartPar
Dual Transformation 1.
Reference:  Microstrip Filters for RF\sphinxhyphen{}Microwave Applications, s.25, Figure 2.6b
\begin{quote}\begin{description}
\sphinxlineitem{Parameters}\begin{itemize}
\item {} 
\sphinxAtStartPar
\sphinxstyleliteralstrong{\sphinxupquote{arg}} (\sphinxstyleliteralemphasis{\sphinxupquote{list}}) \textendash{} 
\sphinxAtStartPar
First 4 arguments are inputs.
\begin{enumerate}
\sphinxsetlistlabels{\arabic}{enumi}{enumii}{}{.}%
\item {} 
\sphinxAtStartPar
L1 ; inductance

\item {} 
\sphinxAtStartPar
C1 ; capacitance

\item {} 
\sphinxAtStartPar
L2 ; inductance

\item {} 
\sphinxAtStartPar
C2 ; capacitance

\item {} 
\sphinxAtStartPar
L1’ ; inductance

\item {} 
\sphinxAtStartPar
C1’ ; capacitance

\item {} 
\sphinxAtStartPar
L2’ ; inductance

\item {} 
\sphinxAtStartPar
C2’ ; capacitance

\end{enumerate}


\item {} 
\sphinxAtStartPar
\sphinxstyleliteralstrong{\sphinxupquote{defaultunits}} (\sphinxstyleliteralemphasis{\sphinxupquote{list}}\sphinxstyleliteralemphasis{\sphinxupquote{, }}\sphinxstyleliteralemphasis{\sphinxupquote{optional}}) \textendash{} Default units for quantities in \sphinxstyleemphasis{arg} list. Default is {[}{]} which means SI units will be used if no unit is given in \sphinxstyleemphasis{arg}.

\end{itemize}

\sphinxlineitem{Returns}
\sphinxAtStartPar
arg

\sphinxlineitem{Return type}
\sphinxAtStartPar
list

\end{description}\end{quote}

\end{fulllineitems}

\index{EWG\_ABCD() (in module components)@\spxentry{EWG\_ABCD()}\spxextra{in module components}}

\begin{fulllineitems}
\phantomsection\label{\detokenize{components:components.EWG_ABCD}}
\pysigstartsignatures
\pysiglinewithargsret{\sphinxcode{\sphinxupquote{components.}}\sphinxbfcode{\sphinxupquote{EWG\_ABCD}}}{\emph{\DUrole{n}{a}}, \emph{\DUrole{n}{b}}, \emph{\DUrole{n}{er}}, \emph{\DUrole{n}{length}}, \emph{\DUrole{n}{frek}}}{}
\pysigstopsignatures
\end{fulllineitems}

\index{EWG\_inv() (in module components)@\spxentry{EWG\_inv()}\spxextra{in module components}}

\begin{fulllineitems}
\phantomsection\label{\detokenize{components:components.EWG_inv}}
\pysigstartsignatures
\pysiglinewithargsret{\sphinxcode{\sphinxupquote{components.}}\sphinxbfcode{\sphinxupquote{EWG\_inv}}}{\emph{\DUrole{n}{a}}, \emph{\DUrole{n}{b}}, \emph{\DUrole{n}{er}}, \emph{\DUrole{n}{length}}, \emph{\DUrole{n}{frek}}}{}
\pysigstopsignatures
\end{fulllineitems}

\index{EvanescentWGEquivalent() (in module components)@\spxentry{EvanescentWGEquivalent()}\spxextra{in module components}}

\begin{fulllineitems}
\phantomsection\label{\detokenize{components:components.EvanescentWGEquivalent}}
\pysigstartsignatures
\pysiglinewithargsret{\sphinxcode{\sphinxupquote{components.}}\sphinxbfcode{\sphinxupquote{EvanescentWGEquivalent}}}{\emph{\DUrole{n}{arg}}, \emph{\DUrole{n}{defaultunits}\DUrole{o}{=}\DUrole{default_value}{{[}{]}}}}{}
\pysigstopsignatures
\sphinxAtStartPar
Waveguide Width Step from Rectangular Waveguide to Evanescent Mode Rectangular Waveguide.
Reference:  The Design of Evanescent Mode Waveguide Bandpass Filters for a Prescribed Insertion Loss Characteristic.pdf
\begin{quote}

\sphinxAtStartPar
Model= Xp1,Xs1,Xp1 ya da Xs2,Xp2,Xs2 (p: shunt, s: series)
Zo=jXo
\end{quote}
\begin{quote}\begin{description}
\sphinxlineitem{Parameters}\begin{itemize}
\item {} 
\sphinxAtStartPar
\sphinxstyleliteralstrong{\sphinxupquote{arg}} (\sphinxstyleliteralemphasis{\sphinxupquote{list}}) \textendash{} 
\sphinxAtStartPar
First 5 arguments are inputs.
\begin{enumerate}
\sphinxsetlistlabels{\arabic}{enumi}{enumii}{}{.}%
\item {} 
\sphinxAtStartPar
Waveguide Width;length

\item {} 
\sphinxAtStartPar
Waveguide Height;length

\item {} 
\sphinxAtStartPar
Dielectric Permittivity;

\item {} 
\sphinxAtStartPar
Waveguide Length;length

\item {} 
\sphinxAtStartPar
Frequency; frequency

\item {} 
\sphinxAtStartPar
Series Inductance For Shunt\sphinxhyphen{}Series\sphinxhyphen{}Shunt Model; inductance

\item {} 
\sphinxAtStartPar
Shunt Inductance For Shunt\sphinxhyphen{}Series\sphinxhyphen{}Shunt Model; inductance

\item {} 
\sphinxAtStartPar
Series Inductance For Series\sphinxhyphen{}Shunt\sphinxhyphen{}Series Model; inductance

\item {} 
\sphinxAtStartPar
Shunt Inductance For Series\sphinxhyphen{}Shunt\sphinxhyphen{}Series Model; inductance

\item {} 
\sphinxAtStartPar
Characteristic Impedance; impedance

\end{enumerate}


\item {} 
\sphinxAtStartPar
\sphinxstyleliteralstrong{\sphinxupquote{defaultunits}} (\sphinxstyleliteralemphasis{\sphinxupquote{list}}\sphinxstyleliteralemphasis{\sphinxupquote{, }}\sphinxstyleliteralemphasis{\sphinxupquote{optional}}) \textendash{} Default units for quantities in \sphinxstyleemphasis{arg} list. Default is {[}{]} which means SI units will be used if no unit is given in \sphinxstyleemphasis{arg}.

\end{itemize}

\sphinxlineitem{Returns}
\sphinxAtStartPar
arg

\sphinxlineitem{Return type}
\sphinxAtStartPar
list

\end{description}\end{quote}

\end{fulllineitems}

\index{Exponential\_Taper\_Impedance\_Transformer() (in module components)@\spxentry{Exponential\_Taper\_Impedance\_Transformer()}\spxextra{in module components}}

\begin{fulllineitems}
\phantomsection\label{\detokenize{components:components.Exponential_Taper_Impedance_Transformer}}
\pysigstartsignatures
\pysiglinewithargsret{\sphinxcode{\sphinxupquote{components.}}\sphinxbfcode{\sphinxupquote{Exponential\_Taper\_Impedance\_Transformer}}}{\emph{\DUrole{n}{arg}}, \emph{\DUrole{n}{defaultunits}\DUrole{o}{=}\DUrole{default_value}{{[}{]}}}}{}
\pysigstopsignatures
\sphinxAtStartPar
Exponential Impedance Taper.
Reference:  Foundations for Microwave Engineering, Collin
\begin{quote}\begin{description}
\sphinxlineitem{Parameters}\begin{itemize}
\item {} 
\sphinxAtStartPar
\sphinxstyleliteralstrong{\sphinxupquote{arg}} (\sphinxstyleliteralemphasis{\sphinxupquote{list}}) \textendash{} 
\sphinxAtStartPar
First 5 arguments are inputs.
\begin{enumerate}
\sphinxsetlistlabels{\arabic}{enumi}{enumii}{}{.}%
\item {} 
\sphinxAtStartPar
Source Impedance ; impedance

\item {} 
\sphinxAtStartPar
Load Impedance ; impedance

\item {} 
\sphinxAtStartPar
Number Of Sections ;

\item {} 
\sphinxAtStartPar
Fractional Bandwidth (F2/F1) ;

\item {} 
\sphinxAtStartPar
Length (normalized to Lambda at fcenter) ;

\item {} 
\sphinxAtStartPar
Impedances ; impedance

\item {} 
\sphinxAtStartPar
Return Loss ;

\end{enumerate}


\item {} 
\sphinxAtStartPar
\sphinxstyleliteralstrong{\sphinxupquote{defaultunits}} (\sphinxstyleliteralemphasis{\sphinxupquote{list}}\sphinxstyleliteralemphasis{\sphinxupquote{, }}\sphinxstyleliteralemphasis{\sphinxupquote{optional}}) \textendash{} Default units for quantities in \sphinxstyleemphasis{arg} list. Default is {[}{]} which means SI units will be used if no unit is given in \sphinxstyleemphasis{arg}.

\end{itemize}

\sphinxlineitem{Returns}
\sphinxAtStartPar
arg

\sphinxlineitem{Return type}
\sphinxAtStartPar
list

\end{description}\end{quote}

\end{fulllineitems}

\index{GyselPowerDivider() (in module components)@\spxentry{GyselPowerDivider()}\spxextra{in module components}}

\begin{fulllineitems}
\phantomsection\label{\detokenize{components:components.GyselPowerDivider}}
\pysigstartsignatures
\pysiglinewithargsret{\sphinxcode{\sphinxupquote{components.}}\sphinxbfcode{\sphinxupquote{GyselPowerDivider}}}{\emph{\DUrole{n}{arg}}, \emph{\DUrole{n}{defaultunits}\DUrole{o}{=}\DUrole{default_value}{{[}{]}}}}{}
\pysigstopsignatures
\sphinxAtStartPar
Triangle network to Star network transformation.
Reference:
\begin{quote}

\sphinxAtStartPar
Zo1: 1. port impedance
Zo2: 2. port impedance
Zo3: 3. port impedance
R1: first isolation resistor (2.porta yakin)
R2: second isolation resistor (3.porta yakin)
ratio: P2/P3 power ratio
Z1: impedance of transmission line between 1.port and 2.port
Z2: impedance of transmission line between 1.port and 3.port
Z3: impedance of transmission line between 2.port and isolation resistor
Z4: impedance of transmission line between 3.port and isolation resistor
\end{quote}
\begin{quote}\begin{description}
\sphinxlineitem{Parameters}\begin{itemize}
\item {} 
\sphinxAtStartPar
\sphinxstyleliteralstrong{\sphinxupquote{arg}} (\sphinxstyleliteralemphasis{\sphinxupquote{list}}) \textendash{} 
\sphinxAtStartPar
First 6 arguments are inputs.
\begin{enumerate}
\sphinxsetlistlabels{\arabic}{enumi}{enumii}{}{.}%
\item {} 
\sphinxAtStartPar
Zo1;  impedance

\item {} 
\sphinxAtStartPar
Zo2;  impedance

\item {} 
\sphinxAtStartPar
Zo3;  impedance

\item {} 
\sphinxAtStartPar
R1; impedance

\item {} 
\sphinxAtStartPar
R2; impedance

\item {} 
\sphinxAtStartPar
P2/P3 ratio;

\item {} 
\sphinxAtStartPar
Z1; impedance

\item {} 
\sphinxAtStartPar
Z2; impedance

\item {} 
\sphinxAtStartPar
Z3; impedance

\item {} 
\sphinxAtStartPar
Z4; impedance

\end{enumerate}


\item {} 
\sphinxAtStartPar
\sphinxstyleliteralstrong{\sphinxupquote{defaultunits}} (\sphinxstyleliteralemphasis{\sphinxupquote{list}}\sphinxstyleliteralemphasis{\sphinxupquote{, }}\sphinxstyleliteralemphasis{\sphinxupquote{optional}}) \textendash{} Default units for quantities in \sphinxstyleemphasis{arg} list. Default is {[}{]} which means SI units will be used if no unit is given in \sphinxstyleemphasis{arg}.

\end{itemize}

\sphinxlineitem{Returns}
\sphinxAtStartPar
arg

\sphinxlineitem{Return type}
\sphinxAtStartPar
list

\end{description}\end{quote}

\end{fulllineitems}

\index{HomogeneousRectWaveguideParameters\_TE() (in module components)@\spxentry{HomogeneousRectWaveguideParameters\_TE()}\spxextra{in module components}}

\begin{fulllineitems}
\phantomsection\label{\detokenize{components:components.HomogeneousRectWaveguideParameters_TE}}
\pysigstartsignatures
\pysiglinewithargsret{\sphinxcode{\sphinxupquote{components.}}\sphinxbfcode{\sphinxupquote{HomogeneousRectWaveguideParameters\_TE}}}{\emph{\DUrole{n}{arg}}, \emph{\DUrole{n}{defaultunits}\DUrole{o}{=}\DUrole{default_value}{{[}{]}}}}{}
\pysigstopsignatures
\sphinxAtStartPar
Homogeneous Rectangular Waveguide Parameters.
Reference:  Marcuvitz Waveguide Handbook s.253
\begin{quote}\begin{description}
\sphinxlineitem{Parameters}\begin{itemize}
\item {} 
\sphinxAtStartPar
\sphinxstyleliteralstrong{\sphinxupquote{arg}} (\sphinxstyleliteralemphasis{\sphinxupquote{list}}) \textendash{} 
\sphinxAtStartPar
First 10 arguments are inputs.
\begin{enumerate}
\sphinxsetlistlabels{\arabic}{enumi}{enumii}{}{.}%
\item {} 
\sphinxAtStartPar
Dielectric Permittivity in Waveguide;

\item {} 
\sphinxAtStartPar
Waveguide Width;length

\item {} 
\sphinxAtStartPar
Waveguide Height;length

\item {} 
\sphinxAtStartPar
Mode (0: Te, 1: Tm);

\item {} 
\sphinxAtStartPar
M;

\item {} 
\sphinxAtStartPar
N;

\item {} 
\sphinxAtStartPar
Tand Of Dielectric;

\item {} 
\sphinxAtStartPar
Electrical Conductivity Of Walls; electrical conductivity

\item {} 
\sphinxAtStartPar
Frequency; frequency

\item {} 
\sphinxAtStartPar
Physical Length;length

\item {} 
\sphinxAtStartPar
Cond Loss; loss per length

\item {} 
\sphinxAtStartPar
Diel Loss; loss per length

\item {} 
\sphinxAtStartPar
Cutoff Freq; frequency

\item {} 
\sphinxAtStartPar
Lambda\_Guided;length

\item {} 
\sphinxAtStartPar
Impedance; impedance

\item {} 
\sphinxAtStartPar
Electrical Length; angle

\end{enumerate}

\sphinxAtStartPar
17. Group Velocity;
17. Group Delay; time


\item {} 
\sphinxAtStartPar
\sphinxstyleliteralstrong{\sphinxupquote{defaultunits}} (\sphinxstyleliteralemphasis{\sphinxupquote{list}}\sphinxstyleliteralemphasis{\sphinxupquote{, }}\sphinxstyleliteralemphasis{\sphinxupquote{optional}}) \textendash{} Default units for quantities in \sphinxstyleemphasis{arg} list. Default is {[}{]} which means SI units will be used if no unit is given in \sphinxstyleemphasis{arg}.

\end{itemize}

\sphinxlineitem{Returns}
\sphinxAtStartPar
arg

\sphinxlineitem{Return type}
\sphinxAtStartPar
list

\end{description}\end{quote}

\end{fulllineitems}

\index{InductivePostInWaveguide() (in module components)@\spxentry{InductivePostInWaveguide()}\spxextra{in module components}}

\begin{fulllineitems}
\phantomsection\label{\detokenize{components:components.InductivePostInWaveguide}}
\pysigstartsignatures
\pysiglinewithargsret{\sphinxcode{\sphinxupquote{components.}}\sphinxbfcode{\sphinxupquote{InductivePostInWaveguide}}}{\emph{\DUrole{n}{arg}}, \emph{\DUrole{n}{defaultunits}\DUrole{o}{=}\DUrole{default_value}{{[}{]}}}}{}
\pysigstopsignatures
\sphinxAtStartPar
Inductive Post In Waveguide.
\begin{quote}\begin{description}
\sphinxlineitem{Parameters}\begin{itemize}
\item {} 
\sphinxAtStartPar
\sphinxstyleliteralstrong{\sphinxupquote{arg}} (\sphinxstyleliteralemphasis{\sphinxupquote{list}}) \textendash{} 
\sphinxAtStartPar
First 6 arguments are inputs.
\begin{enumerate}
\sphinxsetlistlabels{\arabic}{enumi}{enumii}{}{.}%
\item {} 
\sphinxAtStartPar
Dielectric Permittivity in Waveguide ;

\item {} 
\sphinxAtStartPar
Waveguide Width (a);length

\item {} 
\sphinxAtStartPar
Waveguide Height (b);length

\item {} 
\sphinxAtStartPar
Post Diameter (d);length

\item {} 
\sphinxAtStartPar
Waveguide Sidewall To Post Center (s);length

\item {} 
\sphinxAtStartPar
Frequency; frequency

\item {} 
\sphinxAtStartPar
Inductance;inductance

\item {} 
\sphinxAtStartPar
Capacitance; capacitance

\end{enumerate}

\sphinxAtStartPar
9. Impedance; impedance
Reference:  Marcuvitz Waveguide Handbook s.257


\item {} 
\sphinxAtStartPar
\sphinxstyleliteralstrong{\sphinxupquote{defaultunits}} (\sphinxstyleliteralemphasis{\sphinxupquote{list}}\sphinxstyleliteralemphasis{\sphinxupquote{, }}\sphinxstyleliteralemphasis{\sphinxupquote{optional}}) \textendash{} Default units for quantities in \sphinxstyleemphasis{arg} list. Default is {[}{]} which means SI units will be used if no unit is given in \sphinxstyleemphasis{arg}.

\end{itemize}

\sphinxlineitem{Returns}
\sphinxAtStartPar
arg

\sphinxlineitem{Return type}
\sphinxAtStartPar
list

\end{description}\end{quote}

\end{fulllineitems}

\index{InductiveWindowInWaveguide() (in module components)@\spxentry{InductiveWindowInWaveguide()}\spxextra{in module components}}

\begin{fulllineitems}
\phantomsection\label{\detokenize{components:components.InductiveWindowInWaveguide}}
\pysigstartsignatures
\pysiglinewithargsret{\sphinxcode{\sphinxupquote{components.}}\sphinxbfcode{\sphinxupquote{InductiveWindowInWaveguide}}}{\emph{\DUrole{n}{arg}}, \emph{\DUrole{n}{defaultunits}\DUrole{o}{=}\DUrole{default_value}{{[}{]}}}}{}
\pysigstopsignatures
\sphinxAtStartPar
Waveguide Width Step from Rectangular Waveguide to Evanescent Mode Rectangular Waveguide.
Reference:  Marcuvitz Waveguide Handbook s.253
\begin{quote}\begin{description}
\sphinxlineitem{Parameters}\begin{itemize}
\item {} 
\sphinxAtStartPar
\sphinxstyleliteralstrong{\sphinxupquote{arg}} (\sphinxstyleliteralemphasis{\sphinxupquote{list}}) \textendash{} 
\sphinxAtStartPar
First 6 arguments are inputs.
\begin{enumerate}
\sphinxsetlistlabels{\arabic}{enumi}{enumii}{}{.}%
\item {} 
\sphinxAtStartPar
Dielectric Permittivity in Waveguide ;

\item {} 
\sphinxAtStartPar
Waveguide Width (a);length

\item {} 
\sphinxAtStartPar
Waveguide Height (b);length

\item {} 
\sphinxAtStartPar
Difference Of Waveguide Width To Window Width;length

\item {} 
\sphinxAtStartPar
Window Thickness;length

\item {} 
\sphinxAtStartPar
Frequency; frequency

\item {} 
\sphinxAtStartPar
Inductance;inductance

\item {} 
\sphinxAtStartPar
Capacitance; capacitance

\item {} 
\sphinxAtStartPar
Impedance; impedance

\end{enumerate}


\item {} 
\sphinxAtStartPar
\sphinxstyleliteralstrong{\sphinxupquote{defaultunits}} (\sphinxstyleliteralemphasis{\sphinxupquote{list}}\sphinxstyleliteralemphasis{\sphinxupquote{, }}\sphinxstyleliteralemphasis{\sphinxupquote{optional}}) \textendash{} Default units for quantities in \sphinxstyleemphasis{arg} list. Default is {[}{]} which means SI units will be used if no unit is given in \sphinxstyleemphasis{arg}.

\end{itemize}

\sphinxlineitem{Returns}
\sphinxAtStartPar
arg

\sphinxlineitem{Return type}
\sphinxAtStartPar
list

\end{description}\end{quote}

\end{fulllineitems}

\index{Interference\_Phase\_Amp\_Error() (in module components)@\spxentry{Interference\_Phase\_Amp\_Error()}\spxextra{in module components}}

\begin{fulllineitems}
\phantomsection\label{\detokenize{components:components.Interference_Phase_Amp_Error}}
\pysigstartsignatures
\pysiglinewithargsret{\sphinxcode{\sphinxupquote{components.}}\sphinxbfcode{\sphinxupquote{Interference\_Phase\_Amp\_Error}}}{\emph{\DUrole{n}{arg}}, \emph{\DUrole{n}{defaultunits}\DUrole{o}{=}\DUrole{default_value}{{[}{]}}}}{}
\pysigstopsignatures
\sphinxAtStartPar
Maximum phase and amplitude variation of a signal in presence of an interfering signal.
\begin{quote}\begin{description}
\sphinxlineitem{Parameters}\begin{itemize}
\item {} 
\sphinxAtStartPar
\sphinxstyleliteralstrong{\sphinxupquote{arg}} (\sphinxstyleliteralemphasis{\sphinxupquote{list}}) \textendash{} 
\sphinxAtStartPar
First 1 arguments are inputs.
\begin{enumerate}
\sphinxsetlistlabels{\arabic}{enumi}{enumii}{}{.}%
\item {} 
\sphinxAtStartPar
Difference in dB ;

\item {} 
\sphinxAtStartPar
Amplitude Error;

\item {} 
\sphinxAtStartPar
Phase Error; angle

\end{enumerate}


\item {} 
\sphinxAtStartPar
\sphinxstyleliteralstrong{\sphinxupquote{defaultunits}} (\sphinxstyleliteralemphasis{\sphinxupquote{list}}\sphinxstyleliteralemphasis{\sphinxupquote{, }}\sphinxstyleliteralemphasis{\sphinxupquote{optional}}) \textendash{} Default units for quantities in \sphinxstyleemphasis{arg} list. Default is {[}{]} which means SI units will be used if no unit is given in \sphinxstyleemphasis{arg}.

\end{itemize}

\sphinxlineitem{Returns}
\sphinxAtStartPar
arg

\sphinxlineitem{Return type}
\sphinxAtStartPar
list

\end{description}\end{quote}

\end{fulllineitems}

\index{Klopfenstein\_Taper\_Impedance\_Transformer() (in module components)@\spxentry{Klopfenstein\_Taper\_Impedance\_Transformer()}\spxextra{in module components}}

\begin{fulllineitems}
\phantomsection\label{\detokenize{components:components.Klopfenstein_Taper_Impedance_Transformer}}
\pysigstartsignatures
\pysiglinewithargsret{\sphinxcode{\sphinxupquote{components.}}\sphinxbfcode{\sphinxupquote{Klopfenstein\_Taper\_Impedance\_Transformer}}}{\emph{\DUrole{n}{arg}}, \emph{\DUrole{n}{defaultunits}\DUrole{o}{=}\DUrole{default_value}{{[}{]}}}}{}
\pysigstopsignatures
\sphinxAtStartPar
Calculates performance and impedance values for an N\sphinxhyphen{}section Klopfenstein Impedance Taper.
\begin{quote}\begin{description}
\sphinxlineitem{Parameters}\begin{itemize}
\item {} 
\sphinxAtStartPar
\sphinxstyleliteralstrong{\sphinxupquote{arg}} (\sphinxstyleliteralemphasis{\sphinxupquote{list}}) \textendash{} 
\sphinxAtStartPar
First 6 arguments are inputs.
\begin{enumerate}
\sphinxsetlistlabels{\arabic}{enumi}{enumii}{}{.}%
\item {} 
\sphinxAtStartPar
Source Impedance ; impedance

\item {} 
\sphinxAtStartPar
Load Impedance ; impedance

\item {} 
\sphinxAtStartPar
Maximum Reflection Coefficient (dB) ;

\item {} 
\sphinxAtStartPar
Number Of Sections ;

\item {} 
\sphinxAtStartPar
Minimum Frequency ; frequency

\item {} 
\sphinxAtStartPar
Test Frequency ; frequency

\item {} 
\sphinxAtStartPar
Minimum Total Phase at Minimum Frequency ; angle ;

\item {} 
\sphinxAtStartPar
Impedances ; impedance

\end{enumerate}

\sphinxAtStartPar
9.  MAG(Reflection Coefficient) ;
Reference:  Microwave Engineering, Pozar


\item {} 
\sphinxAtStartPar
\sphinxstyleliteralstrong{\sphinxupquote{defaultunits}} (\sphinxstyleliteralemphasis{\sphinxupquote{list}}\sphinxstyleliteralemphasis{\sphinxupquote{, }}\sphinxstyleliteralemphasis{\sphinxupquote{optional}}) \textendash{} Default units for quantities in \sphinxstyleemphasis{arg} list. Default is {[}{]} which means SI units will be used if no unit is given in \sphinxstyleemphasis{arg}.

\end{itemize}

\sphinxlineitem{Returns}
\sphinxAtStartPar
arg

\sphinxlineitem{Return type}
\sphinxAtStartPar
list

\end{description}\end{quote}

\end{fulllineitems}

\index{LC\_Balun() (in module components)@\spxentry{LC\_Balun()}\spxextra{in module components}}

\begin{fulllineitems}
\phantomsection\label{\detokenize{components:components.LC_Balun}}
\pysigstartsignatures
\pysiglinewithargsret{\sphinxcode{\sphinxupquote{components.}}\sphinxbfcode{\sphinxupquote{LC\_Balun}}}{\emph{\DUrole{n}{arg}}, \emph{\DUrole{n}{defaultunits}\DUrole{o}{=}\DUrole{default_value}{{[}{]}}}}{}
\pysigstopsignatures
\sphinxAtStartPar
Calculate LC Balun.
\begin{quote}\begin{description}
\sphinxlineitem{Parameters}\begin{itemize}
\item {} 
\sphinxAtStartPar
\sphinxstyleliteralstrong{\sphinxupquote{arg}} (\sphinxstyleliteralemphasis{\sphinxupquote{list}}) \textendash{} 
\sphinxAtStartPar
First 4 arguments are inputs.
\begin{enumerate}
\sphinxsetlistlabels{\arabic}{enumi}{enumii}{}{.}%
\item {} 
\sphinxAtStartPar
Source Impedance (Rin) ; impedance

\item {} 
\sphinxAtStartPar
Load Impedances (RL) ; impedance

\item {} 
\sphinxAtStartPar
Frequency; frequency

\item {} 
\sphinxAtStartPar
Test Frequency ; frequency

\item {} 
\sphinxAtStartPar
Inductance ; inductance

\item {} 
\sphinxAtStartPar
Capacitance ; capacitance

\item {} 
\sphinxAtStartPar
S11 (dB) ;

\item {} 
\sphinxAtStartPar
S21 (dB) ;

\end{enumerate}

\sphinxAtStartPar
9.  S31 (dB) ;
Reference:


\item {} 
\sphinxAtStartPar
\sphinxstyleliteralstrong{\sphinxupquote{defaultunits}} (\sphinxstyleliteralemphasis{\sphinxupquote{list}}\sphinxstyleliteralemphasis{\sphinxupquote{, }}\sphinxstyleliteralemphasis{\sphinxupquote{optional}}) \textendash{} Default units for quantities in \sphinxstyleemphasis{arg} list. Default is {[}{]} which means SI units will be used if no unit is given in \sphinxstyleemphasis{arg}.

\end{itemize}

\sphinxlineitem{Returns}
\sphinxAtStartPar
arg

\sphinxlineitem{Return type}
\sphinxAtStartPar
list

\end{description}\end{quote}

\end{fulllineitems}

\index{L\_BondWire() (in module components)@\spxentry{L\_BondWire()}\spxextra{in module components}}

\begin{fulllineitems}
\phantomsection\label{\detokenize{components:components.L_BondWire}}
\pysigstartsignatures
\pysiglinewithargsret{\sphinxcode{\sphinxupquote{components.}}\sphinxbfcode{\sphinxupquote{L\_BondWire}}}{\emph{\DUrole{n}{arg}}, \emph{\DUrole{n}{defaultunits}\DUrole{o}{=}\DUrole{default_value}{{[}{]}}}}{}
\pysigstopsignatures
\sphinxAtStartPar
Inductance of a bond wire.
\begin{quote}\begin{description}
\sphinxlineitem{Parameters}\begin{itemize}
\item {} 
\sphinxAtStartPar
\sphinxstyleliteralstrong{\sphinxupquote{arg}} (\sphinxstyleliteralemphasis{\sphinxupquote{list}}) \textendash{} 
\sphinxAtStartPar
First 4 arguments are inputs.
\begin{enumerate}
\sphinxsetlistlabels{\arabic}{enumi}{enumii}{}{.}%
\item {} 
\sphinxAtStartPar
Bondwire Radius ;length

\item {} 
\sphinxAtStartPar
Substrate Thickness ;length

\item {} 
\sphinxAtStartPar
Distance Between End Points ;length

\item {} 
\sphinxAtStartPar
Angle At End Points In Degrees ; angle

\end{enumerate}

\sphinxAtStartPar
5. Inductance ;inductance
Reference:  Transmission Line Design Handbook, Wadell, s.153


\item {} 
\sphinxAtStartPar
\sphinxstyleliteralstrong{\sphinxupquote{defaultunits}} (\sphinxstyleliteralemphasis{\sphinxupquote{list}}\sphinxstyleliteralemphasis{\sphinxupquote{, }}\sphinxstyleliteralemphasis{\sphinxupquote{optional}}) \textendash{} Default units for quantities in \sphinxstyleemphasis{arg} list. Default is {[}{]} which means SI units will be used if no unit is given in \sphinxstyleemphasis{arg}.

\end{itemize}

\sphinxlineitem{Returns}
\sphinxAtStartPar
arg

\sphinxlineitem{Return type}
\sphinxAtStartPar
list

\end{description}\end{quote}

\end{fulllineitems}

\index{L\_StraightFlatWire() (in module components)@\spxentry{L\_StraightFlatWire()}\spxextra{in module components}}

\begin{fulllineitems}
\phantomsection\label{\detokenize{components:components.L_StraightFlatWire}}
\pysigstartsignatures
\pysiglinewithargsret{\sphinxcode{\sphinxupquote{components.}}\sphinxbfcode{\sphinxupquote{L\_StraightFlatWire}}}{\emph{\DUrole{n}{arg}}, \emph{\DUrole{n}{defaultunits}\DUrole{o}{=}\DUrole{default_value}{{[}{]}}}}{}
\pysigstopsignatures
\sphinxAtStartPar
Inductance of a flat wire.
\begin{quote}\begin{description}
\sphinxlineitem{Parameters}\begin{itemize}
\item {} 
\sphinxAtStartPar
\sphinxstyleliteralstrong{\sphinxupquote{arg}} (\sphinxstyleliteralemphasis{\sphinxupquote{list}}) \textendash{} 
\sphinxAtStartPar
First 6 arguments are inputs.
\begin{enumerate}
\sphinxsetlistlabels{\arabic}{enumi}{enumii}{}{.}%
\item {} 
\sphinxAtStartPar
Wire Width ;length

\item {} 
\sphinxAtStartPar
Wire Thickness ;length

\item {} 
\sphinxAtStartPar
Wire Length ;length

\item {} 
\sphinxAtStartPar
Frequency ; frequency

\item {} 
\sphinxAtStartPar
Relative Permeability ;

\item {} 
\sphinxAtStartPar
Conductivity ; electrical conductivity

\item {} 
\sphinxAtStartPar
Inductance ;inductance

\end{enumerate}

\sphinxAtStartPar
8.  Impedance ;impedance
Reference:  Transmission Line Design Handbook, Wadell, s.382


\item {} 
\sphinxAtStartPar
\sphinxstyleliteralstrong{\sphinxupquote{defaultunits}} (\sphinxstyleliteralemphasis{\sphinxupquote{list}}\sphinxstyleliteralemphasis{\sphinxupquote{, }}\sphinxstyleliteralemphasis{\sphinxupquote{optional}}) \textendash{} Default units for quantities in \sphinxstyleemphasis{arg} list. Default is {[}{]} which means SI units will be used if no unit is given in \sphinxstyleemphasis{arg}.

\end{itemize}

\sphinxlineitem{Returns}
\sphinxAtStartPar
arg

\sphinxlineitem{Return type}
\sphinxAtStartPar
list

\end{description}\end{quote}

\end{fulllineitems}

\index{L\_StraightRoundWire() (in module components)@\spxentry{L\_StraightRoundWire()}\spxextra{in module components}}

\begin{fulllineitems}
\phantomsection\label{\detokenize{components:components.L_StraightRoundWire}}
\pysigstartsignatures
\pysiglinewithargsret{\sphinxcode{\sphinxupquote{components.}}\sphinxbfcode{\sphinxupquote{L\_StraightRoundWire}}}{\emph{\DUrole{n}{arg}}, \emph{\DUrole{n}{defaultunits}\DUrole{o}{=}\DUrole{default_value}{{[}{]}}}}{}
\pysigstopsignatures
\sphinxAtStartPar
Inductance of a straight round wire.
\begin{quote}\begin{description}
\sphinxlineitem{Parameters}\begin{itemize}
\item {} 
\sphinxAtStartPar
\sphinxstyleliteralstrong{\sphinxupquote{arg}} (\sphinxstyleliteralemphasis{\sphinxupquote{list}}) \textendash{} 
\sphinxAtStartPar
First 5 arguments are inputs.
\begin{enumerate}
\sphinxsetlistlabels{\arabic}{enumi}{enumii}{}{.}%
\item {} 
\sphinxAtStartPar
Wire Diameter ;length

\item {} 
\sphinxAtStartPar
Wire Length ;length

\item {} 
\sphinxAtStartPar
Frequency ; frequency

\item {} 
\sphinxAtStartPar
Dielectric Permeability  ;

\item {} 
\sphinxAtStartPar
Conductivity ; electrical conductivity

\item {} 
\sphinxAtStartPar
Inductance ;inductance

\end{enumerate}

\sphinxAtStartPar
7. Impedance ; impedance
Reference:  Transmission Line Design Handbook, Wadell, s.380


\item {} 
\sphinxAtStartPar
\sphinxstyleliteralstrong{\sphinxupquote{defaultunits}} (\sphinxstyleliteralemphasis{\sphinxupquote{list}}\sphinxstyleliteralemphasis{\sphinxupquote{, }}\sphinxstyleliteralemphasis{\sphinxupquote{optional}}) \textendash{} Default units for quantities in \sphinxstyleemphasis{arg} list. Default is {[}{]} which means SI units will be used if no unit is given in \sphinxstyleemphasis{arg}.

\end{itemize}

\sphinxlineitem{Returns}
\sphinxAtStartPar
arg

\sphinxlineitem{Return type}
\sphinxAtStartPar
list

\end{description}\end{quote}

\end{fulllineitems}

\index{L\_air\_core\_coil() (in module components)@\spxentry{L\_air\_core\_coil()}\spxextra{in module components}}

\begin{fulllineitems}
\phantomsection\label{\detokenize{components:components.L_air_core_coil}}
\pysigstartsignatures
\pysiglinewithargsret{\sphinxcode{\sphinxupquote{components.}}\sphinxbfcode{\sphinxupquote{L\_air\_core\_coil}}}{\emph{\DUrole{n}{arg}}, \emph{\DUrole{n}{defaultunits}\DUrole{o}{=}\DUrole{default_value}{{[}{]}}}}{}
\pysigstopsignatures
\sphinxAtStartPar
Inductance of a via hole in microstrip.
\begin{quote}\begin{description}
\sphinxlineitem{Parameters}\begin{itemize}
\item {} 
\sphinxAtStartPar
\sphinxstyleliteralstrong{\sphinxupquote{arg}} (\sphinxstyleliteralemphasis{\sphinxupquote{list}}) \textendash{} 
\sphinxAtStartPar
First 4 arguments are inputs.
\begin{enumerate}
\sphinxsetlistlabels{\arabic}{enumi}{enumii}{}{.}%
\item {} 
\sphinxAtStartPar
Wire Diameter (d) ;length

\item {} 
\sphinxAtStartPar
Coil Inner Diameter (d\_in) ;length

\item {} 
\sphinxAtStartPar
Spacing Between Turns (s) ; length

\item {} 
\sphinxAtStartPar
Number Of Turns ;

\item {} 
\sphinxAtStartPar
Inductance ; inductance

\end{enumerate}

\sphinxAtStartPar
6. Resonance Frequency ; frequency
Reference:  www.microwavecoil.com , Microwave Components Inc.


\item {} 
\sphinxAtStartPar
\sphinxstyleliteralstrong{\sphinxupquote{defaultunits}} (\sphinxstyleliteralemphasis{\sphinxupquote{list}}\sphinxstyleliteralemphasis{\sphinxupquote{, }}\sphinxstyleliteralemphasis{\sphinxupquote{optional}}) \textendash{} Default units for quantities in \sphinxstyleemphasis{arg} list. Default is {[}{]} which means SI units will be used if no unit is given in \sphinxstyleemphasis{arg}.

\end{itemize}

\sphinxlineitem{Returns}
\sphinxAtStartPar
arg

\sphinxlineitem{Return type}
\sphinxAtStartPar
list

\end{description}\end{quote}

\end{fulllineitems}

\index{L\_microstrip\_via\_hole() (in module components)@\spxentry{L\_microstrip\_via\_hole()}\spxextra{in module components}}

\begin{fulllineitems}
\phantomsection\label{\detokenize{components:components.L_microstrip_via_hole}}
\pysigstartsignatures
\pysiglinewithargsret{\sphinxcode{\sphinxupquote{components.}}\sphinxbfcode{\sphinxupquote{L\_microstrip\_via\_hole}}}{\emph{\DUrole{n}{arg}}, \emph{\DUrole{n}{defaultunits}\DUrole{o}{=}\DUrole{default_value}{{[}{]}}}}{}
\pysigstopsignatures
\sphinxAtStartPar
Inductance of a via hole in microstrip.
\begin{quote}\begin{description}
\sphinxlineitem{Parameters}\begin{itemize}
\item {} 
\sphinxAtStartPar
\sphinxstyleliteralstrong{\sphinxupquote{arg}} (\sphinxstyleliteralemphasis{\sphinxupquote{list}}) \textendash{} 
\sphinxAtStartPar
First 2 arguments are inputs.
\begin{enumerate}
\sphinxsetlistlabels{\arabic}{enumi}{enumii}{}{.}%
\item {} 
\sphinxAtStartPar
Via Radius ;length

\item {} 
\sphinxAtStartPar
Substrate Thickness ;length

\end{enumerate}

\sphinxAtStartPar
3. Inductance ; inductance
Reference:  Microstrip Via Hole Grounds in Microstrip.pdf


\item {} 
\sphinxAtStartPar
\sphinxstyleliteralstrong{\sphinxupquote{defaultunits}} (\sphinxstyleliteralemphasis{\sphinxupquote{list}}\sphinxstyleliteralemphasis{\sphinxupquote{, }}\sphinxstyleliteralemphasis{\sphinxupquote{optional}}) \textendash{} Default units for quantities in \sphinxstyleemphasis{arg} list. Default is {[}{]} which means SI units will be used if no unit is given in \sphinxstyleemphasis{arg}.

\end{itemize}

\sphinxlineitem{Returns}
\sphinxAtStartPar
arg

\sphinxlineitem{Return type}
\sphinxAtStartPar
list

\end{description}\end{quote}

\end{fulllineitems}

\index{OptimumMitered90DegMicrostripBend() (in module components)@\spxentry{OptimumMitered90DegMicrostripBend()}\spxextra{in module components}}

\begin{fulllineitems}
\phantomsection\label{\detokenize{components:components.OptimumMitered90DegMicrostripBend}}
\pysigstartsignatures
\pysiglinewithargsret{\sphinxcode{\sphinxupquote{components.}}\sphinxbfcode{\sphinxupquote{OptimumMitered90DegMicrostripBend}}}{\emph{\DUrole{n}{arg}}, \emph{\DUrole{n}{defaultunits}\DUrole{o}{=}\DUrole{default_value}{{[}{]}}}}{}
\pysigstopsignatures
\sphinxAtStartPar
Optimum Mitered Microstrip Bend Parameters.
Reference: Tranmission line design handbook, p.290
\begin{quote}\begin{description}
\sphinxlineitem{Parameters}\begin{itemize}
\item {} 
\sphinxAtStartPar
\sphinxstyleliteralstrong{\sphinxupquote{arg}} (\sphinxstyleliteralemphasis{\sphinxupquote{list}}) \textendash{} 
\sphinxAtStartPar
First 2 arguments are inputs.
\begin{enumerate}
\sphinxsetlistlabels{\arabic}{enumi}{enumii}{}{.}%
\item {} 
\sphinxAtStartPar
Microstrip Width;length

\item {} 
\sphinxAtStartPar
Substrate Height;length

\item {} 
\sphinxAtStartPar
Miter Length; length

\end{enumerate}


\item {} 
\sphinxAtStartPar
\sphinxstyleliteralstrong{\sphinxupquote{defaultunits}} (\sphinxstyleliteralemphasis{\sphinxupquote{list}}\sphinxstyleliteralemphasis{\sphinxupquote{, }}\sphinxstyleliteralemphasis{\sphinxupquote{optional}}) \textendash{} Default units for quantities in \sphinxstyleemphasis{arg} list. Default is {[}{]} which means SI units will be used if no unit is given in \sphinxstyleemphasis{arg}.

\end{itemize}

\sphinxlineitem{Returns}
\sphinxAtStartPar
arg

\sphinxlineitem{Return type}
\sphinxAtStartPar
list

\end{description}\end{quote}

\end{fulllineitems}

\index{OptimumMiteredArbitraryAngleMicrostripBend() (in module components)@\spxentry{OptimumMiteredArbitraryAngleMicrostripBend()}\spxextra{in module components}}

\begin{fulllineitems}
\phantomsection\label{\detokenize{components:components.OptimumMiteredArbitraryAngleMicrostripBend}}
\pysigstartsignatures
\pysiglinewithargsret{\sphinxcode{\sphinxupquote{components.}}\sphinxbfcode{\sphinxupquote{OptimumMiteredArbitraryAngleMicrostripBend}}}{\emph{\DUrole{n}{arg}}, \emph{\DUrole{n}{defaultunits}\DUrole{o}{=}\DUrole{default_value}{{[}{]}}}}{}
\pysigstopsignatures
\sphinxAtStartPar
Optimum Mitered Microstrip Bend Parameters.
Reference: MWOHELP, MBENDA model
\begin{quote}\begin{description}
\sphinxlineitem{Parameters}
\sphinxAtStartPar
\sphinxstyleliteralstrong{\sphinxupquote{arg}} (\sphinxstyleliteralemphasis{\sphinxupquote{list}}) \textendash{} 
\sphinxAtStartPar
First 2 arguments are inputs.
\begin{enumerate}
\sphinxsetlistlabels{\arabic}{enumi}{enumii}{}{.}%
\item {} 
\sphinxAtStartPar
Microstrip Width;length;

\item {} 
\sphinxAtStartPar
Substrate Height;length;

\item {} 
\sphinxAtStartPar
Angle (0\sphinxhyphen{}180 degrees); angle ;

\item {} 
\sphinxAtStartPar
Miter Length; length ;

\end{enumerate}


\end{description}\end{quote}

\sphinxAtStartPar
Burada scipy.interpolate.griddata kullanildi ve maalesef extrapolation yapmiyor. Sinir disi degerlerde dogrudan en yakin deger kullanildi.
\begin{quote}

\sphinxAtStartPar
defaultunits(list, optional): Default units for quantities in \sphinxstyleemphasis{arg} list. Default is {[}{]} which means SI units will be used if no unit is given in \sphinxstyleemphasis{arg}.
\end{quote}
\begin{quote}\begin{description}
\sphinxlineitem{Returns}
\sphinxAtStartPar
arg

\sphinxlineitem{Return type}
\sphinxAtStartPar
list

\end{description}\end{quote}

\end{fulllineitems}

\index{PCBTrackCurrentCapacity() (in module components)@\spxentry{PCBTrackCurrentCapacity()}\spxextra{in module components}}

\begin{fulllineitems}
\phantomsection\label{\detokenize{components:components.PCBTrackCurrentCapacity}}
\pysigstartsignatures
\pysiglinewithargsret{\sphinxcode{\sphinxupquote{components.}}\sphinxbfcode{\sphinxupquote{PCBTrackCurrentCapacity}}}{\emph{\DUrole{n}{arg}}, \emph{\DUrole{n}{defaultunits}\DUrole{o}{=}\DUrole{default_value}{{[}{]}}}}{}
\pysigstopsignatures
\sphinxAtStartPar
PCB Track Current Capacity.
\begin{quote}\begin{description}
\sphinxlineitem{Parameters}\begin{itemize}
\item {} 
\sphinxAtStartPar
\sphinxstyleliteralstrong{\sphinxupquote{arg}} (\sphinxstyleliteralemphasis{\sphinxupquote{list}}) \textendash{} 
\sphinxAtStartPar
First 7 arguments are inputs.
\begin{enumerate}
\sphinxsetlistlabels{\arabic}{enumi}{enumii}{}{.}%
\item {} 
\sphinxAtStartPar
Metal Width;  length

\item {} 
\sphinxAtStartPar
PCB Height;     length

\item {} 
\sphinxAtStartPar
Metal Thickness;        length

\item {} 
\sphinxAtStartPar
Allowable Temperature Rise; temperature

\item {} 
\sphinxAtStartPar
Thermal Conductivity;  thermal conductivity

\item {} 
\sphinxAtStartPar
Electrical Conductivity; electrical conductivity

\item {} 
\sphinxAtStartPar
External if 1, Internal if 0;

\item {} 
\sphinxAtStartPar
Current ; current

\end{enumerate}


\item {} 
\sphinxAtStartPar
\sphinxstyleliteralstrong{\sphinxupquote{defaultunits}} (\sphinxstyleliteralemphasis{\sphinxupquote{list}}\sphinxstyleliteralemphasis{\sphinxupquote{, }}\sphinxstyleliteralemphasis{\sphinxupquote{optional}}) \textendash{} Default units for quantities in \sphinxstyleemphasis{arg} list. Default is {[}{]} which means SI units will be used if no unit is given in \sphinxstyleemphasis{arg}.

\end{itemize}

\sphinxlineitem{Returns}
\sphinxAtStartPar
arg

\sphinxlineitem{Return type}
\sphinxAtStartPar
list

\end{description}\end{quote}

\end{fulllineitems}

\index{PCBTrackCurrentCapacityIPC() (in module components)@\spxentry{PCBTrackCurrentCapacityIPC()}\spxextra{in module components}}

\begin{fulllineitems}
\phantomsection\label{\detokenize{components:components.PCBTrackCurrentCapacityIPC}}
\pysigstartsignatures
\pysiglinewithargsret{\sphinxcode{\sphinxupquote{components.}}\sphinxbfcode{\sphinxupquote{PCBTrackCurrentCapacityIPC}}}{\emph{\DUrole{n}{arg}}, \emph{\DUrole{n}{defaultunits}\DUrole{o}{=}\DUrole{default_value}{{[}{]}}}}{}
\pysigstopsignatures
\sphinxAtStartPar
PCB Track Current Capacity, IPC.
Reference: IPC2221A
\begin{quote}\begin{description}
\sphinxlineitem{Parameters}\begin{itemize}
\item {} 
\sphinxAtStartPar
\sphinxstyleliteralstrong{\sphinxupquote{arg}} (\sphinxstyleliteralemphasis{\sphinxupquote{list}}) \textendash{} 
\sphinxAtStartPar
First 4 arguments are inputs.
\begin{enumerate}
\sphinxsetlistlabels{\arabic}{enumi}{enumii}{}{.}%
\item {} 
\sphinxAtStartPar
Metal Width;length

\item {} 
\sphinxAtStartPar
Metal Thickness;length

\item {} 
\sphinxAtStartPar
Allowable Temperature Rise; temperature

\item {} 
\sphinxAtStartPar
External if 1, Internal if 0;

\item {} 
\sphinxAtStartPar
Current ; current

\end{enumerate}


\item {} 
\sphinxAtStartPar
\sphinxstyleliteralstrong{\sphinxupquote{defaultunits}} (\sphinxstyleliteralemphasis{\sphinxupquote{list}}\sphinxstyleliteralemphasis{\sphinxupquote{, }}\sphinxstyleliteralemphasis{\sphinxupquote{optional}}) \textendash{} Default units for quantities in \sphinxstyleemphasis{arg} list. Default is {[}{]} which means SI units will be used if no unit is given in \sphinxstyleemphasis{arg}.

\end{itemize}

\sphinxlineitem{Returns}
\sphinxAtStartPar
arg

\sphinxlineitem{Return type}
\sphinxAtStartPar
list

\end{description}\end{quote}

\end{fulllineitems}

\index{ParallelPlateCap() (in module components)@\spxentry{ParallelPlateCap()}\spxextra{in module components}}

\begin{fulllineitems}
\phantomsection\label{\detokenize{components:components.ParallelPlateCap}}
\pysigstartsignatures
\pysiglinewithargsret{\sphinxcode{\sphinxupquote{components.}}\sphinxbfcode{\sphinxupquote{ParallelPlateCap}}}{\emph{\DUrole{n}{arg}}, \emph{\DUrole{n}{defaultunits}\DUrole{o}{=}\DUrole{default_value}{{[}{]}}}}{}
\pysigstopsignatures
\sphinxAtStartPar
Parallel Plate Capacitance.
\begin{quote}\begin{description}
\sphinxlineitem{Parameters}\begin{itemize}
\item {} 
\sphinxAtStartPar
\sphinxstyleliteralstrong{\sphinxupquote{arg}} (\sphinxstyleliteralemphasis{\sphinxupquote{list}}) \textendash{} 
\sphinxAtStartPar
First 4 arguments are inputs.
\begin{enumerate}
\sphinxsetlistlabels{\arabic}{enumi}{enumii}{}{.}%
\item {} 
\sphinxAtStartPar
Width;length

\item {} 
\sphinxAtStartPar
Length;length

\item {} 
\sphinxAtStartPar
Height;length

\item {} 
\sphinxAtStartPar
Dielectric Permittivity;

\item {} 
\sphinxAtStartPar
Frequency; frequency

\item {} 
\sphinxAtStartPar
Capacitance; capacitance

\item {} 
\sphinxAtStartPar
Impedance; impedance

\end{enumerate}


\item {} 
\sphinxAtStartPar
\sphinxstyleliteralstrong{\sphinxupquote{defaultunits}} (\sphinxstyleliteralemphasis{\sphinxupquote{list}}\sphinxstyleliteralemphasis{\sphinxupquote{, }}\sphinxstyleliteralemphasis{\sphinxupquote{optional}}) \textendash{} Default units for quantities in \sphinxstyleemphasis{arg} list. Default is {[}{]} which means SI units will be used if no unit is given in \sphinxstyleemphasis{arg}.

\end{itemize}

\sphinxlineitem{Returns}
\sphinxAtStartPar
arg

\sphinxlineitem{Return type}
\sphinxAtStartPar
list

\end{description}\end{quote}

\end{fulllineitems}

\index{Patch\_Antenna\_Analysis() (in module components)@\spxentry{Patch\_Antenna\_Analysis()}\spxextra{in module components}}

\begin{fulllineitems}
\phantomsection\label{\detokenize{components:components.Patch_Antenna_Analysis}}
\pysigstartsignatures
\pysiglinewithargsret{\sphinxcode{\sphinxupquote{components.}}\sphinxbfcode{\sphinxupquote{Patch\_Antenna\_Analysis}}}{\emph{\DUrole{n}{arg}}, \emph{\DUrole{n}{defaultunits}\DUrole{o}{=}\DUrole{default_value}{{[}{]}}}}{}
\pysigstopsignatures
\sphinxAtStartPar
Calculates performance and impedance values for an N\sphinxhyphen{}section Chebyshev Impedance Taper.
Ref: Overview of Microstrip Antennas (Jackson) (Presentation)
Reference:  Foundations for Microwave Engineering, Collin
\begin{quote}\begin{description}
\sphinxlineitem{Parameters}\begin{itemize}
\item {} 
\sphinxAtStartPar
\sphinxstyleliteralstrong{\sphinxupquote{arg}} (\sphinxstyleliteralemphasis{\sphinxupquote{list}}) \textendash{} 
\sphinxAtStartPar
First 6 arguments are inputs.
\begin{enumerate}
\sphinxsetlistlabels{\arabic}{enumi}{enumii}{}{.}%
\item {} 
\sphinxAtStartPar
Width (W) ; length

\item {} 
\sphinxAtStartPar
Length (L) ; length

\item {} 
\sphinxAtStartPar
Substrate Thickness (h);length

\item {} 
\sphinxAtStartPar
Dielectric Permittivity ;

\item {} 
\sphinxAtStartPar
Dielectric Loss Tangent ;

\item {} 
\sphinxAtStartPar
Metal Conductivity ; electrical conductivity

\item {} 
\sphinxAtStartPar
Resonance Frequency (f) ; frequency

\item {} 
\sphinxAtStartPar
Bandwidth ; frequency

\end{enumerate}


\item {} 
\sphinxAtStartPar
\sphinxstyleliteralstrong{\sphinxupquote{defaultunits}} (\sphinxstyleliteralemphasis{\sphinxupquote{list}}\sphinxstyleliteralemphasis{\sphinxupquote{, }}\sphinxstyleliteralemphasis{\sphinxupquote{optional}}) \textendash{} Default units for quantities in \sphinxstyleemphasis{arg} list. Default is {[}{]} which means SI units will be used if no unit is given in \sphinxstyleemphasis{arg}.

\end{itemize}

\sphinxlineitem{Returns}
\sphinxAtStartPar
arg

\sphinxlineitem{Return type}
\sphinxAtStartPar
list

\end{description}\end{quote}

\end{fulllineitems}

\index{Pi\_Attenuator\_Analysis() (in module components)@\spxentry{Pi\_Attenuator\_Analysis()}\spxextra{in module components}}

\begin{fulllineitems}
\phantomsection\label{\detokenize{components:components.Pi_Attenuator_Analysis}}
\pysigstartsignatures
\pysiglinewithargsret{\sphinxcode{\sphinxupquote{components.}}\sphinxbfcode{\sphinxupquote{Pi\_Attenuator\_Analysis}}}{\emph{\DUrole{n}{arg}}, \emph{\DUrole{n}{defaultunits}\DUrole{o}{=}\DUrole{default_value}{{[}{]}}}}{}
\pysigstopsignatures
\sphinxAtStartPar
Pi Attenuator Analysis.
\begin{quote}\begin{description}
\sphinxlineitem{Parameters}\begin{itemize}
\item {} 
\sphinxAtStartPar
\sphinxstyleliteralstrong{\sphinxupquote{arg}} (\sphinxstyleliteralemphasis{\sphinxupquote{list}}) \textendash{} 
\sphinxAtStartPar
First 3 arguments are inputs.
\begin{enumerate}
\sphinxsetlistlabels{\arabic}{enumi}{enumii}{}{.}%
\item {} 
\sphinxAtStartPar
Reference Impedance (Zo); impedance

\item {} 
\sphinxAtStartPar
Series Impedance (Rs); impedance

\item {} 
\sphinxAtStartPar
Parallel Impedance (Rp); impedance

\item {} 
\sphinxAtStartPar
S(1,1) ;

\item {} 
\sphinxAtStartPar
S(2,1) ;

\item {} 
\sphinxAtStartPar
P1 ;

\item {} 
\sphinxAtStartPar
P2 ;

\end{enumerate}

\sphinxAtStartPar
8. P3 ;
Reference:


\item {} 
\sphinxAtStartPar
\sphinxstyleliteralstrong{\sphinxupquote{defaultunits}} (\sphinxstyleliteralemphasis{\sphinxupquote{list}}\sphinxstyleliteralemphasis{\sphinxupquote{, }}\sphinxstyleliteralemphasis{\sphinxupquote{optional}}) \textendash{} Default units for quantities in \sphinxstyleemphasis{arg} list. Default is {[}{]} which means SI units will be used if no unit is given in \sphinxstyleemphasis{arg}.

\end{itemize}

\sphinxlineitem{Returns}
\sphinxAtStartPar
arg

\sphinxlineitem{Return type}
\sphinxAtStartPar
list

\end{description}\end{quote}

\end{fulllineitems}

\index{Pi\_Attenuator\_Synthesis() (in module components)@\spxentry{Pi\_Attenuator\_Synthesis()}\spxextra{in module components}}

\begin{fulllineitems}
\phantomsection\label{\detokenize{components:components.Pi_Attenuator_Synthesis}}
\pysigstartsignatures
\pysiglinewithargsret{\sphinxcode{\sphinxupquote{components.}}\sphinxbfcode{\sphinxupquote{Pi\_Attenuator\_Synthesis}}}{\emph{\DUrole{n}{arg}}, \emph{\DUrole{n}{defaultunits}\DUrole{o}{=}\DUrole{default_value}{{[}{]}}}}{}
\pysigstopsignatures
\sphinxAtStartPar
Pi Attenuator Analysis.
\begin{quote}\begin{description}
\sphinxlineitem{Parameters}\begin{itemize}
\item {} 
\sphinxAtStartPar
\sphinxstyleliteralstrong{\sphinxupquote{arg}} (\sphinxstyleliteralemphasis{\sphinxupquote{list}}) \textendash{} 
\sphinxAtStartPar
First 3 arguments are inputs.
\begin{enumerate}
\sphinxsetlistlabels{\arabic}{enumi}{enumii}{}{.}%
\item {} 
\sphinxAtStartPar
Reference Impedance (Zo); impedance

\item {} 
\sphinxAtStartPar
Series Impedance (Rs); impedance

\item {} 
\sphinxAtStartPar
Parallel Impedance (Rp); impedance

\item {} 
\sphinxAtStartPar
S(1,1) ;

\item {} 
\sphinxAtStartPar
S(2,1) ;

\item {} 
\sphinxAtStartPar
P1 ;

\item {} 
\sphinxAtStartPar
P2 ;

\end{enumerate}

\sphinxAtStartPar
8. P3 ;
Reference:


\item {} 
\sphinxAtStartPar
\sphinxstyleliteralstrong{\sphinxupquote{defaultunits}} (\sphinxstyleliteralemphasis{\sphinxupquote{list}}\sphinxstyleliteralemphasis{\sphinxupquote{, }}\sphinxstyleliteralemphasis{\sphinxupquote{optional}}) \textendash{} Default units for quantities in \sphinxstyleemphasis{arg} list. Default is {[}{]} which means SI units will be used if no unit is given in \sphinxstyleemphasis{arg}.

\end{itemize}

\sphinxlineitem{Returns}
\sphinxAtStartPar
arg

\sphinxlineitem{Return type}
\sphinxAtStartPar
list

\end{description}\end{quote}

\end{fulllineitems}

\index{RectWG2EvanescentRectWGStep() (in module components)@\spxentry{RectWG2EvanescentRectWGStep()}\spxextra{in module components}}

\begin{fulllineitems}
\phantomsection\label{\detokenize{components:components.RectWG2EvanescentRectWGStep}}
\pysigstartsignatures
\pysiglinewithargsret{\sphinxcode{\sphinxupquote{components.}}\sphinxbfcode{\sphinxupquote{RectWG2EvanescentRectWGStep}}}{\emph{\DUrole{n}{a1}}, \emph{\DUrole{n}{a2}}}{}
\pysigstopsignatures
\sphinxAtStartPar
Waveguide Width Step from Rectangular Waveguide to Evanescent Mode Rectangular Waveguide.
Reference:  The Design of Evanescent Mode Waveguide Bandpass Filters for a Prescribed Insertion Loss Characteristic.pdf
\begin{quote}\begin{description}
\sphinxlineitem{Parameters}\begin{itemize}
\item {} 
\sphinxAtStartPar
\sphinxstyleliteralstrong{\sphinxupquote{arg}} (\sphinxstyleliteralemphasis{\sphinxupquote{list}}) \textendash{} 
\sphinxAtStartPar
First 2 arguments are inputs.
\begin{enumerate}
\sphinxsetlistlabels{\arabic}{enumi}{enumii}{}{.}%
\item {} 
\sphinxAtStartPar
Width of Rectangular Waveguide;length;

\item {} 
\sphinxAtStartPar
Width of Evanescent Mode Rectangular Waveguide;length;

\item {} 
\sphinxAtStartPar
Inductance; inductance

\item {} 
\sphinxAtStartPar
Turns Ratio;

\end{enumerate}


\item {} 
\sphinxAtStartPar
\sphinxstyleliteralstrong{\sphinxupquote{defaultunits}} (\sphinxstyleliteralemphasis{\sphinxupquote{list}}\sphinxstyleliteralemphasis{\sphinxupquote{, }}\sphinxstyleliteralemphasis{\sphinxupquote{optional}}) \textendash{} Default units for quantities in \sphinxstyleemphasis{arg} list. Default is {[}{]} which means SI units will be used if no unit is given in \sphinxstyleemphasis{arg}.

\end{itemize}

\sphinxlineitem{Returns}
\sphinxAtStartPar
arg

\sphinxlineitem{Return type}
\sphinxAtStartPar
list

\end{description}\end{quote}

\end{fulllineitems}

\index{SIW\_EquivalentWidth() (in module components)@\spxentry{SIW\_EquivalentWidth()}\spxextra{in module components}}

\begin{fulllineitems}
\phantomsection\label{\detokenize{components:components.SIW_EquivalentWidth}}
\pysigstartsignatures
\pysiglinewithargsret{\sphinxcode{\sphinxupquote{components.}}\sphinxbfcode{\sphinxupquote{SIW\_EquivalentWidth}}}{\emph{\DUrole{n}{w}}, \emph{\DUrole{n}{d}}, \emph{\DUrole{n}{s}}}{}
\pysigstopsignatures
\sphinxAtStartPar
Equivalent width of substrate integrated waveguide.
\begin{quote}\begin{description}
\sphinxlineitem{Parameters}\begin{itemize}
\item {} 
\sphinxAtStartPar
\sphinxstyleliteralstrong{\sphinxupquote{w}} (\sphinxstyleliteralemphasis{\sphinxupquote{float}}) \textendash{} Distance between the centers of two via arrays.

\item {} 
\sphinxAtStartPar
\sphinxstyleliteralstrong{\sphinxupquote{d}} (\sphinxstyleliteralemphasis{\sphinxupquote{float}}) \textendash{} Diameter of vias.

\item {} 
\sphinxAtStartPar
\sphinxstyleliteralstrong{\sphinxupquote{s}} (\sphinxstyleliteralemphasis{\sphinxupquote{float}}) \textendash{} Distance between the centers of consecutive vias of via arrays.

\end{itemize}

\sphinxlineitem{Returns}
\sphinxAtStartPar
Equivalent width of waveguide.

\sphinxlineitem{Return type}
\sphinxAtStartPar
float

\end{description}\end{quote}

\end{fulllineitems}

\index{Shorten90DegreeLine() (in module components)@\spxentry{Shorten90DegreeLine()}\spxextra{in module components}}

\begin{fulllineitems}
\phantomsection\label{\detokenize{components:components.Shorten90DegreeLine}}
\pysigstartsignatures
\pysiglinewithargsret{\sphinxcode{\sphinxupquote{components.}}\sphinxbfcode{\sphinxupquote{Shorten90DegreeLine}}}{\emph{\DUrole{n}{arg}}, \emph{\DUrole{n}{defaultunits}\DUrole{o}{=}\DUrole{default_value}{{[}{]}}}}{}
\pysigstopsignatures
\sphinxAtStartPar
Shortening 90 Degree Line with a capacitive load.
\begin{quote}\begin{description}
\sphinxlineitem{Parameters}\begin{itemize}
\item {} 
\sphinxAtStartPar
\sphinxstyleliteralstrong{\sphinxupquote{arg}} (\sphinxstyleliteralemphasis{\sphinxupquote{list}}) \textendash{} 
\sphinxAtStartPar
First 3 arguments are inputs.
\begin{enumerate}
\sphinxsetlistlabels{\arabic}{enumi}{enumii}{}{.}%
\item {} 
\sphinxAtStartPar
Impedance (Zo); impedance

\item {} 
\sphinxAtStartPar
Center Frequency ;  frequency

\item {} 
\sphinxAtStartPar
Electrical Length (theta) ; angle

\item {} 
\sphinxAtStartPar
Impedance (Z); impedance

\item {} 
\sphinxAtStartPar
Capacitance ; capacitance

\end{enumerate}


\item {} 
\sphinxAtStartPar
\sphinxstyleliteralstrong{\sphinxupquote{defaultunits}} (\sphinxstyleliteralemphasis{\sphinxupquote{list}}\sphinxstyleliteralemphasis{\sphinxupquote{, }}\sphinxstyleliteralemphasis{\sphinxupquote{optional}}) \textendash{} Default units for quantities in \sphinxstyleemphasis{arg} list. Default is {[}{]} which means SI units will be used if no unit is given in \sphinxstyleemphasis{arg}.

\end{itemize}

\sphinxlineitem{Returns}
\sphinxAtStartPar
arg

\sphinxlineitem{Return type}
\sphinxAtStartPar
list

\end{description}\end{quote}

\end{fulllineitems}

\index{Star2TriangleTransformation() (in module components)@\spxentry{Star2TriangleTransformation()}\spxextra{in module components}}

\begin{fulllineitems}
\phantomsection\label{\detokenize{components:components.Star2TriangleTransformation}}
\pysigstartsignatures
\pysiglinewithargsret{\sphinxcode{\sphinxupquote{components.}}\sphinxbfcode{\sphinxupquote{Star2TriangleTransformation}}}{\emph{\DUrole{n}{arg}}, \emph{\DUrole{n}{defaultunits}\DUrole{o}{=}\DUrole{default_value}{{[}{]}}}}{}
\pysigstopsignatures
\sphinxAtStartPar
Star network to Triangle network transformation.
\begin{quote}\begin{description}
\sphinxlineitem{Parameters}\begin{itemize}
\item {} 
\sphinxAtStartPar
\sphinxstyleliteralstrong{\sphinxupquote{arg}} (\sphinxstyleliteralemphasis{\sphinxupquote{list}}) \textendash{} 
\sphinxAtStartPar
First 3 arguments are inputs.
\begin{enumerate}
\sphinxsetlistlabels{\arabic}{enumi}{enumii}{}{.}%
\item {} 
\sphinxAtStartPar
Z1; impedance

\item {} 
\sphinxAtStartPar
Z2; impedance

\item {} 
\sphinxAtStartPar
Z3; impedance

\item {} 
\sphinxAtStartPar
Z1’; impedance

\item {} 
\sphinxAtStartPar
Z2’; impedance

\end{enumerate}

\sphinxAtStartPar
6. Z3’; impedance
Reference:
At star, z1 is connected to A\sphinxhyphen{}node, z2 is connected to B\sphinxhyphen{}node, z3 is connected to C\sphinxhyphen{}node
At triangle, z1 is between A\sphinxhyphen{}B, z2 is between A\sphinxhyphen{}C, z3 is between B\sphinxhyphen{}C


\item {} 
\sphinxAtStartPar
\sphinxstyleliteralstrong{\sphinxupquote{defaultunits}} (\sphinxstyleliteralemphasis{\sphinxupquote{list}}\sphinxstyleliteralemphasis{\sphinxupquote{, }}\sphinxstyleliteralemphasis{\sphinxupquote{optional}}) \textendash{} Default units for quantities in \sphinxstyleemphasis{arg} list. Default is {[}{]} which means SI units will be used if no unit is given in \sphinxstyleemphasis{arg}.

\end{itemize}

\sphinxlineitem{Returns}
\sphinxAtStartPar
arg

\sphinxlineitem{Return type}
\sphinxAtStartPar
list

\end{description}\end{quote}

\end{fulllineitems}

\index{SymmetricLangeCoupler() (in module components)@\spxentry{SymmetricLangeCoupler()}\spxextra{in module components}}

\begin{fulllineitems}
\phantomsection\label{\detokenize{components:components.SymmetricLangeCoupler}}
\pysigstartsignatures
\pysiglinewithargsret{\sphinxcode{\sphinxupquote{components.}}\sphinxbfcode{\sphinxupquote{SymmetricLangeCoupler}}}{\emph{\DUrole{n}{arg}}, \emph{\DUrole{n}{defaultunits}\DUrole{o}{=}\DUrole{default_value}{{[}{]}}}}{}
\pysigstopsignatures
\sphinxAtStartPar
Symmetric Lange Coupler.
\begin{quote}\begin{description}
\sphinxlineitem{Parameters}\begin{itemize}
\item {} 
\sphinxAtStartPar
\sphinxstyleliteralstrong{\sphinxupquote{arg}} (\sphinxstyleliteralemphasis{\sphinxupquote{list}}) \textendash{} 
\sphinxAtStartPar
First 3 arguments are inputs.
\begin{enumerate}
\sphinxsetlistlabels{\arabic}{enumi}{enumii}{}{.}%
\item {} 
\sphinxAtStartPar
C: Voltage coupling coefficient in dB (positive);

\item {} 
\sphinxAtStartPar
n: Number of fingers (should be even);

\item {} 
\sphinxAtStartPar
Reference Impedance;impedance

\item {} 
\sphinxAtStartPar
Zoo;impedance

\end{enumerate}

\sphinxAtStartPar
5. Zoe;impedance
Reference:  Microwave Circuits, Analysis and Computer\sphinxhyphen{}Aided Design, Fusco


\item {} 
\sphinxAtStartPar
\sphinxstyleliteralstrong{\sphinxupquote{defaultunits}} (\sphinxstyleliteralemphasis{\sphinxupquote{list}}\sphinxstyleliteralemphasis{\sphinxupquote{, }}\sphinxstyleliteralemphasis{\sphinxupquote{optional}}) \textendash{} Default units for quantities in \sphinxstyleemphasis{arg} list. Default is {[}{]} which means SI units will be used if no unit is given in \sphinxstyleemphasis{arg}.

\end{itemize}

\sphinxlineitem{Returns}
\sphinxAtStartPar
arg

\sphinxlineitem{Return type}
\sphinxAtStartPar
list

\end{description}\end{quote}

\end{fulllineitems}

\index{Tee\_Attenuator\_Analysis() (in module components)@\spxentry{Tee\_Attenuator\_Analysis()}\spxextra{in module components}}

\begin{fulllineitems}
\phantomsection\label{\detokenize{components:components.Tee_Attenuator_Analysis}}
\pysigstartsignatures
\pysiglinewithargsret{\sphinxcode{\sphinxupquote{components.}}\sphinxbfcode{\sphinxupquote{Tee\_Attenuator\_Analysis}}}{\emph{\DUrole{n}{arg}}, \emph{\DUrole{n}{defaultunits}\DUrole{o}{=}\DUrole{default_value}{{[}{]}}}}{}
\pysigstopsignatures
\sphinxAtStartPar
Tee Attenuator Analysis.
\begin{quote}\begin{description}
\sphinxlineitem{Parameters}\begin{itemize}
\item {} 
\sphinxAtStartPar
\sphinxstyleliteralstrong{\sphinxupquote{arg}} (\sphinxstyleliteralemphasis{\sphinxupquote{list}}) \textendash{} 
\sphinxAtStartPar
First 3 arguments are inputs.
\begin{enumerate}
\sphinxsetlistlabels{\arabic}{enumi}{enumii}{}{.}%
\item {} 
\sphinxAtStartPar
Reference Impedance (Zo); impedance

\item {} 
\sphinxAtStartPar
Series Impedance (Rs); impedance

\item {} 
\sphinxAtStartPar
Parallel Impedance (Rp); impedance

\item {} 
\sphinxAtStartPar
S(1,1) ;

\item {} 
\sphinxAtStartPar
S(2,1) ;

\item {} 
\sphinxAtStartPar
P1 ;

\item {} 
\sphinxAtStartPar
P2 ;

\end{enumerate}

\sphinxAtStartPar
8. P3 ;
Reference:


\item {} 
\sphinxAtStartPar
\sphinxstyleliteralstrong{\sphinxupquote{defaultunits}} (\sphinxstyleliteralemphasis{\sphinxupquote{list}}\sphinxstyleliteralemphasis{\sphinxupquote{, }}\sphinxstyleliteralemphasis{\sphinxupquote{optional}}) \textendash{} Default units for quantities in \sphinxstyleemphasis{arg} list. Default is {[}{]} which means SI units will be used if no unit is given in \sphinxstyleemphasis{arg}.

\end{itemize}

\sphinxlineitem{Returns}
\sphinxAtStartPar
arg

\sphinxlineitem{Return type}
\sphinxAtStartPar
list

\end{description}\end{quote}

\end{fulllineitems}

\index{Tee\_Attenuator\_Synthesis() (in module components)@\spxentry{Tee\_Attenuator\_Synthesis()}\spxextra{in module components}}

\begin{fulllineitems}
\phantomsection\label{\detokenize{components:components.Tee_Attenuator_Synthesis}}
\pysigstartsignatures
\pysiglinewithargsret{\sphinxcode{\sphinxupquote{components.}}\sphinxbfcode{\sphinxupquote{Tee\_Attenuator\_Synthesis}}}{\emph{\DUrole{n}{arg}}, \emph{\DUrole{n}{defaultunits}\DUrole{o}{=}\DUrole{default_value}{{[}{]}}}}{}
\pysigstopsignatures
\sphinxAtStartPar
Tee Attenuator Synthesis.
\begin{quote}\begin{description}
\sphinxlineitem{Parameters}\begin{itemize}
\item {} 
\sphinxAtStartPar
\sphinxstyleliteralstrong{\sphinxupquote{arg}} (\sphinxstyleliteralemphasis{\sphinxupquote{list}}) \textendash{} 
\sphinxAtStartPar
First 5 arguments are inputs.
\begin{enumerate}
\sphinxsetlistlabels{\arabic}{enumi}{enumii}{}{.}%
\item {} 
\sphinxAtStartPar
Reference Impedance (Zo); impedance

\item {} 
\sphinxAtStartPar
Series Impedance (Rs); impedance

\item {} 
\sphinxAtStartPar
Parallel Impedance (Rp); impedance

\item {} 
\sphinxAtStartPar
S(1,1) ;

\item {} 
\sphinxAtStartPar
S(2,1) ;

\item {} 
\sphinxAtStartPar
P1 ;

\item {} 
\sphinxAtStartPar
P2 ;

\end{enumerate}

\sphinxAtStartPar
8. P3 ;
Reference:


\item {} 
\sphinxAtStartPar
\sphinxstyleliteralstrong{\sphinxupquote{defaultunits}} (\sphinxstyleliteralemphasis{\sphinxupquote{list}}\sphinxstyleliteralemphasis{\sphinxupquote{, }}\sphinxstyleliteralemphasis{\sphinxupquote{optional}}) \textendash{} Default units for quantities in \sphinxstyleemphasis{arg} list. Default is {[}{]} which means SI units will be used if no unit is given in \sphinxstyleemphasis{arg}.

\end{itemize}

\sphinxlineitem{Returns}
\sphinxAtStartPar
arg

\sphinxlineitem{Return type}
\sphinxAtStartPar
list

\end{description}\end{quote}

\end{fulllineitems}

\index{Triangle2StarTransformation() (in module components)@\spxentry{Triangle2StarTransformation()}\spxextra{in module components}}

\begin{fulllineitems}
\phantomsection\label{\detokenize{components:components.Triangle2StarTransformation}}
\pysigstartsignatures
\pysiglinewithargsret{\sphinxcode{\sphinxupquote{components.}}\sphinxbfcode{\sphinxupquote{Triangle2StarTransformation}}}{\emph{\DUrole{n}{arg}}, \emph{\DUrole{n}{defaultunits}\DUrole{o}{=}\DUrole{default_value}{{[}{]}}}}{}
\pysigstopsignatures
\sphinxAtStartPar
Triangle network to Star network transformation.
At star, z1 is connected to A\sphinxhyphen{}node, z2 is connected to B\sphinxhyphen{}node, z3 is connected to C\sphinxhyphen{}node
At triangle, z1’ is between A\sphinxhyphen{}B, z2’ is between A\sphinxhyphen{}C, z3’ is between B\sphinxhyphen{}C
\begin{quote}\begin{description}
\sphinxlineitem{Parameters}\begin{itemize}
\item {} 
\sphinxAtStartPar
\sphinxstyleliteralstrong{\sphinxupquote{arg}} (\sphinxstyleliteralemphasis{\sphinxupquote{list}}) \textendash{} 
\sphinxAtStartPar
Last 3 arguments are inputs.
\begin{enumerate}
\sphinxsetlistlabels{\arabic}{enumi}{enumii}{}{.}%
\item {} 
\sphinxAtStartPar
Z1; impedance

\item {} 
\sphinxAtStartPar
Z2; impedance

\item {} 
\sphinxAtStartPar
Z3; impedance

\item {} 
\sphinxAtStartPar
Z1’; impedance

\item {} 
\sphinxAtStartPar
Z2’; impedance

\item {} 
\sphinxAtStartPar
Z3’; impedance

\end{enumerate}


\item {} 
\sphinxAtStartPar
\sphinxstyleliteralstrong{\sphinxupquote{defaultunits}} (\sphinxstyleliteralemphasis{\sphinxupquote{list}}\sphinxstyleliteralemphasis{\sphinxupquote{, }}\sphinxstyleliteralemphasis{\sphinxupquote{optional}}) \textendash{} Default units for quantities in \sphinxstyleemphasis{arg} list. Default is {[}{]} which means SI units will be used if no unit is given in \sphinxstyleemphasis{arg}.

\end{itemize}

\sphinxlineitem{Returns}
\sphinxAtStartPar
arg

\sphinxlineitem{Return type}
\sphinxAtStartPar
list

\end{description}\end{quote}

\end{fulllineitems}

\index{Triangular\_Taper\_Impedance\_Transformer() (in module components)@\spxentry{Triangular\_Taper\_Impedance\_Transformer()}\spxextra{in module components}}

\begin{fulllineitems}
\phantomsection\label{\detokenize{components:components.Triangular_Taper_Impedance_Transformer}}
\pysigstartsignatures
\pysiglinewithargsret{\sphinxcode{\sphinxupquote{components.}}\sphinxbfcode{\sphinxupquote{Triangular\_Taper\_Impedance\_Transformer}}}{\emph{\DUrole{n}{arg}}, \emph{\DUrole{n}{defaultunits}\DUrole{o}{=}\DUrole{default_value}{{[}{]}}}}{}
\pysigstopsignatures
\sphinxAtStartPar
Triangular Impedance Taper.
Reference:  Foundations for Microwave Engineering, Collin
\begin{quote}\begin{description}
\sphinxlineitem{Parameters}\begin{itemize}
\item {} 
\sphinxAtStartPar
\sphinxstyleliteralstrong{\sphinxupquote{arg}} (\sphinxstyleliteralemphasis{\sphinxupquote{list}}) \textendash{} 
\sphinxAtStartPar
First 5 arguments are inputs.
\begin{enumerate}
\sphinxsetlistlabels{\arabic}{enumi}{enumii}{}{.}%
\item {} 
\sphinxAtStartPar
Source Impedance ; impedance

\item {} 
\sphinxAtStartPar
Load Impedance ; impedance

\item {} 
\sphinxAtStartPar
Number Of Sections (Even) ;

\item {} 
\sphinxAtStartPar
Fractional Bandwidth (F2/F1) ;

\item {} 
\sphinxAtStartPar
Length (normalized to Lambda at fcenter) ;

\item {} 
\sphinxAtStartPar
Impedances ; impedance

\item {} 
\sphinxAtStartPar
Return Loss ;

\end{enumerate}


\item {} 
\sphinxAtStartPar
\sphinxstyleliteralstrong{\sphinxupquote{defaultunits}} (\sphinxstyleliteralemphasis{\sphinxupquote{list}}\sphinxstyleliteralemphasis{\sphinxupquote{, }}\sphinxstyleliteralemphasis{\sphinxupquote{optional}}) \textendash{} Default units for quantities in \sphinxstyleemphasis{arg} list. Default is {[}{]} which means SI units will be used if no unit is given in \sphinxstyleemphasis{arg}.

\end{itemize}

\sphinxlineitem{Returns}
\sphinxAtStartPar
arg

\sphinxlineitem{Return type}
\sphinxAtStartPar
list

\end{description}\end{quote}

\end{fulllineitems}

\index{Z\_CWG() (in module components)@\spxentry{Z\_CWG()}\spxextra{in module components}}

\begin{fulllineitems}
\phantomsection\label{\detokenize{components:components.Z_CWG}}
\pysigstartsignatures
\pysiglinewithargsret{\sphinxcode{\sphinxupquote{components.}}\sphinxbfcode{\sphinxupquote{Z\_CWG}}}{\emph{\DUrole{n}{rad}}, \emph{\DUrole{n}{freq}}, \emph{\DUrole{n}{eps\_r}\DUrole{o}{=}\DUrole{default_value}{1}}, \emph{\DUrole{n}{v}\DUrole{o}{=}\DUrole{default_value}{0}}, \emph{\DUrole{n}{n}\DUrole{o}{=}\DUrole{default_value}{1}}, \emph{\DUrole{n}{mode}\DUrole{o}{=}\DUrole{default_value}{\textquotesingle{}TE\textquotesingle{}}}}{}
\pysigstopsignatures
\sphinxAtStartPar
Computes the wave impedance of circular waveguide.
\begin{quote}\begin{description}
\sphinxlineitem{Parameters}\begin{itemize}
\item {} 
\sphinxAtStartPar
\sphinxstyleliteralstrong{\sphinxupquote{v}} (\sphinxstyleliteralemphasis{\sphinxupquote{int}}) \textendash{} Mode number of \(\phi\).

\item {} 
\sphinxAtStartPar
\sphinxstyleliteralstrong{\sphinxupquote{n}} (\sphinxstyleliteralemphasis{\sphinxupquote{int}}) \textendash{} Radial mode number.

\item {} 
\sphinxAtStartPar
\sphinxstyleliteralstrong{\sphinxupquote{eps\_r}} (\sphinxstyleliteralemphasis{\sphinxupquote{float}}) \textendash{} Permittivity of filling material.

\item {} 
\sphinxAtStartPar
\sphinxstyleliteralstrong{\sphinxupquote{freq}} (\sphinxstyleliteralemphasis{\sphinxupquote{float}}) \textendash{} Frequency (Hz).

\item {} 
\sphinxAtStartPar
\sphinxstyleliteralstrong{\sphinxupquote{mode}} (\sphinxstyleliteralemphasis{\sphinxupquote{str}}) \textendash{} “TE” or “TM”.

\item {} 
\sphinxAtStartPar
\sphinxstyleliteralstrong{\sphinxupquote{rad}} (\sphinxstyleliteralemphasis{\sphinxupquote{float}}) \textendash{} Radius.

\end{itemize}

\sphinxlineitem{Returns}
\sphinxAtStartPar
Impedance.

\sphinxlineitem{Return type}
\sphinxAtStartPar
Z (float)

\end{description}\end{quote}

\end{fulllineitems}

\index{Z\_WG\_TE10() (in module components)@\spxentry{Z\_WG\_TE10()}\spxextra{in module components}}

\begin{fulllineitems}
\phantomsection\label{\detokenize{components:components.Z_WG_TE10}}
\pysigstartsignatures
\pysiglinewithargsret{\sphinxcode{\sphinxupquote{components.}}\sphinxbfcode{\sphinxupquote{Z\_WG\_TE10}}}{\emph{\DUrole{n}{er}}, \emph{\DUrole{n}{a}}, \emph{\DUrole{n}{b}}, \emph{\DUrole{n}{freq}}, \emph{\DUrole{n}{formulation}\DUrole{o}{=}\DUrole{default_value}{1}}}{}
\pysigstopsignatures
\end{fulllineitems}

\index{Zo\_eeff\_StraightWireOverSubstrate() (in module components)@\spxentry{Zo\_eeff\_StraightWireOverSubstrate()}\spxextra{in module components}}

\begin{fulllineitems}
\phantomsection\label{\detokenize{components:components.Zo_eeff_StraightWireOverSubstrate}}
\pysigstartsignatures
\pysiglinewithargsret{\sphinxcode{\sphinxupquote{components.}}\sphinxbfcode{\sphinxupquote{Zo\_eeff\_StraightWireOverSubstrate}}}{\emph{\DUrole{n}{arg}}, \emph{\DUrole{n}{defaultunits}\DUrole{o}{=}\DUrole{default_value}{{[}{]}}}}{}
\pysigstopsignatures
\sphinxAtStartPar
Impedance and Effective Permittivity of Straight Wire Over Substrate.
\begin{quote}\begin{description}
\sphinxlineitem{Parameters}\begin{itemize}
\item {} 
\sphinxAtStartPar
\sphinxstyleliteralstrong{\sphinxupquote{arg}} (\sphinxstyleliteralemphasis{\sphinxupquote{list}}) \textendash{} 
\sphinxAtStartPar
First 4 arguments are inputs.
\begin{enumerate}
\sphinxsetlistlabels{\arabic}{enumi}{enumii}{}{.}%
\item {} 
\sphinxAtStartPar
Wire Diameter (d);length

\item {} 
\sphinxAtStartPar
Height Of Wire Center Above Ground (h);length

\item {} 
\sphinxAtStartPar
Dielectric Thickness (t);length

\item {} 
\sphinxAtStartPar
Dielectric Permittivity ;

\item {} 
\sphinxAtStartPar
Impedance ; impedance

\end{enumerate}

\sphinxAtStartPar
6.  Effective Diel. Permittivity ;
Reference:  Transmission Line Design Handbook, Wadell, s.151


\item {} 
\sphinxAtStartPar
\sphinxstyleliteralstrong{\sphinxupquote{defaultunits}} (\sphinxstyleliteralemphasis{\sphinxupquote{list}}\sphinxstyleliteralemphasis{\sphinxupquote{, }}\sphinxstyleliteralemphasis{\sphinxupquote{optional}}) \textendash{} Default units for quantities in \sphinxstyleemphasis{arg} list. Default is {[}{]} which means SI units will be used if no unit is given in \sphinxstyleemphasis{arg}.

\end{itemize}

\sphinxlineitem{Returns}
\sphinxAtStartPar
arg

\sphinxlineitem{Return type}
\sphinxAtStartPar
list

\end{description}\end{quote}

\end{fulllineitems}

\index{Zo\_eeff\_WireOnGroundedSubstrate() (in module components)@\spxentry{Zo\_eeff\_WireOnGroundedSubstrate()}\spxextra{in module components}}

\begin{fulllineitems}
\phantomsection\label{\detokenize{components:components.Zo_eeff_WireOnGroundedSubstrate}}
\pysigstartsignatures
\pysiglinewithargsret{\sphinxcode{\sphinxupquote{components.}}\sphinxbfcode{\sphinxupquote{Zo\_eeff\_WireOnGroundedSubstrate}}}{\emph{\DUrole{n}{arg}}, \emph{\DUrole{n}{defaultunits}\DUrole{o}{=}\DUrole{default_value}{{[}{]}}}}{}
\pysigstopsignatures
\sphinxAtStartPar
Impedance and Effective Permittivity of Straight Wire Over Substrate.
\begin{quote}\begin{description}
\sphinxlineitem{Parameters}\begin{itemize}
\item {} 
\sphinxAtStartPar
\sphinxstyleliteralstrong{\sphinxupquote{arg}} (\sphinxstyleliteralemphasis{\sphinxupquote{list}}) \textendash{} 
\sphinxAtStartPar
First 4 arguments are inputs.
\begin{enumerate}
\sphinxsetlistlabels{\arabic}{enumi}{enumii}{}{.}%
\item {} 
\sphinxAtStartPar
Wire Diameter (d);length

\item {} 
\sphinxAtStartPar
Dielectric Thickness (t);length

\item {} 
\sphinxAtStartPar
Dielectric Permittivity ;

\item {} 
\sphinxAtStartPar
Impedance ; impedance

\end{enumerate}

\sphinxAtStartPar
5.  Effective Diel. Permittivity ;
Reference:  Transmission Line Design Handbook, Wadell, s.151
Note: eeff is the same as eeff of microstrip with w=2*d, t=0


\item {} 
\sphinxAtStartPar
\sphinxstyleliteralstrong{\sphinxupquote{defaultunits}} (\sphinxstyleliteralemphasis{\sphinxupquote{list}}\sphinxstyleliteralemphasis{\sphinxupquote{, }}\sphinxstyleliteralemphasis{\sphinxupquote{optional}}) \textendash{} Default units for quantities in \sphinxstyleemphasis{arg} list. Default is {[}{]} which means SI units will be used if no unit is given in \sphinxstyleemphasis{arg}.

\end{itemize}

\sphinxlineitem{Returns}
\sphinxAtStartPar
arg

\sphinxlineitem{Return type}
\sphinxAtStartPar
list

\end{description}\end{quote}

\end{fulllineitems}

\index{fcutoff\_CWG() (in module components)@\spxentry{fcutoff\_CWG()}\spxextra{in module components}}

\begin{fulllineitems}
\phantomsection\label{\detokenize{components:components.fcutoff_CWG}}
\pysigstartsignatures
\pysiglinewithargsret{\sphinxcode{\sphinxupquote{components.}}\sphinxbfcode{\sphinxupquote{fcutoff\_CWG}}}{\emph{\DUrole{n}{rad}}, \emph{\DUrole{n}{eps\_r}\DUrole{o}{=}\DUrole{default_value}{1}}, \emph{\DUrole{n}{v}\DUrole{o}{=}\DUrole{default_value}{0}}, \emph{\DUrole{n}{n}\DUrole{o}{=}\DUrole{default_value}{1}}, \emph{\DUrole{n}{mode}\DUrole{o}{=}\DUrole{default_value}{\textquotesingle{}TE\textquotesingle{}}}}{}
\pysigstopsignatures
\sphinxAtStartPar
Computes the cutoff frequency of circular waveguide.
\begin{quote}\begin{description}
\sphinxlineitem{Parameters}\begin{itemize}
\item {} 
\sphinxAtStartPar
\sphinxstyleliteralstrong{\sphinxupquote{v}} (\sphinxstyleliteralemphasis{\sphinxupquote{int}}) \textendash{} Mode number of \(\phi\).

\item {} 
\sphinxAtStartPar
\sphinxstyleliteralstrong{\sphinxupquote{n}} (\sphinxstyleliteralemphasis{\sphinxupquote{int}}) \textendash{} Radial mode number.

\item {} 
\sphinxAtStartPar
\sphinxstyleliteralstrong{\sphinxupquote{eps\_r}} (\sphinxstyleliteralemphasis{\sphinxupquote{float}}) \textendash{} Permittivity of filling material.

\item {} 
\sphinxAtStartPar
\sphinxstyleliteralstrong{\sphinxupquote{mode}} (\sphinxstyleliteralemphasis{\sphinxupquote{str}}) \textendash{} “TE” or “TM”.

\item {} 
\sphinxAtStartPar
\sphinxstyleliteralstrong{\sphinxupquote{rad}} (\sphinxstyleliteralemphasis{\sphinxupquote{float}}) \textendash{} Radius.

\end{itemize}

\sphinxlineitem{Returns}
\sphinxAtStartPar
Cutoff frequency (Hz).

\sphinxlineitem{Return type}
\sphinxAtStartPar
fc (float)

\end{description}\end{quote}

\end{fulllineitems}

\index{thermal\_conductance\_of\_via\_farm() (in module components)@\spxentry{thermal\_conductance\_of\_via\_farm()}\spxextra{in module components}}

\begin{fulllineitems}
\phantomsection\label{\detokenize{components:components.thermal_conductance_of_via_farm}}
\pysigstartsignatures
\pysiglinewithargsret{\sphinxcode{\sphinxupquote{components.}}\sphinxbfcode{\sphinxupquote{thermal\_conductance\_of\_via\_farm}}}{\emph{\DUrole{n}{arg}}, \emph{\DUrole{n}{defaultunits}}}{}
\pysigstopsignatures
\sphinxAtStartPar
Thermal conductance of an array of vias in PCB.
\begin{quote}\begin{description}
\sphinxlineitem{Parameters}\begin{itemize}
\item {} 
\sphinxAtStartPar
\sphinxstyleliteralstrong{\sphinxupquote{arg}} (\sphinxstyleliteralemphasis{\sphinxupquote{list}}) \textendash{} 
\sphinxAtStartPar
First 7 arguments are inputs.
\begin{enumerate}
\sphinxsetlistlabels{\arabic}{enumi}{enumii}{}{.}%
\item {} 
\sphinxAtStartPar
Plated Via Diameter (d);length

\item {} 
\sphinxAtStartPar
Plating Thickness (t);length

\item {} 
\sphinxAtStartPar
Area Width (w);length

\item {} 
\sphinxAtStartPar
Area Height (l);length

\item {} 
\sphinxAtStartPar
Dielectric Height (h);length

\item {} 
\sphinxAtStartPar
Number Of Vias (n);

\item {} 
\sphinxAtStartPar
Dielectric Thermal Conductivity ;   thermal conductivity

\item {} 
\sphinxAtStartPar
Metal Thermal Conductivity ; thermal conductivity

\item {} 
\sphinxAtStartPar
Thermal Conductance (W/K) ;

\item {} 
\sphinxAtStartPar
Thermal Resistance (K/W) ;

\end{enumerate}


\item {} 
\sphinxAtStartPar
\sphinxstyleliteralstrong{\sphinxupquote{defaultunits}} (\sphinxstyleliteralemphasis{\sphinxupquote{list}}\sphinxstyleliteralemphasis{\sphinxupquote{, }}\sphinxstyleliteralemphasis{\sphinxupquote{optional}}) \textendash{} Default units for quantities in \sphinxstyleemphasis{arg} list. Default is {[}{]} which means SI units will be used if no unit is given in \sphinxstyleemphasis{arg}.

\end{itemize}

\sphinxlineitem{Returns}
\sphinxAtStartPar
arg

\sphinxlineitem{Return type}
\sphinxAtStartPar
list

\end{description}\end{quote}

\end{fulllineitems}

\index{thermal\_conductance\_of\_via\_farm\_view() (in module components)@\spxentry{thermal\_conductance\_of\_via\_farm\_view()}\spxextra{in module components}}

\begin{fulllineitems}
\phantomsection\label{\detokenize{components:components.thermal_conductance_of_via_farm_view}}
\pysigstartsignatures
\pysiglinewithargsret{\sphinxcode{\sphinxupquote{components.}}\sphinxbfcode{\sphinxupquote{thermal\_conductance\_of\_via\_farm\_view}}}{\emph{\DUrole{n}{arg}}, \emph{\DUrole{n}{defaultunits}}}{}
\pysigstopsignatures
\end{fulllineitems}



\chapter{Indices and tables}
\label{\detokenize{index:indices-and-tables}}\begin{itemize}
\item {} 
\sphinxAtStartPar
\DUrole{xref,std,std-ref}{genindex}

\item {} 
\sphinxAtStartPar
\DUrole{xref,std,std-ref}{modindex}

\item {} 
\sphinxAtStartPar
\DUrole{xref,std,std-ref}{search}

\end{itemize}


\renewcommand{\indexname}{Python Module Index}
\begin{sphinxtheindex}
\let\bigletter\sphinxstyleindexlettergroup
\bigletter{c}
\item\relax\sphinxstyleindexentry{components}\sphinxstyleindexpageref{components:\detokenize{module-components}}
\indexspace
\bigletter{t}
\item\relax\sphinxstyleindexentry{touchstone}\sphinxstyleindexpageref{touchstone:\detokenize{module-touchstone}}
\end{sphinxtheindex}

\renewcommand{\indexname}{Index}
\printindex
\end{document}