%% Generated by Sphinx.
\def\sphinxdocclass{report}
\documentclass[letterpaper,10pt,english]{sphinxmanual}
\ifdefined\pdfpxdimen
   \let\sphinxpxdimen\pdfpxdimen\else\newdimen\sphinxpxdimen
\fi \sphinxpxdimen=.75bp\relax

\PassOptionsToPackage{warn}{textcomp}
\usepackage[utf8]{inputenc}
\ifdefined\DeclareUnicodeCharacter
% support both utf8 and utf8x syntaxes
  \ifdefined\DeclareUnicodeCharacterAsOptional
    \def\sphinxDUC#1{\DeclareUnicodeCharacter{"#1}}
  \else
    \let\sphinxDUC\DeclareUnicodeCharacter
  \fi
  \sphinxDUC{00A0}{\nobreakspace}
  \sphinxDUC{2500}{\sphinxunichar{2500}}
  \sphinxDUC{2502}{\sphinxunichar{2502}}
  \sphinxDUC{2514}{\sphinxunichar{2514}}
  \sphinxDUC{251C}{\sphinxunichar{251C}}
  \sphinxDUC{2572}{\textbackslash}
\fi
\usepackage{cmap}
\usepackage[T1]{fontenc}
\usepackage{amsmath,amssymb,amstext}
\usepackage{babel}



\usepackage{times}
\expandafter\ifx\csname T@LGR\endcsname\relax
\else
% LGR was declared as font encoding
  \substitutefont{LGR}{\rmdefault}{cmr}
  \substitutefont{LGR}{\sfdefault}{cmss}
  \substitutefont{LGR}{\ttdefault}{cmtt}
\fi
\expandafter\ifx\csname T@X2\endcsname\relax
  \expandafter\ifx\csname T@T2A\endcsname\relax
  \else
  % T2A was declared as font encoding
    \substitutefont{T2A}{\rmdefault}{cmr}
    \substitutefont{T2A}{\sfdefault}{cmss}
    \substitutefont{T2A}{\ttdefault}{cmtt}
  \fi
\else
% X2 was declared as font encoding
  \substitutefont{X2}{\rmdefault}{cmr}
  \substitutefont{X2}{\sfdefault}{cmss}
  \substitutefont{X2}{\ttdefault}{cmtt}
\fi


\usepackage[Bjarne]{fncychap}
\usepackage{sphinx}

\fvset{fontsize=\small}
\usepackage{geometry}


% Include hyperref last.
\usepackage{hyperref}
% Fix anchor placement for figures with captions.
\usepackage{hypcap}% it must be loaded after hyperref.
% Set up styles of URL: it should be placed after hyperref.
\urlstyle{same}

\usepackage{sphinxmessages}
\setcounter{tocdepth}{3}
\setcounter{secnumdepth}{3}


\title{spfile}
\date{Jul 29, 2021}
\release{0.1}
\author{Tuncay Erdöl}
\newcommand{\sphinxlogo}{\vbox{}}
\renewcommand{\releasename}{Release}
\makeindex
\begin{document}

\pagestyle{empty}
\sphinxmaketitle
\pagestyle{plain}
\sphinxtableofcontents
\pagestyle{normal}
\phantomsection\label{\detokenize{index::doc}}



\chapter{touchstone module}
\label{\detokenize{touchstone:module-touchstone}}\label{\detokenize{touchstone:touchstone-module}}\label{\detokenize{touchstone::doc}}\index{module@\spxentry{module}!touchstone@\spxentry{touchstone}}\index{touchstone@\spxentry{touchstone}!module@\spxentry{module}}\index{average\_networks() (in module touchstone)@\spxentry{average\_networks()}\spxextra{in module touchstone}}

\begin{fulllineitems}
\phantomsection\label{\detokenize{touchstone:touchstone.average_networks}}\pysiglinewithargsret{\sphinxcode{\sphinxupquote{touchstone.}}\sphinxbfcode{\sphinxupquote{average\_networks}}}{\emph{\DUrole{n}{networks}}}{}
\end{fulllineitems}

\index{cascade\_2ports() (in module touchstone)@\spxentry{cascade\_2ports()}\spxextra{in module touchstone}}

\begin{fulllineitems}
\phantomsection\label{\detokenize{touchstone:touchstone.cascade_2ports}}\pysiglinewithargsret{\sphinxcode{\sphinxupquote{touchstone.}}\sphinxbfcode{\sphinxupquote{cascade\_2ports}}}{\emph{\DUrole{n}{filenames}}}{}
\end{fulllineitems}

\index{extract\_gamma\_ereff() (in module touchstone)@\spxentry{extract\_gamma\_ereff()}\spxextra{in module touchstone}}

\begin{fulllineitems}
\phantomsection\label{\detokenize{touchstone:touchstone.extract_gamma_ereff}}\pysiglinewithargsret{\sphinxcode{\sphinxupquote{touchstone.}}\sphinxbfcode{\sphinxupquote{extract\_gamma\_ereff}}}{\emph{\DUrole{n}{file\_long}}, \emph{\DUrole{n}{file\_short}}, \emph{\DUrole{n}{dL}}, \emph{\DUrole{n}{sm}\DUrole{o}{=}\DUrole{default_value}{1}}}{}
Extraction of complex propagation constant (gamma) and complex effective permittivity from the S\sphinxhyphen{}Parameters of 2 uniform transmission lines with different lengths.
\begin{quote}\begin{description}
\item[{Parameters}] \leavevmode\begin{itemize}
\item {} 
\sphinxstyleliteralstrong{\sphinxupquote{file\_long}} (\sphinxstyleliteralemphasis{\sphinxupquote{str}}) \textendash{} S\sphinxhyphen{}Parameter filename of longer line.

\item {} 
\sphinxstyleliteralstrong{\sphinxupquote{file\_short}} (\sphinxstyleliteralemphasis{\sphinxupquote{str}}) \textendash{} S\sphinxhyphen{}Parameter filename of shorter line.

\item {} 
\sphinxstyleliteralstrong{\sphinxupquote{dL}} (\sphinxstyleliteralemphasis{\sphinxupquote{float}}) \textendash{} Difference of lengths of two lines (positive in meter).

\item {} 
\sphinxstyleliteralstrong{\sphinxupquote{sm}} (\sphinxstyleliteralemphasis{\sphinxupquote{int}}\sphinxstyleliteralemphasis{\sphinxupquote{, }}\sphinxstyleliteralemphasis{\sphinxupquote{optional}}) \textendash{} If this is larger than 1, this is used as number of points for smoothing. Defaults to 1.

\end{itemize}

\item[{Returns}] \leavevmode
tuple of two complex numpy arrays (gamma, er\_eff).

\item[{Return type}] \leavevmode
tuple

\end{description}\end{quote}

\end{fulllineitems}

\index{extract\_gamma\_ereff\_all() (in module touchstone)@\spxentry{extract\_gamma\_ereff\_all()}\spxextra{in module touchstone}}

\begin{fulllineitems}
\phantomsection\label{\detokenize{touchstone:touchstone.extract_gamma_ereff_all}}\pysiglinewithargsret{\sphinxcode{\sphinxupquote{touchstone.}}\sphinxbfcode{\sphinxupquote{extract\_gamma\_ereff\_all}}}{\emph{\DUrole{n}{files}}, \emph{\DUrole{n}{Ls}}, \emph{\DUrole{n}{sm}\DUrole{o}{=}\DUrole{default_value}{1}}}{}
Extraction of average complex propagation constant (gamma) and complex effective permittivity from the S\sphinxhyphen{}Parameters of multiple uniform transmission lines with different lengths.
\begin{quote}\begin{description}
\item[{Parameters}] \leavevmode\begin{itemize}
\item {} 
\sphinxstyleliteralstrong{\sphinxupquote{files}} (\sphinxstyleliteralemphasis{\sphinxupquote{list}}) \textendash{} List of S\sphinxhyphen{}Parameter filenames of transmission lines.

\item {} 
\sphinxstyleliteralstrong{\sphinxupquote{Ls}} (\sphinxstyleliteralemphasis{\sphinxupquote{list}}) \textendash{} List of lengths of transmission lines in the same order as \sphinxstyleemphasis{files} parameter.

\item {} 
\sphinxstyleliteralstrong{\sphinxupquote{sm}} (\sphinxstyleliteralemphasis{\sphinxupquote{int}}\sphinxstyleliteralemphasis{\sphinxupquote{, }}\sphinxstyleliteralemphasis{\sphinxupquote{optional}}) \textendash{} If this is larger than 1, this is used as number of points for smoothing. Defaults to 1.

\end{itemize}

\item[{Returns}] \leavevmode
tuple of two complex numpy arrays (gamma, er\_eff).

\item[{Return type}] \leavevmode
tuple

\end{description}\end{quote}

\end{fulllineitems}

\index{generate\_multiport\_spfile() (in module touchstone)@\spxentry{generate\_multiport\_spfile()}\spxextra{in module touchstone}}

\begin{fulllineitems}
\phantomsection\label{\detokenize{touchstone:touchstone.generate_multiport_spfile}}\pysiglinewithargsret{\sphinxcode{\sphinxupquote{touchstone.}}\sphinxbfcode{\sphinxupquote{generate\_multiport\_spfile}}}{\emph{\DUrole{n}{conffile}\DUrole{o}{=}\DUrole{default_value}{\textquotesingle{}\textquotesingle{}}}, \emph{\DUrole{n}{outputfilename}\DUrole{o}{=}\DUrole{default_value}{\textquotesingle{}\textquotesingle{}}}}{}
Configuration file format:
\sphinxhyphen{} comments start by “\#”
\sphinxhyphen{} every line’s format is:
\begin{quote}

i,j ? filename ? is, js
meaning:
S(is,js) of touchstone file filename is S(i,j) of outputfilename
\end{quote}

\end{fulllineitems}

\index{parseformat() (in module touchstone)@\spxentry{parseformat()}\spxextra{in module touchstone}}

\begin{fulllineitems}
\phantomsection\label{\detokenize{touchstone:touchstone.parseformat}}\pysiglinewithargsret{\sphinxcode{\sphinxupquote{touchstone.}}\sphinxbfcode{\sphinxupquote{parseformat}}}{\emph{\DUrole{n}{cumle}}}{}
\end{fulllineitems}

\index{spfile (class in touchstone)@\spxentry{spfile}\spxextra{class in touchstone}}

\begin{fulllineitems}
\phantomsection\label{\detokenize{touchstone:touchstone.spfile}}\pysiglinewithargsret{\sphinxbfcode{\sphinxupquote{class }}\sphinxcode{\sphinxupquote{touchstone.}}\sphinxbfcode{\sphinxupquote{spfile}}}{\emph{\DUrole{n}{dosya}\DUrole{o}{=}\DUrole{default_value}{\textquotesingle{}\textquotesingle{}}}, \emph{\DUrole{n}{freqs}\DUrole{o}{=}\DUrole{default_value}{None}}, \emph{\DUrole{n}{portsayisi}\DUrole{o}{=}\DUrole{default_value}{1}}, \emph{\DUrole{n}{satiratla}\DUrole{o}{=}\DUrole{default_value}{0}}}{}
Bases: \sphinxcode{\sphinxupquote{object}}

Class to process Touchstone files.
\begin{quote}

odo
\end{quote}

TODO:
\index{Extraction() (touchstone.spfile method)@\spxentry{Extraction()}\spxextra{touchstone.spfile method}}

\begin{fulllineitems}
\phantomsection\label{\detokenize{touchstone:touchstone.spfile.Extraction}}\pysiglinewithargsret{\sphinxbfcode{\sphinxupquote{Extraction}}}{\emph{\DUrole{n}{measspfile}}}{}
Extract die S\sphinxhyphen{}Parameters using measurement data and simulated S\sphinxhyphen{}Parameters
Port ordering in \sphinxstyleemphasis{measspfile} is assumed to be the same as this \sphinxstyleemphasis{spfile}.
Remaining ports are ports of block to be extracted.
See “Extracting multiport S\sphinxhyphen{}Parameters of chip” in technical document.
\begin{quote}\begin{description}
\item[{Parameters}] \leavevmode
\sphinxstyleliteralstrong{\sphinxupquote{measspfile}} ({\hyperref[\detokenize{touchstone:touchstone.spfile}]{\sphinxcrossref{\sphinxstyleliteralemphasis{\sphinxupquote{spfile}}}}}) \textendash{} \sphinxstyleemphasis{SPFILE} object of measured S\sphinxhyphen{}Parameters of first k ports

\item[{Returns}] \leavevmode
\sphinxstyleemphasis{SPFILE} object of die’s S\sphinxhyphen{}Parameters

\item[{Return type}] \leavevmode
{\hyperref[\detokenize{touchstone:touchstone.spfile}]{\sphinxcrossref{spfile}}}

\end{description}\end{quote}

\end{fulllineitems}

\index{Ffunc() (touchstone.spfile method)@\spxentry{Ffunc()}\spxextra{touchstone.spfile method}}

\begin{fulllineitems}
\phantomsection\label{\detokenize{touchstone:touchstone.spfile.Ffunc}}\pysiglinewithargsret{\sphinxbfcode{\sphinxupquote{Ffunc}}}{\emph{\DUrole{n}{imp}}}{}
Calculates F\sphinxhyphen{}matrix in a, b definition of S\sphinxhyphen{}Parameters. For internal use of the library.
\begin{quote}
\begin{align*}\!\begin{aligned}
a=F(V+Z_rI)\\
b=F(V-Z_r^*I)\\
\end{aligned}\end{align*}\end{quote}
\begin{quote}\begin{description}
\item[{Parameters}] \leavevmode
\sphinxstyleliteralstrong{\sphinxupquote{imp}} (\sphinxstyleliteralemphasis{\sphinxupquote{ndarray}}) \textendash{} Zref, Reference impedance array for which includes the reference impedance for each port.

\item[{Returns}] \leavevmode
F\sphinxhyphen{}Matrix

\item[{Return type}] \leavevmode
numpy.matrix

\end{description}\end{quote}

\end{fulllineitems}

\index{ImpulseResponse() (touchstone.spfile method)@\spxentry{ImpulseResponse()}\spxextra{touchstone.spfile method}}

\begin{fulllineitems}
\phantomsection\label{\detokenize{touchstone:touchstone.spfile.ImpulseResponse}}\pysiglinewithargsret{\sphinxbfcode{\sphinxupquote{ImpulseResponse}}}{\emph{\DUrole{n}{i}\DUrole{o}{=}\DUrole{default_value}{2}}, \emph{\DUrole{n}{j}\DUrole{o}{=}\DUrole{default_value}{1}}, \emph{\DUrole{n}{dcinterp}\DUrole{o}{=}\DUrole{default_value}{1}}, \emph{\DUrole{n}{dcvalue}\DUrole{o}{=}\DUrole{default_value}{0.0}}, \emph{\DUrole{n}{MaxTimeStep}\DUrole{o}{=}\DUrole{default_value}{1.0}}, \emph{\DUrole{n}{FreqResCoef}\DUrole{o}{=}\DUrole{default_value}{1.0}}, \emph{\DUrole{n}{Window}\DUrole{o}{=}\DUrole{default_value}{\textquotesingle{}blackman\textquotesingle{}}}}{}
Calculates impulse response of \(S_{i j}\)
\begin{quote}\begin{description}
\item[{Parameters}] \leavevmode\begin{itemize}
\item {} 
\sphinxstyleliteralstrong{\sphinxupquote{i}} (\sphinxstyleliteralemphasis{\sphinxupquote{int}}\sphinxstyleliteralemphasis{\sphinxupquote{, }}\sphinxstyleliteralemphasis{\sphinxupquote{optional}}) \textendash{} Port\sphinxhyphen{}1. Defaults to 2.

\item {} 
\sphinxstyleliteralstrong{\sphinxupquote{j}} (\sphinxstyleliteralemphasis{\sphinxupquote{int}}\sphinxstyleliteralemphasis{\sphinxupquote{, }}\sphinxstyleliteralemphasis{\sphinxupquote{optional}}) \textendash{} Port\sphinxhyphen{}2. Defaults to 1.

\item {} 
\sphinxstyleliteralstrong{\sphinxupquote{dcinterp}} (\sphinxstyleliteralemphasis{\sphinxupquote{int}}\sphinxstyleliteralemphasis{\sphinxupquote{, }}\sphinxstyleliteralemphasis{\sphinxupquote{optional}}) \textendash{} If 1, add DC point to interpolation. Defaults to 1.

\item {} 
\sphinxstyleliteralstrong{\sphinxupquote{dcvalue}} (\sphinxstyleliteralemphasis{\sphinxupquote{float}}\sphinxstyleliteralemphasis{\sphinxupquote{, }}\sphinxstyleliteralemphasis{\sphinxupquote{optional}}) \textendash{} dcvalue to be used at interpolation if \sphinxstyleemphasis{dcinterp=0}. Defaults to 0.0. This value is appended to \(S_{i j}\) and the rest is left to interpolation in \sphinxstyleemphasis{data\_array} function.

\item {} 
\sphinxstyleliteralstrong{\sphinxupquote{MaxTimeStep}} (\sphinxstyleliteralemphasis{\sphinxupquote{float}}\sphinxstyleliteralemphasis{\sphinxupquote{, }}\sphinxstyleliteralemphasis{\sphinxupquote{optional}}) \textendash{} Not used for now. Defaults to 1.0.

\item {} 
\sphinxstyleliteralstrong{\sphinxupquote{FreqResCoef}} (\sphinxstyleliteralemphasis{\sphinxupquote{float}}\sphinxstyleliteralemphasis{\sphinxupquote{, }}\sphinxstyleliteralemphasis{\sphinxupquote{optional}}) \textendash{} Coeeficient to increase the frequency resolution by interpolation. Defaults to 1.0 (no interpolation).

\item {} 
\sphinxstyleliteralstrong{\sphinxupquote{Window}} (\sphinxstyleliteralemphasis{\sphinxupquote{str}}\sphinxstyleliteralemphasis{\sphinxupquote{, }}\sphinxstyleliteralemphasis{\sphinxupquote{optional}}) \textendash{} Windows function to prevent ringing. Defaults to “blackman”. Other windows will be added later.

\end{itemize}

\item[{Returns}] \leavevmode
\begin{description}
\item[{The elements of the tuple are the following in order:}] \leavevmode\begin{enumerate}
\sphinxsetlistlabels{\arabic}{enumi}{enumii}{}{.}%
\item {} 
Raw frequency data used as input

\item {} 
Window array

\item {} 
Time array

\item {} 
Time\sphinxhyphen{}Domain Waveform of Impulse Response

\item {} 
Time\sphinxhyphen{}Domain Waveform of Impulse Input

\item {} 
Time step

\item {} 
Frequency step

\item {} 
Size of input array

\item {} 
Max Value of Impulse Input

\end{enumerate}

\end{description}


\item[{Return type}] \leavevmode
9\sphinxhyphen{}tuple

\end{description}\end{quote}

\end{fulllineitems}

\index{S() (touchstone.spfile method)@\spxentry{S()}\spxextra{touchstone.spfile method}}

\begin{fulllineitems}
\phantomsection\label{\detokenize{touchstone:touchstone.spfile.S}}\pysiglinewithargsret{\sphinxbfcode{\sphinxupquote{S}}}{\emph{\DUrole{n}{i}\DUrole{o}{=}\DUrole{default_value}{1}}, \emph{\DUrole{n}{j}\DUrole{o}{=}\DUrole{default_value}{1}}, \emph{\DUrole{n}{dataformat}\DUrole{o}{=}\DUrole{default_value}{\textquotesingle{}COMPLEX\textquotesingle{}}}, \emph{\DUrole{n}{freqs}\DUrole{o}{=}\DUrole{default_value}{None}}}{}
Return \(S_{i j}\) in format \sphinxstyleemphasis{dataformat}
Uses \sphinxstyleemphasis{data\_array} method internally. A convenience function for practical use.
\begin{quote}\begin{description}
\item[{Parameters}] \leavevmode\begin{itemize}
\item {} 
\sphinxstyleliteralstrong{\sphinxupquote{i}} (\sphinxstyleliteralemphasis{\sphinxupquote{int}}\sphinxstyleliteralemphasis{\sphinxupquote{, }}\sphinxstyleliteralemphasis{\sphinxupquote{optional}}) \textendash{} Port\sphinxhyphen{}1. Defaults to 1.

\item {} 
\sphinxstyleliteralstrong{\sphinxupquote{j}} (\sphinxstyleliteralemphasis{\sphinxupquote{int}}\sphinxstyleliteralemphasis{\sphinxupquote{, }}\sphinxstyleliteralemphasis{\sphinxupquote{optional}}) \textendash{} Port\sphinxhyphen{}2. Defaults to 1.

\item {} 
\sphinxstyleliteralstrong{\sphinxupquote{dataformat}} (\sphinxstyleliteralemphasis{\sphinxupquote{str}}\sphinxstyleliteralemphasis{\sphinxupquote{, }}\sphinxstyleliteralemphasis{\sphinxupquote{optional}}) \textendash{} See \sphinxstyleemphasis{dataformat} parameter of \sphinxstyleemphasis{data\_array} method. Defaults to “COMPLEX”.

\end{itemize}

\item[{Returns}] \leavevmode
\(S_{i j}\) as \sphinxstyleemphasis{dataformat}

\item[{Return type}] \leavevmode
numpy.array

\end{description}\end{quote}

\end{fulllineitems}

\index{T() (touchstone.spfile method)@\spxentry{T()}\spxextra{touchstone.spfile method}}

\begin{fulllineitems}
\phantomsection\label{\detokenize{touchstone:touchstone.spfile.T}}\pysiglinewithargsret{\sphinxbfcode{\sphinxupquote{T}}}{\emph{\DUrole{n}{i}\DUrole{o}{=}\DUrole{default_value}{1}}, \emph{\DUrole{n}{j}\DUrole{o}{=}\DUrole{default_value}{1}}, \emph{\DUrole{n}{dataformat}\DUrole{o}{=}\DUrole{default_value}{\textquotesingle{}COMPLEX\textquotesingle{}}}, \emph{\DUrole{n}{freqs}\DUrole{o}{=}\DUrole{default_value}{None}}}{}
Return \(T_{i j}\) in format \sphinxstyleemphasis{dataformat}
Uses \sphinxstyleemphasis{data\_array} method internally. A convenience function for practical use.
\begin{quote}\begin{description}
\item[{Parameters}] \leavevmode\begin{itemize}
\item {} 
\sphinxstyleliteralstrong{\sphinxupquote{i}} (\sphinxstyleliteralemphasis{\sphinxupquote{int}}\sphinxstyleliteralemphasis{\sphinxupquote{, }}\sphinxstyleliteralemphasis{\sphinxupquote{optional}}) \textendash{} Port\sphinxhyphen{}1. Defaults to 1.

\item {} 
\sphinxstyleliteralstrong{\sphinxupquote{j}} (\sphinxstyleliteralemphasis{\sphinxupquote{int}}\sphinxstyleliteralemphasis{\sphinxupquote{, }}\sphinxstyleliteralemphasis{\sphinxupquote{optional}}) \textendash{} Port\sphinxhyphen{}2. Defaults to 1.

\item {} 
\sphinxstyleliteralstrong{\sphinxupquote{dataformat}} (\sphinxstyleliteralemphasis{\sphinxupquote{str}}\sphinxstyleliteralemphasis{\sphinxupquote{, }}\sphinxstyleliteralemphasis{\sphinxupquote{optional}}) \textendash{} See \sphinxstyleemphasis{dataformat} parameter of \sphinxstyleemphasis{data\_array} method. Defaults to “COMPLEX”.

\end{itemize}

\item[{Returns}] \leavevmode
\(T_{i j}\) as \sphinxstyleemphasis{dataformat}

\item[{Return type}] \leavevmode
numpy.array

\end{description}\end{quote}

\end{fulllineitems}

\index{UniformDeembed() (touchstone.spfile method)@\spxentry{UniformDeembed()}\spxextra{touchstone.spfile method}}

\begin{fulllineitems}
\phantomsection\label{\detokenize{touchstone:touchstone.spfile.UniformDeembed}}\pysiglinewithargsret{\sphinxbfcode{\sphinxupquote{UniformDeembed}}}{\emph{\DUrole{n}{phase}}, \emph{\DUrole{n}{delay}\DUrole{o}{=}\DUrole{default_value}{False}}, \emph{\DUrole{n}{deg}\DUrole{o}{=}\DUrole{default_value}{True}}, \emph{\DUrole{n}{inplace}\DUrole{o}{=}\DUrole{default_value}{\sphinxhyphen{} 1}}}{}
This function deembeds a phase length from all ports of S\sphinxhyphen{}Parameters.A positive phase means deembedding into the circuit.
The Zc of de\sphinxhyphen{}embedding lines is the reference impedances of each port.
\begin{quote}\begin{description}
\item[{Parameters}] \leavevmode\begin{itemize}
\item {} 
\sphinxstyleliteralstrong{\sphinxupquote{phase}} (\sphinxstyleliteralemphasis{\sphinxupquote{float}}\sphinxstyleliteralemphasis{\sphinxupquote{ or }}\sphinxstyleliteralemphasis{\sphinxupquote{list}}) \textendash{} Phase or delay to be deembedded.
\sphinxhyphen{} If a number is given, it is used for all frequencies and ports
\sphinxhyphen{} If a list is given, its size should be equal to the number of frequencies. If an element of list is number, it is used for all ports. If an element of the list is also a list, the elements size should be same as the number of ports.

\item {} 
\sphinxstyleliteralstrong{\sphinxupquote{delay}} (\sphinxstyleliteralemphasis{\sphinxupquote{bool}}\sphinxstyleliteralemphasis{\sphinxupquote{, }}\sphinxstyleliteralemphasis{\sphinxupquote{optional}}) \textendash{} If True, \sphinxstyleemphasis{phase} is assumed to be time delay, phase otherwise. Defaults to False.

\item {} 
\sphinxstyleliteralstrong{\sphinxupquote{deg}} (\sphinxstyleliteralemphasis{\sphinxupquote{bool}}\sphinxstyleliteralemphasis{\sphinxupquote{, }}\sphinxstyleliteralemphasis{\sphinxupquote{optional}}) \textendash{} if True, \sphinxstyleemphasis{phase} is assumed to be in radians if \sphinxstyleemphasis{delay} is False. Defaults to 0.

\item {} 
\sphinxstyleliteralstrong{\sphinxupquote{inplace}} (\sphinxstyleliteralemphasis{\sphinxupquote{int}}\sphinxstyleliteralemphasis{\sphinxupquote{, }}\sphinxstyleliteralemphasis{\sphinxupquote{optional}}) \textendash{} Object editing mode. Defaults to \sphinxhyphen{}1.

\end{itemize}

\item[{Returns}] \leavevmode
De\sphinxhyphen{}embedded spfile

\item[{Return type}] \leavevmode
{\hyperref[\detokenize{touchstone:touchstone.spfile}]{\sphinxcrossref{spfile}}}

\end{description}\end{quote}

\end{fulllineitems}

\index{Y() (touchstone.spfile method)@\spxentry{Y()}\spxextra{touchstone.spfile method}}

\begin{fulllineitems}
\phantomsection\label{\detokenize{touchstone:touchstone.spfile.Y}}\pysiglinewithargsret{\sphinxbfcode{\sphinxupquote{Y}}}{\emph{\DUrole{n}{i}\DUrole{o}{=}\DUrole{default_value}{1}}, \emph{\DUrole{n}{j}\DUrole{o}{=}\DUrole{default_value}{1}}, \emph{\DUrole{n}{dataformat}\DUrole{o}{=}\DUrole{default_value}{\textquotesingle{}COMPLEX\textquotesingle{}}}, \emph{\DUrole{n}{freqs}\DUrole{o}{=}\DUrole{default_value}{None}}}{}
Return \(Y_{i j}\) in format \sphinxstyleemphasis{dataformat}
Uses \sphinxstyleemphasis{data\_array} method internally. A convenience function for practical use.
\begin{quote}\begin{description}
\item[{Parameters}] \leavevmode\begin{itemize}
\item {} 
\sphinxstyleliteralstrong{\sphinxupquote{i}} (\sphinxstyleliteralemphasis{\sphinxupquote{int}}\sphinxstyleliteralemphasis{\sphinxupquote{, }}\sphinxstyleliteralemphasis{\sphinxupquote{optional}}) \textendash{} Port\sphinxhyphen{}1. Defaults to 1.

\item {} 
\sphinxstyleliteralstrong{\sphinxupquote{j}} (\sphinxstyleliteralemphasis{\sphinxupquote{int}}\sphinxstyleliteralemphasis{\sphinxupquote{, }}\sphinxstyleliteralemphasis{\sphinxupquote{optional}}) \textendash{} Port\sphinxhyphen{}2. Defaults to 1.

\item {} 
\sphinxstyleliteralstrong{\sphinxupquote{dataformat}} (\sphinxstyleliteralemphasis{\sphinxupquote{str}}\sphinxstyleliteralemphasis{\sphinxupquote{, }}\sphinxstyleliteralemphasis{\sphinxupquote{optional}}) \textendash{} See \sphinxstyleemphasis{dataformat} parameter of \sphinxstyleemphasis{data\_array} method. Defaults to “COMPLEX”.

\end{itemize}

\item[{Returns}] \leavevmode
\(Y_{i j}\) as \sphinxstyleemphasis{dataformat}

\item[{Return type}] \leavevmode
numpy.array

\end{description}\end{quote}

\end{fulllineitems}

\index{Z() (touchstone.spfile method)@\spxentry{Z()}\spxextra{touchstone.spfile method}}

\begin{fulllineitems}
\phantomsection\label{\detokenize{touchstone:touchstone.spfile.Z}}\pysiglinewithargsret{\sphinxbfcode{\sphinxupquote{Z}}}{\emph{\DUrole{n}{i}\DUrole{o}{=}\DUrole{default_value}{1}}, \emph{\DUrole{n}{j}\DUrole{o}{=}\DUrole{default_value}{1}}, \emph{\DUrole{n}{dataformat}\DUrole{o}{=}\DUrole{default_value}{\textquotesingle{}COMPLEX\textquotesingle{}}}, \emph{\DUrole{n}{freqs}\DUrole{o}{=}\DUrole{default_value}{None}}}{}
Return \(Z_{i j}\) in format \sphinxstyleemphasis{dataformat}
Uses \sphinxstyleemphasis{data\_array} method internally. A convenience function for practical use.
\begin{quote}\begin{description}
\item[{Parameters}] \leavevmode\begin{itemize}
\item {} 
\sphinxstyleliteralstrong{\sphinxupquote{i}} (\sphinxstyleliteralemphasis{\sphinxupquote{int}}\sphinxstyleliteralemphasis{\sphinxupquote{, }}\sphinxstyleliteralemphasis{\sphinxupquote{optional}}) \textendash{} Port\sphinxhyphen{}1. Defaults to 1.

\item {} 
\sphinxstyleliteralstrong{\sphinxupquote{j}} (\sphinxstyleliteralemphasis{\sphinxupquote{int}}\sphinxstyleliteralemphasis{\sphinxupquote{, }}\sphinxstyleliteralemphasis{\sphinxupquote{optional}}) \textendash{} Port\sphinxhyphen{}2. Defaults to 1.

\item {} 
\sphinxstyleliteralstrong{\sphinxupquote{dataformat}} (\sphinxstyleliteralemphasis{\sphinxupquote{str}}\sphinxstyleliteralemphasis{\sphinxupquote{, }}\sphinxstyleliteralemphasis{\sphinxupquote{optional}}) \textendash{} See \sphinxstyleemphasis{dataformat} parameter of \sphinxstyleemphasis{data\_array} method. Defaults to “COMPLEX”.

\end{itemize}

\item[{Returns}] \leavevmode
\(Z_{i j}\) as \sphinxstyleemphasis{dataformat}

\item[{Return type}] \leavevmode
numpy.array

\end{description}\end{quote}

\end{fulllineitems}

\index{Z\_conjmatch() (touchstone.spfile method)@\spxentry{Z\_conjmatch()}\spxextra{touchstone.spfile method}}

\begin{fulllineitems}
\phantomsection\label{\detokenize{touchstone:touchstone.spfile.Z_conjmatch}}\pysiglinewithargsret{\sphinxbfcode{\sphinxupquote{Z\_conjmatch}}}{\emph{\DUrole{n}{port1}\DUrole{o}{=}\DUrole{default_value}{1}}, \emph{\DUrole{n}{port2}\DUrole{o}{=}\DUrole{default_value}{2}}}{}
Calculates source and load reflection coefficients for simultaneous conjugate match.
\begin{quote}\begin{description}
\item[{Parameters}] \leavevmode\begin{itemize}
\item {} 
\sphinxstyleliteralstrong{\sphinxupquote{port1}} (\sphinxstyleliteralemphasis{\sphinxupquote{int}}\sphinxstyleliteralemphasis{\sphinxupquote{, }}\sphinxstyleliteralemphasis{\sphinxupquote{optional}}) \textendash{} {[}description{]}. Defaults to 1.

\item {} 
\sphinxstyleliteralstrong{\sphinxupquote{port2}} (\sphinxstyleliteralemphasis{\sphinxupquote{int}}\sphinxstyleliteralemphasis{\sphinxupquote{, }}\sphinxstyleliteralemphasis{\sphinxupquote{optional}}) \textendash{} {[}description{]}. Defaults to 2.

\end{itemize}

\item[{Returns}] \leavevmode
\begin{itemize}
\item {} 
GS: Reflection coefficient at Port\sphinxhyphen{}1

\item {} 
GL: Reflection coefficient at Port\sphinxhyphen{}2

\end{itemize}


\item[{Return type}] \leavevmode
2\sphinxhyphen{}tuple of numpy.arrays (GS, GL)

\end{description}\end{quote}

\end{fulllineitems}

\index{\_\_add\_\_() (touchstone.spfile method)@\spxentry{\_\_add\_\_()}\spxextra{touchstone.spfile method}}

\begin{fulllineitems}
\phantomsection\label{\detokenize{touchstone:touchstone.spfile.__add__}}\pysiglinewithargsret{\sphinxbfcode{\sphinxupquote{\_\_add\_\_}}}{\emph{\DUrole{n}{SP2}}}{}
Implements SP1+SP2.
Cascades SP2 to port\sphinxhyphen{}2 of SP1.
Port ordering is as follows:
(1)\sphinxhyphen{}SP1\sphinxhyphen{}(2)—(1)\sphinxhyphen{}SP2\sphinxhyphen{}(2)
SP1 is \sphinxstyleemphasis{self}.
\begin{quote}\begin{description}
\item[{Parameters}] \leavevmode
\sphinxstyleliteralstrong{\sphinxupquote{SP2}} ({\hyperref[\detokenize{touchstone:touchstone.spfile}]{\sphinxcrossref{\sphinxstyleliteralemphasis{\sphinxupquote{spfile}}}}}) \textendash{} Appended spfile network

\item[{Returns}] \leavevmode
The result of cascade of 2 networks

\item[{Return type}] \leavevmode
{\hyperref[\detokenize{touchstone:touchstone.spfile}]{\sphinxcrossref{spfile}}}

\end{description}\end{quote}

\end{fulllineitems}

\index{\_\_neg\_\_() (touchstone.spfile method)@\spxentry{\_\_neg\_\_()}\spxextra{touchstone.spfile method}}

\begin{fulllineitems}
\phantomsection\label{\detokenize{touchstone:touchstone.spfile.__neg__}}\pysiglinewithargsret{\sphinxbfcode{\sphinxupquote{\_\_neg\_\_}}}{}{}
Calculates an spfile object for two\sphinxhyphen{}port networks which is the inverse of this network. This is used to use + and \sphinxhyphen{} signs to cascade or deembed 2\sphinxhyphen{}port blocks.
\begin{quote}\begin{description}
\item[{Returns}] \leavevmode
\begin{enumerate}
\sphinxsetlistlabels{\arabic}{enumi}{enumii}{}{.}%
\item {} 
\sphinxstyleemphasis{None} if number of ports is not 2.

\item {} 
\sphinxstyleemphasis{spfile} which is the inverse of the spfile object operated on.

\end{enumerate}


\item[{Return type}] \leavevmode
{\hyperref[\detokenize{touchstone:touchstone.spfile}]{\sphinxcrossref{spfile}}}

\end{description}\end{quote}

\end{fulllineitems}

\index{\_\_sub\_\_() (touchstone.spfile method)@\spxentry{\_\_sub\_\_()}\spxextra{touchstone.spfile method}}

\begin{fulllineitems}
\phantomsection\label{\detokenize{touchstone:touchstone.spfile.__sub__}}\pysiglinewithargsret{\sphinxbfcode{\sphinxupquote{\_\_sub\_\_}}}{\emph{\DUrole{n}{SP2}}}{}
Implements SP1\sphinxhyphen{}SP2.
Deembeds SP2 from port\sphinxhyphen{}2 of SP1.
Port ordering is as follows:
(1)\sphinxhyphen{}SP1\sphinxhyphen{}(2)—(1)\sphinxhyphen{}SP2\sphinxhyphen{}(2)
SP1 is \sphinxstyleemphasis{self}.
\begin{quote}\begin{description}
\item[{Parameters}] \leavevmode
\sphinxstyleliteralstrong{\sphinxupquote{SP2}} ({\hyperref[\detokenize{touchstone:touchstone.spfile}]{\sphinxcrossref{\sphinxstyleliteralemphasis{\sphinxupquote{spfile}}}}}) \textendash{} Deembedded spfile network

\item[{Returns}] \leavevmode
The resulting of deembedding process

\item[{Return type}] \leavevmode
{\hyperref[\detokenize{touchstone:touchstone.spfile}]{\sphinxcrossref{spfile}}}

\end{description}\end{quote}

\end{fulllineitems}

\index{add\_abs\_noise() (touchstone.spfile method)@\spxentry{add\_abs\_noise()}\spxextra{touchstone.spfile method}}

\begin{fulllineitems}
\phantomsection\label{\detokenize{touchstone:touchstone.spfile.add_abs_noise}}\pysiglinewithargsret{\sphinxbfcode{\sphinxupquote{add\_abs\_noise}}}{\emph{\DUrole{n}{dbnoise}\DUrole{o}{=}\DUrole{default_value}{0.1}}, \emph{\DUrole{n}{phasenoise}\DUrole{o}{=}\DUrole{default_value}{0.1}}, \emph{\DUrole{n}{inplace}\DUrole{o}{=}\DUrole{default_value}{\sphinxhyphen{} 1}}}{}
This method adds random amplitude and phase noise to the s\sphinxhyphen{}parameter data.
Mean value for both noises are 0.
\begin{quote}\begin{description}
\item[{Parameters}] \leavevmode\begin{itemize}
\item {} 
\sphinxstyleliteralstrong{\sphinxupquote{dbnoise}} (\sphinxstyleliteralemphasis{\sphinxupquote{float}}\sphinxstyleliteralemphasis{\sphinxupquote{, }}\sphinxstyleliteralemphasis{\sphinxupquote{optional}}) \textendash{} Standard deviation of amplitude noise in dB. Defaults to 0.1.

\item {} 
\sphinxstyleliteralstrong{\sphinxupquote{phasenoise}} (\sphinxstyleliteralemphasis{\sphinxupquote{float}}\sphinxstyleliteralemphasis{\sphinxupquote{, }}\sphinxstyleliteralemphasis{\sphinxupquote{optional}}) \textendash{} Standard deviation of phase noise in degrees. Defaults to 0.1.

\item {} 
\sphinxstyleliteralstrong{\sphinxupquote{inplace}} (\sphinxstyleliteralemphasis{\sphinxupquote{int}}\sphinxstyleliteralemphasis{\sphinxupquote{, }}\sphinxstyleliteralemphasis{\sphinxupquote{optional}}) \textendash{} object editing mode. Defaults to \sphinxhyphen{}1.

\end{itemize}

\item[{Returns}] \leavevmode
object with noisy data

\item[{Return type}] \leavevmode
{\hyperref[\detokenize{touchstone:touchstone.spfile}]{\sphinxcrossref{spfile}}}

\end{description}\end{quote}

\end{fulllineitems}

\index{calc\_syz() (touchstone.spfile method)@\spxentry{calc\_syz()}\spxextra{touchstone.spfile method}}

\begin{fulllineitems}
\phantomsection\label{\detokenize{touchstone:touchstone.spfile.calc_syz}}\pysiglinewithargsret{\sphinxbfcode{\sphinxupquote{calc\_syz}}}{\emph{\DUrole{n}{input}\DUrole{o}{=}\DUrole{default_value}{\textquotesingle{}S\textquotesingle{}}}, \emph{\DUrole{n}{indices}\DUrole{o}{=}\DUrole{default_value}{None}}}{}
This function calculates 2 of S, Y and Z parameters by the remaining parameter given.
Y and Z\sphinxhyphen{}matrices calculated separately instead of calculating one and taking inverse. Because one of them may be undefined for some circuits.
\begin{quote}\begin{description}
\item[{Parameters}] \leavevmode\begin{itemize}
\item {} 
\sphinxstyleliteralstrong{\sphinxupquote{input}} (\sphinxstyleliteralemphasis{\sphinxupquote{str}}\sphinxstyleliteralemphasis{\sphinxupquote{, }}\sphinxstyleliteralemphasis{\sphinxupquote{optional}}) \textendash{} Input parameter type (should be S, Y or Z). Defaults to “S”.

\item {} 
\sphinxstyleliteralstrong{\sphinxupquote{indices}} (\sphinxstyleliteralemphasis{\sphinxupquote{list}}\sphinxstyleliteralemphasis{\sphinxupquote{, }}\sphinxstyleliteralemphasis{\sphinxupquote{optional}}) \textendash{} If given, output matrices are calculated only at the indices given by this list. If it is None, then output matrices are calculated at all frequencies. Defaults to None.

\end{itemize}

\end{description}\end{quote}

\end{fulllineitems}

\index{calc\_t\_eigs() (touchstone.spfile method)@\spxentry{calc\_t\_eigs()}\spxextra{touchstone.spfile method}}

\begin{fulllineitems}
\phantomsection\label{\detokenize{touchstone:touchstone.spfile.calc_t_eigs}}\pysiglinewithargsret{\sphinxbfcode{\sphinxupquote{calc\_t\_eigs}}}{\emph{\DUrole{n}{port1}\DUrole{o}{=}\DUrole{default_value}{1}}, \emph{\DUrole{n}{port2}\DUrole{o}{=}\DUrole{default_value}{2}}}{}
Eigenfunctions and Eigenvector of T\sphinxhyphen{}Matrix is calculated.
Only power\sphinxhyphen{}wave formulation is implemented

\end{fulllineitems}

\index{change\_formulation() (touchstone.spfile method)@\spxentry{change\_formulation()}\spxextra{touchstone.spfile method}}

\begin{fulllineitems}
\phantomsection\label{\detokenize{touchstone:touchstone.spfile.change_formulation}}\pysiglinewithargsret{\sphinxbfcode{\sphinxupquote{change\_formulation}}}{\emph{\DUrole{n}{formulation}}}{}
\end{fulllineitems}

\index{change\_ref\_impedance() (touchstone.spfile method)@\spxentry{change\_ref\_impedance()}\spxextra{touchstone.spfile method}}

\begin{fulllineitems}
\phantomsection\label{\detokenize{touchstone:touchstone.spfile.change_ref_impedance}}\pysiglinewithargsret{\sphinxbfcode{\sphinxupquote{change\_ref\_impedance}}}{\emph{\DUrole{n}{Znew}}, \emph{\DUrole{n}{inplace}\DUrole{o}{=}\DUrole{default_value}{\sphinxhyphen{} 1}}}{}
Changes reference impedance and re\sphinxhyphen{}calculates S\sphinxhyphen{}Parameters.
\begin{quote}\begin{description}
\item[{Parameters}] \leavevmode
\sphinxstyleliteralstrong{\sphinxupquote{Znew}} (\sphinxstyleliteralemphasis{\sphinxupquote{float}}\sphinxstyleliteralemphasis{\sphinxupquote{ or }}\sphinxstyleliteralemphasis{\sphinxupquote{list}}) \textendash{} New Reference Impedance. Its type can be:
\sphinxhyphen{} float: In this case Znew value is used for all ports
\sphinxhyphen{} list: In this case each element of this list is assgined to different ports in order as reference impedance. Length of \sphinxstyleemphasis{Znew} should be equal to number of ports. If an element of the list is None, then the reference impedance for corresponding port is not changed.

\item[{Returns}] \leavevmode
The spfile object with new reference impedance

\item[{Return type}] \leavevmode
{\hyperref[\detokenize{touchstone:touchstone.spfile}]{\sphinxcrossref{spfile}}}

\end{description}\end{quote}

\end{fulllineitems}

\index{check\_passivity() (touchstone.spfile method)@\spxentry{check\_passivity()}\spxextra{touchstone.spfile method}}

\begin{fulllineitems}
\phantomsection\label{\detokenize{touchstone:touchstone.spfile.check_passivity}}\pysiglinewithargsret{\sphinxbfcode{\sphinxupquote{check\_passivity}}}{}{}
This method determines the frequencies and frequency indices at which the network is not passive. Condition written in:
Fast Passivity Enforcement of S\sphinxhyphen{}Parameter Macromodels by Pole Perturbation.pdf
For a better discussion: “S\sphinxhyphen{}Parameter Quality Metrics (Yuriy Shlepnev)”
\begin{quote}\begin{description}
\item[{Returns}] \leavevmode
For non\sphinxhyphen{}passive frequencies (indices, frequencies, eigenvalues)

\item[{Return type}] \leavevmode
3\sphinxhyphen{}tuple of lists

\end{description}\end{quote}

\end{fulllineitems}

\index{column\_of\_data() (touchstone.spfile method)@\spxentry{column\_of\_data()}\spxextra{touchstone.spfile method}}

\begin{fulllineitems}
\phantomsection\label{\detokenize{touchstone:touchstone.spfile.column_of_data}}\pysiglinewithargsret{\sphinxbfcode{\sphinxupquote{column\_of\_data}}}{\emph{\DUrole{n}{i}}, \emph{\DUrole{n}{j}}}{}
Gets the indice of column at \sphinxstyleemphasis{sdata} matrix corresponding to \(S_{i j}\)
For internal use of the library.
\begin{quote}\begin{description}
\item[{Parameters}] \leavevmode\begin{itemize}
\item {} 
\sphinxstyleliteralstrong{\sphinxupquote{i}} (\sphinxstyleliteralemphasis{\sphinxupquote{int}}) \textendash{} First index

\item {} 
\sphinxstyleliteralstrong{\sphinxupquote{j}} (\sphinxstyleliteralemphasis{\sphinxupquote{int}}) \textendash{} Second index

\end{itemize}

\item[{Returns}] \leavevmode
Index of column

\item[{Return type}] \leavevmode
int

\end{description}\end{quote}

\end{fulllineitems}

\index{conj\_match\_uncoupled() (touchstone.spfile method)@\spxentry{conj\_match\_uncoupled()}\spxextra{touchstone.spfile method}}

\begin{fulllineitems}
\phantomsection\label{\detokenize{touchstone:touchstone.spfile.conj_match_uncoupled}}\pysiglinewithargsret{\sphinxbfcode{\sphinxupquote{conj\_match\_uncoupled}}}{\emph{\DUrole{n}{ports}\DUrole{o}{=}\DUrole{default_value}{{[}{]}}}, \emph{\DUrole{n}{inplace}\DUrole{o}{=}\DUrole{default_value}{\sphinxhyphen{} 1}}, \emph{\DUrole{n}{noofiters}\DUrole{o}{=}\DUrole{default_value}{50}}}{}
Sets the reference impedance for given ports as the complex conjugate of output impedance at each port.
The ports are assumed to be uncoupled. Coupling is taken care of by doing the same operation multiple times.
\begin{quote}\begin{description}
\item[{Parameters}] \leavevmode\begin{itemize}
\item {} 
\sphinxstyleliteralstrong{\sphinxupquote{ports}} (\sphinxstyleliteralemphasis{\sphinxupquote{list}}\sphinxstyleliteralemphasis{\sphinxupquote{,}}\sphinxstyleliteralemphasis{\sphinxupquote{optional}}) \textendash{} {[}description{]}. Defaults to all ports.

\item {} 
\sphinxstyleliteralstrong{\sphinxupquote{inplace}} (\sphinxstyleliteralemphasis{\sphinxupquote{int}}\sphinxstyleliteralemphasis{\sphinxupquote{, }}\sphinxstyleliteralemphasis{\sphinxupquote{optional}}) \textendash{} Object editing mode. Defaults to \sphinxhyphen{}1.

\item {} 
\sphinxstyleliteralstrong{\sphinxupquote{noofiters}} (\sphinxstyleliteralemphasis{\sphinxupquote{int}}\sphinxstyleliteralemphasis{\sphinxupquote{, }}\sphinxstyleliteralemphasis{\sphinxupquote{optional}}) \textendash{} Numberof iterations. Defaults to 50.

\end{itemize}

\item[{Returns}] \leavevmode
spfile object with new s\sphinxhyphen{}parameters

\end{description}\end{quote}

\end{fulllineitems}

\index{connect\_2\_ports() (touchstone.spfile method)@\spxentry{connect\_2\_ports()}\spxextra{touchstone.spfile method}}

\begin{fulllineitems}
\phantomsection\label{\detokenize{touchstone:touchstone.spfile.connect_2_ports}}\pysiglinewithargsret{\sphinxbfcode{\sphinxupquote{connect\_2\_ports}}}{\emph{\DUrole{n}{k}}, \emph{\DUrole{n}{m}}, \emph{\DUrole{n}{inplace}\DUrole{o}{=}\DUrole{default_value}{\sphinxhyphen{} 1}}}{}
Port\sphinxhyphen{}m is connected to port\sphinxhyphen{}k and both ports are removed.
Reference: QUCS technical.pdf, S\sphinxhyphen{}parameters in CAE programs, p.29
\begin{quote}\begin{description}
\item[{Parameters}] \leavevmode\begin{itemize}
\item {} 
\sphinxstyleliteralstrong{\sphinxupquote{k}} (\sphinxstyleliteralemphasis{\sphinxupquote{int}}) \textendash{} First port index to be connected.

\item {} 
\sphinxstyleliteralstrong{\sphinxupquote{m}} (\sphinxstyleliteralemphasis{\sphinxupquote{int}}) \textendash{} Second port index to be connected.

\item {} 
\sphinxstyleliteralstrong{\sphinxupquote{inplace}} (\sphinxstyleliteralemphasis{\sphinxupquote{int}}\sphinxstyleliteralemphasis{\sphinxupquote{, }}\sphinxstyleliteralemphasis{\sphinxupquote{optional}}) \textendash{} Object editing mode. Defaults to \sphinxhyphen{}1.

\end{itemize}

\item[{Returns}] \leavevmode
New spfile object

\item[{Return type}] \leavevmode
{\hyperref[\detokenize{touchstone:touchstone.spfile}]{\sphinxcrossref{spfile}}}

\end{description}\end{quote}

\end{fulllineitems}

\index{connect\_2\_ports\_list() (touchstone.spfile method)@\spxentry{connect\_2\_ports\_list()}\spxextra{touchstone.spfile method}}

\begin{fulllineitems}
\phantomsection\label{\detokenize{touchstone:touchstone.spfile.connect_2_ports_list}}\pysiglinewithargsret{\sphinxbfcode{\sphinxupquote{connect\_2\_ports\_list}}}{\emph{\DUrole{n}{conns}}, \emph{\DUrole{n}{inplace}\DUrole{o}{=}\DUrole{default_value}{\sphinxhyphen{} 1}}}{}
Short circuit ports together one\sphinxhyphen{}to\sphinxhyphen{}one. Short circuited ports are removed.
Ports that will be connected are given as tuples in list \sphinxstyleemphasis{conns}
i.e. conns={[}(p1,p2),(p3,p4),..{]}
The order of remaining ports is kept.
Reference: QUCS technical.pdf, S\sphinxhyphen{}parameters in CAE programs, p.29
\begin{quote}\begin{description}
\item[{Parameters}] \leavevmode\begin{itemize}
\item {} 
\sphinxstyleliteralstrong{\sphinxupquote{conns}} (\sphinxstyleliteralemphasis{\sphinxupquote{list of tuples}}) \textendash{} A list of 2\sphinxhyphen{}tuples of integers showing the ports connected

\item {} 
\sphinxstyleliteralstrong{\sphinxupquote{inplace}} (\sphinxstyleliteralemphasis{\sphinxupquote{int}}\sphinxstyleliteralemphasis{\sphinxupquote{, }}\sphinxstyleliteralemphasis{\sphinxupquote{optional}}) \textendash{} Object editing mode. Defaults to \sphinxhyphen{}1.

\end{itemize}

\item[{Returns}] \leavevmode
New spfile object

\item[{Return type}] \leavevmode
{\hyperref[\detokenize{touchstone:touchstone.spfile}]{\sphinxcrossref{spfile}}}

\end{description}\end{quote}

\end{fulllineitems}

\index{connect\_2\_ports\_retain() (touchstone.spfile method)@\spxentry{connect\_2\_ports\_retain()}\spxextra{touchstone.spfile method}}

\begin{fulllineitems}
\phantomsection\label{\detokenize{touchstone:touchstone.spfile.connect_2_ports_retain}}\pysiglinewithargsret{\sphinxbfcode{\sphinxupquote{connect\_2\_ports\_retain}}}{\emph{\DUrole{n}{k}}, \emph{\DUrole{n}{m}}, \emph{\DUrole{n}{inplace}\DUrole{o}{=}\DUrole{default_value}{\sphinxhyphen{} 1}}}{}
Port\sphinxhyphen{}m is connected to port\sphinxhyphen{}k and both ports are removed. New port becomes the last port of the circuit.
Reference: QUCS technical.pdf, S\sphinxhyphen{}parameters in CAE programs, p.29
\begin{quote}\begin{description}
\item[{Parameters}] \leavevmode\begin{itemize}
\item {} 
\sphinxstyleliteralstrong{\sphinxupquote{k}} (\sphinxstyleliteralemphasis{\sphinxupquote{int}}) \textendash{} First port index to be connected.

\item {} 
\sphinxstyleliteralstrong{\sphinxupquote{m}} (\sphinxstyleliteralemphasis{\sphinxupquote{int}}) \textendash{} Second port index to be connected.

\item {} 
\sphinxstyleliteralstrong{\sphinxupquote{inplace}} (\sphinxstyleliteralemphasis{\sphinxupquote{int}}\sphinxstyleliteralemphasis{\sphinxupquote{, }}\sphinxstyleliteralemphasis{\sphinxupquote{optional}}) \textendash{} Object editing mode. Defaults to \sphinxhyphen{}1.

\end{itemize}

\item[{Returns}] \leavevmode
New \sphinxstyleemphasis{spfile} object

\item[{Return type}] \leavevmode
{\hyperref[\detokenize{touchstone:touchstone.spfile}]{\sphinxcrossref{spfile}}}

\end{description}\end{quote}

\end{fulllineitems}

\index{connect\_network\_1\_conn() (touchstone.spfile method)@\spxentry{connect\_network\_1\_conn()}\spxextra{touchstone.spfile method}}

\begin{fulllineitems}
\phantomsection\label{\detokenize{touchstone:touchstone.spfile.connect_network_1_conn}}\pysiglinewithargsret{\sphinxbfcode{\sphinxupquote{connect\_network\_1\_conn}}}{\emph{\DUrole{n}{EX}}, \emph{\DUrole{n}{k}}, \emph{\DUrole{n}{m}}, \emph{\DUrole{n}{inplace}\DUrole{o}{=}\DUrole{default_value}{\sphinxhyphen{} 1}}, \emph{\DUrole{n}{preserveportnumbers1}\DUrole{o}{=}\DUrole{default_value}{False}}}{}
Port\sphinxhyphen{}m of EX circuit is connected to port\sphinxhyphen{}k of this circuit. Both of these ports will be removed.
Remaining ports of EX are added to the port list of this circuit in order.
Reference: QUCS technical.pdf, S\sphinxhyphen{}parameters in CAE programs, p.29
\begin{quote}\begin{description}
\item[{Parameters}] \leavevmode\begin{itemize}
\item {} 
\sphinxstyleliteralstrong{\sphinxupquote{EX}} ({\hyperref[\detokenize{touchstone:touchstone.spfile}]{\sphinxcrossref{\sphinxstyleliteralemphasis{\sphinxupquote{spfile}}}}}) \textendash{} External network to be connected to this.

\item {} 
\sphinxstyleliteralstrong{\sphinxupquote{k}} (\sphinxstyleliteralemphasis{\sphinxupquote{int}}) \textendash{} Port number of self to be connected.

\item {} 
\sphinxstyleliteralstrong{\sphinxupquote{m}} (\sphinxstyleliteralemphasis{\sphinxupquote{int}}) \textendash{} Port number of EX to be connected.

\item {} 
\sphinxstyleliteralstrong{\sphinxupquote{inplace}} (\sphinxstyleliteralemphasis{\sphinxupquote{int}}\sphinxstyleliteralemphasis{\sphinxupquote{, }}\sphinxstyleliteralemphasis{\sphinxupquote{optional}}) \textendash{} Object editing mode. Defaults to \sphinxhyphen{}1.

\item {} 
\sphinxstyleliteralstrong{\sphinxupquote{preserveportnumbers1}} (\sphinxstyleliteralemphasis{\sphinxupquote{bool}}\sphinxstyleliteralemphasis{\sphinxupquote{, }}\sphinxstyleliteralemphasis{\sphinxupquote{optional}}) \textendash{} if True, the number of the first added port will be k. Defaults to False.

\end{itemize}

\item[{Returns}] \leavevmode
Connected network

\item[{Return type}] \leavevmode
{\hyperref[\detokenize{touchstone:touchstone.spfile}]{\sphinxcrossref{spfile}}}

\end{description}\end{quote}

\end{fulllineitems}

\index{connect\_network\_1\_conn\_retain() (touchstone.spfile method)@\spxentry{connect\_network\_1\_conn\_retain()}\spxextra{touchstone.spfile method}}

\begin{fulllineitems}
\phantomsection\label{\detokenize{touchstone:touchstone.spfile.connect_network_1_conn_retain}}\pysiglinewithargsret{\sphinxbfcode{\sphinxupquote{connect\_network\_1\_conn\_retain}}}{\emph{\DUrole{n}{EX}}, \emph{\DUrole{n}{k}}, \emph{\DUrole{n}{m}}, \emph{\DUrole{n}{inplace}\DUrole{o}{=}\DUrole{default_value}{\sphinxhyphen{} 1}}}{}
Port\sphinxhyphen{}m of EX circuit is connected to port\sphinxhyphen{}k of this circuit. This connection point will also be a port.
Remaining ports of EX are added to the port list of this circuit in order.
Reference: QUCS technical.pdf, S\sphinxhyphen{}parameters in CAE programs, p.29
\begin{quote}\begin{description}
\item[{Parameters}] \leavevmode\begin{itemize}
\item {} 
\sphinxstyleliteralstrong{\sphinxupquote{EX}} ({\hyperref[\detokenize{touchstone:touchstone.spfile}]{\sphinxcrossref{\sphinxstyleliteralemphasis{\sphinxupquote{spfile}}}}}) \textendash{} External network to be connected to this.

\item {} 
\sphinxstyleliteralstrong{\sphinxupquote{k}} (\sphinxstyleliteralemphasis{\sphinxupquote{int}}) \textendash{} Port number of self to be connected.

\item {} 
\sphinxstyleliteralstrong{\sphinxupquote{m}} (\sphinxstyleliteralemphasis{\sphinxupquote{int}}) \textendash{} Port number of EX to be connected.

\item {} 
\sphinxstyleliteralstrong{\sphinxupquote{inplace}} (\sphinxstyleliteralemphasis{\sphinxupquote{int}}\sphinxstyleliteralemphasis{\sphinxupquote{, }}\sphinxstyleliteralemphasis{\sphinxupquote{optional}}) \textendash{} Object editing mode. Defaults to \sphinxhyphen{}1.

\item {} 
\sphinxstyleliteralstrong{\sphinxupquote{preserveportnumbers1}} (\sphinxstyleliteralemphasis{\sphinxupquote{bool}}\sphinxstyleliteralemphasis{\sphinxupquote{, }}\sphinxstyleliteralemphasis{\sphinxupquote{optional}}) \textendash{} if True, the number of the first added port will be k. Defaults to False.

\end{itemize}

\item[{Returns}] \leavevmode
Connected network

\item[{Return type}] \leavevmode
{\hyperref[\detokenize{touchstone:touchstone.spfile}]{\sphinxcrossref{spfile}}}

\end{description}\end{quote}

\end{fulllineitems}

\index{convert\_s1p\_to\_s2p() (touchstone.spfile method)@\spxentry{convert\_s1p\_to\_s2p()}\spxextra{touchstone.spfile method}}

\begin{fulllineitems}
\phantomsection\label{\detokenize{touchstone:touchstone.spfile.convert_s1p_to_s2p}}\pysiglinewithargsret{\sphinxbfcode{\sphinxupquote{convert\_s1p\_to\_s2p}}}{}{}
\end{fulllineitems}

\index{copy() (touchstone.spfile method)@\spxentry{copy()}\spxextra{touchstone.spfile method}}

\begin{fulllineitems}
\phantomsection\label{\detokenize{touchstone:touchstone.spfile.copy}}\pysiglinewithargsret{\sphinxbfcode{\sphinxupquote{copy}}}{}{}
\end{fulllineitems}

\index{copy\_data\_from\_spfile() (touchstone.spfile method)@\spxentry{copy\_data\_from\_spfile()}\spxextra{touchstone.spfile method}}

\begin{fulllineitems}
\phantomsection\label{\detokenize{touchstone:touchstone.spfile.copy_data_from_spfile}}\pysiglinewithargsret{\sphinxbfcode{\sphinxupquote{copy\_data\_from\_spfile}}}{\emph{\DUrole{n}{local\_i}}, \emph{\DUrole{n}{local\_j}}, \emph{\DUrole{n}{source\_i}}, \emph{\DUrole{n}{source\_j}}, \emph{\DUrole{n}{sourcespfile}}}{}
This method copies S\sphinxhyphen{}Parameter data from another SPFILE object

\end{fulllineitems}

\index{cpwgline() (touchstone.spfile class method)@\spxentry{cpwgline()}\spxextra{touchstone.spfile class method}}

\begin{fulllineitems}
\phantomsection\label{\detokenize{touchstone:touchstone.spfile.cpwgline}}\pysiglinewithargsret{\sphinxbfcode{\sphinxupquote{classmethod }}\sphinxbfcode{\sphinxupquote{cpwgline}}}{\emph{\DUrole{n}{length}}, \emph{\DUrole{n}{w}}, \emph{\DUrole{n}{th}}, \emph{\DUrole{n}{er}}, \emph{\DUrole{n}{s}}, \emph{\DUrole{n}{h}}, \emph{\DUrole{n}{freqs}\DUrole{o}{=}\DUrole{default_value}{None}}}{}
Create an \sphinxcode{\sphinxupquote{spfile}} object corresponding to a cpwg transmission line.
\begin{quote}\begin{description}
\item[{Parameters}] \leavevmode\begin{itemize}
\item {} 
\sphinxstyleliteralstrong{\sphinxupquote{length}} (\sphinxstyleliteralemphasis{\sphinxupquote{float}}) \textendash{} Length of cpwg line.

\item {} 
\sphinxstyleliteralstrong{\sphinxupquote{w}} (\sphinxstyleliteralemphasis{\sphinxupquote{float}}) \textendash{} Width of cpwg line.

\item {} 
\sphinxstyleliteralstrong{\sphinxupquote{th}} (\sphinxstyleliteralemphasis{\sphinxupquote{float}}) \textendash{} Thickness of metal.

\item {} 
\sphinxstyleliteralstrong{\sphinxupquote{er}} (\sphinxstyleliteralemphasis{\sphinxupquote{float}}) \textendash{} Relative permittivity of substrate.

\item {} 
\sphinxstyleliteralstrong{\sphinxupquote{s}} (\sphinxstyleliteralemphasis{\sphinxupquote{float}}) \textendash{} Gap of cpwg line.

\item {} 
\sphinxstyleliteralstrong{\sphinxupquote{h}} (\sphinxstyleliteralemphasis{\sphinxupquote{float}}) \textendash{} Thickness of substrate.

\item {} 
\sphinxstyleliteralstrong{\sphinxupquote{freqs}} (\sphinxstyleliteralemphasis{\sphinxupquote{float}}\sphinxstyleliteralemphasis{\sphinxupquote{, }}\sphinxstyleliteralemphasis{\sphinxupquote{optional}}) \textendash{} Frequency list of object. Defaults to None. If not given, frequencies should be set later.

\end{itemize}

\item[{Returns}] \leavevmode
An spfile object.

\item[{Return type}] \leavevmode
{\hyperref[\detokenize{touchstone:touchstone.spfile}]{\sphinxcrossref{spfile}}}

\end{description}\end{quote}

\end{fulllineitems}

\index{data\_array() (touchstone.spfile method)@\spxentry{data\_array()}\spxextra{touchstone.spfile method}}

\begin{fulllineitems}
\phantomsection\label{\detokenize{touchstone:touchstone.spfile.data_array}}\pysiglinewithargsret{\sphinxbfcode{\sphinxupquote{data\_array}}}{\emph{\DUrole{n}{dataformat}\DUrole{o}{=}\DUrole{default_value}{\textquotesingle{}DB\textquotesingle{}}}, \emph{\DUrole{n}{M}\DUrole{o}{=}\DUrole{default_value}{\textquotesingle{}S\textquotesingle{}}}, \emph{\DUrole{n}{i}\DUrole{o}{=}\DUrole{default_value}{1}}, \emph{\DUrole{n}{j}\DUrole{o}{=}\DUrole{default_value}{1}}, \emph{\DUrole{n}{frekanslar}\DUrole{o}{=}\DUrole{default_value}{None}}, \emph{\DUrole{n}{ref}\DUrole{o}{=}\DUrole{default_value}{None}}, \emph{\DUrole{n}{DCInt}\DUrole{o}{=}\DUrole{default_value}{0}}, \emph{\DUrole{n}{DCValue}\DUrole{o}{=}\DUrole{default_value}{0.0}}, \emph{\DUrole{n}{smoothing}\DUrole{o}{=}\DUrole{default_value}{0}}, \emph{\DUrole{n}{InterpolationConstant}\DUrole{o}{=}\DUrole{default_value}{0}}}{}
Return a network parameter between ports \sphinxstyleemphasis{i} and \sphinxstyleemphasis{j} (\(M_{i j}\)) at specified frequencies in specified format.
\begin{quote}\begin{description}
\item[{Parameters}] \leavevmode\begin{itemize}
\item {} 
\sphinxstyleliteralstrong{\sphinxupquote{dataformat}} (\sphinxstyleliteralemphasis{\sphinxupquote{str}}\sphinxstyleliteralemphasis{\sphinxupquote{, }}\sphinxstyleliteralemphasis{\sphinxupquote{optional}}) \textendash{} Defaults to “DB”. The format of the data returned. Possible values (case insensitive):
\sphinxhyphen{}   “K”: Stability factor of 2\sphinxhyphen{}port
\sphinxhyphen{}   “MU1”: Input stability factor of 2\sphinxhyphen{}port
\sphinxhyphen{}   “MU2”: Output stability factor of 2\sphinxhyphen{}port
\sphinxhyphen{}   “VSWR”: VSWR ar port i
\sphinxhyphen{}   “MAG”: Magnitude of \(M_{i j}\)
\sphinxhyphen{}   “DB”: Magnitude of \(M_{i j}\) in dB
\sphinxhyphen{}   “REAL”: Real part of \(M_{i j}\)
\sphinxhyphen{}   “IMAG”: Imaginary part of \(M_{i j}\)
\sphinxhyphen{}   “PHASE”: Phase of \(M_{i j}\) in degrees between 0\sphinxhyphen{}360
\sphinxhyphen{}   “UNWRAPPEDPHASE”: Unwrapped Phase of \(M_{i j}\) in degrees
\sphinxhyphen{}   “GROUPDELAY”: Group Delay of \(M_{i j}\) in degrees

\item {} 
\sphinxstyleliteralstrong{\sphinxupquote{M}} (\sphinxstyleliteralemphasis{\sphinxupquote{str}}\sphinxstyleliteralemphasis{\sphinxupquote{, }}\sphinxstyleliteralemphasis{\sphinxupquote{optional}}) \textendash{} Defaults to “S”. Possible values (case insensitive):
\sphinxhyphen{}   “S”: Return S\sphinxhyphen{}parameter data
\sphinxhyphen{}   “Y”: Return Y\sphinxhyphen{}parameter data
\sphinxhyphen{}   “Z”: Return Z\sphinxhyphen{}parameter data
\sphinxhyphen{}   “ABCD”: Return ABCD\sphinxhyphen{}parameter data

\item {} 
\sphinxstyleliteralstrong{\sphinxupquote{i}} (\sphinxstyleliteralemphasis{\sphinxupquote{int}}\sphinxstyleliteralemphasis{\sphinxupquote{, }}\sphinxstyleliteralemphasis{\sphinxupquote{optional}}) \textendash{} First port number. Defaults to 1.

\item {} 
\sphinxstyleliteralstrong{\sphinxupquote{j}} (\sphinxstyleliteralemphasis{\sphinxupquote{int}}\sphinxstyleliteralemphasis{\sphinxupquote{, }}\sphinxstyleliteralemphasis{\sphinxupquote{optional}}) \textendash{} Second port number. Defaults to 1. Ignored for {\color{red}\bfseries{}*}dataformat*=”VSWR”

\item {} 
\sphinxstyleliteralstrong{\sphinxupquote{frekanslar}} (\sphinxstyleliteralemphasis{\sphinxupquote{list}}\sphinxstyleliteralemphasis{\sphinxupquote{, }}\sphinxstyleliteralemphasis{\sphinxupquote{optional}}) \textendash{} Defaults to {[}{]}. List of frequencies in Hz. If an empty list is given, networks whole frequency range is used.

\item {} 
\sphinxstyleliteralstrong{\sphinxupquote{ref}} ({\hyperref[\detokenize{touchstone:touchstone.spfile}]{\sphinxcrossref{\sphinxstyleliteralemphasis{\sphinxupquote{spfile}}}}}\sphinxstyleliteralemphasis{\sphinxupquote{, }}\sphinxstyleliteralemphasis{\sphinxupquote{optional}}) \textendash{} Defaults to None. If given the data of this network is subtracted from the same data of \sphinxstyleemphasis{ref} object.

\item {} 
\sphinxstyleliteralstrong{\sphinxupquote{DCInt}} (\sphinxstyleliteralemphasis{\sphinxupquote{int}}\sphinxstyleliteralemphasis{\sphinxupquote{, }}\sphinxstyleliteralemphasis{\sphinxupquote{optional}}) \textendash{} Defaults to 0. If 1, DC point given by \sphinxstyleemphasis{DCValue} is used at frequency interpolation if \sphinxstyleemphasis{frekanslar} is not {[}{]}.

\item {} 
\sphinxstyleliteralstrong{\sphinxupquote{DCValue}} (\sphinxstyleliteralemphasis{\sphinxupquote{complex}}\sphinxstyleliteralemphasis{\sphinxupquote{, }}\sphinxstyleliteralemphasis{\sphinxupquote{optional}}) \textendash{} Defaults to 0.0. DCValue that can be used for interpolation over frequency.

\item {} 
\sphinxstyleliteralstrong{\sphinxupquote{smoothing}} (\sphinxstyleliteralemphasis{\sphinxupquote{int}}\sphinxstyleliteralemphasis{\sphinxupquote{, }}\sphinxstyleliteralemphasis{\sphinxupquote{optional}}) \textendash{} Defaults to 0. if this is higher than 0, it is used as the number of points for smoothing.

\item {} 
\sphinxstyleliteralstrong{\sphinxupquote{InterpolationConstant}} (\sphinxstyleliteralemphasis{\sphinxupquote{int}}\sphinxstyleliteralemphasis{\sphinxupquote{, }}\sphinxstyleliteralemphasis{\sphinxupquote{optional}}) \textendash{} Defaults to 0. If this is higher than 0, it is taken as the number of frequencies that will be added between 2 consecutive frequency points. By this way, number of frequencies is increased by interpolation.

\end{itemize}

\item[{Returns}] \leavevmode
Network data array

\item[{Return type}] \leavevmode
numpy.array

\end{description}\end{quote}

\end{fulllineitems}

\index{dosyaoku() (touchstone.spfile method)@\spxentry{dosyaoku()}\spxextra{touchstone.spfile method}}

\begin{fulllineitems}
\phantomsection\label{\detokenize{touchstone:touchstone.spfile.dosyaoku}}\pysiglinewithargsret{\sphinxbfcode{\sphinxupquote{dosyaoku}}}{\emph{\DUrole{n}{file\_name}}, \emph{\DUrole{n}{satiratla}\DUrole{o}{=}\DUrole{default_value}{0}}}{}
\end{fulllineitems}

\index{dosyayi\_tekrar\_oku() (touchstone.spfile method)@\spxentry{dosyayi\_tekrar\_oku()}\spxextra{touchstone.spfile method}}

\begin{fulllineitems}
\phantomsection\label{\detokenize{touchstone:touchstone.spfile.dosyayi_tekrar_oku}}\pysiglinewithargsret{\sphinxbfcode{\sphinxupquote{dosyayi\_tekrar\_oku}}}{}{}
\end{fulllineitems}

\index{gav() (touchstone.spfile method)@\spxentry{gav()}\spxextra{touchstone.spfile method}}

\begin{fulllineitems}
\phantomsection\label{\detokenize{touchstone:touchstone.spfile.gav}}\pysiglinewithargsret{\sphinxbfcode{\sphinxupquote{gav}}}{\emph{\DUrole{n}{port1}\DUrole{o}{=}\DUrole{default_value}{1}}, \emph{\DUrole{n}{port2}\DUrole{o}{=}\DUrole{default_value}{2}}, \emph{\DUrole{n}{ZS}\DUrole{o}{=}\DUrole{default_value}{{[}{]}}}, \emph{\DUrole{n}{dB}\DUrole{o}{=}\DUrole{default_value}{True}}}{}
Available gain from port1 to port2. If dB=True, output is in dB, otherwise it is a power ratio.
\begin{quote}
\begin{equation*}
\begin{split}G_{av}=\frac{P_{av,toLoad}}{P_{av,fromSource}}\end{split}
\end{equation*}\end{quote}
\begin{quote}\begin{description}
\item[{Parameters}] \leavevmode\begin{itemize}
\item {} 
\sphinxstyleliteralstrong{\sphinxupquote{port1}} (\sphinxstyleliteralemphasis{\sphinxupquote{int}}\sphinxstyleliteralemphasis{\sphinxupquote{, }}\sphinxstyleliteralemphasis{\sphinxupquote{optional}}) \textendash{} Index of input port. Defaults to 1.

\item {} 
\sphinxstyleliteralstrong{\sphinxupquote{port2}} (\sphinxstyleliteralemphasis{\sphinxupquote{int}}\sphinxstyleliteralemphasis{\sphinxupquote{, }}\sphinxstyleliteralemphasis{\sphinxupquote{optional}}) \textendash{} Index of output port. Defaults to 2.

\item {} 
\sphinxstyleliteralstrong{\sphinxupquote{ZS}} (\sphinxstyleliteralemphasis{\sphinxupquote{list}}\sphinxstyleliteralemphasis{\sphinxupquote{ or }}\sphinxstyleliteralemphasis{\sphinxupquote{numpy.ndarray}}\sphinxstyleliteralemphasis{\sphinxupquote{, }}\sphinxstyleliteralemphasis{\sphinxupquote{optional}}) \textendash{} Impedance of input port. Defaults to current reference impedance.

\item {} 
\sphinxstyleliteralstrong{\sphinxupquote{dB}} (\sphinxstyleliteralemphasis{\sphinxupquote{bool}}\sphinxstyleliteralemphasis{\sphinxupquote{, }}\sphinxstyleliteralemphasis{\sphinxupquote{optional}}) \textendash{} Enable dB output. Defaults to True.

\end{itemize}

\item[{Returns}] \leavevmode
Array of Gmax values for all frequencies

\item[{Return type}] \leavevmode
numpy.ndarray

\end{description}\end{quote}

\end{fulllineitems}

\index{get\_formulation() (touchstone.spfile method)@\spxentry{get\_formulation()}\spxextra{touchstone.spfile method}}

\begin{fulllineitems}
\phantomsection\label{\detokenize{touchstone:touchstone.spfile.get_formulation}}\pysiglinewithargsret{\sphinxbfcode{\sphinxupquote{get\_formulation}}}{}{}
\end{fulllineitems}

\index{get\_frequency\_list() (touchstone.spfile method)@\spxentry{get\_frequency\_list()}\spxextra{touchstone.spfile method}}

\begin{fulllineitems}
\phantomsection\label{\detokenize{touchstone:touchstone.spfile.get_frequency_list}}\pysiglinewithargsret{\sphinxbfcode{\sphinxupquote{get\_frequency\_list}}}{}{}
Returns the frequency list of network
\begin{quote}\begin{description}
\item[{Returns}] \leavevmode
Frequency list of network

\item[{Return type}] \leavevmode
numpy.array

\end{description}\end{quote}

\end{fulllineitems}

\index{get\_no\_of\_ports() (touchstone.spfile method)@\spxentry{get\_no\_of\_ports()}\spxextra{touchstone.spfile method}}

\begin{fulllineitems}
\phantomsection\label{\detokenize{touchstone:touchstone.spfile.get_no_of_ports}}\pysiglinewithargsret{\sphinxbfcode{\sphinxupquote{get\_no\_of\_ports}}}{}{}
\end{fulllineitems}

\index{get\_port\_names() (touchstone.spfile method)@\spxentry{get\_port\_names()}\spxextra{touchstone.spfile method}}

\begin{fulllineitems}
\phantomsection\label{\detokenize{touchstone:touchstone.spfile.get_port_names}}\pysiglinewithargsret{\sphinxbfcode{\sphinxupquote{get\_port\_names}}}{}{}
Get list of port names.

\end{fulllineitems}

\index{get\_port\_number\_from\_name() (touchstone.spfile method)@\spxentry{get\_port\_number\_from\_name()}\spxextra{touchstone.spfile method}}

\begin{fulllineitems}
\phantomsection\label{\detokenize{touchstone:touchstone.spfile.get_port_number_from_name}}\pysiglinewithargsret{\sphinxbfcode{\sphinxupquote{get\_port\_number\_from\_name}}}{\emph{\DUrole{n}{isim}}}{}
Index of first port index with name \sphinxstyleemphasis{isim}
\begin{quote}\begin{description}
\item[{Parameters}] \leavevmode
\sphinxstyleliteralstrong{\sphinxupquote{isim}} (\sphinxstyleliteralemphasis{\sphinxupquote{bool}}) \textendash{} Name of the port

\item[{Returns}] \leavevmode
Port index if port is found, 0 otherwise

\item[{Return type}] \leavevmode
int

\end{description}\end{quote}

\end{fulllineitems}

\index{get\_sym\_parameters() (touchstone.spfile method)@\spxentry{get\_sym\_parameters()}\spxextra{touchstone.spfile method}}

\begin{fulllineitems}
\phantomsection\label{\detokenize{touchstone:touchstone.spfile.get_sym_parameters}}\pysiglinewithargsret{\sphinxbfcode{\sphinxupquote{get\_sym\_parameters}}}{}{}
This function is used to get the values of symbolic variables of the network.
\begin{quote}\begin{description}
\item[{Returns}] \leavevmode
This is a dictionary containing the values of symbolic variables of the network

\item[{Return type}] \leavevmode
dict

\end{description}\end{quote}

\end{fulllineitems}

\index{get\_sym\_smatrix() (touchstone.spfile method)@\spxentry{get\_sym\_smatrix()}\spxextra{touchstone.spfile method}}

\begin{fulllineitems}
\phantomsection\label{\detokenize{touchstone:touchstone.spfile.get_sym_smatrix}}\pysiglinewithargsret{\sphinxbfcode{\sphinxupquote{get\_sym\_smatrix}}}{}{}
\end{fulllineitems}

\index{get\_undefinedYindices() (touchstone.spfile method)@\spxentry{get\_undefinedYindices()}\spxextra{touchstone.spfile method}}

\begin{fulllineitems}
\phantomsection\label{\detokenize{touchstone:touchstone.spfile.get_undefinedYindices}}\pysiglinewithargsret{\sphinxbfcode{\sphinxupquote{get\_undefinedYindices}}}{}{}
\end{fulllineitems}

\index{get\_undefinedZindices() (touchstone.spfile method)@\spxentry{get\_undefinedZindices()}\spxextra{touchstone.spfile method}}

\begin{fulllineitems}
\phantomsection\label{\detokenize{touchstone:touchstone.spfile.get_undefinedZindices}}\pysiglinewithargsret{\sphinxbfcode{\sphinxupquote{get\_undefinedZindices}}}{}{}
\end{fulllineitems}

\index{getdataformat() (touchstone.spfile method)@\spxentry{getdataformat()}\spxextra{touchstone.spfile method}}

\begin{fulllineitems}
\phantomsection\label{\detokenize{touchstone:touchstone.spfile.getdataformat}}\pysiglinewithargsret{\sphinxbfcode{\sphinxupquote{getdataformat}}}{}{}
\end{fulllineitems}

\index{getfilename() (touchstone.spfile method)@\spxentry{getfilename()}\spxextra{touchstone.spfile method}}

\begin{fulllineitems}
\phantomsection\label{\detokenize{touchstone:touchstone.spfile.getfilename}}\pysiglinewithargsret{\sphinxbfcode{\sphinxupquote{getfilename}}}{}{}
\end{fulllineitems}

\index{gmax() (touchstone.spfile method)@\spxentry{gmax()}\spxextra{touchstone.spfile method}}

\begin{fulllineitems}
\phantomsection\label{\detokenize{touchstone:touchstone.spfile.gmax}}\pysiglinewithargsret{\sphinxbfcode{\sphinxupquote{gmax}}}{\emph{\DUrole{n}{port1}\DUrole{o}{=}\DUrole{default_value}{1}}, \emph{\DUrole{n}{port2}\DUrole{o}{=}\DUrole{default_value}{2}}, \emph{\DUrole{n}{dB}\DUrole{o}{=}\DUrole{default_value}{True}}}{}
Calculates Gmax from port1 to port2. Other ports are terminated with current reference impedances. If dB=True, output is in dB, otherwise it is a power ratio.
\begin{quote}\begin{description}
\item[{Parameters}] \leavevmode\begin{itemize}
\item {} 
\sphinxstyleliteralstrong{\sphinxupquote{port1}} (\sphinxstyleliteralemphasis{\sphinxupquote{int}}\sphinxstyleliteralemphasis{\sphinxupquote{, }}\sphinxstyleliteralemphasis{\sphinxupquote{optional}}) \textendash{} Index of input port. Defaults to 1.

\item {} 
\sphinxstyleliteralstrong{\sphinxupquote{port2}} (\sphinxstyleliteralemphasis{\sphinxupquote{int}}\sphinxstyleliteralemphasis{\sphinxupquote{, }}\sphinxstyleliteralemphasis{\sphinxupquote{optional}}) \textendash{} Index of output port. Defaults to 2.

\item {} 
\sphinxstyleliteralstrong{\sphinxupquote{dB}} (\sphinxstyleliteralemphasis{\sphinxupquote{bool}}\sphinxstyleliteralemphasis{\sphinxupquote{, }}\sphinxstyleliteralemphasis{\sphinxupquote{optional}}) \textendash{} Enable dB output. Defaults to True.

\end{itemize}

\item[{Returns}] \leavevmode
Array of Gmax values for all frequencies

\item[{Return type}] \leavevmode
numpy.ndarray

\end{description}\end{quote}

\end{fulllineitems}

\index{gop() (touchstone.spfile method)@\spxentry{gop()}\spxextra{touchstone.spfile method}}

\begin{fulllineitems}
\phantomsection\label{\detokenize{touchstone:touchstone.spfile.gop}}\pysiglinewithargsret{\sphinxbfcode{\sphinxupquote{gop}}}{\emph{\DUrole{n}{port1}\DUrole{o}{=}\DUrole{default_value}{1}}, \emph{\DUrole{n}{port2}\DUrole{o}{=}\DUrole{default_value}{2}}, \emph{\DUrole{n}{ZL}\DUrole{o}{=}\DUrole{default_value}{None}}, \emph{\DUrole{n}{dB}\DUrole{o}{=}\DUrole{default_value}{True}}}{}
Operating power gain from port1 to port2 with load impedance of ZL. If dB=True, output is in dB, otherwise it is a power ratio.
\begin{quote}
\begin{equation*}
\begin{split}G_{op}=\frac{P_{toLoad}}{P_{toNetwork}}\end{split}
\end{equation*}\end{quote}
\begin{quote}\begin{description}
\item[{Parameters}] \leavevmode\begin{itemize}
\item {} 
\sphinxstyleliteralstrong{\sphinxupquote{port1}} (\sphinxstyleliteralemphasis{\sphinxupquote{int}}\sphinxstyleliteralemphasis{\sphinxupquote{, }}\sphinxstyleliteralemphasis{\sphinxupquote{optional}}) \textendash{} Index of input port. Defaults to 1.

\item {} 
\sphinxstyleliteralstrong{\sphinxupquote{port2}} (\sphinxstyleliteralemphasis{\sphinxupquote{int}}\sphinxstyleliteralemphasis{\sphinxupquote{, }}\sphinxstyleliteralemphasis{\sphinxupquote{optional}}) \textendash{} Index of output port. Defaults to 2.

\item {} 
\sphinxstyleliteralstrong{\sphinxupquote{ZL}} (\sphinxstyleliteralemphasis{\sphinxupquote{ndarray}}\sphinxstyleliteralemphasis{\sphinxupquote{ or }}\sphinxstyleliteralemphasis{\sphinxupquote{float}}\sphinxstyleliteralemphasis{\sphinxupquote{, }}\sphinxstyleliteralemphasis{\sphinxupquote{optional}}) \textendash{} Load impedance. Defaults to current port impedance at port2.

\item {} 
\sphinxstyleliteralstrong{\sphinxupquote{dB}} (\sphinxstyleliteralemphasis{\sphinxupquote{bool}}\sphinxstyleliteralemphasis{\sphinxupquote{, }}\sphinxstyleliteralemphasis{\sphinxupquote{optional}}) \textendash{} Enable dB output. Defaults to True.

\end{itemize}

\item[{Returns}] \leavevmode
Array of Gop values for all frequencies

\item[{Return type}] \leavevmode
numpy.ndarray

\end{description}\end{quote}

\end{fulllineitems}

\index{gop2() (touchstone.spfile method)@\spxentry{gop2()}\spxextra{touchstone.spfile method}}

\begin{fulllineitems}
\phantomsection\label{\detokenize{touchstone:touchstone.spfile.gop2}}\pysiglinewithargsret{\sphinxbfcode{\sphinxupquote{gop2}}}{\emph{\DUrole{n}{port1}\DUrole{o}{=}\DUrole{default_value}{1}}, \emph{\DUrole{n}{port2}\DUrole{o}{=}\DUrole{default_value}{2}}, \emph{\DUrole{n}{ZL}\DUrole{o}{=}\DUrole{default_value}{50.0}}, \emph{\DUrole{n}{dB}\DUrole{o}{=}\DUrole{default_value}{True}}}{}
Operating power gain from port1 to port2 with load impedance of ZL. If dB=True, output is in dB, otherwise it is a power ratio.
\begin{quote}
\begin{equation*}
\begin{split}G_{op}=\frac{P_{toLoad}}{P_{toNetwork}}\end{split}
\end{equation*}\end{quote}
\begin{quote}\begin{description}
\item[{Parameters}] \leavevmode\begin{itemize}
\item {} 
\sphinxstyleliteralstrong{\sphinxupquote{port1}} (\sphinxstyleliteralemphasis{\sphinxupquote{int}}\sphinxstyleliteralemphasis{\sphinxupquote{, }}\sphinxstyleliteralemphasis{\sphinxupquote{optional}}) \textendash{} Index of input port. Defaults to 1.

\item {} 
\sphinxstyleliteralstrong{\sphinxupquote{port2}} (\sphinxstyleliteralemphasis{\sphinxupquote{int}}\sphinxstyleliteralemphasis{\sphinxupquote{, }}\sphinxstyleliteralemphasis{\sphinxupquote{optional}}) \textendash{} Index of output port. Defaults to 2.

\item {} 
\sphinxstyleliteralstrong{\sphinxupquote{ZL}} (\sphinxstyleliteralemphasis{\sphinxupquote{ndarray}}\sphinxstyleliteralemphasis{\sphinxupquote{ or }}\sphinxstyleliteralemphasis{\sphinxupquote{float}}\sphinxstyleliteralemphasis{\sphinxupquote{, }}\sphinxstyleliteralemphasis{\sphinxupquote{optional}}) \textendash{} Load impedance. Defaults to current port impedance at port2.

\item {} 
\sphinxstyleliteralstrong{\sphinxupquote{dB}} (\sphinxstyleliteralemphasis{\sphinxupquote{bool}}\sphinxstyleliteralemphasis{\sphinxupquote{, }}\sphinxstyleliteralemphasis{\sphinxupquote{optional}}) \textendash{} Enable dB output. Defaults to True.

\end{itemize}

\item[{Returns}] \leavevmode
Array of Gop values for all frequencies

\item[{Return type}] \leavevmode
numpy.ndarray

\end{description}\end{quote}

\end{fulllineitems}

\index{gt() (touchstone.spfile method)@\spxentry{gt()}\spxextra{touchstone.spfile method}}

\begin{fulllineitems}
\phantomsection\label{\detokenize{touchstone:touchstone.spfile.gt}}\pysiglinewithargsret{\sphinxbfcode{\sphinxupquote{gt}}}{\emph{\DUrole{n}{port1}\DUrole{o}{=}\DUrole{default_value}{1}}, \emph{\DUrole{n}{port2}\DUrole{o}{=}\DUrole{default_value}{2}}, \emph{\DUrole{n}{ZS}\DUrole{o}{=}\DUrole{default_value}{{[}{]}}}, \emph{\DUrole{n}{ZL}\DUrole{o}{=}\DUrole{default_value}{{[}{]}}}, \emph{\DUrole{n}{dB}\DUrole{o}{=}\DUrole{default_value}{True}}}{}
This method calculates transducer gain (GT) from port1 to port2. Source and load impedances can be specified independently. If any one of them is not specified, current reference impedance is used for that port. Other ports are terminated by reference impedances. This calculation can also be done using impedance renormalization.
\begin{quote}
\begin{equation*}
\begin{split}G_{av}=\frac{P_{load}}{P_{av,fromSource}}\end{split}
\end{equation*}\end{quote}
\begin{quote}\begin{description}
\item[{Parameters}] \leavevmode\begin{itemize}
\item {} 
\sphinxstyleliteralstrong{\sphinxupquote{port1}} (\sphinxstyleliteralemphasis{\sphinxupquote{int}}\sphinxstyleliteralemphasis{\sphinxupquote{, }}\sphinxstyleliteralemphasis{\sphinxupquote{optional}}) \textendash{} Index of source port. Defaults to 1.

\item {} 
\sphinxstyleliteralstrong{\sphinxupquote{port2}} (\sphinxstyleliteralemphasis{\sphinxupquote{int}}\sphinxstyleliteralemphasis{\sphinxupquote{, }}\sphinxstyleliteralemphasis{\sphinxupquote{optional}}) \textendash{} Index of load port. Defaults to 2.

\item {} 
\sphinxstyleliteralstrong{\sphinxupquote{dB}} (\sphinxstyleliteralemphasis{\sphinxupquote{bool}}\sphinxstyleliteralemphasis{\sphinxupquote{, }}\sphinxstyleliteralemphasis{\sphinxupquote{optional}}) \textendash{} Enable dB output. Defaults to True.

\item {} 
\sphinxstyleliteralstrong{\sphinxupquote{ZS}} (\sphinxstyleliteralemphasis{\sphinxupquote{float}}\sphinxstyleliteralemphasis{\sphinxupquote{, }}\sphinxstyleliteralemphasis{\sphinxupquote{optional}}) \textendash{} Source impedance. Defaults to 50.0.

\item {} 
\sphinxstyleliteralstrong{\sphinxupquote{ZL}} (\sphinxstyleliteralemphasis{\sphinxupquote{float}}\sphinxstyleliteralemphasis{\sphinxupquote{, }}\sphinxstyleliteralemphasis{\sphinxupquote{optional}}) \textendash{} Load impedance. Defaults to 50.0.

\end{itemize}

\item[{Returns}] \leavevmode
Array of GT values for all frequencies

\item[{Return type}] \leavevmode
numpy.ndarray

\end{description}\end{quote}

\end{fulllineitems}

\index{input\_impedance() (touchstone.spfile method)@\spxentry{input\_impedance()}\spxextra{touchstone.spfile method}}

\begin{fulllineitems}
\phantomsection\label{\detokenize{touchstone:touchstone.spfile.input_impedance}}\pysiglinewithargsret{\sphinxbfcode{\sphinxupquote{input\_impedance}}}{\emph{\DUrole{n}{k}}}{}
Input impedance at port k. All ports are terminated with reference impedances.
\begin{quote}\begin{description}
\item[{Parameters}] \leavevmode
\sphinxstyleliteralstrong{\sphinxupquote{port}} (\sphinxstyleliteralemphasis{\sphinxupquote{int}}) \textendash{} Port number for input impedance.

\item[{Returns}] \leavevmode
Array of impedance values for all frequencies

\item[{Return type}] \leavevmode
numpy.ndarray

\end{description}\end{quote}

\end{fulllineitems}

\index{interpolate\_data() (touchstone.spfile method)@\spxentry{interpolate\_data()}\spxextra{touchstone.spfile method}}

\begin{fulllineitems}
\phantomsection\label{\detokenize{touchstone:touchstone.spfile.interpolate_data}}\pysiglinewithargsret{\sphinxbfcode{\sphinxupquote{interpolate\_data}}}{\emph{\DUrole{n}{datain}}, \emph{\DUrole{n}{freqs}}}{}
Calculate new data corresponding to new frequency points \sphinxstyleemphasis{freqs} by interpolation from original data corresponding to current frequency points of the network.
\begin{quote}\begin{description}
\item[{Parameters}] \leavevmode\begin{itemize}
\item {} 
\sphinxstyleliteralstrong{\sphinxupquote{data}} (\sphinxstyleliteralemphasis{\sphinxupquote{numpy.ndarray}}\sphinxstyleliteralemphasis{\sphinxupquote{ or }}\sphinxstyleliteralemphasis{\sphinxupquote{list}}) \textendash{} Original data specified at current frequency points of the network.

\item {} 
\sphinxstyleliteralstrong{\sphinxupquote{freqs}} (\sphinxstyleliteralemphasis{\sphinxupquote{numpy.ndarray}}\sphinxstyleliteralemphasis{\sphinxupquote{ or }}\sphinxstyleliteralemphasis{\sphinxupquote{list}}) \textendash{} New frequency list.

\end{itemize}

\item[{Returns}] \leavevmode
New data corresponding to \sphinxstyleemphasis{freqs}

\item[{Return type}] \leavevmode
numpy.ndarray

\end{description}\end{quote}

\end{fulllineitems}

\index{inverse\_2port() (touchstone.spfile method)@\spxentry{inverse\_2port()}\spxextra{touchstone.spfile method}}

\begin{fulllineitems}
\phantomsection\label{\detokenize{touchstone:touchstone.spfile.inverse_2port}}\pysiglinewithargsret{\sphinxbfcode{\sphinxupquote{inverse\_2port}}}{\emph{\DUrole{n}{inplace}\DUrole{o}{=}\DUrole{default_value}{\sphinxhyphen{} 1}}}{}
Take inverse of 2\sphinxhyphen{}port data for de\sphinxhyphen{}embedding purposes.
\begin{quote}\begin{description}
\item[{Parameters}] \leavevmode
\sphinxstyleliteralstrong{\sphinxupquote{inplace}} (\sphinxstyleliteralemphasis{\sphinxupquote{int}}\sphinxstyleliteralemphasis{\sphinxupquote{, }}\sphinxstyleliteralemphasis{\sphinxupquote{optional}}) \textendash{} Object editing mode. Defaults to \sphinxhyphen{}1.

\item[{Returns}] \leavevmode
Inverted 2\sphinxhyphen{}port spfile

\item[{Return type}] \leavevmode
{\hyperref[\detokenize{touchstone:touchstone.spfile}]{\sphinxcrossref{spfile}}}

\end{description}\end{quote}

\end{fulllineitems}

\index{load\_impedance() (touchstone.spfile method)@\spxentry{load\_impedance()}\spxextra{touchstone.spfile method}}

\begin{fulllineitems}
\phantomsection\label{\detokenize{touchstone:touchstone.spfile.load_impedance}}\pysiglinewithargsret{\sphinxbfcode{\sphinxupquote{load\_impedance}}}{\emph{\DUrole{n}{Gamma\_in}}, \emph{\DUrole{n}{port1}\DUrole{o}{=}\DUrole{default_value}{1}}, \emph{\DUrole{n}{port2}\DUrole{o}{=}\DUrole{default_value}{2}}}{}
Calculates termination impedance at port2 that gives Gamma\_in reflection coefficient at port1.
\begin{quote}\begin{description}
\item[{Parameters}] \leavevmode\begin{itemize}
\item {} 
\sphinxstyleliteralstrong{\sphinxupquote{Gamma\_in}} (\sphinxstyleliteralemphasis{\sphinxupquote{float}}\sphinxstyleliteralemphasis{\sphinxupquote{,}}\sphinxstyleliteralemphasis{\sphinxupquote{ndarray}}) \textendash{} Required reflection coefficient.

\item {} 
\sphinxstyleliteralstrong{\sphinxupquote{port1}} (\sphinxstyleliteralemphasis{\sphinxupquote{int}}) \textendash{} Source port.

\item {} 
\sphinxstyleliteralstrong{\sphinxupquote{port2}} (\sphinxstyleliteralemphasis{\sphinxupquote{int}}) \textendash{} Load port.

\end{itemize}

\item[{Returns}] \leavevmode
Array of reflection coeeficient of termination at port2

\item[{Return type}] \leavevmode
numpy.ndarray

\end{description}\end{quote}

\end{fulllineitems}

\index{microstripline() (touchstone.spfile class method)@\spxentry{microstripline()}\spxextra{touchstone.spfile class method}}

\begin{fulllineitems}
\phantomsection\label{\detokenize{touchstone:touchstone.spfile.microstripline}}\pysiglinewithargsret{\sphinxbfcode{\sphinxupquote{classmethod }}\sphinxbfcode{\sphinxupquote{microstripline}}}{\emph{\DUrole{n}{length}}, \emph{\DUrole{n}{w}}, \emph{\DUrole{n}{h}}, \emph{\DUrole{n}{t}}, \emph{\DUrole{n}{er}}, \emph{\DUrole{n}{freqs}\DUrole{o}{=}\DUrole{default_value}{None}}}{}
Create an \sphinxcode{\sphinxupquote{spfile}} object corresponding to a microstrip line.
\begin{quote}\begin{description}
\item[{Parameters}] \leavevmode\begin{itemize}
\item {} 
\sphinxstyleliteralstrong{\sphinxupquote{length}} (\sphinxstyleliteralemphasis{\sphinxupquote{float}}) \textendash{} Length of microstrip line.

\item {} 
\sphinxstyleliteralstrong{\sphinxupquote{w}} (\sphinxstyleliteralemphasis{\sphinxupquote{float}}) \textendash{} Width of microstrip line.

\item {} 
\sphinxstyleliteralstrong{\sphinxupquote{h}} (\sphinxstyleliteralemphasis{\sphinxupquote{float}}) \textendash{} Thickness of substrate.

\item {} 
\sphinxstyleliteralstrong{\sphinxupquote{t}} (\sphinxstyleliteralemphasis{\sphinxupquote{float}}) \textendash{} Thickness of metal.

\item {} 
\sphinxstyleliteralstrong{\sphinxupquote{er}} (\sphinxstyleliteralemphasis{\sphinxupquote{float}}) \textendash{} Relative permittivity of microstrip substrate.

\item {} 
\sphinxstyleliteralstrong{\sphinxupquote{freqs}} (\sphinxstyleliteralemphasis{\sphinxupquote{float}}\sphinxstyleliteralemphasis{\sphinxupquote{, }}\sphinxstyleliteralemphasis{\sphinxupquote{optional}}) \textendash{} Frequency list of object. Defaults to None. If not given, frequencies should be set later.

\end{itemize}

\item[{Returns}] \leavevmode
An spfile object.

\item[{Return type}] \leavevmode
{\hyperref[\detokenize{touchstone:touchstone.spfile}]{\sphinxcrossref{spfile}}}

\end{description}\end{quote}

\end{fulllineitems}

\index{microstripstep() (touchstone.spfile class method)@\spxentry{microstripstep()}\spxextra{touchstone.spfile class method}}

\begin{fulllineitems}
\phantomsection\label{\detokenize{touchstone:touchstone.spfile.microstripstep}}\pysiglinewithargsret{\sphinxbfcode{\sphinxupquote{classmethod }}\sphinxbfcode{\sphinxupquote{microstripstep}}}{\emph{\DUrole{n}{w1}}, \emph{\DUrole{n}{w2}}, \emph{\DUrole{n}{eps\_r}}, \emph{\DUrole{n}{h}}, \emph{\DUrole{n}{t}}, \emph{\DUrole{n}{freqs}\DUrole{o}{=}\DUrole{default_value}{None}}}{}
Create an \sphinxcode{\sphinxupquote{spfile}} object corresponding to a microstrip step.
\begin{quote}\begin{description}
\item[{Parameters}] \leavevmode\begin{itemize}
\item {} 
\sphinxstyleliteralstrong{\sphinxupquote{w1}} (\sphinxstyleliteralemphasis{\sphinxupquote{float}}) \textendash{} Width of microstrip line at port\sphinxhyphen{}1.

\item {} 
\sphinxstyleliteralstrong{\sphinxupquote{w2}} (\sphinxstyleliteralemphasis{\sphinxupquote{float}}) \textendash{} Width of microstrip line at port\sphinxhyphen{}2.

\item {} 
\sphinxstyleliteralstrong{\sphinxupquote{eps\_r}} (\sphinxstyleliteralemphasis{\sphinxupquote{float}}) \textendash{} Relative permittivity of microstrip substrate.

\item {} 
\sphinxstyleliteralstrong{\sphinxupquote{h}} (\sphinxstyleliteralemphasis{\sphinxupquote{float}}) \textendash{} Thickness of microstrip substrate.

\item {} 
\sphinxstyleliteralstrong{\sphinxupquote{t}} (\sphinxstyleliteralemphasis{\sphinxupquote{float}}) \textendash{} Thickness of metal.

\item {} 
\sphinxstyleliteralstrong{\sphinxupquote{freqs}} (\sphinxstyleliteralemphasis{\sphinxupquote{float}}\sphinxstyleliteralemphasis{\sphinxupquote{, }}\sphinxstyleliteralemphasis{\sphinxupquote{optional}}) \textendash{} Frequency list of object. Defaults to None. If not given, frequencies should be set later.

\end{itemize}

\item[{Returns}] \leavevmode
An spfile object equivalent to microstrip step.

\item[{Return type}] \leavevmode
{\hyperref[\detokenize{touchstone:touchstone.spfile}]{\sphinxcrossref{spfile}}}

\end{description}\end{quote}

\end{fulllineitems}

\index{prepare\_ref\_impedance\_array() (touchstone.spfile method)@\spxentry{prepare\_ref\_impedance\_array()}\spxextra{touchstone.spfile method}}

\begin{fulllineitems}
\phantomsection\label{\detokenize{touchstone:touchstone.spfile.prepare_ref_impedance_array}}\pysiglinewithargsret{\sphinxbfcode{\sphinxupquote{prepare\_ref\_impedance\_array}}}{\emph{\DUrole{n}{imparray}\DUrole{o}{=}\DUrole{default_value}{None}}}{}~\begin{description}
\item[{Turns reference impedance array which is composed of numbers,arrays, functions or 1\sphinxhyphen{}ports to numerical array which}] \leavevmode
is composed of numbers and arrays. It is made sure that :math:{\color{red}\bfseries{}\textasciigrave{}}Re(Z)

\end{description}

eq 0\textasciigrave{}. Mainly for internal use.
\begin{quote}
\begin{description}
\item[{Args:}] \leavevmode
imparray (list): List of impedance array

\item[{Returns:}] \leavevmode
numpy.ndarray: Calculated impedance array

\end{description}
\end{quote}

\end{fulllineitems}

\index{restore\_passivity() (touchstone.spfile method)@\spxentry{restore\_passivity()}\spxextra{touchstone.spfile method}}

\begin{fulllineitems}
\phantomsection\label{\detokenize{touchstone:touchstone.spfile.restore_passivity}}\pysiglinewithargsret{\sphinxbfcode{\sphinxupquote{restore\_passivity}}}{\emph{\DUrole{n}{inplace}\DUrole{o}{=}\DUrole{default_value}{\sphinxhyphen{} 1}}}{}
Make the network passive by minimum modification.
Method reference:
“Fast and Optimal Algorithms for Enforcing Reciprocity, Passivity and Causality in S\sphinxhyphen{}parameters.pdf”
\begin{quote}\begin{description}
\item[{Parameters}] \leavevmode
\sphinxstyleliteralstrong{\sphinxupquote{inplace}} (\sphinxstyleliteralemphasis{\sphinxupquote{int}}\sphinxstyleliteralemphasis{\sphinxupquote{, }}\sphinxstyleliteralemphasis{\sphinxupquote{optional}}) \textendash{} Object editing mode. Defaults to \sphinxhyphen{}1.

\item[{Returns}] \leavevmode
Passive network object

\item[{Return type}] \leavevmode
{\hyperref[\detokenize{touchstone:touchstone.spfile}]{\sphinxcrossref{spfile}}}

\end{description}\end{quote}

\end{fulllineitems}

\index{restore\_passivity2() (touchstone.spfile method)@\spxentry{restore\_passivity2()}\spxextra{touchstone.spfile method}}

\begin{fulllineitems}
\phantomsection\label{\detokenize{touchstone:touchstone.spfile.restore_passivity2}}\pysiglinewithargsret{\sphinxbfcode{\sphinxupquote{restore\_passivity2}}}{}{}~\begin{description}
\item[{\sphinxstylestrong{Obsolete}}] \leavevmode
Bu metod S\sphinxhyphen{}parametre datasinin pasif olmadigi frekanslarda

\end{description}

S\sphinxhyphen{}parametre datasina mumkun olan en kucuk degisikligi yaparak
S\sphinxhyphen{}parametre datasini pasif hale getirir.
Referans:
Restoration of Passivity In S\sphinxhyphen{}parameter Data of Microwave Measurements.pdf

\end{fulllineitems}

\index{return\_s2p() (touchstone.spfile method)@\spxentry{return\_s2p()}\spxextra{touchstone.spfile method}}

\begin{fulllineitems}
\phantomsection\label{\detokenize{touchstone:touchstone.spfile.return_s2p}}\pysiglinewithargsret{\sphinxbfcode{\sphinxupquote{return\_s2p}}}{\emph{\DUrole{n}{port1}\DUrole{o}{=}\DUrole{default_value}{1}}, \emph{\DUrole{n}{port2}\DUrole{o}{=}\DUrole{default_value}{2}}}{}
\end{fulllineitems}

\index{s2abcd() (touchstone.spfile method)@\spxentry{s2abcd()}\spxextra{touchstone.spfile method}}

\begin{fulllineitems}
\phantomsection\label{\detokenize{touchstone:touchstone.spfile.s2abcd}}\pysiglinewithargsret{\sphinxbfcode{\sphinxupquote{s2abcd}}}{\emph{\DUrole{n}{port1}\DUrole{o}{=}\DUrole{default_value}{1}}, \emph{\DUrole{n}{port2}\DUrole{o}{=}\DUrole{default_value}{2}}}{}
S\sphinxhyphen{}Matrix to ABCD matrix conversion between 2 chosen ports. Other ports are terminated with reference impedances
\begin{quote}\begin{description}
\item[{Parameters}] \leavevmode\begin{itemize}
\item {} 
\sphinxstyleliteralstrong{\sphinxupquote{port1}} (\sphinxstyleliteralemphasis{\sphinxupquote{int}}\sphinxstyleliteralemphasis{\sphinxupquote{, }}\sphinxstyleliteralemphasis{\sphinxupquote{optional}}) \textendash{} Index of Port\sphinxhyphen{}1. Defaults to 1.

\item {} 
\sphinxstyleliteralstrong{\sphinxupquote{port2}} (\sphinxstyleliteralemphasis{\sphinxupquote{int}}\sphinxstyleliteralemphasis{\sphinxupquote{, }}\sphinxstyleliteralemphasis{\sphinxupquote{optional}}) \textendash{} Index of Port\sphinxhyphen{}2. Defaults to 2.

\end{itemize}

\item[{Returns}] \leavevmode
ABCD data. Numpy.matrix of size (ns,4) (ns: number of frequencies). Each row contains (A,B,C,D) numbers in order.

\item[{Return type}] \leavevmode
numpy.matrix

\end{description}\end{quote}

\end{fulllineitems}

\index{s2abcd2() (touchstone.spfile method)@\spxentry{s2abcd2()}\spxextra{touchstone.spfile method}}

\begin{fulllineitems}
\phantomsection\label{\detokenize{touchstone:touchstone.spfile.s2abcd2}}\pysiglinewithargsret{\sphinxbfcode{\sphinxupquote{s2abcd2}}}{\emph{\DUrole{n}{port1}\DUrole{o}{=}\DUrole{default_value}{1}}, \emph{\DUrole{n}{port2}\DUrole{o}{=}\DUrole{default_value}{2}}}{}
S\sphinxhyphen{}Matrix to ABCD matrix conversion between 2 chosen ports. Other ports are terminated with reference impedances
\begin{quote}\begin{description}
\item[{Parameters}] \leavevmode\begin{itemize}
\item {} 
\sphinxstyleliteralstrong{\sphinxupquote{port1}} (\sphinxstyleliteralemphasis{\sphinxupquote{int}}\sphinxstyleliteralemphasis{\sphinxupquote{, }}\sphinxstyleliteralemphasis{\sphinxupquote{optional}}) \textendash{} Index of Port\sphinxhyphen{}1. Defaults to 1.

\item {} 
\sphinxstyleliteralstrong{\sphinxupquote{port2}} (\sphinxstyleliteralemphasis{\sphinxupquote{int}}\sphinxstyleliteralemphasis{\sphinxupquote{, }}\sphinxstyleliteralemphasis{\sphinxupquote{optional}}) \textendash{} Index of Port\sphinxhyphen{}2. Defaults to 2.

\end{itemize}

\item[{Returns}] \leavevmode
ABCD data. Numpy.matrix of size (ns,4) (ns: number of frequencies). Each row contains (A,B,C,D) numbers in order.

\item[{Return type}] \leavevmode
numpy.matrix

\end{description}\end{quote}

\end{fulllineitems}

\index{s2t() (touchstone.spfile method)@\spxentry{s2t()}\spxextra{touchstone.spfile method}}

\begin{fulllineitems}
\phantomsection\label{\detokenize{touchstone:touchstone.spfile.s2t}}\pysiglinewithargsret{\sphinxbfcode{\sphinxupquote{s2t}}}{}{}
Take inverse of 2\sphinxhyphen{}port data for de\sphinxhyphen{}embedding purposes.
\begin{quote}\begin{description}
\item[{Parameters}] \leavevmode
\sphinxstyleliteralstrong{\sphinxupquote{inplace}} (\sphinxstyleliteralemphasis{\sphinxupquote{int}}\sphinxstyleliteralemphasis{\sphinxupquote{, }}\sphinxstyleliteralemphasis{\sphinxupquote{optional}}) \textendash{} Object editing mode. Defaults to \sphinxhyphen{}1.

\item[{Returns}] \leavevmode
Inverted 2\sphinxhyphen{}port spfile

\item[{Return type}] \leavevmode
{\hyperref[\detokenize{touchstone:touchstone.spfile}]{\sphinxcrossref{spfile}}}

\end{description}\end{quote}

\end{fulllineitems}

\index{scaledata() (touchstone.spfile method)@\spxentry{scaledata()}\spxextra{touchstone.spfile method}}

\begin{fulllineitems}
\phantomsection\label{\detokenize{touchstone:touchstone.spfile.scaledata}}\pysiglinewithargsret{\sphinxbfcode{\sphinxupquote{scaledata}}}{\emph{\DUrole{n}{scale}\DUrole{o}{=}\DUrole{default_value}{1.0}}, \emph{\DUrole{n}{dataindices}\DUrole{o}{=}\DUrole{default_value}{None}}}{}
\end{fulllineitems}

\index{set\_formulation() (touchstone.spfile method)@\spxentry{set\_formulation()}\spxextra{touchstone.spfile method}}

\begin{fulllineitems}
\phantomsection\label{\detokenize{touchstone:touchstone.spfile.set_formulation}}\pysiglinewithargsret{\sphinxbfcode{\sphinxupquote{set\_formulation}}}{\emph{\DUrole{n}{formulation}}}{}
\end{fulllineitems}

\index{set\_frequencies\_wo\_recalc() (touchstone.spfile method)@\spxentry{set\_frequencies\_wo\_recalc()}\spxextra{touchstone.spfile method}}

\begin{fulllineitems}
\phantomsection\label{\detokenize{touchstone:touchstone.spfile.set_frequencies_wo_recalc}}\pysiglinewithargsret{\sphinxbfcode{\sphinxupquote{set\_frequencies\_wo\_recalc}}}{\emph{\DUrole{n}{freqs}}}{}
Directly sets the frequencies of this network, but does not re\sphinxhyphen{}calculate s\sphinxhyphen{}parameters.
\begin{quote}\begin{description}
\item[{Parameters}] \leavevmode
\sphinxstyleliteralstrong{\sphinxupquote{freqs}} (\sphinxstyleliteralemphasis{\sphinxupquote{list}}\sphinxstyleliteralemphasis{\sphinxupquote{ or }}\sphinxstyleliteralemphasis{\sphinxupquote{ndarray}}) \textendash{} New frequency values

\end{description}\end{quote}

\end{fulllineitems}

\index{set\_frequency\_limits() (touchstone.spfile method)@\spxentry{set\_frequency\_limits()}\spxextra{touchstone.spfile method}}

\begin{fulllineitems}
\phantomsection\label{\detokenize{touchstone:touchstone.spfile.set_frequency_limits}}\pysiglinewithargsret{\sphinxbfcode{\sphinxupquote{set\_frequency\_limits}}}{\emph{\DUrole{n}{flow}}, \emph{\DUrole{n}{fhigh}}, \emph{\DUrole{n}{inplace}\DUrole{o}{=}\DUrole{default_value}{\sphinxhyphen{} 1}}}{}
Remove frequency points higher than \sphinxstyleemphasis{fhigh} and lower than \sphinxstyleemphasis{flow}.
\begin{quote}\begin{description}
\item[{Parameters}] \leavevmode\begin{itemize}
\item {} 
\sphinxstyleliteralstrong{\sphinxupquote{flow}} (\sphinxstyleliteralemphasis{\sphinxupquote{float}}) \textendash{} Lowest Frequency (Hz)

\item {} 
\sphinxstyleliteralstrong{\sphinxupquote{fhigh}} (\sphinxstyleliteralemphasis{\sphinxupquote{float}}) \textendash{} Highest Frequency (Hz)

\item {} 
\sphinxstyleliteralstrong{\sphinxupquote{inplace}} (\sphinxstyleliteralemphasis{\sphinxupquote{int}}\sphinxstyleliteralemphasis{\sphinxupquote{, }}\sphinxstyleliteralemphasis{\sphinxupquote{optional}}) \textendash{} Object editing mode. Defaults to \sphinxhyphen{}1.

\end{itemize}

\item[{Returns}] \leavevmode
spfile object with new frequency points.

\item[{Return type}] \leavevmode
{\hyperref[\detokenize{touchstone:touchstone.spfile}]{\sphinxcrossref{spfile}}}

\end{description}\end{quote}

\end{fulllineitems}

\index{set\_frequency\_points() (touchstone.spfile method)@\spxentry{set\_frequency\_points()}\spxextra{touchstone.spfile method}}

\begin{fulllineitems}
\phantomsection\label{\detokenize{touchstone:touchstone.spfile.set_frequency_points}}\pysiglinewithargsret{\sphinxbfcode{\sphinxupquote{set\_frequency\_points}}}{\emph{\DUrole{n}{frekanslar}}, \emph{\DUrole{n}{inplace}\DUrole{o}{=}\DUrole{default_value}{\sphinxhyphen{} 1}}}{}
Set new frequency points. if S\sphinxhyphen{}Parameter data generator function is available, use that to calculate new s\sphinxhyphen{}parameter data. If not, use interpolation/extrapolation.
For new frequency points, S\sphinxhyphen{}Parameters and reference impedances which are in the form of array are re\sphinxhyphen{}calculated.
\begin{quote}\begin{description}
\item[{Parameters}] \leavevmode\begin{itemize}
\item {} 
\sphinxstyleliteralstrong{\sphinxupquote{frekanslar}} (\sphinxstyleliteralemphasis{\sphinxupquote{list}}) \textendash{} New frequency array

\item {} 
\sphinxstyleliteralstrong{\sphinxupquote{inplace}} (\sphinxstyleliteralemphasis{\sphinxupquote{int}}\sphinxstyleliteralemphasis{\sphinxupquote{, }}\sphinxstyleliteralemphasis{\sphinxupquote{optional}}) \textendash{} Object editing mode. Defaults to \sphinxhyphen{}1.

\end{itemize}

\item[{Returns}] \leavevmode
spfile object with new frequency points.

\item[{Return type}] \leavevmode
{\hyperref[\detokenize{touchstone:touchstone.spfile}]{\sphinxcrossref{spfile}}}

\end{description}\end{quote}

\end{fulllineitems}

\index{set\_frequency\_points\_array() (touchstone.spfile method)@\spxentry{set\_frequency\_points\_array()}\spxextra{touchstone.spfile method}}

\begin{fulllineitems}
\phantomsection\label{\detokenize{touchstone:touchstone.spfile.set_frequency_points_array}}\pysiglinewithargsret{\sphinxbfcode{\sphinxupquote{set\_frequency\_points\_array}}}{\emph{\DUrole{n}{fstart}}, \emph{\DUrole{n}{fstop}}, \emph{\DUrole{n}{NumberOfPoints}}, \emph{\DUrole{n}{inplace}\DUrole{o}{=}\DUrole{default_value}{\sphinxhyphen{} 1}}}{}
Set the frequencies of the object using start\sphinxhyphen{}end frequencies and number of points.
\begin{quote}\begin{description}
\item[{Parameters}] \leavevmode\begin{itemize}
\item {} 
\sphinxstyleliteralstrong{\sphinxupquote{fstart}} (\sphinxstyleliteralemphasis{\sphinxupquote{{[}}}\sphinxstyleliteralemphasis{\sphinxupquote{type}}\sphinxstyleliteralemphasis{\sphinxupquote{{]}}}) \textendash{} Start frequency.

\item {} 
\sphinxstyleliteralstrong{\sphinxupquote{fstop}} (\sphinxstyleliteralemphasis{\sphinxupquote{{[}}}\sphinxstyleliteralemphasis{\sphinxupquote{type}}\sphinxstyleliteralemphasis{\sphinxupquote{{]}}}) \textendash{} End frequency.

\item {} 
\sphinxstyleliteralstrong{\sphinxupquote{NumberOfPoints}} (\sphinxstyleliteralemphasis{\sphinxupquote{int}}) \textendash{} Number of frequencies.

\item {} 
\sphinxstyleliteralstrong{\sphinxupquote{inplace}} (\sphinxstyleliteralemphasis{\sphinxupquote{int}}\sphinxstyleliteralemphasis{\sphinxupquote{, }}\sphinxstyleliteralemphasis{\sphinxupquote{optional}}) \textendash{} Object editing mode. Defaults to \sphinxhyphen{}1.

\end{itemize}

\item[{Returns}] \leavevmode
spfile object with new frequency points.

\item[{Return type}] \leavevmode
{\hyperref[\detokenize{touchstone:touchstone.spfile}]{\sphinxcrossref{spfile}}}

\end{description}\end{quote}

\end{fulllineitems}

\index{set\_inplace() (touchstone.spfile method)@\spxentry{set\_inplace()}\spxextra{touchstone.spfile method}}

\begin{fulllineitems}
\phantomsection\label{\detokenize{touchstone:touchstone.spfile.set_inplace}}\pysiglinewithargsret{\sphinxbfcode{\sphinxupquote{set\_inplace}}}{\emph{\DUrole{n}{inplace}}}{}
\end{fulllineitems}

\index{set\_port\_name() (touchstone.spfile method)@\spxentry{set\_port\_name()}\spxextra{touchstone.spfile method}}

\begin{fulllineitems}
\phantomsection\label{\detokenize{touchstone:touchstone.spfile.set_port_name}}\pysiglinewithargsret{\sphinxbfcode{\sphinxupquote{set\_port\_name}}}{\emph{\DUrole{n}{name}}, \emph{\DUrole{n}{i}}}{}
Set name of a specific port.
\begin{quote}\begin{description}
\item[{Parameters}] \leavevmode\begin{itemize}
\item {} 
\sphinxstyleliteralstrong{\sphinxupquote{name}} (\sphinxstyleliteralemphasis{\sphinxupquote{str}}) \textendash{} New name of the port

\item {} 
\sphinxstyleliteralstrong{\sphinxupquote{i}} (\sphinxstyleliteralemphasis{\sphinxupquote{int}}) \textendash{} Port number

\end{itemize}

\end{description}\end{quote}

\end{fulllineitems}

\index{set\_port\_names() (touchstone.spfile method)@\spxentry{set\_port\_names()}\spxextra{touchstone.spfile method}}

\begin{fulllineitems}
\phantomsection\label{\detokenize{touchstone:touchstone.spfile.set_port_names}}\pysiglinewithargsret{\sphinxbfcode{\sphinxupquote{set\_port\_names}}}{\emph{\DUrole{n}{names}}}{}
Set port names with a list.
\begin{quote}\begin{description}
\item[{Parameters}] \leavevmode
\sphinxstyleliteralstrong{\sphinxupquote{names}} (\sphinxstyleliteralemphasis{\sphinxupquote{list}}) \textendash{} List of new names of the ports

\end{description}\end{quote}

\end{fulllineitems}

\index{set\_smatrix\_at\_frequency\_point() (touchstone.spfile method)@\spxentry{set\_smatrix\_at\_frequency\_point()}\spxextra{touchstone.spfile method}}

\begin{fulllineitems}
\phantomsection\label{\detokenize{touchstone:touchstone.spfile.set_smatrix_at_frequency_point}}\pysiglinewithargsret{\sphinxbfcode{\sphinxupquote{set\_smatrix\_at\_frequency\_point}}}{\emph{\DUrole{n}{indices}}, \emph{\DUrole{n}{smatrix}}}{}
Set S\sphinxhyphen{}Matrix at frequency indices
\begin{quote}\begin{description}
\item[{Parameters}] \leavevmode\begin{itemize}
\item {} 
\sphinxstyleliteralstrong{\sphinxupquote{indices}} (\sphinxstyleliteralemphasis{\sphinxupquote{list}}) \textendash{} List of frequency indices

\item {} 
\sphinxstyleliteralstrong{\sphinxupquote{smatrix}} (\sphinxstyleliteralemphasis{\sphinxupquote{numpy.matrix}}) \textendash{} New S\sphinxhyphen{}Matrix value which is to be set at all \sphinxstyleemphasis{indices}

\end{itemize}

\end{description}\end{quote}

\end{fulllineitems}

\index{set\_sparam\_gen\_func() (touchstone.spfile method)@\spxentry{set\_sparam\_gen\_func()}\spxextra{touchstone.spfile method}}

\begin{fulllineitems}
\phantomsection\label{\detokenize{touchstone:touchstone.spfile.set_sparam_gen_func}}\pysiglinewithargsret{\sphinxbfcode{\sphinxupquote{set\_sparam\_gen\_func}}}{\emph{\DUrole{n}{func}\DUrole{o}{=}\DUrole{default_value}{None}}}{}
This function is used to set the function that generates s\sphinxhyphen{}parameters from frequency.
\begin{quote}\begin{description}
\item[{Parameters}] \leavevmode
\sphinxstyleliteralstrong{\sphinxupquote{func}} (\sphinxstyleliteralemphasis{\sphinxupquote{function}}\sphinxstyleliteralemphasis{\sphinxupquote{, }}\sphinxstyleliteralemphasis{\sphinxupquote{optional}}) \textendash{} function to be set. Defaults to None.

\end{description}\end{quote}

\end{fulllineitems}

\index{set\_sparam\_mod\_func() (touchstone.spfile method)@\spxentry{set\_sparam\_mod\_func()}\spxextra{touchstone.spfile method}}

\begin{fulllineitems}
\phantomsection\label{\detokenize{touchstone:touchstone.spfile.set_sparam_mod_func}}\pysiglinewithargsret{\sphinxbfcode{\sphinxupquote{set\_sparam\_mod\_func}}}{\emph{\DUrole{n}{func}\DUrole{o}{=}\DUrole{default_value}{None}}}{}
This function is used to set the function that generates s\sphinxhyphen{}parameters from frequency.
\begin{quote}\begin{description}
\item[{Parameters}] \leavevmode
\sphinxstyleliteralstrong{\sphinxupquote{func}} (\sphinxstyleliteralemphasis{\sphinxupquote{function}}\sphinxstyleliteralemphasis{\sphinxupquote{, }}\sphinxstyleliteralemphasis{\sphinxupquote{optional}}) \textendash{} function to be set. Defaults to None.

\end{description}\end{quote}

\end{fulllineitems}

\index{set\_sym\_parameters() (touchstone.spfile method)@\spxentry{set\_sym\_parameters()}\spxextra{touchstone.spfile method}}

\begin{fulllineitems}
\phantomsection\label{\detokenize{touchstone:touchstone.spfile.set_sym_parameters}}\pysiglinewithargsret{\sphinxbfcode{\sphinxupquote{set\_sym\_parameters}}}{\emph{\DUrole{n}{paramdict}}}{}
This function is used to set the values of symbolic variables of the network. This is used if the S\sphinxhyphen{}Matrix of the network is defined by an arithmetic expression containing symbolic variables. This property is used in conjunction with \sphinxstyleemphasis{sympy} library for symbolic manipulation. Arithmetic expression for S\sphinxhyphen{}Matrix is defined by \sphinxcode{\sphinxupquote{set\_sym\_smatrix}} function.
\begin{quote}\begin{description}
\item[{Parameters}] \leavevmode
\sphinxstyleliteralstrong{\sphinxupquote{paramdict}} (\sphinxstyleliteralemphasis{\sphinxupquote{dict}}) \textendash{} This is a dictionary containing the values of symbolic variables of the network

\end{description}\end{quote}

\end{fulllineitems}

\index{set\_sym\_smatrix() (touchstone.spfile method)@\spxentry{set\_sym\_smatrix()}\spxextra{touchstone.spfile method}}

\begin{fulllineitems}
\phantomsection\label{\detokenize{touchstone:touchstone.spfile.set_sym_smatrix}}\pysiglinewithargsret{\sphinxbfcode{\sphinxupquote{set\_sym\_smatrix}}}{\emph{\DUrole{n}{SM}}}{}
This function is used to set arithmetic expression for S\sphinxhyphen{}Matrix, if S\sphinxhyphen{}Matrix is defined using symbolic variables.
\begin{quote}\begin{description}
\item[{Parameters}] \leavevmode
\sphinxstyleliteralstrong{\sphinxupquote{SM}} (\sphinxstyleliteralemphasis{\sphinxupquote{sympy.Matrix}}) \textendash{} Symbolic \sphinxcode{\sphinxupquote{sympy.Matrix}} expression for S\sphinxhyphen{}Parameter matrix

\end{description}\end{quote}

\end{fulllineitems}

\index{setdataformat() (touchstone.spfile method)@\spxentry{setdataformat()}\spxextra{touchstone.spfile method}}

\begin{fulllineitems}
\phantomsection\label{\detokenize{touchstone:touchstone.spfile.setdataformat}}\pysiglinewithargsret{\sphinxbfcode{\sphinxupquote{setdataformat}}}{\emph{\DUrole{n}{dataformat}}}{}
\end{fulllineitems}

\index{setdatapoint() (touchstone.spfile method)@\spxentry{setdatapoint()}\spxextra{touchstone.spfile method}}

\begin{fulllineitems}
\phantomsection\label{\detokenize{touchstone:touchstone.spfile.setdatapoint}}\pysiglinewithargsret{\sphinxbfcode{\sphinxupquote{setdatapoint}}}{\emph{\DUrole{n}{m}}, \emph{\DUrole{n}{indices}}, \emph{\DUrole{n}{x}}}{}
Set the value for some part of S\sphinxhyphen{}Parameter data.
\begin{quote}
\begin{equation*}
\begin{split}S_{i j}[m:m+len(x)]=x\end{split}
\end{equation*}\end{quote}
\begin{quote}\begin{description}
\item[{Parameters}] \leavevmode\begin{itemize}
\item {} 
\sphinxstyleliteralstrong{\sphinxupquote{m}} (\sphinxstyleliteralemphasis{\sphinxupquote{int}}) \textendash{} Starting frequency indice

\item {} 
\sphinxstyleliteralstrong{\sphinxupquote{indices}} (\sphinxstyleliteralemphasis{\sphinxupquote{tuple of int}}) \textendash{} Parameters to be set (i,j)

\item {} 
\sphinxstyleliteralstrong{\sphinxupquote{x}} (\sphinxstyleliteralemphasis{\sphinxupquote{number}}\sphinxstyleliteralemphasis{\sphinxupquote{ or }}\sphinxstyleliteralemphasis{\sphinxupquote{list}}) \textendash{} New value. If this is a number, it is converted to a list.

\end{itemize}

\end{description}\end{quote}

\end{fulllineitems}

\index{smoothing() (touchstone.spfile method)@\spxentry{smoothing()}\spxextra{touchstone.spfile method}}

\begin{fulllineitems}
\phantomsection\label{\detokenize{touchstone:touchstone.spfile.smoothing}}\pysiglinewithargsret{\sphinxbfcode{\sphinxupquote{smoothing}}}{\emph{\DUrole{n}{smoothing\_length}\DUrole{o}{=}\DUrole{default_value}{5}}, \emph{\DUrole{n}{inplace}\DUrole{o}{=}\DUrole{default_value}{\sphinxhyphen{} 1}}}{}
This method applies moving average smoothing to the s\sphinxhyphen{}parameter data
\begin{quote}\begin{description}
\item[{Parameters}] \leavevmode\begin{itemize}
\item {} 
\sphinxstyleliteralstrong{\sphinxupquote{smoothing\_length}} (\sphinxstyleliteralemphasis{\sphinxupquote{int}}\sphinxstyleliteralemphasis{\sphinxupquote{, }}\sphinxstyleliteralemphasis{\sphinxupquote{optional}}) \textendash{} Number of points used for smoothing. Defaults to 5.

\item {} 
\sphinxstyleliteralstrong{\sphinxupquote{inplace}} (\sphinxstyleliteralemphasis{\sphinxupquote{int}}\sphinxstyleliteralemphasis{\sphinxupquote{, }}\sphinxstyleliteralemphasis{\sphinxupquote{optional}}) \textendash{} object editing mode. Defaults to \sphinxhyphen{}1.

\end{itemize}

\item[{Returns}] \leavevmode
Network object with smooth data

\item[{Return type}] \leavevmode
{\hyperref[\detokenize{touchstone:touchstone.spfile}]{\sphinxcrossref{spfile}}}

\end{description}\end{quote}

\end{fulllineitems}

\index{snp2smp() (touchstone.spfile method)@\spxentry{snp2smp()}\spxextra{touchstone.spfile method}}

\begin{fulllineitems}
\phantomsection\label{\detokenize{touchstone:touchstone.spfile.snp2smp}}\pysiglinewithargsret{\sphinxbfcode{\sphinxupquote{snp2smp}}}{\emph{\DUrole{n}{ports}}, \emph{\DUrole{n}{inplace}\DUrole{o}{=}\DUrole{default_value}{\sphinxhyphen{} 1}}}{}
This method changes the port numbering of the network
port j of new network corresponds to ports{[}j{]} in old network.

if the length of “ports” argument is lower than number of ports, remaining ports are terminated with current reference impedances and number of ports are reduced.
\begin{quote}\begin{description}
\item[{Parameters}] \leavevmode\begin{itemize}
\item {} 
\sphinxstyleliteralstrong{\sphinxupquote{ports}} (\sphinxstyleliteralemphasis{\sphinxupquote{list}}) \textendash{} New port order

\item {} 
\sphinxstyleliteralstrong{\sphinxupquote{inplace}} (\sphinxstyleliteralemphasis{\sphinxupquote{int}}\sphinxstyleliteralemphasis{\sphinxupquote{, }}\sphinxstyleliteralemphasis{\sphinxupquote{optional}}) \textendash{} Object editing mode. Defaults to \sphinxhyphen{}1.

\end{itemize}

\item[{Returns}] \leavevmode
Modified spfile object

\item[{Return type}] \leavevmode
{\hyperref[\detokenize{touchstone:touchstone.spfile}]{\sphinxcrossref{spfile}}}

\end{description}\end{quote}

\end{fulllineitems}

\index{stability\_factor\_k() (touchstone.spfile method)@\spxentry{stability\_factor\_k()}\spxextra{touchstone.spfile method}}

\begin{fulllineitems}
\phantomsection\label{\detokenize{touchstone:touchstone.spfile.stability_factor_k}}\pysiglinewithargsret{\sphinxbfcode{\sphinxupquote{stability\_factor\_k}}}{\emph{\DUrole{n}{port1}\DUrole{o}{=}\DUrole{default_value}{1}}, \emph{\DUrole{n}{port2}\DUrole{o}{=}\DUrole{default_value}{2}}}{}
Calculates \sphinxstyleemphasis{k} stability factor, from port1 to port2. Other ports are terminated with reference impedances.
\begin{quote}\begin{description}
\item[{Parameters}] \leavevmode\begin{itemize}
\item {} 
\sphinxstyleliteralstrong{\sphinxupquote{port1}} (\sphinxstyleliteralemphasis{\sphinxupquote{int}}\sphinxstyleliteralemphasis{\sphinxupquote{, }}\sphinxstyleliteralemphasis{\sphinxupquote{optional}}) \textendash{} Index of source port. Defaults to 1.

\item {} 
\sphinxstyleliteralstrong{\sphinxupquote{port2}} (\sphinxstyleliteralemphasis{\sphinxupquote{int}}\sphinxstyleliteralemphasis{\sphinxupquote{, }}\sphinxstyleliteralemphasis{\sphinxupquote{optional}}) \textendash{} Index of load port. Defaults to 2.

\end{itemize}

\item[{Returns}] \leavevmode
Array of stability factor for all frequencies

\item[{Return type}] \leavevmode
numpy.ndarray

\end{description}\end{quote}

\end{fulllineitems}

\index{stability\_factor\_mu1() (touchstone.spfile method)@\spxentry{stability\_factor\_mu1()}\spxextra{touchstone.spfile method}}

\begin{fulllineitems}
\phantomsection\label{\detokenize{touchstone:touchstone.spfile.stability_factor_mu1}}\pysiglinewithargsret{\sphinxbfcode{\sphinxupquote{stability\_factor\_mu1}}}{\emph{\DUrole{n}{port1}\DUrole{o}{=}\DUrole{default_value}{1}}, \emph{\DUrole{n}{port2}\DUrole{o}{=}\DUrole{default_value}{2}}}{}
Calculates \(\mu_1\) stability factor, from port1 to port2. Other ports are terminated with reference impedances.
\begin{quote}\begin{description}
\item[{Parameters}] \leavevmode\begin{itemize}
\item {} 
\sphinxstyleliteralstrong{\sphinxupquote{port1}} (\sphinxstyleliteralemphasis{\sphinxupquote{int}}\sphinxstyleliteralemphasis{\sphinxupquote{, }}\sphinxstyleliteralemphasis{\sphinxupquote{optional}}) \textendash{} Index of source port. Defaults to 1.

\item {} 
\sphinxstyleliteralstrong{\sphinxupquote{port2}} (\sphinxstyleliteralemphasis{\sphinxupquote{int}}\sphinxstyleliteralemphasis{\sphinxupquote{, }}\sphinxstyleliteralemphasis{\sphinxupquote{optional}}) \textendash{} Index of load port. Defaults to 2.

\end{itemize}

\item[{Returns}] \leavevmode
Array of stability factor for all frequencies

\item[{Return type}] \leavevmode
numpy.ndarray

\end{description}\end{quote}

\end{fulllineitems}

\index{stability\_factor\_mu2() (touchstone.spfile method)@\spxentry{stability\_factor\_mu2()}\spxextra{touchstone.spfile method}}

\begin{fulllineitems}
\phantomsection\label{\detokenize{touchstone:touchstone.spfile.stability_factor_mu2}}\pysiglinewithargsret{\sphinxbfcode{\sphinxupquote{stability\_factor\_mu2}}}{\emph{\DUrole{n}{port1}\DUrole{o}{=}\DUrole{default_value}{1}}, \emph{\DUrole{n}{port2}\DUrole{o}{=}\DUrole{default_value}{2}}}{}
Calculates \(\mu_2\) stability factor, from port1 to port2. Other ports are terminated with reference impedances.
\begin{quote}\begin{description}
\item[{Parameters}] \leavevmode\begin{itemize}
\item {} 
\sphinxstyleliteralstrong{\sphinxupquote{port1}} (\sphinxstyleliteralemphasis{\sphinxupquote{int}}\sphinxstyleliteralemphasis{\sphinxupquote{, }}\sphinxstyleliteralemphasis{\sphinxupquote{optional}}) \textendash{} Index of source port. Defaults to 1.

\item {} 
\sphinxstyleliteralstrong{\sphinxupquote{port2}} (\sphinxstyleliteralemphasis{\sphinxupquote{int}}\sphinxstyleliteralemphasis{\sphinxupquote{, }}\sphinxstyleliteralemphasis{\sphinxupquote{optional}}) \textendash{} Index of load port. Defaults to 2.

\end{itemize}

\item[{Returns}] \leavevmode
Array of stability factor for all frequencies

\item[{Return type}] \leavevmode
numpy.ndarray

\end{description}\end{quote}

\end{fulllineitems}

\index{stripline() (touchstone.spfile class method)@\spxentry{stripline()}\spxextra{touchstone.spfile class method}}

\begin{fulllineitems}
\phantomsection\label{\detokenize{touchstone:touchstone.spfile.stripline}}\pysiglinewithargsret{\sphinxbfcode{\sphinxupquote{classmethod }}\sphinxbfcode{\sphinxupquote{stripline}}}{\emph{\DUrole{n}{length}}, \emph{\DUrole{n}{w}}, \emph{\DUrole{n}{er}}, \emph{\DUrole{n}{h1}}, \emph{\DUrole{n}{h2}}, \emph{\DUrole{n}{t}}, \emph{\DUrole{n}{freqs}\DUrole{o}{=}\DUrole{default_value}{None}}}{}
Create an \sphinxcode{\sphinxupquote{spfile}} object corresponding to a stripline transmission line.
\begin{quote}\begin{description}
\item[{Parameters}] \leavevmode\begin{itemize}
\item {} 
\sphinxstyleliteralstrong{\sphinxupquote{length}} (\sphinxstyleliteralemphasis{\sphinxupquote{float}}) \textendash{} Length of cpwg line.

\item {} 
\sphinxstyleliteralstrong{\sphinxupquote{w}} (\sphinxstyleliteralemphasis{\sphinxupquote{float}}) \textendash{} Width of stripline.

\item {} 
\sphinxstyleliteralstrong{\sphinxupquote{er}} (\sphinxstyleliteralemphasis{\sphinxupquote{float}}) \textendash{} Relative permittivity of substrate.

\item {} 
\sphinxstyleliteralstrong{\sphinxupquote{h1}} (\sphinxstyleliteralemphasis{\sphinxupquote{float}}) \textendash{} Thickness of substrate from bottom ground to bottom of line.

\item {} 
\sphinxstyleliteralstrong{\sphinxupquote{h2}} (\sphinxstyleliteralemphasis{\sphinxupquote{float}}) \textendash{} Thickness of substrate from top line to top ground.

\item {} 
\sphinxstyleliteralstrong{\sphinxupquote{t}} (\sphinxstyleliteralemphasis{\sphinxupquote{float}}) \textendash{} Thickness of metal.

\item {} 
\sphinxstyleliteralstrong{\sphinxupquote{freqs}} (\sphinxstyleliteralemphasis{\sphinxupquote{float}}\sphinxstyleliteralemphasis{\sphinxupquote{, }}\sphinxstyleliteralemphasis{\sphinxupquote{optional}}) \textendash{} Frequency list of object. Defaults to None. If not given, frequencies should be set later.

\end{itemize}

\item[{Returns}] \leavevmode
An spfile object.

\item[{Return type}] \leavevmode
{\hyperref[\detokenize{touchstone:touchstone.spfile}]{\sphinxcrossref{spfile}}}

\end{description}\end{quote}

\end{fulllineitems}

\index{striplinestep() (touchstone.spfile class method)@\spxentry{striplinestep()}\spxextra{touchstone.spfile class method}}

\begin{fulllineitems}
\phantomsection\label{\detokenize{touchstone:touchstone.spfile.striplinestep}}\pysiglinewithargsret{\sphinxbfcode{\sphinxupquote{classmethod }}\sphinxbfcode{\sphinxupquote{striplinestep}}}{\emph{\DUrole{n}{w1}}, \emph{\DUrole{n}{w2}}, \emph{\DUrole{n}{eps\_r}}, \emph{\DUrole{n}{h1}}, \emph{\DUrole{n}{h2}}, \emph{\DUrole{n}{t}}, \emph{\DUrole{n}{freqs}\DUrole{o}{=}\DUrole{default_value}{None}}}{}
Create an \sphinxcode{\sphinxupquote{spfile}} object corresponding to a stripline step
\begin{quote}\begin{description}
\item[{Parameters}] \leavevmode\begin{itemize}
\item {} 
\sphinxstyleliteralstrong{\sphinxupquote{w1}} (\sphinxstyleliteralemphasis{\sphinxupquote{float}}) \textendash{} Width of stripline line at port\sphinxhyphen{}1.

\item {} 
\sphinxstyleliteralstrong{\sphinxupquote{w2}} (\sphinxstyleliteralemphasis{\sphinxupquote{float}}) \textendash{} Width of stripline line at port\sphinxhyphen{}2.

\item {} 
\sphinxstyleliteralstrong{\sphinxupquote{eps\_r}} (\sphinxstyleliteralemphasis{\sphinxupquote{float}}) \textendash{} Relative permittivity of stripline substrate.

\item {} 
\sphinxstyleliteralstrong{\sphinxupquote{h}} (\sphinxstyleliteralemphasis{\sphinxupquote{float}}) \textendash{} Thickness of stripline substrate.

\item {} 
\sphinxstyleliteralstrong{\sphinxupquote{t}} (\sphinxstyleliteralemphasis{\sphinxupquote{float}}) \textendash{} Thickness of metal.

\item {} 
\sphinxstyleliteralstrong{\sphinxupquote{freqs}} (\sphinxstyleliteralemphasis{\sphinxupquote{float}}\sphinxstyleliteralemphasis{\sphinxupquote{, }}\sphinxstyleliteralemphasis{\sphinxupquote{optional}}) \textendash{} Frequency list of object. Defaults to None. If not given, frequencies should be set later.

\end{itemize}

\item[{Returns}] \leavevmode
An spfile object.

\item[{Return type}] \leavevmode
{\hyperref[\detokenize{touchstone:touchstone.spfile}]{\sphinxcrossref{spfile}}}

\end{description}\end{quote}

\end{fulllineitems}

\index{write2file() (touchstone.spfile method)@\spxentry{write2file()}\spxextra{touchstone.spfile method}}

\begin{fulllineitems}
\phantomsection\label{\detokenize{touchstone:touchstone.spfile.write2file}}\pysiglinewithargsret{\sphinxbfcode{\sphinxupquote{write2file}}}{\emph{\DUrole{n}{newfilename}\DUrole{o}{=}\DUrole{default_value}{\textquotesingle{}\textquotesingle{}}}, \emph{\DUrole{n}{parameter}\DUrole{o}{=}\DUrole{default_value}{\textquotesingle{}S\textquotesingle{}}}, \emph{\DUrole{n}{freq\_unit}\DUrole{o}{=}\DUrole{default_value}{\textquotesingle{}\textquotesingle{}}}, \emph{\DUrole{n}{dataformat}\DUrole{o}{=}\DUrole{default_value}{\textquotesingle{}\textquotesingle{}}}}{}
This function writes a parameter (S, Y or Z) file. If the filename given does not have the proper filename extension, it is corrected.
\begin{quote}\begin{description}
\item[{Parameters}] \leavevmode\begin{itemize}
\item {} 
\sphinxstyleliteralstrong{\sphinxupquote{newfilename}} (\sphinxstyleliteralemphasis{\sphinxupquote{str}}\sphinxstyleliteralemphasis{\sphinxupquote{, }}\sphinxstyleliteralemphasis{\sphinxupquote{optional}}) \textendash{} Filename to be written. Defaults to “”.

\item {} 
\sphinxstyleliteralstrong{\sphinxupquote{parameter}} (\sphinxstyleliteralemphasis{\sphinxupquote{str}}\sphinxstyleliteralemphasis{\sphinxupquote{, }}\sphinxstyleliteralemphasis{\sphinxupquote{optional}}) \textendash{} Parameter to be written (S, Y or Z). Defaults to “S”.

\item {} 
\sphinxstyleliteralstrong{\sphinxupquote{freq\_unit}} (\sphinxstyleliteralemphasis{\sphinxupquote{str}}\sphinxstyleliteralemphasis{\sphinxupquote{, }}\sphinxstyleliteralemphasis{\sphinxupquote{optional}}) \textendash{} Frequency unit (GHz, MHz, kHz or Hz). Defaults to “Hz”.

\item {} 
\sphinxstyleliteralstrong{\sphinxupquote{dataformat}} (\sphinxstyleliteralemphasis{\sphinxupquote{str}}\sphinxstyleliteralemphasis{\sphinxupquote{, }}\sphinxstyleliteralemphasis{\sphinxupquote{optional}}) \textendash{} Format of file DB, RI or MA. Defaults to “”.

\end{itemize}

\end{description}\end{quote}

\end{fulllineitems}


\end{fulllineitems}

\index{thru\_line\_deembedding() (in module touchstone)@\spxentry{thru\_line\_deembedding()}\spxextra{in module touchstone}}

\begin{fulllineitems}
\phantomsection\label{\detokenize{touchstone:touchstone.thru_line_deembedding}}\pysiglinewithargsret{\sphinxcode{\sphinxupquote{touchstone.}}\sphinxbfcode{\sphinxupquote{thru\_line\_deembedding}}}{\emph{\DUrole{n}{thru\_filename}}, \emph{\DUrole{n}{line\_filename}}}{}
Extraction of transition s\sphinxhyphen{}parameters from THRU and LINE measurements. Transitions on both sides are assumed to be identical. For output \sphinxstyleemphasis{spfile} objects, port\sphinxhyphen{}1 is launcher side and port\sphinxhyphen{}2 is transmission line side. The length difference between LINE and THRU should be ideally \(\lambda/4\).
The reference impedance for the 2. port of the transition should be the same as the characteristic impedance of the interconnecting line. So the reference impedances of the output \sphinxstyleemphasis{spfile} should be adjusted (without renormalizing s\sphinxhyphen{}parameters) after calling this function. The minimum frequency in the S\sphinxhyphen{}Parameter files should be such that the phase difference between the measurements should be smaller than 2:math:\sphinxtitleref{pi}.
\begin{quote}\begin{description}
\item[{Parameters}] \leavevmode\begin{itemize}
\item {} 
\sphinxstyleliteralstrong{\sphinxupquote{thru\_filename}} (\sphinxstyleliteralemphasis{\sphinxupquote{str}}) \textendash{} 2\sphinxhyphen{}Port S\sphinxhyphen{}Parameter filename of THRU measurement

\item {} 
\sphinxstyleliteralstrong{\sphinxupquote{line\_filename}} (\sphinxstyleliteralemphasis{\sphinxupquote{str}}) \textendash{} 2\sphinxhyphen{}Port S\sphinxhyphen{}Parameter filename of LINE measurement

\end{itemize}

\item[{Returns}] \leavevmode
2\sphinxhyphen{}Element tuple of (transition spfile, complex phase vector (\(-\gamma l\)) of connecting line of LINE standard (in radian))

\item[{Return type}] \leavevmode
tuple({\hyperref[\detokenize{touchstone:touchstone.spfile}]{\sphinxcrossref{spfile}}}, numpy.ndarray)

\end{description}\end{quote}

\end{fulllineitems}

\index{trl\_launcher\_extraction() (in module touchstone)@\spxentry{trl\_launcher\_extraction()}\spxextra{in module touchstone}}

\begin{fulllineitems}
\phantomsection\label{\detokenize{touchstone:touchstone.trl_launcher_extraction}}\pysiglinewithargsret{\sphinxcode{\sphinxupquote{touchstone.}}\sphinxbfcode{\sphinxupquote{trl\_launcher\_extraction}}}{\emph{\DUrole{n}{thru\_filename}}, \emph{\DUrole{n}{line\_filename}}, \emph{\DUrole{n}{reflect\_filename}}, \emph{\DUrole{n}{refstd}\DUrole{o}{=}\DUrole{default_value}{False}}}{}
Extraction of launcher s\sphinxhyphen{}parameters by THRU, LINE, REFLECT calibration. For output \sphinxstyleemphasis{spfile} objects, port\sphinxhyphen{}1 is launcher side and port\sphinxhyphen{}2 is transmission line side.
Reference: TRL algorithm to de\sphinxhyphen{}embed a RF test fixture.pdf
\begin{quote}\begin{description}
\item[{Parameters}] \leavevmode\begin{itemize}
\item {} 
\sphinxstyleliteralstrong{\sphinxupquote{thru\_filename}} (\sphinxstyleliteralemphasis{\sphinxupquote{str}}) \textendash{} 2\sphinxhyphen{}Port S\sphinxhyphen{}Parameter filename of THRU measurement

\item {} 
\sphinxstyleliteralstrong{\sphinxupquote{line\_filename}} (\sphinxstyleliteralemphasis{\sphinxupquote{str}}) \textendash{} 2\sphinxhyphen{}Port S\sphinxhyphen{}Parameter filename of LINE measurement

\item {} 
\sphinxstyleliteralstrong{\sphinxupquote{reflect\_filename}} (\sphinxstyleliteralemphasis{\sphinxupquote{str}}) \textendash{} 2\sphinxhyphen{}Port S\sphinxhyphen{}Parameter filename of REFLECT measurement

\item {} 
\sphinxstyleliteralstrong{\sphinxupquote{refstd}} (\sphinxstyleliteralemphasis{\sphinxupquote{boolean}}) \textendash{} True if OPEN is used as REFLECT standard and False (default) if SHORT is used

\end{itemize}

\item[{Returns}] \leavevmode
3\sphinxhyphen{}Element tuple of (left side launcher spfile, right side launcher spfile, phase vector of connecting line of LINE standard (in radian) )

\item[{Return type}] \leavevmode
tuple({\hyperref[\detokenize{touchstone:touchstone.spfile}]{\sphinxcrossref{spfile}}}, sofile, numpy.ndarray)

\end{description}\end{quote}

\end{fulllineitems}

\index{untermination\_method() (in module touchstone)@\spxentry{untermination\_method()}\spxextra{in module touchstone}}

\begin{fulllineitems}
\phantomsection\label{\detokenize{touchstone:touchstone.untermination_method}}\pysiglinewithargsret{\sphinxcode{\sphinxupquote{touchstone.}}\sphinxbfcode{\sphinxupquote{untermination\_method}}}{\emph{\DUrole{n}{g1}}, \emph{\DUrole{n}{g2}}, \emph{\DUrole{n}{g3}}, \emph{\DUrole{n}{gL1}}, \emph{\DUrole{n}{gL2}}, \emph{\DUrole{n}{gL3}}, \emph{\DUrole{n}{returnS2P}\DUrole{o}{=}\DUrole{default_value}{False}}, \emph{\DUrole{n}{freqs}\DUrole{o}{=}\DUrole{default_value}{None}}}{}
Determination of \(S_{11}\), \(S_{22}\) and \(S_{21}=S_{12}\) for a 2\sphinxhyphen{}port network network using 3 reflection coefficient values at port\sphinxhyphen{}1 for 3 terminations at port\sphinxhyphen{}2. \(S_{21}\) can only be calculated with a sign ambiguity because it exists only as square in the formulae.

Port\sphinxhyphen{}1: Input port.
Port\sphinxhyphen{}2: Output port where load impedances are switched.
\begin{quote}\begin{description}
\item[{Parameters}] \leavevmode\begin{itemize}
\item {} 
\sphinxstyleliteralstrong{\sphinxupquote{g1}} (\sphinxstyleliteralemphasis{\sphinxupquote{float}}\sphinxstyleliteralemphasis{\sphinxupquote{, }}\sphinxstyleliteralemphasis{\sphinxupquote{complex}}\sphinxstyleliteralemphasis{\sphinxupquote{ or }}\sphinxstyleliteralemphasis{\sphinxupquote{ndarray}}) \textendash{} Reflection coefficient at port\sphinxhyphen{}1 when port\sphinxhyphen{}2 is terminated by a load with reflection coefficient gL1

\item {} 
\sphinxstyleliteralstrong{\sphinxupquote{g2}} (\sphinxstyleliteralemphasis{\sphinxupquote{float}}\sphinxstyleliteralemphasis{\sphinxupquote{, }}\sphinxstyleliteralemphasis{\sphinxupquote{complex}}\sphinxstyleliteralemphasis{\sphinxupquote{ or }}\sphinxstyleliteralemphasis{\sphinxupquote{ndarray}}) \textendash{} Reflection coefficient at port\sphinxhyphen{}1 when port\sphinxhyphen{}2 is terminated by a load with reflection coefficient gL2

\item {} 
\sphinxstyleliteralstrong{\sphinxupquote{g3}} (\sphinxstyleliteralemphasis{\sphinxupquote{float}}\sphinxstyleliteralemphasis{\sphinxupquote{, }}\sphinxstyleliteralemphasis{\sphinxupquote{complex}}\sphinxstyleliteralemphasis{\sphinxupquote{ or }}\sphinxstyleliteralemphasis{\sphinxupquote{ndarray}}) \textendash{} Reflection coefficient at port\sphinxhyphen{}1 when port\sphinxhyphen{}2 is terminated by a load with reflection coefficient gL3

\item {} 
\sphinxstyleliteralstrong{\sphinxupquote{gL1}} (\sphinxstyleliteralemphasis{\sphinxupquote{float}}\sphinxstyleliteralemphasis{\sphinxupquote{, }}\sphinxstyleliteralemphasis{\sphinxupquote{complex}}\sphinxstyleliteralemphasis{\sphinxupquote{ or }}\sphinxstyleliteralemphasis{\sphinxupquote{ndarray}}) \textendash{} Reflection coefficient of load at port\sphinxhyphen{}2 that gives g1 reflection coefficient at port\sphinxhyphen{}1

\item {} 
\sphinxstyleliteralstrong{\sphinxupquote{gL2}} (\sphinxstyleliteralemphasis{\sphinxupquote{float}}\sphinxstyleliteralemphasis{\sphinxupquote{, }}\sphinxstyleliteralemphasis{\sphinxupquote{complex}}\sphinxstyleliteralemphasis{\sphinxupquote{ or }}\sphinxstyleliteralemphasis{\sphinxupquote{ndarray}}) \textendash{} Reflection coefficient of load at port\sphinxhyphen{}2 that gives g2 reflection coefficient at port\sphinxhyphen{}1

\item {} 
\sphinxstyleliteralstrong{\sphinxupquote{gL3}} (\sphinxstyleliteralemphasis{\sphinxupquote{float}}\sphinxstyleliteralemphasis{\sphinxupquote{, }}\sphinxstyleliteralemphasis{\sphinxupquote{complex}}\sphinxstyleliteralemphasis{\sphinxupquote{ or }}\sphinxstyleliteralemphasis{\sphinxupquote{ndarray}}) \textendash{} Reflection coefficient of load at port\sphinxhyphen{}2 that gives g3 reflection coefficient at port\sphinxhyphen{}1

\item {} 
\sphinxstyleliteralstrong{\sphinxupquote{returnS2P}} (\sphinxstyleliteralemphasis{\sphinxupquote{boolean}}) \textendash{} If True, function returns an \sphinxstyleemphasis{spfile} object of the 2\sphinxhyphen{}port network, if False, it returns 3\sphinxhyphen{}tuple of S\sphinxhyphen{}Parameter arrays. Default is False.

\item {} 
\sphinxstyleliteralstrong{\sphinxupquote{freqs}} (\sphinxstyleliteralemphasis{\sphinxupquote{numpy.ndarray}}\sphinxstyleliteralemphasis{\sphinxupquote{, }}\sphinxstyleliteralemphasis{\sphinxupquote{list}}) \textendash{} If returnS2P is True, this input is used as the frequency points of the returned \sphinxstyleemphasis{spfile} object. Default is None.

\end{itemize}

\item[{Returns}] \leavevmode
Either 3\sphinxhyphen{}Element tuple of (S11, S22, S21) or \sphinxstyleemphasis{spfile} object, depending on returnS2P input.

\item[{Return type}] \leavevmode
tuple

\end{description}\end{quote}

\end{fulllineitems}

\index{writeImpedanceAsS1p() (in module touchstone)@\spxentry{writeImpedanceAsS1p()}\spxextra{in module touchstone}}

\begin{fulllineitems}
\phantomsection\label{\detokenize{touchstone:touchstone.writeImpedanceAsS1p}}\pysiglinewithargsret{\sphinxcode{\sphinxupquote{touchstone.}}\sphinxbfcode{\sphinxupquote{writeImpedanceAsS1p}}}{\emph{\DUrole{n}{filename}}, \emph{\DUrole{n}{frequencies}}, \emph{\DUrole{n}{Z}}}{}
\end{fulllineitems}



\chapter{components module}
\label{\detokenize{components:module-components}}\label{\detokenize{components:components-module}}\label{\detokenize{components::doc}}\index{module@\spxentry{module}!components@\spxentry{components}}\index{components@\spxentry{components}!module@\spxentry{module}}
Created on Tue Nov 17 11:52:33 2009

@author: Tuncay
\index{AWG2Dia() (in module components)@\spxentry{AWG2Dia()}\spxextra{in module components}}

\begin{fulllineitems}
\phantomsection\label{\detokenize{components:components.AWG2Dia}}\pysiglinewithargsret{\sphinxcode{\sphinxupquote{components.}}\sphinxbfcode{\sphinxupquote{AWG2Dia}}}{\emph{\DUrole{n}{arg}}, \emph{\DUrole{n}{defaultunits}\DUrole{o}{=}\DUrole{default_value}{{[}{]}}}}{}
Convert AWG to Diameter.
\begin{quote}\begin{description}
\item[{Parameters}] \leavevmode\begin{itemize}
\item {} 
\sphinxstyleliteralstrong{\sphinxupquote{arg}} (\sphinxstyleliteralemphasis{\sphinxupquote{list}}) \textendash{} 
First 1 arguments are inputs.
\begin{enumerate}
\sphinxsetlistlabels{\arabic}{enumi}{enumii}{}{.}%
\item {} 
AWG ;

\item {} 
Diameter ;length

\end{enumerate}

3. Current rating in still air ; current
Reference:  Wikipedia, Current rating is calculated through curve fit from online data


\item {} 
\sphinxstyleliteralstrong{\sphinxupquote{defaultunits}} (\sphinxstyleliteralemphasis{\sphinxupquote{list}}\sphinxstyleliteralemphasis{\sphinxupquote{, }}\sphinxstyleliteralemphasis{\sphinxupquote{optional}}) \textendash{} Default units for quantities in \sphinxstyleemphasis{arg} list. Default is {[}{]} which means SI units will be used if no unit is given in \sphinxstyleemphasis{arg}.

\end{itemize}

\item[{Returns}] \leavevmode
arg

\item[{Return type}] \leavevmode
list

\end{description}\end{quote}

\end{fulllineitems}

\index{Absorptive\_Filter\_Equalizer() (in module components)@\spxentry{Absorptive\_Filter\_Equalizer()}\spxextra{in module components}}

\begin{fulllineitems}
\phantomsection\label{\detokenize{components:components.Absorptive_Filter_Equalizer}}\pysiglinewithargsret{\sphinxcode{\sphinxupquote{components.}}\sphinxbfcode{\sphinxupquote{Absorptive\_Filter\_Equalizer}}}{\emph{\DUrole{n}{arg}}, \emph{\DUrole{n}{defaultunits}\DUrole{o}{=}\DUrole{default_value}{{[}{]}}}}{}
Equalizer using an absorptive filter composed of two coupled lines.
\begin{quote}\begin{description}
\item[{Parameters}] \leavevmode\begin{itemize}
\item {} 
\sphinxstyleliteralstrong{\sphinxupquote{arg}} (\sphinxstyleliteralemphasis{\sphinxupquote{list}}) \textendash{} 
First 4 arguments are inputs.
\begin{enumerate}
\sphinxsetlistlabels{\arabic}{enumi}{enumii}{}{.}%
\item {} 
Reference Impedance ; impedance

\item {} 
Coupling (dB) ;

\item {} 
Center Frequency ; frequency

\item {} 
Test Frequency ; frequency

\item {} 
S21 (dB) ;

\item {} 
Zeven ;  impedance

\item {} 
Zodd ;  impedance

\end{enumerate}

Reference:


\item {} 
\sphinxstyleliteralstrong{\sphinxupquote{defaultunits}} (\sphinxstyleliteralemphasis{\sphinxupquote{list}}\sphinxstyleliteralemphasis{\sphinxupquote{, }}\sphinxstyleliteralemphasis{\sphinxupquote{optional}}) \textendash{} Default units for quantities in \sphinxstyleemphasis{arg} list. Default is {[}{]} which means SI units will be used if no unit is given in \sphinxstyleemphasis{arg}.

\end{itemize}

\item[{Returns}] \leavevmode
arg

\item[{Return type}] \leavevmode
list

\end{description}\end{quote}

\end{fulllineitems}

\index{Binomial\_QWave\_Impedance\_Transformer() (in module components)@\spxentry{Binomial\_QWave\_Impedance\_Transformer()}\spxextra{in module components}}

\begin{fulllineitems}
\phantomsection\label{\detokenize{components:components.Binomial_QWave_Impedance_Transformer}}\pysiglinewithargsret{\sphinxcode{\sphinxupquote{components.}}\sphinxbfcode{\sphinxupquote{Binomial\_QWave\_Impedance\_Transformer}}}{\emph{\DUrole{n}{arg}}, \emph{\DUrole{n}{defaultunits}\DUrole{o}{=}\DUrole{default_value}{{[}{]}}}}{}
Binomial Quarter Wave Impedance Transformer.
\begin{quote}\begin{description}
\item[{Parameters}] \leavevmode\begin{itemize}
\item {} 
\sphinxstyleliteralstrong{\sphinxupquote{arg}} (\sphinxstyleliteralemphasis{\sphinxupquote{list}}) \textendash{} 
First 5 arguments are inputs.
\begin{enumerate}
\sphinxsetlistlabels{\arabic}{enumi}{enumii}{}{.}%
\item {} 
Source Impedance;impedance

\item {} 
Load Impedance;impedance

\item {} 
Number Of Matching Sections;

\item {} 
Max(dB(S\textless{}sub\textgreater{}11\textless{}/sub\textgreater{})) In Frequency Band ;

\item {} 
Center Frequency ; frequency

\item {} 
Impedances ; impedance

\end{enumerate}

7.  Bandwidth ; frequency
Reference:  Impedance Matching and Transformation.pdf


\item {} 
\sphinxstyleliteralstrong{\sphinxupquote{defaultunits}} (\sphinxstyleliteralemphasis{\sphinxupquote{list}}\sphinxstyleliteralemphasis{\sphinxupquote{, }}\sphinxstyleliteralemphasis{\sphinxupquote{optional}}) \textendash{} Default units for quantities in \sphinxstyleemphasis{arg} list. Default is {[}{]} which means SI units will be used if no unit is given in \sphinxstyleemphasis{arg}.

\end{itemize}

\item[{Returns}] \leavevmode
arg

\item[{Return type}] \leavevmode
list

\end{description}\end{quote}

\end{fulllineitems}

\index{Bridged\_Tee\_Attenuator\_Analysis() (in module components)@\spxentry{Bridged\_Tee\_Attenuator\_Analysis()}\spxextra{in module components}}

\begin{fulllineitems}
\phantomsection\label{\detokenize{components:components.Bridged_Tee_Attenuator_Analysis}}\pysiglinewithargsret{\sphinxcode{\sphinxupquote{components.}}\sphinxbfcode{\sphinxupquote{Bridged\_Tee\_Attenuator\_Analysis}}}{\emph{\DUrole{n}{arg}}, \emph{\DUrole{n}{defaultunits}\DUrole{o}{=}\DUrole{default_value}{{[}{]}}}}{}
Bridged Tee Attenuator Analysis.
\begin{quote}\begin{description}
\item[{Parameters}] \leavevmode\begin{itemize}
\item {} 
\sphinxstyleliteralstrong{\sphinxupquote{arg}} (\sphinxstyleliteralemphasis{\sphinxupquote{list}}) \textendash{} 
First 3 arguments are inputs.
\begin{enumerate}
\sphinxsetlistlabels{\arabic}{enumi}{enumii}{}{.}%
\item {} 
Reference Impedance (Zo); impedance

\item {} 
Series Impedance (Rs); impedance

\item {} 
Parallel Impedance (Rp); impedance

\item {} 
S(1,1) ;

\end{enumerate}

5. S(2,1) ;
Reference:


\item {} 
\sphinxstyleliteralstrong{\sphinxupquote{defaultunits}} (\sphinxstyleliteralemphasis{\sphinxupquote{list}}\sphinxstyleliteralemphasis{\sphinxupquote{, }}\sphinxstyleliteralemphasis{\sphinxupquote{optional}}) \textendash{} Default units for quantities in \sphinxstyleemphasis{arg} list. Default is {[}{]} which means SI units will be used if no unit is given in \sphinxstyleemphasis{arg}.

\end{itemize}

\item[{Returns}] \leavevmode
arg

\item[{Return type}] \leavevmode
list

\end{description}\end{quote}

\end{fulllineitems}

\index{Bridged\_Tee\_Attenuator\_Synthesis() (in module components)@\spxentry{Bridged\_Tee\_Attenuator\_Synthesis()}\spxextra{in module components}}

\begin{fulllineitems}
\phantomsection\label{\detokenize{components:components.Bridged_Tee_Attenuator_Synthesis}}\pysiglinewithargsret{\sphinxcode{\sphinxupquote{components.}}\sphinxbfcode{\sphinxupquote{Bridged\_Tee\_Attenuator\_Synthesis}}}{\emph{\DUrole{n}{arg}}, \emph{\DUrole{n}{defaultunits}\DUrole{o}{=}\DUrole{default_value}{{[}{]}}}}{}
Bridged Tee Attenuator Synthesis.
\begin{quote}\begin{description}
\item[{Parameters}] \leavevmode\begin{itemize}
\item {} 
\sphinxstyleliteralstrong{\sphinxupquote{arg}} (\sphinxstyleliteralemphasis{\sphinxupquote{list}}) \textendash{} 
First 3 arguments are inputs.
\begin{enumerate}
\sphinxsetlistlabels{\arabic}{enumi}{enumii}{}{.}%
\item {} 
Reference Impedance (Zo); impedance

\item {} 
Series Impedance (Rs); impedance

\item {} 
Parallel Impedance (Rp); impedance

\item {} 
S(1,1) ;

\end{enumerate}

5. S(2,1) ;
Reference:


\item {} 
\sphinxstyleliteralstrong{\sphinxupquote{defaultunits}} (\sphinxstyleliteralemphasis{\sphinxupquote{list}}\sphinxstyleliteralemphasis{\sphinxupquote{, }}\sphinxstyleliteralemphasis{\sphinxupquote{optional}}) \textendash{} Default units for quantities in \sphinxstyleemphasis{arg} list. Default is {[}{]} which means SI units will be used if no unit is given in \sphinxstyleemphasis{arg}.

\end{itemize}

\item[{Returns}] \leavevmode
arg

\item[{Return type}] \leavevmode
list

\end{description}\end{quote}

\end{fulllineitems}

\index{Chebyshev\_QWave\_Impedance\_Transformer() (in module components)@\spxentry{Chebyshev\_QWave\_Impedance\_Transformer()}\spxextra{in module components}}

\begin{fulllineitems}
\phantomsection\label{\detokenize{components:components.Chebyshev_QWave_Impedance_Transformer}}\pysiglinewithargsret{\sphinxcode{\sphinxupquote{components.}}\sphinxbfcode{\sphinxupquote{Chebyshev\_QWave\_Impedance\_Transformer}}}{\emph{\DUrole{n}{arg}}, \emph{\DUrole{n}{defaultunits}\DUrole{o}{=}\DUrole{default_value}{{[}{]}}}}{}
Chebyshev Quarter Wave Impedance Transformer.
\begin{quote}\begin{description}
\item[{Parameters}] \leavevmode\begin{itemize}
\item {} 
\sphinxstyleliteralstrong{\sphinxupquote{arg}} (\sphinxstyleliteralemphasis{\sphinxupquote{list}}) \textendash{} 
First 6 arguments are inputs.
\begin{enumerate}
\sphinxsetlistlabels{\arabic}{enumi}{enumii}{}{.}%
\item {} 
Source Impedance ; impedance

\item {} 
Load Impedance ; impedance

\item {} 
Number Of Matching Sections ;

\item {} 
Minimum Frequency ; frequency

\item {} 
Maximum Frequency ; frequency

\item {} 
Test Frequency ; frequency

\item {} 
Impedances ; impedance

\end{enumerate}

8.  Return Loss at Test Frequency ;
Reference:  Impedance Matching and Transformation.pdf + eski kod


\item {} 
\sphinxstyleliteralstrong{\sphinxupquote{defaultunits}} (\sphinxstyleliteralemphasis{\sphinxupquote{list}}\sphinxstyleliteralemphasis{\sphinxupquote{, }}\sphinxstyleliteralemphasis{\sphinxupquote{optional}}) \textendash{} Default units for quantities in \sphinxstyleemphasis{arg} list. Default is {[}{]} which means SI units will be used if no unit is given in \sphinxstyleemphasis{arg}.

\end{itemize}

\item[{Returns}] \leavevmode
arg

\item[{Return type}] \leavevmode
list

\end{description}\end{quote}

\end{fulllineitems}

\index{Chebyshev\_Taper\_Impedance\_Transformer() (in module components)@\spxentry{Chebyshev\_Taper\_Impedance\_Transformer()}\spxextra{in module components}}

\begin{fulllineitems}
\phantomsection\label{\detokenize{components:components.Chebyshev_Taper_Impedance_Transformer}}\pysiglinewithargsret{\sphinxcode{\sphinxupquote{components.}}\sphinxbfcode{\sphinxupquote{Chebyshev\_Taper\_Impedance\_Transformer}}}{\emph{\DUrole{n}{arg}}, \emph{\DUrole{n}{defaultunits}\DUrole{o}{=}\DUrole{default_value}{{[}{]}}}}{}
Calculates performance and impedance values for an N\sphinxhyphen{}section Chebyshev Impedance Taper.
\begin{quote}\begin{description}
\item[{Parameters}] \leavevmode\begin{itemize}
\item {} 
\sphinxstyleliteralstrong{\sphinxupquote{arg}} (\sphinxstyleliteralemphasis{\sphinxupquote{list}}) \textendash{} 
First 5 arguments are inputs.
\begin{enumerate}
\sphinxsetlistlabels{\arabic}{enumi}{enumii}{}{.}%
\item {} 
Source Impedance ; impedance

\item {} 
Load Impedance ; impedance

\item {} 
Number Of Sections (Even) ;

\item {} 
Fractional Bandwidth (F2/F1) ;

\item {} 
Length (normalized to Lambda at fcenter) ;

\item {} 
Impedances ; impedance

\end{enumerate}

7.  Return Loss ;
Reference:  Foundations for Microwave Engineering, Collin


\item {} 
\sphinxstyleliteralstrong{\sphinxupquote{defaultunits}} (\sphinxstyleliteralemphasis{\sphinxupquote{list}}\sphinxstyleliteralemphasis{\sphinxupquote{, }}\sphinxstyleliteralemphasis{\sphinxupquote{optional}}) \textendash{} Default units for quantities in \sphinxstyleemphasis{arg} list. Default is {[}{]} which means SI units will be used if no unit is given in \sphinxstyleemphasis{arg}.

\end{itemize}

\item[{Returns}] \leavevmode
arg

\item[{Return type}] \leavevmode
list

\end{description}\end{quote}

\end{fulllineitems}

\index{CircularPlateCap() (in module components)@\spxentry{CircularPlateCap()}\spxextra{in module components}}

\begin{fulllineitems}
\phantomsection\label{\detokenize{components:components.CircularPlateCap}}\pysiglinewithargsret{\sphinxcode{\sphinxupquote{components.}}\sphinxbfcode{\sphinxupquote{CircularPlateCap}}}{\emph{\DUrole{n}{arg}}, \emph{\DUrole{n}{defaultunits}\DUrole{o}{=}\DUrole{default_value}{{[}{]}}}}{}
Circular Plate Capacitance.
\begin{quote}\begin{description}
\item[{Parameters}] \leavevmode\begin{itemize}
\item {} 
\sphinxstyleliteralstrong{\sphinxupquote{arg}} (\sphinxstyleliteralemphasis{\sphinxupquote{list}}) \textendash{} 
First 3 arguments are inputs.
\begin{enumerate}
\sphinxsetlistlabels{\arabic}{enumi}{enumii}{}{.}%
\item {} 
Radius;length

\item {} 
Height;length

\item {} 
Dielectric Permittivity;

\end{enumerate}

4. Capacitance; capacitance
Reference:


\item {} 
\sphinxstyleliteralstrong{\sphinxupquote{defaultunits}} (\sphinxstyleliteralemphasis{\sphinxupquote{list}}\sphinxstyleliteralemphasis{\sphinxupquote{, }}\sphinxstyleliteralemphasis{\sphinxupquote{optional}}) \textendash{} Default units for quantities in \sphinxstyleemphasis{arg} list. Default is {[}{]} which means SI units will be used if no unit is given in \sphinxstyleemphasis{arg}.

\end{itemize}

\item[{Returns}] \leavevmode
arg

\item[{Return type}] \leavevmode
list

\end{description}\end{quote}

\end{fulllineitems}

\index{Dia2AWG() (in module components)@\spxentry{Dia2AWG()}\spxextra{in module components}}

\begin{fulllineitems}
\phantomsection\label{\detokenize{components:components.Dia2AWG}}\pysiglinewithargsret{\sphinxcode{\sphinxupquote{components.}}\sphinxbfcode{\sphinxupquote{Dia2AWG}}}{\emph{\DUrole{n}{arg}}, \emph{\DUrole{n}{defaultunits}\DUrole{o}{=}\DUrole{default_value}{{[}{]}}}}{}
Convert Diameter to AWG.
\begin{quote}\begin{description}
\item[{Parameters}] \leavevmode\begin{itemize}
\item {} 
\sphinxstyleliteralstrong{\sphinxupquote{arg}} (\sphinxstyleliteralemphasis{\sphinxupquote{list}}) \textendash{} 
First 1 arguments are inputs.
\begin{enumerate}
\sphinxsetlistlabels{\arabic}{enumi}{enumii}{}{.}%
\item {} 
AWG ;

\item {} 
Diameter ;length

\end{enumerate}

3. Current rating in still air ; current
Reference:  Wikipedia


\item {} 
\sphinxstyleliteralstrong{\sphinxupquote{defaultunits}} (\sphinxstyleliteralemphasis{\sphinxupquote{list}}\sphinxstyleliteralemphasis{\sphinxupquote{, }}\sphinxstyleliteralemphasis{\sphinxupquote{optional}}) \textendash{} Default units for quantities in \sphinxstyleemphasis{arg} list. Default is {[}{]} which means SI units will be used if no unit is given in \sphinxstyleemphasis{arg}.

\end{itemize}

\item[{Returns}] \leavevmode
arg

\item[{Return type}] \leavevmode
list

\end{description}\end{quote}

\end{fulllineitems}

\index{DualFrequencyTransformer() (in module components)@\spxentry{DualFrequencyTransformer()}\spxextra{in module components}}

\begin{fulllineitems}
\phantomsection\label{\detokenize{components:components.DualFrequencyTransformer}}\pysiglinewithargsret{\sphinxcode{\sphinxupquote{components.}}\sphinxbfcode{\sphinxupquote{DualFrequencyTransformer}}}{\emph{\DUrole{n}{arg}}, \emph{\DUrole{n}{defaultunits}\DUrole{o}{=}\DUrole{default_value}{{[}{]}}}}{}
Dual Frequency Transformer.
\begin{quote}\begin{description}
\item[{Parameters}] \leavevmode\begin{itemize}
\item {} 
\sphinxstyleliteralstrong{\sphinxupquote{arg}} (\sphinxstyleliteralemphasis{\sphinxupquote{list}}) \textendash{} 
First 4 arguments are inputs.
\begin{enumerate}
\sphinxsetlistlabels{\arabic}{enumi}{enumii}{}{.}%
\item {} 
Source Impedance; impedance

\item {} 
Load Impedance; impedance

\item {} 
f1 Lower Frequency; frequency

\item {} 
f2 Higher Frequency; frequency

\item {} 
Z1; impedance

\item {} 
Z2; impedance

\end{enumerate}

7. Electrical Length ; angle
Reference:  A Small Dual Frequency Transformer in Two Sections


\item {} 
\sphinxstyleliteralstrong{\sphinxupquote{defaultunits}} (\sphinxstyleliteralemphasis{\sphinxupquote{list}}\sphinxstyleliteralemphasis{\sphinxupquote{, }}\sphinxstyleliteralemphasis{\sphinxupquote{optional}}) \textendash{} Default units for quantities in \sphinxstyleemphasis{arg} list. Default is {[}{]} which means SI units will be used if no unit is given in \sphinxstyleemphasis{arg}.

\end{itemize}

\item[{Returns}] \leavevmode
arg

\item[{Return type}] \leavevmode
list

\end{description}\end{quote}

\end{fulllineitems}

\index{DualTransformation1() (in module components)@\spxentry{DualTransformation1()}\spxextra{in module components}}

\begin{fulllineitems}
\phantomsection\label{\detokenize{components:components.DualTransformation1}}\pysiglinewithargsret{\sphinxcode{\sphinxupquote{components.}}\sphinxbfcode{\sphinxupquote{DualTransformation1}}}{\emph{\DUrole{n}{arg}}, \emph{\DUrole{n}{defaultunits}\DUrole{o}{=}\DUrole{default_value}{{[}{]}}}}{}
Dual Transformation 1.
\begin{quote}\begin{description}
\item[{Parameters}] \leavevmode\begin{itemize}
\item {} 
\sphinxstyleliteralstrong{\sphinxupquote{arg}} (\sphinxstyleliteralemphasis{\sphinxupquote{list}}) \textendash{} 
First 4 arguments are inputs.
\begin{enumerate}
\sphinxsetlistlabels{\arabic}{enumi}{enumii}{}{.}%
\item {} 
L1 ; inductance

\item {} 
C1 ; capacitance

\item {} 
L2 ; inductance

\item {} 
C2 ; capacitance

\item {} 
L1’ ; inductance

\item {} 
C1’ ; capacitance

\item {} 
L2’ ; inductance

\end{enumerate}

8.  C2’ ; capacitance
Reference:  Microstrip Filters for RF\sphinxhyphen{}Microwave Applications, s.25, Figure 2.6a


\item {} 
\sphinxstyleliteralstrong{\sphinxupquote{defaultunits}} (\sphinxstyleliteralemphasis{\sphinxupquote{list}}\sphinxstyleliteralemphasis{\sphinxupquote{, }}\sphinxstyleliteralemphasis{\sphinxupquote{optional}}) \textendash{} Default units for quantities in \sphinxstyleemphasis{arg} list. Default is {[}{]} which means SI units will be used if no unit is given in \sphinxstyleemphasis{arg}.

\end{itemize}

\item[{Returns}] \leavevmode
arg

\item[{Return type}] \leavevmode
list

\end{description}\end{quote}

\end{fulllineitems}

\index{DualTransformation2() (in module components)@\spxentry{DualTransformation2()}\spxextra{in module components}}

\begin{fulllineitems}
\phantomsection\label{\detokenize{components:components.DualTransformation2}}\pysiglinewithargsret{\sphinxcode{\sphinxupquote{components.}}\sphinxbfcode{\sphinxupquote{DualTransformation2}}}{\emph{\DUrole{n}{arg}}, \emph{\DUrole{n}{defaultunits}\DUrole{o}{=}\DUrole{default_value}{{[}{]}}}}{}
Dual Transformation 1.
\begin{quote}\begin{description}
\item[{Parameters}] \leavevmode\begin{itemize}
\item {} 
\sphinxstyleliteralstrong{\sphinxupquote{arg}} (\sphinxstyleliteralemphasis{\sphinxupquote{list}}) \textendash{} 
First 4 arguments are inputs.
\begin{enumerate}
\sphinxsetlistlabels{\arabic}{enumi}{enumii}{}{.}%
\item {} 
L1 ; inductance

\item {} 
C1 ; capacitance

\item {} 
L2 ; inductance

\item {} 
C2 ; capacitance

\item {} 
L1’ ; inductance

\item {} 
C1’ ; capacitance

\item {} 
L2’ ; inductance

\end{enumerate}

8.  C2’ ; capacitance
Reference:  Microstrip Filters for RF\sphinxhyphen{}Microwave Applications, s.25, Figure 2.6b


\item {} 
\sphinxstyleliteralstrong{\sphinxupquote{defaultunits}} (\sphinxstyleliteralemphasis{\sphinxupquote{list}}\sphinxstyleliteralemphasis{\sphinxupquote{, }}\sphinxstyleliteralemphasis{\sphinxupquote{optional}}) \textendash{} Default units for quantities in \sphinxstyleemphasis{arg} list. Default is {[}{]} which means SI units will be used if no unit is given in \sphinxstyleemphasis{arg}.

\end{itemize}

\item[{Returns}] \leavevmode
arg

\item[{Return type}] \leavevmode
list

\end{description}\end{quote}

\end{fulllineitems}

\index{EWG\_ABCD() (in module components)@\spxentry{EWG\_ABCD()}\spxextra{in module components}}

\begin{fulllineitems}
\phantomsection\label{\detokenize{components:components.EWG_ABCD}}\pysiglinewithargsret{\sphinxcode{\sphinxupquote{components.}}\sphinxbfcode{\sphinxupquote{EWG\_ABCD}}}{\emph{\DUrole{n}{a}}, \emph{\DUrole{n}{b}}, \emph{\DUrole{n}{er}}, \emph{\DUrole{n}{length}}, \emph{\DUrole{n}{frek}}}{}
\end{fulllineitems}

\index{EWG\_inv() (in module components)@\spxentry{EWG\_inv()}\spxextra{in module components}}

\begin{fulllineitems}
\phantomsection\label{\detokenize{components:components.EWG_inv}}\pysiglinewithargsret{\sphinxcode{\sphinxupquote{components.}}\sphinxbfcode{\sphinxupquote{EWG\_inv}}}{\emph{\DUrole{n}{a}}, \emph{\DUrole{n}{b}}, \emph{\DUrole{n}{er}}, \emph{\DUrole{n}{length}}, \emph{\DUrole{n}{frek}}}{}
\end{fulllineitems}

\index{EvanescentWGEquivalent() (in module components)@\spxentry{EvanescentWGEquivalent()}\spxextra{in module components}}

\begin{fulllineitems}
\phantomsection\label{\detokenize{components:components.EvanescentWGEquivalent}}\pysiglinewithargsret{\sphinxcode{\sphinxupquote{components.}}\sphinxbfcode{\sphinxupquote{EvanescentWGEquivalent}}}{\emph{\DUrole{n}{arg}}, \emph{\DUrole{n}{defaultunits}\DUrole{o}{=}\DUrole{default_value}{{[}{]}}}}{}
Waveguide Width Step from Rectangular Waveguide to Evanescent Mode Rectangular Waveguide.
\begin{quote}\begin{description}
\item[{Parameters}] \leavevmode\begin{itemize}
\item {} 
\sphinxstyleliteralstrong{\sphinxupquote{arg}} (\sphinxstyleliteralemphasis{\sphinxupquote{list}}) \textendash{} 
First 5 arguments are inputs.
\begin{enumerate}
\sphinxsetlistlabels{\arabic}{enumi}{enumii}{}{.}%
\item {} 
Waveguide Width;length

\item {} 
Waveguide Height;length

\item {} 
Dielectric Permittivity;

\item {} 
Waveguide Length;length

\item {} 
Frequency; frequency

\item {} 
Series Inductance For Shunt\sphinxhyphen{}Series\sphinxhyphen{}Shunt Model; inductance

\item {} 
Shunt Inductance For Shunt\sphinxhyphen{}Series\sphinxhyphen{}Shunt Model; inductance

\item {} 
Series Inductance For Series\sphinxhyphen{}Shunt\sphinxhyphen{}Series Model; inductance

\item {} 
Shunt Inductance For Series\sphinxhyphen{}Shunt\sphinxhyphen{}Series Model; inductance

\end{enumerate}

10. Characteristic Impedance; impedance
Reference:  The Design of Evanescent Mode Waveguide Bandpass Filters for a Prescribed Insertion Loss Characteristic.pdf
Model= Xp1,Xs1,Xp1 ya da Xs2,Xp2,Xs2 (p: shunt, s: series)
Zo=jXo


\item {} 
\sphinxstyleliteralstrong{\sphinxupquote{defaultunits}} (\sphinxstyleliteralemphasis{\sphinxupquote{list}}\sphinxstyleliteralemphasis{\sphinxupquote{, }}\sphinxstyleliteralemphasis{\sphinxupquote{optional}}) \textendash{} Default units for quantities in \sphinxstyleemphasis{arg} list. Default is {[}{]} which means SI units will be used if no unit is given in \sphinxstyleemphasis{arg}.

\end{itemize}

\item[{Returns}] \leavevmode
arg

\item[{Return type}] \leavevmode
list

\end{description}\end{quote}

\end{fulllineitems}

\index{Exponential\_Taper\_Impedance\_Transformer() (in module components)@\spxentry{Exponential\_Taper\_Impedance\_Transformer()}\spxextra{in module components}}

\begin{fulllineitems}
\phantomsection\label{\detokenize{components:components.Exponential_Taper_Impedance_Transformer}}\pysiglinewithargsret{\sphinxcode{\sphinxupquote{components.}}\sphinxbfcode{\sphinxupquote{Exponential\_Taper\_Impedance\_Transformer}}}{\emph{\DUrole{n}{arg}}, \emph{\DUrole{n}{defaultunits}\DUrole{o}{=}\DUrole{default_value}{{[}{]}}}}{}
Exponential Impedance Taper.
\begin{quote}\begin{description}
\item[{Parameters}] \leavevmode\begin{itemize}
\item {} 
\sphinxstyleliteralstrong{\sphinxupquote{arg}} (\sphinxstyleliteralemphasis{\sphinxupquote{list}}) \textendash{} 
First 5 arguments are inputs.
\begin{enumerate}
\sphinxsetlistlabels{\arabic}{enumi}{enumii}{}{.}%
\item {} 
Source Impedance ; impedance

\item {} 
Load Impedance ; impedance

\item {} 
Number Of Sections ;

\item {} 
Fractional Bandwidth (F2/F1) ;

\item {} 
Length (normalized to Lambda at fcenter) ;

\item {} 
Impedances ; impedance

\end{enumerate}

7.  Return Loss ;
Reference:  Foundations for Microwave Engineering, Collin


\item {} 
\sphinxstyleliteralstrong{\sphinxupquote{defaultunits}} (\sphinxstyleliteralemphasis{\sphinxupquote{list}}\sphinxstyleliteralemphasis{\sphinxupquote{, }}\sphinxstyleliteralemphasis{\sphinxupquote{optional}}) \textendash{} Default units for quantities in \sphinxstyleemphasis{arg} list. Default is {[}{]} which means SI units will be used if no unit is given in \sphinxstyleemphasis{arg}.

\end{itemize}

\item[{Returns}] \leavevmode
arg

\item[{Return type}] \leavevmode
list

\end{description}\end{quote}

\end{fulllineitems}

\index{GyselPowerDivider() (in module components)@\spxentry{GyselPowerDivider()}\spxextra{in module components}}

\begin{fulllineitems}
\phantomsection\label{\detokenize{components:components.GyselPowerDivider}}\pysiglinewithargsret{\sphinxcode{\sphinxupquote{components.}}\sphinxbfcode{\sphinxupquote{GyselPowerDivider}}}{\emph{\DUrole{n}{arg}}, \emph{\DUrole{n}{defaultunits}\DUrole{o}{=}\DUrole{default_value}{{[}{]}}}}{}
Triangle network to Star network transformation.
\begin{quote}\begin{description}
\item[{Parameters}] \leavevmode\begin{itemize}
\item {} 
\sphinxstyleliteralstrong{\sphinxupquote{arg}} (\sphinxstyleliteralemphasis{\sphinxupquote{list}}) \textendash{} 
First 6 arguments are inputs.
\begin{enumerate}
\sphinxsetlistlabels{\arabic}{enumi}{enumii}{}{.}%
\item {} 
Zo1;  impedance

\item {} 
Zo2;  impedance

\item {} 
Zo3;  impedance

\item {} 
R1; impedance

\item {} 
R2; impedance

\item {} 
P2/P3 ratio;

\item {} 
Z1; impedance

\item {} 
Z2; impedance

\item {} 
Z3; impedance

\end{enumerate}

10. Z4; impedance
Reference:
Zo1: 1. port impedance
Zo2: 2. port impedance
Zo3: 3. port impedance
R1: first isolation resistor (2.porta yakin)
R2: second isolation resistor (3.porta yakin)
ratio: P2/P3 power ratio
Z1: impedance of transmission line between 1.port and 2.port
Z2: impedance of transmission line between 1.port and 3.port
Z3: impedance of transmission line between 2.port and isolation resistor
Z4: impedance of transmission line between 3.port and isolation resistor


\item {} 
\sphinxstyleliteralstrong{\sphinxupquote{defaultunits}} (\sphinxstyleliteralemphasis{\sphinxupquote{list}}\sphinxstyleliteralemphasis{\sphinxupquote{, }}\sphinxstyleliteralemphasis{\sphinxupquote{optional}}) \textendash{} Default units for quantities in \sphinxstyleemphasis{arg} list. Default is {[}{]} which means SI units will be used if no unit is given in \sphinxstyleemphasis{arg}.

\end{itemize}

\item[{Returns}] \leavevmode
arg

\item[{Return type}] \leavevmode
list

\end{description}\end{quote}

\end{fulllineitems}

\index{HomogeneousRectWaveguideParameters\_TE() (in module components)@\spxentry{HomogeneousRectWaveguideParameters\_TE()}\spxextra{in module components}}

\begin{fulllineitems}
\phantomsection\label{\detokenize{components:components.HomogeneousRectWaveguideParameters_TE}}\pysiglinewithargsret{\sphinxcode{\sphinxupquote{components.}}\sphinxbfcode{\sphinxupquote{HomogeneousRectWaveguideParameters\_TE}}}{\emph{\DUrole{n}{arg}}, \emph{\DUrole{n}{defaultunits}\DUrole{o}{=}\DUrole{default_value}{{[}{]}}}}{}
Homogeneous Rectangular Waveguide Parameters.
\begin{quote}\begin{description}
\item[{Parameters}] \leavevmode\begin{itemize}
\item {} 
\sphinxstyleliteralstrong{\sphinxupquote{arg}} (\sphinxstyleliteralemphasis{\sphinxupquote{list}}) \textendash{} 
First 10 arguments are inputs.
\begin{enumerate}
\sphinxsetlistlabels{\arabic}{enumi}{enumii}{}{.}%
\item {} 
Dielectric Permittivity in Waveguide;

\item {} 
Waveguide Width;length

\item {} 
Waveguide Height;length

\item {} 
Mode (0: Te, 1: Tm);

\item {} 
M;

\item {} 
N;

\item {} 
Tand Of Dielectric;

\item {} 
Electrical Conductivity Of Walls; electrical conductivity

\item {} 
Frequency; frequency

\item {} 
Physical Length;length

\item {} 
Cond Loss; loss per length

\item {} 
Diel Loss; loss per length

\item {} 
Cutoff Freq; frequency

\item {} 
Lambda\_Guided;length

\item {} 
Impedance; impedance

\item {} 
Electrical Length; angle

\end{enumerate}

Reference:  Marcuvitz Waveguide Handbook s.253


\item {} 
\sphinxstyleliteralstrong{\sphinxupquote{defaultunits}} (\sphinxstyleliteralemphasis{\sphinxupquote{list}}\sphinxstyleliteralemphasis{\sphinxupquote{, }}\sphinxstyleliteralemphasis{\sphinxupquote{optional}}) \textendash{} Default units for quantities in \sphinxstyleemphasis{arg} list. Default is {[}{]} which means SI units will be used if no unit is given in \sphinxstyleemphasis{arg}.

\end{itemize}

\item[{Returns}] \leavevmode
arg

\item[{Return type}] \leavevmode
list

\end{description}\end{quote}

\end{fulllineitems}

\index{InductivePostInWaveguide() (in module components)@\spxentry{InductivePostInWaveguide()}\spxextra{in module components}}

\begin{fulllineitems}
\phantomsection\label{\detokenize{components:components.InductivePostInWaveguide}}\pysiglinewithargsret{\sphinxcode{\sphinxupquote{components.}}\sphinxbfcode{\sphinxupquote{InductivePostInWaveguide}}}{\emph{\DUrole{n}{arg}}, \emph{\DUrole{n}{defaultunits}\DUrole{o}{=}\DUrole{default_value}{{[}{]}}}}{}
Inductive Post In Waveguide.
\begin{quote}\begin{description}
\item[{Parameters}] \leavevmode\begin{itemize}
\item {} 
\sphinxstyleliteralstrong{\sphinxupquote{arg}} (\sphinxstyleliteralemphasis{\sphinxupquote{list}}) \textendash{} 
First 6 arguments are inputs.
\begin{enumerate}
\sphinxsetlistlabels{\arabic}{enumi}{enumii}{}{.}%
\item {} 
Dielectric Permittivity in Waveguide ;

\item {} 
Waveguide Width (a);length

\item {} 
Waveguide Height (b);length

\item {} 
Post Diameter (d);length

\item {} 
Waveguide Sidewall To Post Center (s);length

\item {} 
Frequency; frequency

\item {} 
Inductance;inductance

\item {} 
Capacitance; capacitance

\end{enumerate}

9. Impedance; impedance
Reference:  Marcuvitz Waveguide Handbook s.257


\item {} 
\sphinxstyleliteralstrong{\sphinxupquote{defaultunits}} (\sphinxstyleliteralemphasis{\sphinxupquote{list}}\sphinxstyleliteralemphasis{\sphinxupquote{, }}\sphinxstyleliteralemphasis{\sphinxupquote{optional}}) \textendash{} Default units for quantities in \sphinxstyleemphasis{arg} list. Default is {[}{]} which means SI units will be used if no unit is given in \sphinxstyleemphasis{arg}.

\end{itemize}

\item[{Returns}] \leavevmode
arg

\item[{Return type}] \leavevmode
list

\end{description}\end{quote}

\end{fulllineitems}

\index{InductiveWindowInWaveguide() (in module components)@\spxentry{InductiveWindowInWaveguide()}\spxextra{in module components}}

\begin{fulllineitems}
\phantomsection\label{\detokenize{components:components.InductiveWindowInWaveguide}}\pysiglinewithargsret{\sphinxcode{\sphinxupquote{components.}}\sphinxbfcode{\sphinxupquote{InductiveWindowInWaveguide}}}{\emph{\DUrole{n}{arg}}, \emph{\DUrole{n}{defaultunits}\DUrole{o}{=}\DUrole{default_value}{{[}{]}}}}{}
Waveguide Width Step from Rectangular Waveguide to Evanescent Mode Rectangular Waveguide.
\begin{quote}\begin{description}
\item[{Parameters}] \leavevmode\begin{itemize}
\item {} 
\sphinxstyleliteralstrong{\sphinxupquote{arg}} (\sphinxstyleliteralemphasis{\sphinxupquote{list}}) \textendash{} 
First 6 arguments are inputs.
\begin{enumerate}
\sphinxsetlistlabels{\arabic}{enumi}{enumii}{}{.}%
\item {} 
Dielectric Permittivity in Waveguide ;

\item {} 
Waveguide Width (a);length

\item {} 
Waveguide Height (b);length

\item {} 
Difference Of Waveguide Width To Window Width;length

\item {} 
Window Thickness;length

\item {} 
Frequency; frequency

\item {} 
Inductance;inductance

\item {} 
Capacitance; capacitance

\item {} 
Impedance; impedance

\end{enumerate}

Reference:  Marcuvitz Waveguide Handbook s.253


\item {} 
\sphinxstyleliteralstrong{\sphinxupquote{defaultunits}} (\sphinxstyleliteralemphasis{\sphinxupquote{list}}\sphinxstyleliteralemphasis{\sphinxupquote{, }}\sphinxstyleliteralemphasis{\sphinxupquote{optional}}) \textendash{} Default units for quantities in \sphinxstyleemphasis{arg} list. Default is {[}{]} which means SI units will be used if no unit is given in \sphinxstyleemphasis{arg}.

\end{itemize}

\item[{Returns}] \leavevmode
arg

\item[{Return type}] \leavevmode
list

\end{description}\end{quote}

\end{fulllineitems}

\index{Interference\_Phase\_Amp\_Error() (in module components)@\spxentry{Interference\_Phase\_Amp\_Error()}\spxextra{in module components}}

\begin{fulllineitems}
\phantomsection\label{\detokenize{components:components.Interference_Phase_Amp_Error}}\pysiglinewithargsret{\sphinxcode{\sphinxupquote{components.}}\sphinxbfcode{\sphinxupquote{Interference\_Phase\_Amp\_Error}}}{\emph{\DUrole{n}{arg}}, \emph{\DUrole{n}{defaultunits}\DUrole{o}{=}\DUrole{default_value}{{[}{]}}}}{}
Maximum phase and amplitude variation of a signal in presence of an interfering signal.
\begin{quote}\begin{description}
\item[{Parameters}] \leavevmode\begin{itemize}
\item {} 
\sphinxstyleliteralstrong{\sphinxupquote{arg}} (\sphinxstyleliteralemphasis{\sphinxupquote{list}}) \textendash{} 
First 1 arguments are inputs.
\begin{enumerate}
\sphinxsetlistlabels{\arabic}{enumi}{enumii}{}{.}%
\item {} 
Difference in dB ;

\item {} 
Amplitude Error;

\end{enumerate}

3. Phase Error; angle
Reference:


\item {} 
\sphinxstyleliteralstrong{\sphinxupquote{defaultunits}} (\sphinxstyleliteralemphasis{\sphinxupquote{list}}\sphinxstyleliteralemphasis{\sphinxupquote{, }}\sphinxstyleliteralemphasis{\sphinxupquote{optional}}) \textendash{} Default units for quantities in \sphinxstyleemphasis{arg} list. Default is {[}{]} which means SI units will be used if no unit is given in \sphinxstyleemphasis{arg}.

\end{itemize}

\item[{Returns}] \leavevmode
arg

\item[{Return type}] \leavevmode
list

\end{description}\end{quote}

\end{fulllineitems}

\index{Klopfenstein\_Taper\_Impedance\_Transformer() (in module components)@\spxentry{Klopfenstein\_Taper\_Impedance\_Transformer()}\spxextra{in module components}}

\begin{fulllineitems}
\phantomsection\label{\detokenize{components:components.Klopfenstein_Taper_Impedance_Transformer}}\pysiglinewithargsret{\sphinxcode{\sphinxupquote{components.}}\sphinxbfcode{\sphinxupquote{Klopfenstein\_Taper\_Impedance\_Transformer}}}{\emph{\DUrole{n}{arg}}, \emph{\DUrole{n}{defaultunits}\DUrole{o}{=}\DUrole{default_value}{{[}{]}}}}{}
Calculates performance and impedance values for an N\sphinxhyphen{}section Klopfenstein Impedance Taper.
\begin{quote}\begin{description}
\item[{Parameters}] \leavevmode\begin{itemize}
\item {} 
\sphinxstyleliteralstrong{\sphinxupquote{arg}} (\sphinxstyleliteralemphasis{\sphinxupquote{list}}) \textendash{} 
First 6 arguments are inputs.
\begin{enumerate}
\sphinxsetlistlabels{\arabic}{enumi}{enumii}{}{.}%
\item {} 
Source Impedance ; impedance

\item {} 
Load Impedance ; impedance

\item {} 
Maximum Reflection Coefficient (dB) ;

\item {} 
Number Of Sections ;

\item {} 
Minimum Frequency ; frequency

\item {} 
Test Frequency ; frequency

\item {} 
Minimum Total Phase at Minimum Frequency ; angle ;

\item {} 
Impedances ; impedance

\end{enumerate}

9.  MAG(Reflection Coefficient) ;
Reference:  Microwave Engineering, Pozar


\item {} 
\sphinxstyleliteralstrong{\sphinxupquote{defaultunits}} (\sphinxstyleliteralemphasis{\sphinxupquote{list}}\sphinxstyleliteralemphasis{\sphinxupquote{, }}\sphinxstyleliteralemphasis{\sphinxupquote{optional}}) \textendash{} Default units for quantities in \sphinxstyleemphasis{arg} list. Default is {[}{]} which means SI units will be used if no unit is given in \sphinxstyleemphasis{arg}.

\end{itemize}

\item[{Returns}] \leavevmode
arg

\item[{Return type}] \leavevmode
list

\end{description}\end{quote}

\end{fulllineitems}

\index{LC\_Balun() (in module components)@\spxentry{LC\_Balun()}\spxextra{in module components}}

\begin{fulllineitems}
\phantomsection\label{\detokenize{components:components.LC_Balun}}\pysiglinewithargsret{\sphinxcode{\sphinxupquote{components.}}\sphinxbfcode{\sphinxupquote{LC\_Balun}}}{\emph{\DUrole{n}{arg}}, \emph{\DUrole{n}{defaultunits}\DUrole{o}{=}\DUrole{default_value}{{[}{]}}}}{}
Calculate LC Balun.
\begin{quote}\begin{description}
\item[{Parameters}] \leavevmode\begin{itemize}
\item {} 
\sphinxstyleliteralstrong{\sphinxupquote{arg}} (\sphinxstyleliteralemphasis{\sphinxupquote{list}}) \textendash{} 
First 4 arguments are inputs.
\begin{enumerate}
\sphinxsetlistlabels{\arabic}{enumi}{enumii}{}{.}%
\item {} 
Source Impedance (Rin) ; impedance

\item {} 
Load Impedances (RL) ; impedance

\item {} 
Frequency; frequency

\item {} 
Test Frequency ; frequency

\item {} 
Inductance ; inductance

\item {} 
Capacitance ; capacitance

\item {} 
S11 (dB) ;

\item {} 
S21 (dB) ;

\end{enumerate}

9.  S31 (dB) ;
Reference:


\item {} 
\sphinxstyleliteralstrong{\sphinxupquote{defaultunits}} (\sphinxstyleliteralemphasis{\sphinxupquote{list}}\sphinxstyleliteralemphasis{\sphinxupquote{, }}\sphinxstyleliteralemphasis{\sphinxupquote{optional}}) \textendash{} Default units for quantities in \sphinxstyleemphasis{arg} list. Default is {[}{]} which means SI units will be used if no unit is given in \sphinxstyleemphasis{arg}.

\end{itemize}

\item[{Returns}] \leavevmode
arg

\item[{Return type}] \leavevmode
list

\end{description}\end{quote}

\end{fulllineitems}

\index{L\_BondWire() (in module components)@\spxentry{L\_BondWire()}\spxextra{in module components}}

\begin{fulllineitems}
\phantomsection\label{\detokenize{components:components.L_BondWire}}\pysiglinewithargsret{\sphinxcode{\sphinxupquote{components.}}\sphinxbfcode{\sphinxupquote{L\_BondWire}}}{\emph{\DUrole{n}{arg}}, \emph{\DUrole{n}{defaultunits}\DUrole{o}{=}\DUrole{default_value}{{[}{]}}}}{}
Inductance of a bond wire.
\begin{quote}\begin{description}
\item[{Parameters}] \leavevmode\begin{itemize}
\item {} 
\sphinxstyleliteralstrong{\sphinxupquote{arg}} (\sphinxstyleliteralemphasis{\sphinxupquote{list}}) \textendash{} 
First 4 arguments are inputs.
\begin{enumerate}
\sphinxsetlistlabels{\arabic}{enumi}{enumii}{}{.}%
\item {} 
Bondwire Radius ;length

\item {} 
Substrate Thickness ;length

\item {} 
Distance Between End Points ;length

\item {} 
Angle At End Points In Degrees ; angle

\end{enumerate}

5. Inductance ;inductance
Reference:  Transmission Line Design Handbook, Wadell, s.153


\item {} 
\sphinxstyleliteralstrong{\sphinxupquote{defaultunits}} (\sphinxstyleliteralemphasis{\sphinxupquote{list}}\sphinxstyleliteralemphasis{\sphinxupquote{, }}\sphinxstyleliteralemphasis{\sphinxupquote{optional}}) \textendash{} Default units for quantities in \sphinxstyleemphasis{arg} list. Default is {[}{]} which means SI units will be used if no unit is given in \sphinxstyleemphasis{arg}.

\end{itemize}

\item[{Returns}] \leavevmode
arg

\item[{Return type}] \leavevmode
list

\end{description}\end{quote}

\end{fulllineitems}

\index{L\_StraightFlatWire() (in module components)@\spxentry{L\_StraightFlatWire()}\spxextra{in module components}}

\begin{fulllineitems}
\phantomsection\label{\detokenize{components:components.L_StraightFlatWire}}\pysiglinewithargsret{\sphinxcode{\sphinxupquote{components.}}\sphinxbfcode{\sphinxupquote{L\_StraightFlatWire}}}{\emph{\DUrole{n}{arg}}, \emph{\DUrole{n}{defaultunits}\DUrole{o}{=}\DUrole{default_value}{{[}{]}}}}{}
Inductance of a flat wire.
\begin{quote}\begin{description}
\item[{Parameters}] \leavevmode\begin{itemize}
\item {} 
\sphinxstyleliteralstrong{\sphinxupquote{arg}} (\sphinxstyleliteralemphasis{\sphinxupquote{list}}) \textendash{} 
First 6 arguments are inputs.
\begin{enumerate}
\sphinxsetlistlabels{\arabic}{enumi}{enumii}{}{.}%
\item {} 
Wire Width ;length

\item {} 
Wire Thickness ;length

\item {} 
Wire Length ;length

\item {} 
Frequency ; frequency

\item {} 
Relative Permeability ;

\item {} 
Conductivity ; electrical conductivity

\item {} 
Inductance ;inductance

\end{enumerate}

8.  Impedance ;impedance
Reference:  Transmission Line Design Handbook, Wadell, s.382


\item {} 
\sphinxstyleliteralstrong{\sphinxupquote{defaultunits}} (\sphinxstyleliteralemphasis{\sphinxupquote{list}}\sphinxstyleliteralemphasis{\sphinxupquote{, }}\sphinxstyleliteralemphasis{\sphinxupquote{optional}}) \textendash{} Default units for quantities in \sphinxstyleemphasis{arg} list. Default is {[}{]} which means SI units will be used if no unit is given in \sphinxstyleemphasis{arg}.

\end{itemize}

\item[{Returns}] \leavevmode
arg

\item[{Return type}] \leavevmode
list

\end{description}\end{quote}

\end{fulllineitems}

\index{L\_StraightRoundWire() (in module components)@\spxentry{L\_StraightRoundWire()}\spxextra{in module components}}

\begin{fulllineitems}
\phantomsection\label{\detokenize{components:components.L_StraightRoundWire}}\pysiglinewithargsret{\sphinxcode{\sphinxupquote{components.}}\sphinxbfcode{\sphinxupquote{L\_StraightRoundWire}}}{\emph{\DUrole{n}{arg}}, \emph{\DUrole{n}{defaultunits}\DUrole{o}{=}\DUrole{default_value}{{[}{]}}}}{}
Inductance of a straight round wire.
\begin{quote}\begin{description}
\item[{Parameters}] \leavevmode\begin{itemize}
\item {} 
\sphinxstyleliteralstrong{\sphinxupquote{arg}} (\sphinxstyleliteralemphasis{\sphinxupquote{list}}) \textendash{} 
First 5 arguments are inputs.
\begin{enumerate}
\sphinxsetlistlabels{\arabic}{enumi}{enumii}{}{.}%
\item {} 
Wire Diameter ;length

\item {} 
Wire Length ;length

\item {} 
Frequency ; frequency

\item {} 
Dielectric Permeability  ;

\item {} 
Conductivity ; electrical conductivity

\item {} 
Inductance ;inductance

\end{enumerate}

7. Impedance ; impedance
Reference:  Transmission Line Design Handbook, Wadell, s.380


\item {} 
\sphinxstyleliteralstrong{\sphinxupquote{defaultunits}} (\sphinxstyleliteralemphasis{\sphinxupquote{list}}\sphinxstyleliteralemphasis{\sphinxupquote{, }}\sphinxstyleliteralemphasis{\sphinxupquote{optional}}) \textendash{} Default units for quantities in \sphinxstyleemphasis{arg} list. Default is {[}{]} which means SI units will be used if no unit is given in \sphinxstyleemphasis{arg}.

\end{itemize}

\item[{Returns}] \leavevmode
arg

\item[{Return type}] \leavevmode
list

\end{description}\end{quote}

\end{fulllineitems}

\index{L\_air\_core\_coil() (in module components)@\spxentry{L\_air\_core\_coil()}\spxextra{in module components}}

\begin{fulllineitems}
\phantomsection\label{\detokenize{components:components.L_air_core_coil}}\pysiglinewithargsret{\sphinxcode{\sphinxupquote{components.}}\sphinxbfcode{\sphinxupquote{L\_air\_core\_coil}}}{\emph{\DUrole{n}{arg}}, \emph{\DUrole{n}{defaultunits}\DUrole{o}{=}\DUrole{default_value}{{[}{]}}}}{}
Inductance of a via hole in microstrip.
\begin{quote}\begin{description}
\item[{Parameters}] \leavevmode\begin{itemize}
\item {} 
\sphinxstyleliteralstrong{\sphinxupquote{arg}} (\sphinxstyleliteralemphasis{\sphinxupquote{list}}) \textendash{} 
First 4 arguments are inputs.
\begin{enumerate}
\sphinxsetlistlabels{\arabic}{enumi}{enumii}{}{.}%
\item {} 
Wire Diameter (d) ;length

\item {} 
Coil Inner Diameter (d\_in) ;length

\item {} 
Spacing Between Turns (s) ; length

\item {} 
Number Of Turns ;

\item {} 
Inductance ; inductance

\end{enumerate}

6. Resonance Frequency ; frequency
Reference:  www.microwavecoil.com , Microwave Components Inc.


\item {} 
\sphinxstyleliteralstrong{\sphinxupquote{defaultunits}} (\sphinxstyleliteralemphasis{\sphinxupquote{list}}\sphinxstyleliteralemphasis{\sphinxupquote{, }}\sphinxstyleliteralemphasis{\sphinxupquote{optional}}) \textendash{} Default units for quantities in \sphinxstyleemphasis{arg} list. Default is {[}{]} which means SI units will be used if no unit is given in \sphinxstyleemphasis{arg}.

\end{itemize}

\item[{Returns}] \leavevmode
arg

\item[{Return type}] \leavevmode
list

\end{description}\end{quote}

\end{fulllineitems}

\index{L\_microstrip\_via\_hole() (in module components)@\spxentry{L\_microstrip\_via\_hole()}\spxextra{in module components}}

\begin{fulllineitems}
\phantomsection\label{\detokenize{components:components.L_microstrip_via_hole}}\pysiglinewithargsret{\sphinxcode{\sphinxupquote{components.}}\sphinxbfcode{\sphinxupquote{L\_microstrip\_via\_hole}}}{\emph{\DUrole{n}{arg}}, \emph{\DUrole{n}{defaultunits}\DUrole{o}{=}\DUrole{default_value}{{[}{]}}}}{}
Inductance of a via hole in microstrip.
\begin{quote}\begin{description}
\item[{Parameters}] \leavevmode\begin{itemize}
\item {} 
\sphinxstyleliteralstrong{\sphinxupquote{arg}} (\sphinxstyleliteralemphasis{\sphinxupquote{list}}) \textendash{} 
First 2 arguments are inputs.
\begin{enumerate}
\sphinxsetlistlabels{\arabic}{enumi}{enumii}{}{.}%
\item {} 
Via Radius ;length

\item {} 
Substrate Thickness ;length

\end{enumerate}

3. Inductance ; inductance
Reference:  Microstrip Via Hole Grounds in Microstrip.pdf


\item {} 
\sphinxstyleliteralstrong{\sphinxupquote{defaultunits}} (\sphinxstyleliteralemphasis{\sphinxupquote{list}}\sphinxstyleliteralemphasis{\sphinxupquote{, }}\sphinxstyleliteralemphasis{\sphinxupquote{optional}}) \textendash{} Default units for quantities in \sphinxstyleemphasis{arg} list. Default is {[}{]} which means SI units will be used if no unit is given in \sphinxstyleemphasis{arg}.

\end{itemize}

\item[{Returns}] \leavevmode
arg

\item[{Return type}] \leavevmode
list

\end{description}\end{quote}

\end{fulllineitems}

\index{OptimumMitered90DegMicrostripBend() (in module components)@\spxentry{OptimumMitered90DegMicrostripBend()}\spxextra{in module components}}

\begin{fulllineitems}
\phantomsection\label{\detokenize{components:components.OptimumMitered90DegMicrostripBend}}\pysiglinewithargsret{\sphinxcode{\sphinxupquote{components.}}\sphinxbfcode{\sphinxupquote{OptimumMitered90DegMicrostripBend}}}{\emph{\DUrole{n}{arg}}, \emph{\DUrole{n}{defaultunits}\DUrole{o}{=}\DUrole{default_value}{{[}{]}}}}{}
Optimum Mitered Microstrip Bend Parameters.
\begin{quote}\begin{description}
\item[{Parameters}] \leavevmode\begin{itemize}
\item {} 
\sphinxstyleliteralstrong{\sphinxupquote{arg}} (\sphinxstyleliteralemphasis{\sphinxupquote{list}}) \textendash{} 
First 2 arguments are inputs.
\begin{enumerate}
\sphinxsetlistlabels{\arabic}{enumi}{enumii}{}{.}%
\item {} 
Microstrip Width;length

\item {} 
Substrate Height;length

\end{enumerate}

3.  Miter Length; length
Reference: Tranmission line design handbook, p.290


\item {} 
\sphinxstyleliteralstrong{\sphinxupquote{defaultunits}} (\sphinxstyleliteralemphasis{\sphinxupquote{list}}\sphinxstyleliteralemphasis{\sphinxupquote{, }}\sphinxstyleliteralemphasis{\sphinxupquote{optional}}) \textendash{} Default units for quantities in \sphinxstyleemphasis{arg} list. Default is {[}{]} which means SI units will be used if no unit is given in \sphinxstyleemphasis{arg}.

\end{itemize}

\item[{Returns}] \leavevmode
arg

\item[{Return type}] \leavevmode
list

\end{description}\end{quote}

\end{fulllineitems}

\index{OptimumMiteredArbitraryAngleMicrostripBend() (in module components)@\spxentry{OptimumMiteredArbitraryAngleMicrostripBend()}\spxextra{in module components}}

\begin{fulllineitems}
\phantomsection\label{\detokenize{components:components.OptimumMiteredArbitraryAngleMicrostripBend}}\pysiglinewithargsret{\sphinxcode{\sphinxupquote{components.}}\sphinxbfcode{\sphinxupquote{OptimumMiteredArbitraryAngleMicrostripBend}}}{\emph{\DUrole{n}{arg}}, \emph{\DUrole{n}{defaultunits}\DUrole{o}{=}\DUrole{default_value}{{[}{]}}}}{}
Optimum Mitered Microstrip Bend Parameters.
\begin{quote}\begin{description}
\item[{Parameters}] \leavevmode
\sphinxstyleliteralstrong{\sphinxupquote{arg}} (\sphinxstyleliteralemphasis{\sphinxupquote{list}}) \textendash{} 
First 2 arguments are inputs.
\begin{enumerate}
\sphinxsetlistlabels{\arabic}{enumi}{enumii}{}{.}%
\item {} 
Microstrip Width;length;

\item {} 
Substrate Height;length;

\item {} 
Angle (0\sphinxhyphen{}180 degrees); angle ;

\end{enumerate}

4.  Miter Length; length ;
Reference: MWOHELP, MBENDA model


\end{description}\end{quote}

Burada scipy.interpolate.griddata kullanildi ve maalesef extrapolation yapmiyor. Sinir disi degerlerde dogrudan en yakin deger kullanildi.
\begin{quote}

defaultunits(list, optional): Default units for quantities in \sphinxstyleemphasis{arg} list. Default is {[}{]} which means SI units will be used if no unit is given in \sphinxstyleemphasis{arg}.
\end{quote}
\begin{quote}\begin{description}
\item[{Returns}] \leavevmode
arg

\item[{Return type}] \leavevmode
list

\end{description}\end{quote}

\end{fulllineitems}

\index{PCBTrackCurrentCapacity() (in module components)@\spxentry{PCBTrackCurrentCapacity()}\spxextra{in module components}}

\begin{fulllineitems}
\phantomsection\label{\detokenize{components:components.PCBTrackCurrentCapacity}}\pysiglinewithargsret{\sphinxcode{\sphinxupquote{components.}}\sphinxbfcode{\sphinxupquote{PCBTrackCurrentCapacity}}}{\emph{\DUrole{n}{arg}}, \emph{\DUrole{n}{defaultunits}\DUrole{o}{=}\DUrole{default_value}{{[}{]}}}}{}
PCB Track Current Capacity.
\begin{quote}\begin{description}
\item[{Parameters}] \leavevmode\begin{itemize}
\item {} 
\sphinxstyleliteralstrong{\sphinxupquote{arg}} (\sphinxstyleliteralemphasis{\sphinxupquote{list}}) \textendash{} 
First 7 arguments are inputs.
\begin{enumerate}
\sphinxsetlistlabels{\arabic}{enumi}{enumii}{}{.}%
\item {} 
Metal Width;  length

\item {} 
PCB Height;     length

\item {} 
Metal Thickness;        length

\item {} 
Allowable Temperature Rise; temperature

\item {} 
Thermal Conductivity;  thermal conductivity

\item {} 
Electrical Conductivity; electrical conductivity

\item {} 
External if 1, Internal if 0;

\end{enumerate}

8. Current ; current
Reference:


\item {} 
\sphinxstyleliteralstrong{\sphinxupquote{defaultunits}} (\sphinxstyleliteralemphasis{\sphinxupquote{list}}\sphinxstyleliteralemphasis{\sphinxupquote{, }}\sphinxstyleliteralemphasis{\sphinxupquote{optional}}) \textendash{} Default units for quantities in \sphinxstyleemphasis{arg} list. Default is {[}{]} which means SI units will be used if no unit is given in \sphinxstyleemphasis{arg}.

\end{itemize}

\item[{Returns}] \leavevmode
arg

\item[{Return type}] \leavevmode
list

\end{description}\end{quote}

\end{fulllineitems}

\index{PCBTrackCurrentCapacityIPC() (in module components)@\spxentry{PCBTrackCurrentCapacityIPC()}\spxextra{in module components}}

\begin{fulllineitems}
\phantomsection\label{\detokenize{components:components.PCBTrackCurrentCapacityIPC}}\pysiglinewithargsret{\sphinxcode{\sphinxupquote{components.}}\sphinxbfcode{\sphinxupquote{PCBTrackCurrentCapacityIPC}}}{\emph{\DUrole{n}{arg}}, \emph{\DUrole{n}{defaultunits}\DUrole{o}{=}\DUrole{default_value}{{[}{]}}}}{}
PCB Track Current Capacity, IPC.
\begin{quote}\begin{description}
\item[{Parameters}] \leavevmode\begin{itemize}
\item {} 
\sphinxstyleliteralstrong{\sphinxupquote{arg}} (\sphinxstyleliteralemphasis{\sphinxupquote{list}}) \textendash{} 
First 4 arguments are inputs.
\begin{enumerate}
\sphinxsetlistlabels{\arabic}{enumi}{enumii}{}{.}%
\item {} 
Metal Width;length

\item {} 
Metal Thickness;length

\item {} 
Allowable Temperature Rise; temperature

\item {} 
External if 1, Internal if 0;

\end{enumerate}

5. Current ; current
Reference:  IPC2221A


\item {} 
\sphinxstyleliteralstrong{\sphinxupquote{defaultunits}} (\sphinxstyleliteralemphasis{\sphinxupquote{list}}\sphinxstyleliteralemphasis{\sphinxupquote{, }}\sphinxstyleliteralemphasis{\sphinxupquote{optional}}) \textendash{} Default units for quantities in \sphinxstyleemphasis{arg} list. Default is {[}{]} which means SI units will be used if no unit is given in \sphinxstyleemphasis{arg}.

\end{itemize}

\item[{Returns}] \leavevmode
arg

\item[{Return type}] \leavevmode
list

\end{description}\end{quote}

\end{fulllineitems}

\index{ParallelPlateCap() (in module components)@\spxentry{ParallelPlateCap()}\spxextra{in module components}}

\begin{fulllineitems}
\phantomsection\label{\detokenize{components:components.ParallelPlateCap}}\pysiglinewithargsret{\sphinxcode{\sphinxupquote{components.}}\sphinxbfcode{\sphinxupquote{ParallelPlateCap}}}{\emph{\DUrole{n}{arg}}, \emph{\DUrole{n}{defaultunits}\DUrole{o}{=}\DUrole{default_value}{{[}{]}}}}{}
Parallel Plate Capacitance.
\begin{quote}\begin{description}
\item[{Parameters}] \leavevmode\begin{itemize}
\item {} 
\sphinxstyleliteralstrong{\sphinxupquote{arg}} (\sphinxstyleliteralemphasis{\sphinxupquote{list}}) \textendash{} 
First 4 arguments are inputs.
\begin{enumerate}
\sphinxsetlistlabels{\arabic}{enumi}{enumii}{}{.}%
\item {} 
Width;length

\item {} 
Length;length

\item {} 
Height;length

\item {} 
Dielectric Permittivity;

\end{enumerate}

5. Capacitance; capacitance
Reference:


\item {} 
\sphinxstyleliteralstrong{\sphinxupquote{defaultunits}} (\sphinxstyleliteralemphasis{\sphinxupquote{list}}\sphinxstyleliteralemphasis{\sphinxupquote{, }}\sphinxstyleliteralemphasis{\sphinxupquote{optional}}) \textendash{} Default units for quantities in \sphinxstyleemphasis{arg} list. Default is {[}{]} which means SI units will be used if no unit is given in \sphinxstyleemphasis{arg}.

\end{itemize}

\item[{Returns}] \leavevmode
arg

\item[{Return type}] \leavevmode
list

\end{description}\end{quote}

\end{fulllineitems}

\index{Patch\_Antenna\_Analysis() (in module components)@\spxentry{Patch\_Antenna\_Analysis()}\spxextra{in module components}}

\begin{fulllineitems}
\phantomsection\label{\detokenize{components:components.Patch_Antenna_Analysis}}\pysiglinewithargsret{\sphinxcode{\sphinxupquote{components.}}\sphinxbfcode{\sphinxupquote{Patch\_Antenna\_Analysis}}}{\emph{\DUrole{n}{arg}}, \emph{\DUrole{n}{defaultunits}\DUrole{o}{=}\DUrole{default_value}{{[}{]}}}}{}
Calculates performance and impedance values for an N\sphinxhyphen{}section Chebyshev Impedance Taper.
Ref: Overview of Microstrip Antennas (Jackson) (Presentation)
\begin{quote}\begin{description}
\item[{Parameters}] \leavevmode\begin{itemize}
\item {} 
\sphinxstyleliteralstrong{\sphinxupquote{arg}} (\sphinxstyleliteralemphasis{\sphinxupquote{list}}) \textendash{} 
First 6 arguments are inputs.
\begin{enumerate}
\sphinxsetlistlabels{\arabic}{enumi}{enumii}{}{.}%
\item {} 
Width (W) ; length

\item {} 
Length (L) ; length

\item {} 
Substrate Thickness (h);length

\item {} 
Dielectric Permittivity ;

\item {} 
Dielectric Loss Tangent ;

\item {} 
Metal Conductivity ; electrical conductivity

\item {} 
Resonance Frequency (f) ; frequency

\end{enumerate}

8.  Bandwidth ; frequency
Reference:  Foundations for Microwave Engineering, Collin


\item {} 
\sphinxstyleliteralstrong{\sphinxupquote{defaultunits}} (\sphinxstyleliteralemphasis{\sphinxupquote{list}}\sphinxstyleliteralemphasis{\sphinxupquote{, }}\sphinxstyleliteralemphasis{\sphinxupquote{optional}}) \textendash{} Default units for quantities in \sphinxstyleemphasis{arg} list. Default is {[}{]} which means SI units will be used if no unit is given in \sphinxstyleemphasis{arg}.

\end{itemize}

\item[{Returns}] \leavevmode
arg

\item[{Return type}] \leavevmode
list

\end{description}\end{quote}

\end{fulllineitems}

\index{Pi\_Attenuator\_Analysis() (in module components)@\spxentry{Pi\_Attenuator\_Analysis()}\spxextra{in module components}}

\begin{fulllineitems}
\phantomsection\label{\detokenize{components:components.Pi_Attenuator_Analysis}}\pysiglinewithargsret{\sphinxcode{\sphinxupquote{components.}}\sphinxbfcode{\sphinxupquote{Pi\_Attenuator\_Analysis}}}{\emph{\DUrole{n}{arg}}, \emph{\DUrole{n}{defaultunits}\DUrole{o}{=}\DUrole{default_value}{{[}{]}}}}{}
Pi Attenuator Analysis.
\begin{quote}\begin{description}
\item[{Parameters}] \leavevmode\begin{itemize}
\item {} 
\sphinxstyleliteralstrong{\sphinxupquote{arg}} (\sphinxstyleliteralemphasis{\sphinxupquote{list}}) \textendash{} 
First 3 arguments are inputs.
\begin{enumerate}
\sphinxsetlistlabels{\arabic}{enumi}{enumii}{}{.}%
\item {} 
Reference Impedance (Zo); impedance

\item {} 
Series Impedance (Rs); impedance

\item {} 
Parallel Impedance (Rp); impedance

\item {} 
S(1,1) ;

\item {} 
S(2,1) ;

\item {} 
P1 ;

\item {} 
P2 ;

\end{enumerate}

8. P3 ;
Reference:


\item {} 
\sphinxstyleliteralstrong{\sphinxupquote{defaultunits}} (\sphinxstyleliteralemphasis{\sphinxupquote{list}}\sphinxstyleliteralemphasis{\sphinxupquote{, }}\sphinxstyleliteralemphasis{\sphinxupquote{optional}}) \textendash{} Default units for quantities in \sphinxstyleemphasis{arg} list. Default is {[}{]} which means SI units will be used if no unit is given in \sphinxstyleemphasis{arg}.

\end{itemize}

\item[{Returns}] \leavevmode
arg

\item[{Return type}] \leavevmode
list

\end{description}\end{quote}

\end{fulllineitems}

\index{Pi\_Attenuator\_Synthesis() (in module components)@\spxentry{Pi\_Attenuator\_Synthesis()}\spxextra{in module components}}

\begin{fulllineitems}
\phantomsection\label{\detokenize{components:components.Pi_Attenuator_Synthesis}}\pysiglinewithargsret{\sphinxcode{\sphinxupquote{components.}}\sphinxbfcode{\sphinxupquote{Pi\_Attenuator\_Synthesis}}}{\emph{\DUrole{n}{arg}}, \emph{\DUrole{n}{defaultunits}\DUrole{o}{=}\DUrole{default_value}{{[}{]}}}}{}
Pi Attenuator Analysis.
\begin{quote}\begin{description}
\item[{Parameters}] \leavevmode\begin{itemize}
\item {} 
\sphinxstyleliteralstrong{\sphinxupquote{arg}} (\sphinxstyleliteralemphasis{\sphinxupquote{list}}) \textendash{} 
First 3 arguments are inputs.
\begin{enumerate}
\sphinxsetlistlabels{\arabic}{enumi}{enumii}{}{.}%
\item {} 
Reference Impedance (Zo); impedance

\item {} 
Series Impedance (Rs); impedance

\item {} 
Parallel Impedance (Rp); impedance

\item {} 
S(1,1) ;

\item {} 
S(2,1) ;

\item {} 
P1 ;

\item {} 
P2 ;

\end{enumerate}

8. P3 ;
Reference:


\item {} 
\sphinxstyleliteralstrong{\sphinxupquote{defaultunits}} (\sphinxstyleliteralemphasis{\sphinxupquote{list}}\sphinxstyleliteralemphasis{\sphinxupquote{, }}\sphinxstyleliteralemphasis{\sphinxupquote{optional}}) \textendash{} Default units for quantities in \sphinxstyleemphasis{arg} list. Default is {[}{]} which means SI units will be used if no unit is given in \sphinxstyleemphasis{arg}.

\end{itemize}

\item[{Returns}] \leavevmode
arg

\item[{Return type}] \leavevmode
list

\end{description}\end{quote}

\end{fulllineitems}

\index{RectWG2EvanescentRectWGStep() (in module components)@\spxentry{RectWG2EvanescentRectWGStep()}\spxextra{in module components}}

\begin{fulllineitems}
\phantomsection\label{\detokenize{components:components.RectWG2EvanescentRectWGStep}}\pysiglinewithargsret{\sphinxcode{\sphinxupquote{components.}}\sphinxbfcode{\sphinxupquote{RectWG2EvanescentRectWGStep}}}{\emph{\DUrole{n}{a1}}, \emph{\DUrole{n}{a2}}}{}
Waveguide Width Step from Rectangular Waveguide to Evanescent Mode Rectangular Waveguide.
\begin{quote}\begin{description}
\item[{Parameters}] \leavevmode\begin{itemize}
\item {} 
\sphinxstyleliteralstrong{\sphinxupquote{arg}} (\sphinxstyleliteralemphasis{\sphinxupquote{list}}) \textendash{} 
First 2 arguments are inputs.
\begin{enumerate}
\sphinxsetlistlabels{\arabic}{enumi}{enumii}{}{.}%
\item {} 
Width of Rectangular Waveguide;length;

\item {} 
Width of Evanescent Mode Rectangular Waveguide;length;

\item {} 
Inductance; inductance

\end{enumerate}

4. Turns Ratio;
Reference:  The Design of Evanescent Mode Waveguide Bandpass Filters for a Prescribed Insertion Loss Characteristic.pdf


\item {} 
\sphinxstyleliteralstrong{\sphinxupquote{defaultunits}} (\sphinxstyleliteralemphasis{\sphinxupquote{list}}\sphinxstyleliteralemphasis{\sphinxupquote{, }}\sphinxstyleliteralemphasis{\sphinxupquote{optional}}) \textendash{} Default units for quantities in \sphinxstyleemphasis{arg} list. Default is {[}{]} which means SI units will be used if no unit is given in \sphinxstyleemphasis{arg}.

\end{itemize}

\item[{Returns}] \leavevmode
arg

\item[{Return type}] \leavevmode
list

\end{description}\end{quote}

\end{fulllineitems}

\index{SIW\_EquivalentWidth() (in module components)@\spxentry{SIW\_EquivalentWidth()}\spxextra{in module components}}

\begin{fulllineitems}
\phantomsection\label{\detokenize{components:components.SIW_EquivalentWidth}}\pysiglinewithargsret{\sphinxcode{\sphinxupquote{components.}}\sphinxbfcode{\sphinxupquote{SIW\_EquivalentWidth}}}{\emph{\DUrole{n}{w}}, \emph{\DUrole{n}{d}}, \emph{\DUrole{n}{s}}}{}
Equivalent width of substrate integrated waveguide.
\begin{quote}\begin{description}
\item[{Parameters}] \leavevmode\begin{itemize}
\item {} 
\sphinxstyleliteralstrong{\sphinxupquote{w}} (\sphinxstyleliteralemphasis{\sphinxupquote{float}}) \textendash{} Distance between the centers of two via arrays.

\item {} 
\sphinxstyleliteralstrong{\sphinxupquote{d}} (\sphinxstyleliteralemphasis{\sphinxupquote{float}}) \textendash{} Diameter of vias.

\item {} 
\sphinxstyleliteralstrong{\sphinxupquote{s}} (\sphinxstyleliteralemphasis{\sphinxupquote{float}}) \textendash{} Distance between the centers of consecutive vias of via arrays.

\end{itemize}

\item[{Returns}] \leavevmode
Equivalent width of waveguide.

\item[{Return type}] \leavevmode
float

\end{description}\end{quote}

\end{fulllineitems}

\index{Shorten90DegreeLine() (in module components)@\spxentry{Shorten90DegreeLine()}\spxextra{in module components}}

\begin{fulllineitems}
\phantomsection\label{\detokenize{components:components.Shorten90DegreeLine}}\pysiglinewithargsret{\sphinxcode{\sphinxupquote{components.}}\sphinxbfcode{\sphinxupquote{Shorten90DegreeLine}}}{\emph{\DUrole{n}{arg}}, \emph{\DUrole{n}{defaultunits}\DUrole{o}{=}\DUrole{default_value}{{[}{]}}}}{}
Shortening 90 Degree Line with a capacitive load.
\begin{quote}\begin{description}
\item[{Parameters}] \leavevmode\begin{itemize}
\item {} 
\sphinxstyleliteralstrong{\sphinxupquote{arg}} (\sphinxstyleliteralemphasis{\sphinxupquote{list}}) \textendash{} 
First 3 arguments are inputs.
\begin{enumerate}
\sphinxsetlistlabels{\arabic}{enumi}{enumii}{}{.}%
\item {} 
Impedance (Zo); impedance

\item {} 
Center Frequency ;  frequency

\item {} 
Electrical Length (theta) ; angle

\item {} 
Impedance (Z); impedance

\end{enumerate}

5. Capacitance ; capacitance
Reference:


\item {} 
\sphinxstyleliteralstrong{\sphinxupquote{defaultunits}} (\sphinxstyleliteralemphasis{\sphinxupquote{list}}\sphinxstyleliteralemphasis{\sphinxupquote{, }}\sphinxstyleliteralemphasis{\sphinxupquote{optional}}) \textendash{} Default units for quantities in \sphinxstyleemphasis{arg} list. Default is {[}{]} which means SI units will be used if no unit is given in \sphinxstyleemphasis{arg}.

\end{itemize}

\item[{Returns}] \leavevmode
arg

\item[{Return type}] \leavevmode
list

\end{description}\end{quote}

\end{fulllineitems}

\index{Star2TriangleTransformation() (in module components)@\spxentry{Star2TriangleTransformation()}\spxextra{in module components}}

\begin{fulllineitems}
\phantomsection\label{\detokenize{components:components.Star2TriangleTransformation}}\pysiglinewithargsret{\sphinxcode{\sphinxupquote{components.}}\sphinxbfcode{\sphinxupquote{Star2TriangleTransformation}}}{\emph{\DUrole{n}{arg}}, \emph{\DUrole{n}{defaultunits}\DUrole{o}{=}\DUrole{default_value}{{[}{]}}}}{}
Star network to Triangle network transformation.
\begin{quote}\begin{description}
\item[{Parameters}] \leavevmode\begin{itemize}
\item {} 
\sphinxstyleliteralstrong{\sphinxupquote{arg}} (\sphinxstyleliteralemphasis{\sphinxupquote{list}}) \textendash{} 
First 3 arguments are inputs.
\begin{enumerate}
\sphinxsetlistlabels{\arabic}{enumi}{enumii}{}{.}%
\item {} 
Z1; impedance

\item {} 
Z2; impedance

\item {} 
Z3; impedance

\item {} 
Z1’; impedance

\item {} 
Z2’; impedance

\end{enumerate}

6. Z3’; impedance
Reference:
At star, z1 is connected to A\sphinxhyphen{}node, z2 is connected to B\sphinxhyphen{}node, z3 is connected to C\sphinxhyphen{}node
At triangle, z1 is between A\sphinxhyphen{}B, z2 is between A\sphinxhyphen{}C, z3 is between B\sphinxhyphen{}C


\item {} 
\sphinxstyleliteralstrong{\sphinxupquote{defaultunits}} (\sphinxstyleliteralemphasis{\sphinxupquote{list}}\sphinxstyleliteralemphasis{\sphinxupquote{, }}\sphinxstyleliteralemphasis{\sphinxupquote{optional}}) \textendash{} Default units for quantities in \sphinxstyleemphasis{arg} list. Default is {[}{]} which means SI units will be used if no unit is given in \sphinxstyleemphasis{arg}.

\end{itemize}

\item[{Returns}] \leavevmode
arg

\item[{Return type}] \leavevmode
list

\end{description}\end{quote}

\end{fulllineitems}

\index{SymmetricLangeCoupler() (in module components)@\spxentry{SymmetricLangeCoupler()}\spxextra{in module components}}

\begin{fulllineitems}
\phantomsection\label{\detokenize{components:components.SymmetricLangeCoupler}}\pysiglinewithargsret{\sphinxcode{\sphinxupquote{components.}}\sphinxbfcode{\sphinxupquote{SymmetricLangeCoupler}}}{\emph{\DUrole{n}{arg}}, \emph{\DUrole{n}{defaultunits}\DUrole{o}{=}\DUrole{default_value}{{[}{]}}}}{}
Symmetric Lange Coupler.
\begin{quote}\begin{description}
\item[{Parameters}] \leavevmode\begin{itemize}
\item {} 
\sphinxstyleliteralstrong{\sphinxupquote{arg}} (\sphinxstyleliteralemphasis{\sphinxupquote{list}}) \textendash{} 
First 3 arguments are inputs.
\begin{enumerate}
\sphinxsetlistlabels{\arabic}{enumi}{enumii}{}{.}%
\item {} 
C: Voltage coupling coefficient in dB (positive);

\item {} 
n: Number of fingers (should be even);

\item {} 
Reference Impedance;impedance

\item {} 
Zoo;impedance

\end{enumerate}

5. Zoe;impedance
Reference:  Microwave Circuits, Analysis and Computer\sphinxhyphen{}Aided Design, Fusco


\item {} 
\sphinxstyleliteralstrong{\sphinxupquote{defaultunits}} (\sphinxstyleliteralemphasis{\sphinxupquote{list}}\sphinxstyleliteralemphasis{\sphinxupquote{, }}\sphinxstyleliteralemphasis{\sphinxupquote{optional}}) \textendash{} Default units for quantities in \sphinxstyleemphasis{arg} list. Default is {[}{]} which means SI units will be used if no unit is given in \sphinxstyleemphasis{arg}.

\end{itemize}

\item[{Returns}] \leavevmode
arg

\item[{Return type}] \leavevmode
list

\end{description}\end{quote}

\end{fulllineitems}

\index{Tee\_Attenuator\_Analysis() (in module components)@\spxentry{Tee\_Attenuator\_Analysis()}\spxextra{in module components}}

\begin{fulllineitems}
\phantomsection\label{\detokenize{components:components.Tee_Attenuator_Analysis}}\pysiglinewithargsret{\sphinxcode{\sphinxupquote{components.}}\sphinxbfcode{\sphinxupquote{Tee\_Attenuator\_Analysis}}}{\emph{\DUrole{n}{arg}}, \emph{\DUrole{n}{defaultunits}\DUrole{o}{=}\DUrole{default_value}{{[}{]}}}}{}
Tee Attenuator Analysis.
\begin{quote}\begin{description}
\item[{Parameters}] \leavevmode\begin{itemize}
\item {} 
\sphinxstyleliteralstrong{\sphinxupquote{arg}} (\sphinxstyleliteralemphasis{\sphinxupquote{list}}) \textendash{} 
First 3 arguments are inputs.
\begin{enumerate}
\sphinxsetlistlabels{\arabic}{enumi}{enumii}{}{.}%
\item {} 
Reference Impedance (Zo); impedance

\item {} 
Series Impedance (Rs); impedance

\item {} 
Parallel Impedance (Rp); impedance

\item {} 
S(1,1) ;

\item {} 
S(2,1) ;

\item {} 
P1 ;

\item {} 
P2 ;

\end{enumerate}

8. P3 ;
Reference:


\item {} 
\sphinxstyleliteralstrong{\sphinxupquote{defaultunits}} (\sphinxstyleliteralemphasis{\sphinxupquote{list}}\sphinxstyleliteralemphasis{\sphinxupquote{, }}\sphinxstyleliteralemphasis{\sphinxupquote{optional}}) \textendash{} Default units for quantities in \sphinxstyleemphasis{arg} list. Default is {[}{]} which means SI units will be used if no unit is given in \sphinxstyleemphasis{arg}.

\end{itemize}

\item[{Returns}] \leavevmode
arg

\item[{Return type}] \leavevmode
list

\end{description}\end{quote}

\end{fulllineitems}

\index{Tee\_Attenuator\_Synthesis() (in module components)@\spxentry{Tee\_Attenuator\_Synthesis()}\spxextra{in module components}}

\begin{fulllineitems}
\phantomsection\label{\detokenize{components:components.Tee_Attenuator_Synthesis}}\pysiglinewithargsret{\sphinxcode{\sphinxupquote{components.}}\sphinxbfcode{\sphinxupquote{Tee\_Attenuator\_Synthesis}}}{\emph{\DUrole{n}{arg}}, \emph{\DUrole{n}{defaultunits}\DUrole{o}{=}\DUrole{default_value}{{[}{]}}}}{}
Tee Attenuator Synthesis.
\begin{quote}\begin{description}
\item[{Parameters}] \leavevmode\begin{itemize}
\item {} 
\sphinxstyleliteralstrong{\sphinxupquote{arg}} (\sphinxstyleliteralemphasis{\sphinxupquote{list}}) \textendash{} 
First 5 arguments are inputs.
\begin{enumerate}
\sphinxsetlistlabels{\arabic}{enumi}{enumii}{}{.}%
\item {} 
Reference Impedance (Zo); impedance

\item {} 
Series Impedance (Rs); impedance

\item {} 
Parallel Impedance (Rp); impedance

\item {} 
S(1,1) ;

\item {} 
S(2,1) ;

\item {} 
P1 ;

\item {} 
P2 ;

\end{enumerate}

8. P3 ;
Reference:


\item {} 
\sphinxstyleliteralstrong{\sphinxupquote{defaultunits}} (\sphinxstyleliteralemphasis{\sphinxupquote{list}}\sphinxstyleliteralemphasis{\sphinxupquote{, }}\sphinxstyleliteralemphasis{\sphinxupquote{optional}}) \textendash{} Default units for quantities in \sphinxstyleemphasis{arg} list. Default is {[}{]} which means SI units will be used if no unit is given in \sphinxstyleemphasis{arg}.

\end{itemize}

\item[{Returns}] \leavevmode
arg

\item[{Return type}] \leavevmode
list

\end{description}\end{quote}

\end{fulllineitems}

\index{Triangle2StarTransformation() (in module components)@\spxentry{Triangle2StarTransformation()}\spxextra{in module components}}

\begin{fulllineitems}
\phantomsection\label{\detokenize{components:components.Triangle2StarTransformation}}\pysiglinewithargsret{\sphinxcode{\sphinxupquote{components.}}\sphinxbfcode{\sphinxupquote{Triangle2StarTransformation}}}{\emph{\DUrole{n}{arg}}, \emph{\DUrole{n}{defaultunits}\DUrole{o}{=}\DUrole{default_value}{{[}{]}}}}{}
Triangle network to Star network transformation.
\begin{quote}\begin{description}
\item[{Parameters}] \leavevmode\begin{itemize}
\item {} 
\sphinxstyleliteralstrong{\sphinxupquote{arg}} (\sphinxstyleliteralemphasis{\sphinxupquote{list}}) \textendash{} 
Last 3 arguments are inputs.
\begin{enumerate}
\sphinxsetlistlabels{\arabic}{enumi}{enumii}{}{.}%
\item {} 
Z1; impedance

\item {} 
Z2; impedance

\item {} 
Z3; impedance

\item {} 
Z1’; impedance

\item {} 
Z2’; impedance

\end{enumerate}

6. Z3’; impedance
Reference:
At star, z1 is connected to A\sphinxhyphen{}node, z2 is connected to B\sphinxhyphen{}node, z3 is connected to C\sphinxhyphen{}node
At triangle, z1’ is between A\sphinxhyphen{}B, z2’ is between A\sphinxhyphen{}C, z3’ is between B\sphinxhyphen{}C


\item {} 
\sphinxstyleliteralstrong{\sphinxupquote{defaultunits}} (\sphinxstyleliteralemphasis{\sphinxupquote{list}}\sphinxstyleliteralemphasis{\sphinxupquote{, }}\sphinxstyleliteralemphasis{\sphinxupquote{optional}}) \textendash{} Default units for quantities in \sphinxstyleemphasis{arg} list. Default is {[}{]} which means SI units will be used if no unit is given in \sphinxstyleemphasis{arg}.

\end{itemize}

\item[{Returns}] \leavevmode
arg

\item[{Return type}] \leavevmode
list

\end{description}\end{quote}

\end{fulllineitems}

\index{Triangular\_Taper\_Impedance\_Transformer() (in module components)@\spxentry{Triangular\_Taper\_Impedance\_Transformer()}\spxextra{in module components}}

\begin{fulllineitems}
\phantomsection\label{\detokenize{components:components.Triangular_Taper_Impedance_Transformer}}\pysiglinewithargsret{\sphinxcode{\sphinxupquote{components.}}\sphinxbfcode{\sphinxupquote{Triangular\_Taper\_Impedance\_Transformer}}}{\emph{\DUrole{n}{arg}}, \emph{\DUrole{n}{defaultunits}\DUrole{o}{=}\DUrole{default_value}{{[}{]}}}}{}
Triangular Impedance Taper.
\begin{quote}\begin{description}
\item[{Parameters}] \leavevmode\begin{itemize}
\item {} 
\sphinxstyleliteralstrong{\sphinxupquote{arg}} (\sphinxstyleliteralemphasis{\sphinxupquote{list}}) \textendash{} 
First 5 arguments are inputs.
\begin{enumerate}
\sphinxsetlistlabels{\arabic}{enumi}{enumii}{}{.}%
\item {} 
Source Impedance ; impedance

\item {} 
Load Impedance ; impedance

\item {} 
Number Of Sections (Even) ;

\item {} 
Fractional Bandwidth (F2/F1) ;

\item {} 
Length (normalized to Lambda at fcenter) ;

\item {} 
Impedances ; impedance

\end{enumerate}

7.  Return Loss ;
Reference:  Foundations for Microwave Engineering, Collin


\item {} 
\sphinxstyleliteralstrong{\sphinxupquote{defaultunits}} (\sphinxstyleliteralemphasis{\sphinxupquote{list}}\sphinxstyleliteralemphasis{\sphinxupquote{, }}\sphinxstyleliteralemphasis{\sphinxupquote{optional}}) \textendash{} Default units for quantities in \sphinxstyleemphasis{arg} list. Default is {[}{]} which means SI units will be used if no unit is given in \sphinxstyleemphasis{arg}.

\end{itemize}

\item[{Returns}] \leavevmode
arg

\item[{Return type}] \leavevmode
list

\end{description}\end{quote}

\end{fulllineitems}

\index{Z\_CWG() (in module components)@\spxentry{Z\_CWG()}\spxextra{in module components}}

\begin{fulllineitems}
\phantomsection\label{\detokenize{components:components.Z_CWG}}\pysiglinewithargsret{\sphinxcode{\sphinxupquote{components.}}\sphinxbfcode{\sphinxupquote{Z\_CWG}}}{\emph{\DUrole{n}{rad}}, \emph{\DUrole{n}{freq}}, \emph{\DUrole{n}{eps\_r}\DUrole{o}{=}\DUrole{default_value}{1}}, \emph{\DUrole{n}{v}\DUrole{o}{=}\DUrole{default_value}{0}}, \emph{\DUrole{n}{n}\DUrole{o}{=}\DUrole{default_value}{1}}, \emph{\DUrole{n}{mode}\DUrole{o}{=}\DUrole{default_value}{\textquotesingle{}TE\textquotesingle{}}}}{}
Computes the wave impedance of circular waveguide.
\begin{quote}\begin{description}
\item[{Parameters}] \leavevmode\begin{itemize}
\item {} 
\sphinxstyleliteralstrong{\sphinxupquote{v}} (\sphinxstyleliteralemphasis{\sphinxupquote{int}}) \textendash{} Mode number of \(\phi\).

\item {} 
\sphinxstyleliteralstrong{\sphinxupquote{n}} (\sphinxstyleliteralemphasis{\sphinxupquote{int}}) \textendash{} Radial mode number.

\item {} 
\sphinxstyleliteralstrong{\sphinxupquote{eps\_r}} (\sphinxstyleliteralemphasis{\sphinxupquote{float}}) \textendash{} Permittivity of filling material.

\item {} 
\sphinxstyleliteralstrong{\sphinxupquote{freq}} (\sphinxstyleliteralemphasis{\sphinxupquote{float}}) \textendash{} Frequency (Hz).

\item {} 
\sphinxstyleliteralstrong{\sphinxupquote{mode}} (\sphinxstyleliteralemphasis{\sphinxupquote{str}}) \textendash{} “TE” or “TM”.

\item {} 
\sphinxstyleliteralstrong{\sphinxupquote{rad}} (\sphinxstyleliteralemphasis{\sphinxupquote{float}}) \textendash{} Radius.

\end{itemize}

\item[{Returns}] \leavevmode
Impedance.

\item[{Return type}] \leavevmode
Z (float)

\end{description}\end{quote}

\end{fulllineitems}

\index{Z\_WG\_TE10() (in module components)@\spxentry{Z\_WG\_TE10()}\spxextra{in module components}}

\begin{fulllineitems}
\phantomsection\label{\detokenize{components:components.Z_WG_TE10}}\pysiglinewithargsret{\sphinxcode{\sphinxupquote{components.}}\sphinxbfcode{\sphinxupquote{Z\_WG\_TE10}}}{\emph{\DUrole{n}{er}}, \emph{\DUrole{n}{a}}, \emph{\DUrole{n}{b}}, \emph{\DUrole{n}{freq}}, \emph{\DUrole{n}{formulation}\DUrole{o}{=}\DUrole{default_value}{1}}}{}
\end{fulllineitems}

\index{Zo\_eeff\_StraightWireOverSubstrate() (in module components)@\spxentry{Zo\_eeff\_StraightWireOverSubstrate()}\spxextra{in module components}}

\begin{fulllineitems}
\phantomsection\label{\detokenize{components:components.Zo_eeff_StraightWireOverSubstrate}}\pysiglinewithargsret{\sphinxcode{\sphinxupquote{components.}}\sphinxbfcode{\sphinxupquote{Zo\_eeff\_StraightWireOverSubstrate}}}{\emph{\DUrole{n}{arg}}, \emph{\DUrole{n}{defaultunits}\DUrole{o}{=}\DUrole{default_value}{{[}{]}}}}{}
Impedance and Effective Permittivity of Straight Wire Over Substrate.
\begin{quote}\begin{description}
\item[{Parameters}] \leavevmode\begin{itemize}
\item {} 
\sphinxstyleliteralstrong{\sphinxupquote{arg}} (\sphinxstyleliteralemphasis{\sphinxupquote{list}}) \textendash{} 
First 4 arguments are inputs.
\begin{enumerate}
\sphinxsetlistlabels{\arabic}{enumi}{enumii}{}{.}%
\item {} 
Wire Diameter (d);length

\item {} 
Height Of Wire Center Above Ground (h);length

\item {} 
Dielectric Thickness (t);length

\item {} 
Dielectric Permittivity ;

\item {} 
Impedance ; impedance

\end{enumerate}

6.  Effective Diel. Permittivity ;
Reference:  Transmission Line Design Handbook, Wadell, s.151


\item {} 
\sphinxstyleliteralstrong{\sphinxupquote{defaultunits}} (\sphinxstyleliteralemphasis{\sphinxupquote{list}}\sphinxstyleliteralemphasis{\sphinxupquote{, }}\sphinxstyleliteralemphasis{\sphinxupquote{optional}}) \textendash{} Default units for quantities in \sphinxstyleemphasis{arg} list. Default is {[}{]} which means SI units will be used if no unit is given in \sphinxstyleemphasis{arg}.

\end{itemize}

\item[{Returns}] \leavevmode
arg

\item[{Return type}] \leavevmode
list

\end{description}\end{quote}

\end{fulllineitems}

\index{Zo\_eeff\_WireOnGroundedSubstrate() (in module components)@\spxentry{Zo\_eeff\_WireOnGroundedSubstrate()}\spxextra{in module components}}

\begin{fulllineitems}
\phantomsection\label{\detokenize{components:components.Zo_eeff_WireOnGroundedSubstrate}}\pysiglinewithargsret{\sphinxcode{\sphinxupquote{components.}}\sphinxbfcode{\sphinxupquote{Zo\_eeff\_WireOnGroundedSubstrate}}}{\emph{\DUrole{n}{arg}}, \emph{\DUrole{n}{defaultunits}\DUrole{o}{=}\DUrole{default_value}{{[}{]}}}}{}
Impedance and Effective Permittivity of Straight Wire Over Substrate.
\begin{quote}\begin{description}
\item[{Parameters}] \leavevmode\begin{itemize}
\item {} 
\sphinxstyleliteralstrong{\sphinxupquote{arg}} (\sphinxstyleliteralemphasis{\sphinxupquote{list}}) \textendash{} 
First 4 arguments are inputs.
\begin{enumerate}
\sphinxsetlistlabels{\arabic}{enumi}{enumii}{}{.}%
\item {} 
Wire Diameter (d);length

\item {} 
Dielectric Thickness (t);length

\item {} 
Dielectric Permittivity ;

\item {} 
Impedance ; impedance

\end{enumerate}

5.  Effective Diel. Permittivity ;
Reference:  Transmission Line Design Handbook, Wadell, s.151
Note: eeff is the same as eeff of microstrip with w=2*d, t=0


\item {} 
\sphinxstyleliteralstrong{\sphinxupquote{defaultunits}} (\sphinxstyleliteralemphasis{\sphinxupquote{list}}\sphinxstyleliteralemphasis{\sphinxupquote{, }}\sphinxstyleliteralemphasis{\sphinxupquote{optional}}) \textendash{} Default units for quantities in \sphinxstyleemphasis{arg} list. Default is {[}{]} which means SI units will be used if no unit is given in \sphinxstyleemphasis{arg}.

\end{itemize}

\item[{Returns}] \leavevmode
arg

\item[{Return type}] \leavevmode
list

\end{description}\end{quote}

\end{fulllineitems}

\index{fcutoff\_CWG() (in module components)@\spxentry{fcutoff\_CWG()}\spxextra{in module components}}

\begin{fulllineitems}
\phantomsection\label{\detokenize{components:components.fcutoff_CWG}}\pysiglinewithargsret{\sphinxcode{\sphinxupquote{components.}}\sphinxbfcode{\sphinxupquote{fcutoff\_CWG}}}{\emph{\DUrole{n}{rad}}, \emph{\DUrole{n}{eps\_r}\DUrole{o}{=}\DUrole{default_value}{1}}, \emph{\DUrole{n}{v}\DUrole{o}{=}\DUrole{default_value}{0}}, \emph{\DUrole{n}{n}\DUrole{o}{=}\DUrole{default_value}{1}}, \emph{\DUrole{n}{mode}\DUrole{o}{=}\DUrole{default_value}{\textquotesingle{}TE\textquotesingle{}}}}{}
Computes the cutoff frequency of circular waveguide.
\begin{quote}\begin{description}
\item[{Parameters}] \leavevmode\begin{itemize}
\item {} 
\sphinxstyleliteralstrong{\sphinxupquote{v}} (\sphinxstyleliteralemphasis{\sphinxupquote{int}}) \textendash{} Mode number of \(\phi\).

\item {} 
\sphinxstyleliteralstrong{\sphinxupquote{n}} (\sphinxstyleliteralemphasis{\sphinxupquote{int}}) \textendash{} Radial mode number.

\item {} 
\sphinxstyleliteralstrong{\sphinxupquote{eps\_r}} (\sphinxstyleliteralemphasis{\sphinxupquote{float}}) \textendash{} Permittivity of filling material.

\item {} 
\sphinxstyleliteralstrong{\sphinxupquote{mode}} (\sphinxstyleliteralemphasis{\sphinxupquote{str}}) \textendash{} “TE” or “TM”.

\item {} 
\sphinxstyleliteralstrong{\sphinxupquote{rad}} (\sphinxstyleliteralemphasis{\sphinxupquote{float}}) \textendash{} Radius.

\end{itemize}

\item[{Returns}] \leavevmode
Cutoff frequency (Hz).

\item[{Return type}] \leavevmode
fc (float)

\end{description}\end{quote}

\end{fulllineitems}

\index{thermal\_conductance\_of\_via\_farm() (in module components)@\spxentry{thermal\_conductance\_of\_via\_farm()}\spxextra{in module components}}

\begin{fulllineitems}
\phantomsection\label{\detokenize{components:components.thermal_conductance_of_via_farm}}\pysiglinewithargsret{\sphinxcode{\sphinxupquote{components.}}\sphinxbfcode{\sphinxupquote{thermal\_conductance\_of\_via\_farm}}}{\emph{\DUrole{n}{arg}}, \emph{\DUrole{n}{defaultunits}}}{}
Thermal conductance of an array of vias in PCB.
\begin{quote}\begin{description}
\item[{Parameters}] \leavevmode\begin{itemize}
\item {} 
\sphinxstyleliteralstrong{\sphinxupquote{arg}} (\sphinxstyleliteralemphasis{\sphinxupquote{list}}) \textendash{} 
First 7 arguments are inputs.
\begin{enumerate}
\sphinxsetlistlabels{\arabic}{enumi}{enumii}{}{.}%
\item {} 
Plated Via Diameter (d);length

\item {} 
Plating Thickness (t);length

\item {} 
Area Width (w);length

\item {} 
Area Height (l);length

\item {} 
Dielectric Height (h);length

\item {} 
Number Of Vias (n);

\item {} 
Dielectric Thermal Conductivity ;   thermal conductivity

\item {} 
Metal Thermal Conductivity ; thermal conductivity

\item {} 
Thermal Conductance (W/K) ;

\item {} 
Thermal Resistance (K/W) ;

\end{enumerate}


\item {} 
\sphinxstyleliteralstrong{\sphinxupquote{defaultunits}} (\sphinxstyleliteralemphasis{\sphinxupquote{list}}\sphinxstyleliteralemphasis{\sphinxupquote{, }}\sphinxstyleliteralemphasis{\sphinxupquote{optional}}) \textendash{} Default units for quantities in \sphinxstyleemphasis{arg} list. Default is {[}{]} which means SI units will be used if no unit is given in \sphinxstyleemphasis{arg}.

\end{itemize}

\item[{Returns}] \leavevmode
arg

\item[{Return type}] \leavevmode
list

\end{description}\end{quote}

\end{fulllineitems}

\index{thermal\_conductance\_of\_via\_farm\_view() (in module components)@\spxentry{thermal\_conductance\_of\_via\_farm\_view()}\spxextra{in module components}}

\begin{fulllineitems}
\phantomsection\label{\detokenize{components:components.thermal_conductance_of_via_farm_view}}\pysiglinewithargsret{\sphinxcode{\sphinxupquote{components.}}\sphinxbfcode{\sphinxupquote{thermal\_conductance\_of\_via\_farm\_view}}}{\emph{\DUrole{n}{arg}}, \emph{\DUrole{n}{defaultunits}}}{}
\end{fulllineitems}



\chapter{Indices and tables}
\label{\detokenize{index:indices-and-tables}}\begin{itemize}
\item {} 
\DUrole{xref,std,std-ref}{genindex}

\item {} 
\DUrole{xref,std,std-ref}{modindex}

\item {} 
\DUrole{xref,std,std-ref}{search}

\end{itemize}


\renewcommand{\indexname}{Python Module Index}
\begin{sphinxtheindex}
\let\bigletter\sphinxstyleindexlettergroup
\bigletter{c}
\item\relax\sphinxstyleindexentry{components}\sphinxstyleindexpageref{components:\detokenize{module-components}}
\indexspace
\bigletter{t}
\item\relax\sphinxstyleindexentry{touchstone}\sphinxstyleindexpageref{touchstone:\detokenize{module-touchstone}}
\end{sphinxtheindex}

\renewcommand{\indexname}{Index}
\printindex
\end{document}